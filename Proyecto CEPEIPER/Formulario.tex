\documentclass[a4paper,10pt]{report}
\usepackage[utf8]{inputenc}
\usepackage{times}
\usepackage{pythontex}



% Title Page
\title{}
\author{}


\begin{document}
 \section*{Identificación de la propuesta y anclaje institucional}
 \begin{enumerate}
  \item \emph{Nombre de la idea-proyecto:} Material digital para 
  \emph{Ecuaciones Diferenciales}.
  \item \emph{Tema:} Ecuaciones Diferenciales Ordinarias.
  \item \emph{Asignatura:} Ecuaciones Diferenciales, Código 1913.
  \item \emph{Año en el plan de estudios:} Cuarto.
  \item \emph{Departamento:} Matemática.
  \item \emph{Facultad:} Facultad de Ciancias Exactas, Físico-Químicas y Naturales.
  
 \end{enumerate}
\section*{Destinatarios y alcance}
\begin{enumerate}
 \item \emph{Cantidad de estudiantes a los que se destinaría el material:} 2-4.
 \item \emph{Cantidad de docentes involucrados en el desarrollo de la asignatura:} 1.
 \item \emph{Otros actores involucrados:} Eventuales usuarios externos de los materiales
\end{enumerate}

\section*{Justificación:} 
\begin{enumerate}
 \item \emph{Problema o necesidad identificada a la cual se quiere 
 dar respuesta con el desarrollo de esta propuesta en general y del material en particular.}
 
\end{enumerate}
\begin{pyblock}
from sympy import *
a=symbols('a0:%s' %5)
print(a)
\end{pyblock}
\end{document}          
