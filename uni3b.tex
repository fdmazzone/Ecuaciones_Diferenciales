%\documentclass{article}
%\documentclass[hyperref={colorlinks=true}]{beamer}
\documentclass[handout,hyperref={colorlinks=true}]{beamer}


%%%%%%%%%%%%%%%%%%%%%%%%%%%%%%Paquetes%%%%%%%%%%%%%%%%%%%%%%%%%%%%%%%%%%%%%%%%%%%%%%%5
%%%%%%%%%%%%%%%%%%%%%%%%%%%%%%%%%%%%%%%%%%%%%%%%%%%%%%%%%%%%%%%%%%%%%%%%%%%%%%%%%%%%%

\usepackage{pgfpages}
%\pgfpagesuselayout{2 on 1}[a4paper,border shrink=5mm]
\usepackage{empheq}
\usepackage[spanish]{babel}
\usepackage[utf8x]{inputenc}
\usepackage{times}
\usepackage[T1]{fontenc}
\usepackage{amssymb,amsmath}
\usepackage{enumerate}
\usepackage{verbatim}
\usepackage{ esint }
\usepackage{pdfsync}
%\usepackage{pst-all}
%\usepackage{pstricks-add}
\usepackage{array}
%\usepackage[T1]{fontenc}
\usepackage{animate}
%\usepackage{media9}
\usepackage{movie15}
\usepackage{xparse}
\usepackage{listings}
\usepackage{ wasysym }
\usepackage{sagetex}
\usepackage{hyperref}




%%%%%%%%%%%%%%%%%%%%%%%%%Configuracion listing
\lstdefinelanguage{Sage}[]{Python}
{morekeywords={False,sage,True},sensitive=true}
\lstset{
  frame=none,
  showtabs=False,
  showspaces=False,
  showstringspaces=False,
  commentstyle={\ttfamily\color{dgreencolor}},
  keywordstyle={\ttfamily\color{dbluecolor}\bfseries},
  stringstyle={\ttfamily\color{dgraycolor}\bfseries},
  language=Sage,
  basicstyle={\fontsize{8pt}{8pt}\ttfamily},
  aboveskip=.3em,
  belowskip=0.1em,
  numbers=none,
  numberstyle=\footnotesize
}


 


%%%%%%%%%%%%%%%%%%%%%%%%Colores
\definecolor{myblue}{rgb}{.8, .8, 1}
\definecolor{dblackcolor}{rgb}{0.0,0.0,0.0}
\definecolor{dbluecolor}{rgb}{0.01,0.02,0.7}
\definecolor{dgreencolor}{rgb}{0.2,0.4,0.0}
\definecolor{dgraycolor}{rgb}{0.30,0.3,0.30}
\newcommand{\dblue}{\color{dbluecolor}\bf}
\newcommand{\dred}{\color{dredcolor}\bf}
\newcommand{\dblack}{\color{dblackcolor}\bf}










%%%%%%%%%%%%%%%%%%%%%%%%%%Nuevos comandos entornos%%%%%%%%%%%%%%%%%%%%%%%%%%%%%%%%
%%%%%%%%%%%%%%%%%%%%%%%%%%%%%%%%%%%%%%%%%%%%%%%%%%%%%%%%%%%%%%%%%%%%%%%%
\newenvironment{demo}{\noindent\emph{Dem.}}{$\square$ \newline\vspace{5pt}}
\newcommand{\com}{\mathbb{C}}
\newcommand{\dis}{\mathbb{D}}
\newcommand{\rr}{\mathbb{R}}
\newcommand{\oo}{\mathcal{O}}
\renewcommand{\emph}[1]{\textcolor[rgb]{1,0,0}{#1}}
\newcommand{\der}[2]{\frac{\partial #1}{\partial #2}}
\renewcommand{\v}[1]{\overrightarrow{#1}}
\renewcommand{\epsilon}{\varepsilon}
\newlength\mytemplen
\newsavebox\mytempbox
\makeatletter
\newcommand\mybluebox{%
    \@ifnextchar[%]
       {\@mybluebox}%
       {\@mybluebox[0pt]}}

\def\@mybluebox[#1]{%
    \@ifnextchar[%]
       {\@@mybluebox[#1]}%
       {\@@mybluebox[#1][0pt]}}

\def\@@mybluebox[#1][#2]#3{
    \sbox\mytempbox{#3}%
    \mytemplen\ht\mytempbox
    \advance\mytemplen #1\relax
    \ht\mytempbox\mytemplen
    \mytemplen\dp\mytempbox
    \advance\mytemplen #2\relax
    \dp\mytempbox\mytemplen
    \colorbox{myblue}{\hspace{1em}\usebox{\mytempbox}\hspace{1em}}}

\makeatother
\DeclareDocumentCommand\boxedeq{ m g }{%
    {\begin{empheq}[box={\mybluebox[2pt][2pt]}]{equation}% #1%
        \IfNoValueF {#2} {\label{#2}}%
       #1
       \end{empheq}
    }%
}
\DeclareMathOperator{\atan2}{atan2}
\DeclareMathOperator{\sen}{sen}
\newtheorem{teorema}{Teorema}[section]
\newtheorem{lema}[teorema]{Lema}
\newtheorem{corolario}[teorema]{Corolario}
\newtheorem{proposicion}[teorema]{Proposici\'on}
\newtheorem{definicion}[teorema]{Definici\'on}
%%%%%%%%%%%%%%%%%%%%%%%%%%%%%%%%%%%%%%%%%%%%%%%%%%%%%%%%%%%%%%%%%%%%%%%%%%%%%%%%%%%%%%%%%%%%%%%%%%%%%%%%%%%




%%%%%%%%%%Para escibir en clase articulo o similar
% \usepackage{color}
% \newcommand{\nl}{ }
% \renewenvironment{frame}[1]{}{}
% \newcommand{\qed}{$\square$}
% %\newcommand{\defverbatim}{\def{#1}}
% \newenvironment{block}[1]{\textbf{#1}}{}
% \title{Ecuaciones lineales de segundo orden}
% \author{Fernando Mazzone}
% 




%%%%%%%%%%%%%%%%%%%%%%%Para clase beamer
\newcommand{\nl}{\onslide<+-> }
% \mode<all>
% {
%   \usetheme{Boadilla}
%   % oder ...
%  
%   \setbeamercovered{wolverine}
%   % oder auch nicht
% }


\mode<handout>{
  \usetheme{default}
  %\setbeamercolor{background canvas}{bg=black!5}
 % \pgfpagesuselayout{4 on 1}[letterpaper,landscape,border shrink=2.5mm]
}

% \mode<all>{
%   \usetheme{default}
%   %\setbeamercolor{background canvas}{bg=black!5}
%   \pgfpagesuselayout{4 on 1}[letterpaper,landscape,border shrink=2.5mm]
% }

\title[Ecuaciones lineales de segundo orden] % (optional, nur bei langen Titeln nötig)
{%
 Ecuaciones lineales de segundo orden
}



\author[] % (optional, nur bei vielen Autoren)
{Fernando Mazzone}

\institute[Depto de Matemática] % (optional, aber oft nötig)
{
 Depto de Matemática\\
Facultad de Ciencias Exactas Físico-Químicas y Naturales\\
Universidad Nacional de Río Cuarto}


\subject{Ecuaciones Diferenciales}

%%%%%%%%%%%%%%%%%%%%%%%%%%%%%%%%%%%%%%%%%%%%%%%%%%%%%%%%%%%%%%%%%%%%%%%%%%%%%%%%%%%%%%




\begin{document}

\section{Ecuaciones lineales de orden superior}

\begin{frame}{Ecuación lineal general de orden $n$}
\begin{block}{Ecuación lineal general de orden $n$}
Es una ecuación de la forma
\boxedeq{y^{(n)}(x)+p_{n-1}(x)y^{(n-1)}(x)+\cdots+p_1(x)y'(x)+p_0(x)y(x)=r(x)}{eq:ecua_lin_gen_orden_n}
donde $p_i,r$, $i=0,\ldots,n-1$ son funciones definidas en un intervalo $I$
\end{block}


Los resultados y técnicas que hemos desarollado para ecuaciones de orden $2$ se aplican con cambios menores a ecuaciones de mayor orden. No vamos a repetir la demostración de estos resultados, dados que es prácticamente la misma. Los exponemos de manera sumaria.
\end{frame}

\begin{frame}{Existencia y unicidad de soluciones}
\begin{block}{Teorema de existencia y unicidad de soluciones}
 Supongamos $p_i,r$, $i=0,\ldots,n-1$ continuas sobre $I$. Sean $x_0\in I$ e $y_0,y_1,\ldots y_{n-1}\in\rr$ dados. Entonces existe una única solución del PVI
 \[\left\{
 \begin{array}{l l l}
   y^{(n)}(x)+&p_{n-1}(x)y^{(n-1)}(x)+\cdots+p_0(x)y(x)=r(x),x\in I\\
   y(x_0)&=y_0&\\
   y'(x_0)&=y_1&\\
   &\vdots\\
  y^{(n-1)}(x_0)&=y_{n-1}&\\
  \end{array}\right.
\]

\end{block}
\end{frame}

\begin{frame}{Estructura del conjunto de soluciones}

\begin{block}{Teorema estructura conjunto de soluciones ecuaciones homogéneas}
Supongamos $y_1,y_2,\ldots,y_n$ soluciones linealmente independientes de 
\[y^{(n)}(x)+p_{n-1}(x)y^{(n-1)}(x)+\cdots+p_0(x)y(x)=0.\]
Entonces 
\[y=c_1y_1+\cdots+c_ny_n,\quad c_i\in\rr, i=1,\ldots,n,\]
es solución general. Vale decir, el conjunto de soluciones es un espacio vectorial $n$-dimensional.
\end{block}
\end{frame}


\begin{frame}{Estructura del conjunto de soluciones}
\begin{block}{Teorema estructura conjunto de soluciones ecuaciones no homogéneas}
Una solución general de la ecuación no homogénea  
\[y^{(n)}(x)+p_{n-1}(x)y^{(n-1)}(x)+\cdots+p_0(x)y(x)=r(x),\]
es la suma de una solución particular de esta ecuación más una solución general de la ecuación homogénea asociada.
\end{block}

\end{frame}




\begin{frame}{Wronskiano}
\begin{block}{Fórmula Abel}
Si $y_1,y_2,\ldots,y_n$ son solución de la ecuación  homogénea  
\boxedeq{y^{(n)}(x)+p_{n-1}(x)y^{(n-1)}(x)+\cdots+p_0(x)y(x)=0,}{eq:lin_hom_orden_n}
Entonces el Wronskiano satisface
\boxedeq{W(y_1,\ldots,y_n)(x)=W(y_1,\ldots,y_n)(x_0)e^{-\int_{x_0}^xp_{n-1}(t)dt}.}{eq:formu_abel}
En particular  $\{y_1,y_2,\ldots,y_n\}$ son linalmente independientes si y solo si $W(x)\neq 0$ para todo $x\in I$.
\end{block}

\textbf{Dem.} Las demostraciones de los resultados anteriores es tan similar a su análogo de orden 2 que no vale la pena invertir tiempo en ellas. La demostración de la fórmula de Abel, si nos parece lo suficientemente interesante para dejarla como \textbf{Ejercicio}.
\end{frame}


\begin{frame}{Ecuaciones homogéneas con coeficientes constantes}
Las ecuaciones lineales de orden $n$, homogeneas con coeficientes constantes  se resuelven por métodos análogos los considerados para eciaciones de segundo orden. Se propone $y(x)=e^{rx}$ como solución. Reemplazando esta función
en \eqref{eq:lin_hom_orden_n} vemos que $y$ sería solución si y solo si $r$ es solución de la ecuación característica
\boxedeq{r^n+p_{n-1}r^{n-1}+\cdots+p_1r+p_0=0.}{eq:ecua_carac}
 
Ahora se presentan casos algo diferentes.


\end{frame}

\begin{frame}{Ecuaciones homogéneas con coeficientes constantes}

\textbf{Raices reales distintas.}
Si la ecuación característica \eqref{eq:ecua_carac} tiene $n$ raíces $r_1,\ldots,r_n$ reales y distintas entonces 
$y_1(x)=e^{r_1x},\ldots,y_n(x)=e^{r_nx}$ son soluciones linealmente independientes y por ende
\[\boxed{y(x)=c_1e^{r_1x}+\cdots+c_ne^{r_nx}}\]
es solución general. 

Dejamos como ejercicio la demostración de la independencia lineal. En lugar de usar el Wronskiano, se puede utilizar  la siguiente idea basada en métodos operacionales. 

Por $D$ vamos a denotar el operador diferenciación, esto es $D$ es sencillamente la función  definida sobre el conjunto de funciones difenciables sobre un intervalo abierto y que actúa derivando, esto es $Dy=y'(x)$. 





\end{frame}





\begin{frame}{Ecuaciones homogéneas con coeficientes constantes}
Dado un polinomio $p(X)=p_nX^n+p_{n-1}X^{n-1}+\cdots+p_1X+p_0$, $p_i\in\rr$, $i=0,\ldots,n$, denotamos por  $p(D)$ el operador definido por
\[p(D)y=p_ny^{(n)}+p_{n-1}y^{(n-1)}+\cdots+p_1y'+p_0y.\]
Diremos que $p(D)$ es un \emph{operador diferencial polinomial}. 

\textbf{Ejercicio 1} Dado que dos polinomios se pueden sumar y multiplicar resultando estas operaciones en un nuevo polinomio, es posible hacer lo propio con operadores diferenciales polinomiales. Demostrar que el producto de dos de tales operadores  $p(D)$ y $q(D)$ es conmutativo. Esto es consecuencia que el producto de polinomios es conmutativo. Analizar que ocurriría si permitiesemos que los coeficientes $p_i$ fuese funciones de $x$, $p_i=p_i(x)$.
 
\end{frame}

\begin{frame}{Ecuaciones homogéneas con coeficientes constantes}
\textbf{Ejercicio 2} Supongamos ahora que $r_1,\ldots,r_n$ son números reales distintos y que 
\[y=c_1e^{r_1x}+c_2e^{r_2x}+\cdots+c_ne^{r_nx}.\]
Considerar el operador diferencial
\[p(D)=(D-r_1)\cdots(D-r_{i-1})(D-r_{i+1})\cdots(D-r_n).\]
Demostrar que
\[p(D)y=(r_i-r_1)\cdots(r_i-r_{i-1})(r_i-r_{i+1})\cdots(r_i-r_n)e^{r_ix}.\]
Deducir de esto la independencia lineal de $\{y_1(x)=e^{r_1x},\ldots,y_n(x)=e^{r_nx}\}$.
\end{frame}

\begin{frame}{Ecuaciones homogéneas con coeficientes constantes}

\textbf{Raices reales repetidas.}
Supongamos que la ecuación característica \eqref{eq:ecua_carac} tiene  raíces reales repetidas. Sean $r_1,\ldots,r_k$ las raices distintas y $m_j$ la multiplicidad de la raíz $r_j$. Por cada raíz $r_j$ considerar las $m_j$  funciones

\[\mathcal{B}_j:=\{y_j^0(x)=e^{r_jx}, y_j^1(x)=xe^{r_jx},\ldots y_j^{m_j-1}(x)=x^{m_j-1}e^{r_jx}\}.\]

 \textbf{Ejericio 2} Demostrar que $\mathcal{B}:=\mathcal{B}_1\cup\cdots\cup \mathcal{B}_k$ forma
 un conjunto de  $n$  soluciones linealmente independientes. 

\end{frame}

\begin{frame}{Ecuaciones homogéneas con coeficientes constantes}
\textbf{Raices complejas.}
Supongamos que la ecuación característica \eqref{eq:ecua_carac} tiene  raíces complejas. Como la ecuación característica tiene coeficientes reales las raices aparecen de a pares conjugados $\mu\pm\nu i$. 

Si las raices son simples, como en el caso de ecuaciones de orden 2, por cada uno de estos pares hay que considerar las soluciones $ e^{\mu x}\cos x$ y $e^{\mu x}\sen x$ . Si son múltiples con multiplicidad $k$ hay que considerar las $2k$ soluciones   $ e^{\mu x}\cos x, xe^{\mu x}\cos x,\ldots, x^{k-1}e^{\mu x}\cos x$ y $e^{\mu x}\sen x,xe^{\mu x}\sen x\ldots x^{k-1}e^{\mu x}\sen x $. Dejamos los detalles que es necesario completar como trabajo práctico.
 
\end{frame}


\begin{frame}{Osciladores armónicos acoplados}
Supongamos dos masas unidas por un resorte.

 \includemovie{osciladores_acoplados/Oscillators.avi}
 
 
\end{frame}
% 
% 
\end{document}

