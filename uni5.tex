\documentclass{article}
%\documentclass[hyperref={colorlinks=true}]{beamer}
%\documentclass[handout,hyperref={colorlinks=true}]{beamer}


%%%%%%%%%%%%%%%%%%%%%%%%%%%%%%Paquetes%%%%%%%%%%%%%%%%%%%%%%%%%%%%%%%%%%%%%%%%%%%%%%%
%%%%%%%%%%%%%%%%%%%%%%%%%%%%%%%%%%%%%%%%%%%%%%%%%%%%%%%%%%%%%%%%%%%%%%%%%%%%%%%%%%%%%

%\usepackage{pgfpages}
%\pgfpagesuselayout{4 on 1}[a4paper,landscape,border shrink=5mm]
\usepackage{empheq}
\usepackage[spanish]{babel}
\usepackage[utf8x]{inputenc}
\usepackage{times}
\usepackage[T1]{fontenc}
\usepackage{amssymb,amsmath}
\usepackage{enumerate}
\usepackage{verbatim}
\usepackage{ esint }
\usepackage{pdfsync}
%\usepackage{pst-all}
%\usepackage{pstricks-add}
\usepackage{array}
%\usepackage[T1]{fontenc}
\usepackage{animate}
%\usepackage{media9}
%\usepackage{movie15}
\usepackage{xparse}
%\usepackage{listings}
\usepackage{ wasysym }
\usepackage{sagetex}
\usepackage{hyperref}
\usepackage{tabularx}
\usepackage{xcolor}
\usepackage{adjustbox}



%%%%%%%%%%%%%%%%%%%%%%%%%%Nuevos comandos entornos%%%%%%%%%%%%%%%%%%%%%%%%%%%%%%%%
%%%%%%%%%%%%%%%%%%%%%%%%%%%%%%%%%%%%%%%%%%%%%%%%%%%%%%%%%%%%%%%%%%%%%%%%
%%%%%%%%%%%%%%%%%%%%%%%%Colores

\definecolor{myblue}{rgb}{.8, .8, 1}
\definecolor{dblackcolor}{rgb}{0.0,0.0,0.0}
\definecolor{dbluecolor}{rgb}{0.01,0.02,0.7}
\definecolor{dgreencolor}{rgb}{0.2,0.4,0.0}
\definecolor{dgraycolor}{rgb}{0.30,0.3,0.30}
\definecolor{color_teor}{HTML}{FFFDA2}
\newcommand{\dblue}{\color{dbluecolor}\bf}
\newcommand{\dred}{\color{dredcolor}\bf}
\newcommand{\dblack}{\color{dblackcolor}\bf}


%%%%%%%%%%%%%%%%%%%Teoremas, definiciones , etc


\newenvironment{demo}{\noindent\emph{Dem.}}{$\square$ \newline\vspace{5pt}}
\newenvironment{colbox}[2]{%
    \begin{adjustbox}{minipage={\linewidth},margin=1ex,bgcolor=#1,env=center}
        #2}{%
    \end{adjustbox}%
}

\newcounter{defi_cont}\stepcounter{defi_cont}
\newenvironment{definicion}[1]{\begin{colbox}{myblue}{\textbf{Definición \arabic{defi_cont}.} #1}}{\stepcounter{defi_cont}\end{colbox}}

\newcounter{teor_cont}\stepcounter{teor_cont}
\newenvironment{teorema}[1]{\begin{colbox}{color_teor}{\textbf{Teorema \arabic{teor_cont}.} #1}}{\stepcounter{teor_cont}\end{colbox}}

\newcounter{ejem_cont}\stepcounter{ejem_cont}
\newenvironment{ejemplo}[1]{\vspace{1ex}\noindent\textbf{Ejemplo \arabic{ejem_cont}.} #1}{\stepcounter{ejem_cont}}

%\newtheorem{teorema}{Teorema}[section]
%\newtheorem{lema}[teorema]{Lema}
%\newtheorem{corolario}[teorema]{Corolario}
%\newtheorem{proposicion}[teorema]{Proposici\'on}
%%%%%%%%%%%%%%%%%%%%%%%%%%%%%%%%%%%%Simbolos

\newcommand{\com}{\mathbb{R}}
\newcommand{\dis}{\mathbb{D}}
\newcommand{\rr}{\mathbb{R}}
\newcommand{\oo}{\mathcal{O}}
\renewcommand{\emph}[1]{\textcolor[rgb]{0,0,1}{#1}}
\newcommand{\der}[2]{\frac{\partial #1}{\partial #2}}
\renewcommand{\v}[1]{\overrightarrow{#1}}
\renewcommand{\epsilon}{\varepsilon}
\DeclareMathOperator{\mcd}{mcd}
\DeclareMathOperator{\atan2}{atan2}
\DeclareMathOperator{\sen}{sen}
%%%%%%%%%%%%%%%%%%%%%%%%%%%%%%Formulas en cajas 
\newlength\mytemplen
\newsavebox\mytempbox
\makeatletter
\newcommand\mybluebox{%
    \@ifnextchar[%]
       {\@mybluebox}%
       {\@mybluebox[0pt]}}

\def\@mybluebox[#1]{%
    \@ifnextchar[%]
       {\@@mybluebox[#1]}%
       {\@@mybluebox[#1][0pt]}}

\def\@@mybluebox[#1][#2]#3{
    \sbox\mytempbox{#3}%
    \mytemplen\ht\mytempbox
    \advance\mytemplen #1\relax
    \ht\mytempbox\mytemplen
    \mytemplen\dp\mytempbox
    \advance\mytemplen #2\relax
    \dp\mytempbox\mytemplen
    \colorbox{myblue}{\hspace{1em}\usebox{\mytempbox}\hspace{1em}}}

\makeatother
\DeclareDocumentCommand\boxedeq{ m g }{%
    {\begin{empheq}[box={\mybluebox[2pt][2pt]}]{equation}% #1%
        \IfNoValueF {#2} {\label{#2}}%
       #1
       \end{empheq}
    }%
}

%%%%%%%%%%%%%%%%%%%%%%%%%%%%%%%%%%%%%%%%%%%%%%%%%%%%%%%%%%%%%%%%%%%%%%%%%%%%%%%%%%%%%%%%%%%%%%%%%%%%%%%%%%%%%%%%%%%%%%%%%%%%%%%%%%%%%%%%%%%%%%%%%%%%%%%%%%%%%%%%%%%%%%%%%%%%%%%%%%%%%%%%%%%%%%%




\begin{document}



\section{Series de potencias}

\subsection{Definición} %Primer Frame para información personal.
\begin{definicion} Una serie de potencias es una serie de la forma:
\[
	\sum\limits_{n=0}^{\infty}a_n(z-z_0)^n
\]
donde $a_n$, $n=0,1,\ldots$, $z_0$ y $z$ son elementos de $\com$.
\end{definicion}


Estamos interesados en determinar los valores de $z$ para los cuales   una serie converge. 

\begin{ejemplo} La serie geométrica
\[
	\sum\limits_{n=0}^{\infty}z^n,
\]
es una serie de potencias. Aquí  $a_n=1$, $n=0,1,\ldots$ y $z_0=0$. Esta serie converge para $|z|<1$ a
\[ \frac{1}{1-z}\]
y no converge para cualquier otro valor de $z\in\com$.
\end{ejemplo}

\begin{ejemplo} Supongamos $f:I\to\rr$, donde $I$ es un intervalo abierto $I=(a,b)$ y que $f$ tiene derivadas de todo orden en  $z_0\in I$. Entonces es posible construir la serie de Taylor de $f$ en $z_0$ que es una serie de potencias. Recordemos que esta serie es
\[S(f,z_0,z)=\sum\limits_{n=0}^{\infty}\frac{f^{(n)}(z_0)}{n!}(z-z_0)^n.\]
\end{ejemplo}

 
 

\section{Límites superior e inferior}


\begin{definicion} Dada una sucesión de números reales $x_n$, 
 consideramos una nueva sucesión:

\[
	A_n=\sup\{x_n, x_{n+1},\ldots \}
\]

La nueva sucesión de reales $A_n$ es siempre nocreciente ($A_n\geq A_{n+1}$), luego tiene un límite (puede ser $\pm\infty$). A este límite lo llamamos \emph{el límite superior de $x_n$}. Lo denotamos por $\limsup$. Es decir:   


\[
\limsup x_n=\lim_{n\to\infty} A_n=\lim\limits_{n\to \infty} \sup\{x_n, x_{n+1},\ldots \}.
\]


Tomando ínfimo en lugar de supremo conseguimos \emph{el límite inferior} ($\liminf$).  
\end{definicion}
\begin{ejemplo}  Si $x_n=(-1)^n$, entonces
\[
	\{x_n,x_{n+1},\ldots\}=\{\pm1,\mp1,\pm1,\ldots\}.
\]
El supremode este conjunto  es para todo $n$ igual a 1 y el ínfimo igual a -1. Luego $\liminf x_n=-1$ y $\limsup=1$. 
\end{ejemplo}



\begin{ejemplo}  Si $x_n=1/n$, si $n$ es par y $x_n=1$ si $n$ es impar, entonces el conjunto

\[
	\{x_n,x_{n+1},\ldots\}
\]
tiene por supremo 1 y el ínfimo igual a $0$ . Luego $\liminf x_n=0$ y $\limsup=1$. 
\end{ejemplo}

\begin{teorema}\textbf{Propiedades} Sea $x_n$  e $y_n$ dos sucesiones de números reales, entonces:

\begin{enumerate}
	\item El $\limsup$ y el $\liminf$  existen siempre si se permite que $\pm\infty$ sean sus posibles valores.
	\item $\liminf x_n \leq \limsup x_n$.
	\item $\liminf x_n=\limsup x_n$ si y solo si el $\lim x_n$ existe. En este caso todos los límites coinciden.
	\item $\liminf (x_n+y_n)\geq \liminf x_n + \liminf y_n$.
	\item $\limsup (x_n+y_n)\leq \limsup x_n + \limsup y_n$

\end{enumerate}
\end{teorema}



\section{Radio de convergencia}


\begin{definicion} Dada la serie de potencias 
\[
	\sum\limits_{n=0}^{\infty}a_n(z-z_0)^n,
\]
definimos el \emph{radio de convergencia} $R$ de la siguiente forma:



\[
\frac{1}{R}=\limsup\limits_{n\to \infty} |a_n|^{1/n}.
\]

\end{definicion}

\begin{ejemplo} La serie  
\[
	\sum\limits_{n=0}^{\infty}z^n,
\]
tiene radio de convergencia:

\[\frac{1}{R}=\limsup\limits_{n\to \infty}1^{1/n}=\lim\limits_{n\to\infty}1^{1/n}=1\]

Luego $R=1$. 
\end{ejemplo}

\begin{ejemplo} La serie  
\[
	\sum\limits_{n=0}^{\infty}\bigg(\frac{1}{M}\bigg)^nz^n,
\]
tiene radio de convergencia:

\[\frac{1}{R}=\limsup\limits_{n\to \infty}\bigg(\bigg(\frac{1}{M}\bigg)^n\bigg)^{1/n}=\lim\limits_{n\to\infty}\bigg(\bigg(\frac{1}{M}\bigg)^n\bigg)^{1/n}=\frac1M\]

Luego $R=M$. 

\end{ejemplo}

\begin{ejemplo} Fijemos $M>0$ y $n$ un natural tal que $[n/2] >M$ (aquí $[x]$ es la parte entera de $x$). Entonces,
como $n-[n/2]\geq [n/2]>M$

\[ 
\begin{split}
n!=n(n-1)\cdots 1 &> n(n-1)\cdots (n-[n/2]) \\
&>\underbrace{M\cdots M }_{[n/2]-\hbox{veces}}\\
&\geq M^{[n/2]}\\
&> M^{n/3}\\
\end{split}
\]
Luego 
\[\frac{1}{R}:=\limsup_{n\to\infty} (1/n!)^{1/n}\leq \lim_{n\to\infty} \bigg(\frac{1}{M^{n/3}}\bigg)^{1/n}=\frac{1}{\sqrt[3]{M}}\]
Como $M$ es arbitrario, haciendo $M\to\infty$ vemos que el radio de convergencia de la serie  $\sum\limits_{n=0}^{\infty}\frac{1}{n!}z^n$ 
es $R=\infty$. 
\end{ejemplo}





 




\textbf{ Primer Teorema Principal}

\textbf{Teorema}  Consideremos la serie:

\[
	\sum\limits_{n=0}^{\infty}a_nz^n,
\]

Entonces:

\begin{enumerate}
	\item<+-> Si $|z|<R$, la serie converge absolutamente en $z$. 
	\item<+-> Si $|z|>R$, la serie diverge.
	\item<+-> Si $|z|=R$, ?????
	
\end{enumerate}

	
 


\textbf{Demostración Primer Teorema Principal}
\textbf{Dem.} Supongamos $0<R<\infty$. Sea $L=1/R$ y tomemos $\epsilon>0$ pequeño. Como
\[\lim_{n\to\infty}\sup\{|a_n|^{1/n},|a_{n+1}|^{1/n+1},\ldots\}=L\]
para $n_0$ suficientemente grande
\[\sup\{|a_n|^{1/n},|a_{n+1}|^{1/n+1},\ldots\}<L+\epsilon.\]
 Así
\[|a_n|^{1/n}<L+\epsilon\quad\hbox{para } n\geq n_0.\]
Elijamos $0<r<1/(L+\epsilon)<1/L=R$. Si $|z|<r$ entonces
\[|a_n||z|^n<(L+\epsilon)^nr^n\quad\hbox{para } n\geq n_0.\]

 


\textbf{Demostración Primer Teorema Principal}

Pero $r(L+\epsilon)<1$. La desigualdad de arriba y el teorema de comparación (notar que el miembro de la derecha forma una serie geométrica) implican que la serie converge absolutamente para este $z$.  Como $epsilon$ es arbitrario, dado cualquier $z$, con $|z|<1$, tenemos un $\epsilon$ lo suficientemente chico para que $|z|<1/(L+\epsilon)$. 

 \textbf{Ejercicio:} Demostrar los casos $R=0$, $R=\infty$ y el segundo inciso.

	
 

\textbf{ Segundo Teorema Principal}

\textbf{Teorema}  La función

\[
f(z)=	\sum\limits_{n=0}^{\infty}a_nz^n,
\]
es holomorfa dentro del círculo con centro en el origen dado por el radio de convergencia. Además


\[
f'(z)=g(z):=	\sum\limits_{n=1}^{\infty}na_nz^{n-1},
\]
teniendo esta serie el mismo radio de convergencia que el de $f$.


	
 


\textbf{Demostración Segundo Teorema Principal}
\textbf{Dem.} La afirmación sobre el radio de convergencia es consecuencia de que $\lim_{n\to\infty}n^{1/n}=1$. Como el radio $R'$ de convergencia de $g$ es:

\[
\begin{split}
\frac{1}{R'}&=\limsup_{n\to\infty}|a_{n+1}(n+1)|^{1/(n+1)}\\
&\underbrace{=}_{\hbox{\emph{ Ejer. Matemáticos}}}\limsup_{n\to\infty}|a_{n+1}|^{1/(n+1)}\lim_{n\to\infty}|(n+1)|^{1/(n+1)}\\
&=\limsup_{n\to\infty}|a_{n+1}|^{1/(n+1)}=\frac{1}{R}
\end{split}\]






 


\textbf{Demostración Segundo Teorema Principal}
Ahora veamos que $f$ es holomorfa y $f'=g$.   Sea $0<r<R$, $|z_0|<r$ y $N\in\mathbb{N}$. Pongamos:
\[f(z)=S_N(z)+E_N(z),\]
\[S_N(z)=\sum_{n=0}^{N}a_nz^n\quad\hbox{y}\quad E_N(z)=\sum_{n=N+1}^{\infty}a_nz^n\]
Tomemos $|h|<r-|z_0|$, así $|z_0+h|<r$.  Tenemos
\[\begin{split}
\frac{f(z_0+h)-f(z_0)}{h}-g(z_0)&= \frac{S_N(z_0+h)-S_N(z_0)}{h}-S_N'(z_0)\\
				&+S_N'(z_0)-g(z_0)\\
				&+\frac{E_N(z_0+h)-E_N(z_0)}{h}
\end{split}\]



 


\textbf{Demostración Segundo Teorema Principal}
Ahora si $\epsilon>0$

\[\begin{split}
				&\bigg|\frac{E_N(z_0+h)-E_N(z_0)}{h}\bigg|\leq\sum_{n=N+1}^{\infty}|a_n|\bigg|\frac{(z_0+h)^n-z_0^n}{h}\bigg|\\
				&=\sum_{n=N+1}^{\infty}|a_n|(|z_0|^{n-1}+|z_0|^{n-2}h+\cdots+h^{n-1})\\
				&\leq2\sum_{n=N+1}^{\infty}|a_n|nr^{n-1}<\epsilon\\
\end{split}\]
Para $N$ suficientemente grande. Además como $S_N'(z)\to g(z)$ cuando $N\to\infty$ podemos elegir, a su vez, $N$ suficientemente grande para que

\[
	|S_N'(z_0)-g(z_0)|<\epsilon
\]




 


\textbf{Demostración Segundo Teorema Principal}
 Fijemos un $N$ que satisfaga las condiciones anteriores. Ahora podemos encontrar $\delta>0$ para que $|h|<\delta$ cumpla que 
\[
\bigg|\frac{S_N(z_0+h)-S_N(z_0)}{h}-S_N'(z_0)\bigg|<\epsilon.
\]

Esto muestra que $f'(z_0)=g(z_0)$ y por consiguiente $f$ es holomorfa.



 


\textbf{Un corolario}

\textbf{Corolario}  Una serie de potencias es infinitamente diferenciable. Las sucesivas derivadas se obtienen derivando término a término la serie. El radio de convergencia se conserva.

	


 

\textbf{Representación de la función exponencial}

\textbf{Ejemplo}  Habímos visto que la serie:
\[
	f(z)=\sum_{n=0}^{\infty}\frac{1}{n!}z^n
\]
tenía radio de convergencia infinito y por ende converge en $\com$. Ahora vemos que
\[f'(z)=\sum_{n=1}^{\infty}\frac{1}{(n-1)!}z^{n-1} =\sum_{n=0}^{\infty}\frac{1}{n!}z^n
\]



	


 


\textbf{Representación de la función exponencial}

\textbf{Ejemplo}  Esta $f$ resuelve la simple ecuación diferencial $f'(z)=f(z)$. La misma ecuación es resuelta por $g(z)=e^z$. Tenemos:
\[
	\frac{d}{dz}f(z)e^{-z}=f'(z)e^{-z}-f(z)e^{-z}=0
\]
Entonces $h(z)=f(z)e^{-z}$ es una función costante. Pero $h(0)=1$, entonces $f(z)e^{-z}=1$ y de allí $f(z)=e^z$. 



	
\textbf{Comentario} Podemos a partir de la exponencial encontrar desarrollos en serie para las funciones trigonométricas.

 



\section{Funciones analíticas}
\textbf{Funciones Analíticas}

Podemos hacer un desarrollo de pontencias centrado en otro punto $z_0\in\com$ diferente que $0$. Nos referimos a considerar series de la forma
\[ 
\sum_{n=0}^{\infty}a_n(z-z_0)^n
\]

Claramente a este tipo de serie se le aplican todos los resultados anteriores. Ahora el disco de convergencia es un disco centrado en $z_0$ y de radio dado por el radio de convergencia dado por la misma fórmula que antes.



 

\textbf{Funciones Analíticas}

\textbf{Definición} Una función $f:\Omega\subset\com\to\com$ se dirá analítica si para cada $z_0\in\Omega$, existe un $R$ y $a_n\in\com$, tal que vale la igualdad:

\[ f(z)=\sum_{n=0}^{\infty}a_n(z-z_0)^n,\quad\hbox{para } |z-z_0|<R \]



\textbf{Ejercicio} Si $f$ es analítica tenemos la siguiente fórmula
\[
	a_n=\frac{f^{(n)}(z_0)}{n!}
\]
para los coeficientes $a_n$. 



 







\end{document}

