\documentclass{article}
%\documentclass[hyperref={colorlinks=true}]{beamer}
%\documentclass[handout,hyperref={colorlinks=true}]{beamer}


%%%%%%%%%%%%%%%%%%%%%%%%%%%%%%Paquetes%%%%%%%%%%%%%%%%%%%%%%%%%%%%%%%%%%%%%%%%%%%%%%%
%%%%%%%%%%%%%%%%%%%%%%%%%%%%%%%%%%%%%%%%%%%%%%%%%%%%%%%%%%%%%%%%%%%%%%%%%%%%%%%%%%%%%

%\usepackage{pgfpages}
%\pgfpagesuselayout{4 on 1}[a4paper,landscape,border shrink=5mm]
\usepackage{empheq}
\usepackage[spanish]{babel}
\usepackage[utf8x]{inputenc}
\usepackage{times}
\usepackage[T1]{fontenc}
\usepackage{amssymb,amsmath}
\usepackage{enumerate}
\usepackage{verbatim}
\usepackage{ esint }
\usepackage{pdfsync}
%\usepackage{pst-all}
%\usepackage{pstricks-add}
\usepackage{array}
%\usepackage[T1]{fontenc}
\usepackage{animate}
%\usepackage{media9}
%\usepackage{movie15}
\usepackage{xparse}
%\usepackage{listings}
\usepackage{ wasysym }
\usepackage{sagetex}
\usepackage{hyperref}
\usepackage{tabularx}
\usepackage{xcolor}
\usepackage{adjustbox}



%%%%%%%%%%%%%%%%%%%%%%%%%%Nuevos comandos entornos%%%%%%%%%%%%%%%%%%%%%%%%%%%%%%%%
%%%%%%%%%%%%%%%%%%%%%%%%%%%%%%%%%%%%%%%%%%%%%%%%%%%%%%%%%%%%%%%%%%%%%%%%
%%%%%%%%%%%%%%%%%%%%%%%%Colores

\definecolor{myblue}{rgb}{.8, .8, 1}
\definecolor{dblackcolor}{rgb}{0.0,0.0,0.0}
\definecolor{dbluecolor}{rgb}{0.01,0.02,0.7}
\definecolor{dgreencolor}{rgb}{0.2,0.4,0.0}
\definecolor{dgraycolor}{rgb}{0.30,0.3,0.30}
\definecolor{color_teor}{HTML}{FFFDA2}
\definecolor{color_ejer}{HTML}{F6BFBF}
\definecolor{color_cor}{HTML}{49E6D8}
\definecolor{color_lem}{HTML}{C6F6AB}
\newcommand{\dblue}{\color{dbluecolor}\bf}
\newcommand{\dred}{\color{dredcolor}\bf}
\newcommand{\dblack}{\color{dblackcolor}\bf}

%%%%%%%%%%%%%%%%%%%Teoremas, definiciones , etc

\newenvironment{demo}{\noindent\emph{Dem.}}{{\hspace*{\fill}$\square$} \newline\vspace{5pt}}
\newenvironment{colbox}[2]{%
    \begin{adjustbox}{minipage={\linewidth},margin=1ex,bgcolor=#1,env=center}
        #2}{%
    \end{adjustbox}%
}

\newcounter{defi_cont}
\newenvironment{definicion}[1]{\begin{colbox}{myblue}{\refstepcounter{defi_cont}\textbf{Definición \arabic{defi_cont}.} #1}}{\end{colbox}}

\newcounter{teor_cont}
\newenvironment{teorema}[1]{\refstepcounter{teor_cont}\begin{colbox}{color_teor}{\textbf{Teorema \arabic{teor_cont}.} #1}}{\end{colbox}}

\newcounter{cor_cont}
\newenvironment{corolario}[1]{\begin{colbox}{color_cor}{\refstepcounter{cor_cont}\textbf{Corolario \arabic{cor_cont}.} #1}}{\end{colbox}}

\newcounter{lem_cont}
\newenvironment{lema}[1]{\begin{colbox}{color_lem}{\refstepcounter{lem_cont}\textbf{Lema \arabic{lem_cont}.} #1}}{\end{colbox}}

\newcounter{ejem_cont}
\newenvironment{ejemplo}[1]{\refstepcounter{ejem_cont}\vspace{1ex}\noindent\textbf{Ejemplo \arabic{ejem_cont}.} #1}{}

\newcounter{ejer_cont}
\newenvironment{ejercicio}[1]{\begin{colbox}{color_ejer}{\refstepcounter{ejer_cont}\textbf{Ejercicio \arabic{ejer_cont}.} #1}}{\end{colbox}}


%%%%%%%%%%%%%%%%%%%%%%%%%%%%%%%%%%%%Simbolos

\newcommand{\com}{\mathbb{R}}
\newcommand{\dis}{\mathbb{D}}
\newcommand{\rr}{\mathbb{R}}
\newcommand{\oo}{\mathcal{O}}
\renewcommand{\emph}[1]{\textcolor[rgb]{0,0,1}{#1}}
\newcommand{\der}[2]{\frac{\partial #1}{\partial #2}}
\renewcommand{\v}[1]{\overrightarrow{#1}}
\renewcommand{\epsilon}{\varepsilon}
\DeclareMathOperator{\mcd}{mcd}
\DeclareMathOperator{\atan2}{atan2}
\DeclareMathOperator{\sen}{sen}
%%%%%%%%%%%%%%%%%%%%%%%%%%%%%%Formulas en cajas 
\newlength\mytemplen
\newsavebox\mytempbox
\makeatletter
\newcommand\mybluebox{%
    \@ifnextchar[%]
       {\@mybluebox}%
       {\@mybluebox[0pt]}}

\def\@mybluebox[#1]{%
    \@ifnextchar[%]
       {\@@mybluebox[#1]}%
       {\@@mybluebox[#1][0pt]}}

\def\@@mybluebox[#1][#2]#3{
    \sbox\mytempbox{#3}%
    \mytemplen\ht\mytempbox
    \advance\mytemplen #1\relax
    \ht\mytempbox\mytemplen
    \mytemplen\dp\mytempbox
    \advance\mytemplen #2\relax
    \dp\mytempbox\mytemplen
    \colorbox{myblue}{\hspace{1em}\usebox{\mytempbox}\hspace{1em}}}

\makeatother
\DeclareDocumentCommand\boxedeq{ m g }{%
    {\begin{empheq}[box={\mybluebox[2pt][2pt]}]{equation}% #1%
        \IfNoValueF {#2} {\label{#2}}%
       #1
       \end{empheq}
    }%
}

%%%%%%%%%%%%%%%%%%%%%%%%%%%%%%%%%%%%%%%%%%%%%%%%%%%%%%%%%%%%%%%%%%%%%%%%%%%%%%%%%%%%%%%%%%%%%%%%%%%%%%%%%%%%%%%%%%%%%%%%%%%%%%%%%%%%%%%%%%%%%%%%%%%%%%%%%%%%%%%%%%%%%%%%%%%%%%%%%%%%%%%%%%%%%%%




\begin{document}



 \section{Series de potencias}

En esta sección recordamos algunos conceptos y teoremas sobre series de potencias. Dado que este tema es motivo de un estudio más profundo en otras materias de la licenciatura en matemática omitimos algunas demostraciones.
\subsection{Definición} %Primer Frame para información personal.
\begin{definicion} Una serie de potencias es una serie de la forma:
\[
	\sum\limits_{n=0}^{\infty}a_n(z-z_0)^n
\]
donde $a_n$, $n=0,1,\ldots$, $z_0$ y $z$ son elementos de $\com$.
\end{definicion}


Estamos interesados en determinar los valores de $z$ para los cuales   una serie converge. 

\begin{ejemplo} La serie geométrica
\[
	\sum\limits_{n=0}^{\infty}z^n,
\]
es una serie de potencias. Aquí  $a_n=1$, $n=0,1,\ldots$ y $z_0=0$. Esta serie converge para $|z|<1$ a
\[ \frac{1}{1-z}\]
y no converge para cualquier otro valor de $z\in\com$.
\end{ejemplo}

\begin{ejemplo} Supongamos $f:I\to\rr$, donde $I$ es un intervalo abierto $I=(a,b)$ y que $f$ tiene derivadas de todo orden en  $z_0\in I$. Entonces es posible construir la serie de Taylor de $f$ en $z_0$ que es una serie de potencias. Recordemos que esta serie es
\[S(f,z_0,z)=\sum\limits_{n=0}^{\infty}\frac{f^{(n)}(z_0)}{n!}(z-z_0)^n.\]
\end{ejemplo}

 
 

\subsection{Límites superior e inferior}


\begin{definicion} Dada una sucesión de números reales $x_n$, 
 consideramos una nueva sucesión:

\[
	A_n=\sup\{x_n, x_{n+1},\ldots \}
\]

La nueva sucesión de reales $A_n$ es siempre nocreciente ($A_n\geq A_{n+1}$), luego tiene un límite (puede ser $\pm\infty$). A este límite lo llamamos \emph{el límite superior de $x_n$}. Lo denotamos por $\limsup$. Es decir:   


\[
\limsup x_n=\lim_{n\to\infty} A_n=\lim\limits_{n\to \infty} \sup\{x_n, x_{n+1},\ldots \}.
\]


Tomando ínfimo en lugar de supremo conseguimos \emph{el límite inferior} ($\liminf$).  
\end{definicion}
\begin{ejemplo}  Si $x_n=(-1)^n$, entonces
\[
	\{x_n,x_{n+1},\ldots\}=\{\pm1,\mp1,\pm1,\ldots\}.
\]
El supremode este conjunto  es para todo $n$ igual a 1 y el ínfimo igual a -1. Luego $\liminf x_n=-1$ y $\limsup=1$. 
\end{ejemplo}



\begin{ejemplo}  Si $x_n=1/n$, si $n$ es par y $x_n=1$ si $n$ es impar, entonces el conjunto

\[
	\{x_n,x_{n+1},\ldots\}
\]
tiene por supremo 1 y el ínfimo igual a $0$ . Luego $\liminf x_n=0$ y $\limsup=1$. 
\end{ejemplo}

\begin{teorema}\textbf{Propiedades} Sea $x_n$  e $y_n$ dos sucesiones de números reales, entonces:

\begin{enumerate}
	\item El $\limsup$ y el $\liminf$  existen siempre si se permite que $\pm\infty$ sean sus posibles valores.
	\item $\liminf x_n \leq \limsup x_n$.
	\item $\liminf x_n=\limsup x_n$ si y solo si el $\lim x_n$ existe. En este caso todos los límites coinciden.
	\item $\liminf (x_n+y_n)\geq \liminf x_n + \liminf y_n$.
	\item $\limsup (x_n+y_n)\leq \limsup x_n + \limsup y_n$

\end{enumerate}
\end{teorema}



\subsection{Radio de convergencia}


\begin{definicion} Dada la serie de potencias 
\[
	\sum\limits_{n=0}^{\infty}a_n(z-z_0)^n,
\]
definimos el \emph{radio de convergencia} $R$ de la siguiente forma:



\[
\frac{1}{R}=\limsup\limits_{n\to \infty} |a_n|^{1/n}.
\]

\end{definicion}

\begin{ejemplo} La serie  
\[
	\sum\limits_{n=0}^{\infty}z^n,
\]
tiene radio de convergencia:

\[\frac{1}{R}=\limsup\limits_{n\to \infty}1^{1/n}=\lim\limits_{n\to\infty}1^{1/n}=1\]

Luego $R=1$. 
\end{ejemplo}

\begin{ejemplo} La serie  
\[
	\sum\limits_{n=0}^{\infty}\bigg(\frac{1}{M}\bigg)^nz^n,
\]
tiene radio de convergencia:

\[\frac{1}{R}=\limsup\limits_{n\to \infty}\bigg(\bigg(\frac{1}{M}\bigg)^n\bigg)^{1/n}=\lim\limits_{n\to\infty}\bigg(\bigg(\frac{1}{M}\bigg)^n\bigg)^{1/n}=\frac1M\]

Luego $R=M$. 

\end{ejemplo}

\begin{ejemplo} Fijemos $M>0$ y $n$ un natural tal que $[n/2] >M$ (aquí $[x]$ es la parte entera de $x$). Entonces,
como $n-[n/2]\geq [n/2]>M$

\[ 
\begin{split}
n!=n(n-1)\cdots 1 &> n(n-1)\cdots (n-[n/2]) \\
&>\underbrace{M\cdots M }_{[n/2]-\hbox{veces}}\\
&\geq M^{[n/2]}\\
&> M^{n/3}\\
\end{split}
\]
Luego 
\[\frac{1}{R}:=\limsup_{n\to\infty} (1/n!)^{1/n}\leq \lim_{n\to\infty} \bigg(\frac{1}{M^{n/3}}\bigg)^{1/n}=\frac{1}{\sqrt[3]{M}}\]
Como $M$ es arbitrario, haciendo $M\to\infty$ vemos que el radio de convergencia de la serie  $\sum\limits_{n=0}^{\infty}\frac{1}{n!}z^n$ 
es $R=\infty$. 
\end{ejemplo}

\begin{teorema}  Consideremos la serie:

\[
	\sum\limits_{n=0}^{\infty}a_n(z-z_0)^n,
\]

Entonces:

\begin{enumerate}
	\item Si $|z-z_0|<R$, la serie converge absolutamente en $z$. 
	\item Si $|z-z_0|>R$, la serie diverge.
	\item Si $|z-z_0|=R$, no se afirma nada.
	
\end{enumerate}
\end{teorema}
	
 \begin{demo} Se puede suponer sin perdida de generalidad $z_0=0$. Supongamos $0<R<\infty$. Sea $L=1/R$ y tomemos $\epsilon>0$ pequeño. Como
\[\lim_{n\to\infty}\sup\{|a_n|^{1/n},|a_{n+1}|^{1/n+1},\ldots\}=L\]
para $n_0$ suficientemente grande
\[\sup\{|a_n|^{1/n},|a_{n+1}|^{1/n+1},\ldots\}<L+\epsilon.\]
 Así
\[|a_n|^{1/n}<L+\epsilon\quad\hbox{para } n\geq n_0.\]
Elijamos $0<r<1/(L+\epsilon)<1/L=R$. Si $|z|<r$ entonces
\[|a_n||z|^n<(L+\epsilon)^nr^n\quad\hbox{para } n\geq n_0.\]
Pero $r(L+\epsilon)<1$. La desigualdad de arriba y el teorema de comparación (notar que el miembro de la derecha forma una serie geométrica) implican que la serie converge absolutamente para este $z$.  Como $\epsilon$ es arbitrario, dado cualquier $z$, con $|z|<1/L$, tenemos un $\epsilon$ lo suficientemente chico para que $|z|<1/(L+\epsilon)$. \end{demo}

 \begin{ejercicio} Demostrar los casos $R=0$, $R=\infty$ y el segundo inciso.
 \end{ejercicio}
  
\begin{teorema}  La función
\[
f(z)=	\sum\limits_{n=0}^{\infty}a_n(z-z_0)^n,
\]
es diferenciable dentro en $\{z:|z-z_0|<R\}$. Además
\[
f'(z)=g(z):=	\sum\limits_{n=1}^{\infty}na_n(z-z_0)^{n-1},
\]
teniendo esta serie el mismo radio de convergencia que el de $f$.
\end{teorema}

\begin{demo} Nuevamente supondremos $z_0=0$. La afirmación sobre el radio de convergencia es consecuencia de que $\lim_{n\to\infty}n^{1/n}=1$. Como el radio $R'$ de convergencia de $g$ es:

\[
\begin{split}
\frac{1}{R'}&=\limsup_{n\to\infty}|a_{n+1}(n+1)|^{1/(n+1)}\\
&\underbrace{=}_{\hbox{\emph{ Ejercicio}}}\limsup_{n\to\infty}|a_{n+1}|^{1/(n+1)}\lim_{n\to\infty}|(n+1)|^{1/(n+1)}\\
&=\limsup_{n\to\infty}|a_{n+1}|^{1/(n+1)}=\frac{1}{R}
\end{split}\]

Ahora veamos que  $f'=g$.   Sea $0<r<R$, $|z_0|<r$ y $N\in\mathbb{N}$. Pongamos:
\[f(z)=S_N(z)+E_N(z),\]
\[S_N(z)=\sum_{n=0}^{N}a_nz^n\quad\hbox{y}\quad E_N(z)=\sum_{n=N+1}^{\infty}a_nz^n\]
Tomemos $|h|<r-|z_0|$, así $|z_0+h|<r$.  Tenemos
\[\begin{split}
\frac{f(z_0+h)-f(z_0)}{h}-g(z_0)&= \frac{S_N(z_0+h)-S_N(z_0)}{h}-S_N'(z_0)\\
				&+S_N'(z_0)-g(z_0)\\
				&+\frac{E_N(z_0+h)-E_N(z_0)}{h}
\end{split}\]

Ahora si $\epsilon>0$

\[\begin{split}
				&\bigg|\frac{E_N(z_0+h)-E_N(z_0)}{h}\bigg|\leq\sum_{n=N+1}^{\infty}|a_n|\bigg|\frac{(z_0+h)^n-z_0^n}{h}\bigg|\\
				&=\sum_{n=N+1}^{\infty}|a_n|(|z_0|^{n-1}+|z_0|^{n-2}h+\cdots+h^{n-1})\\
				&\leq2\sum_{n=N+1}^{\infty}|a_n|nr^{n-1}<\epsilon\\
\end{split}\]
Para $N$ suficientemente grande. Además como $S_N'(z)\to g(z)$ cuando $N\to\infty$ podemos elegir, a su vez, $N$ suficientemente grande para que

\[
	|S_N'(z_0)-g(z_0)|<\epsilon
\]

 Fijemos un $N$ que satisfaga las condiciones anteriores. Ahora podemos encontrar $\delta>0$ para que $|h|<\delta$ cumpla que 
\[
\bigg|\frac{S_N(z_0+h)-S_N(z_0)}{h}-S_N'(z_0)\bigg|<\epsilon.
\]

Esto muestra que $f'(z_0)=g(z_0)$ y por consiguiente $f$ es derivavble.  
\end{demo}


 


\begin{corolario}  Una serie de potencias es infinitamente diferenciable. Las sucesivas derivadas se obtienen derivando término a término la serie. El radio de convergencia se conserva.
\end{corolario}

\begin{ejemplo}  Hemos visto que la serie:
\[
	f(z)=\sum_{n=0}^{\infty}\frac{1}{n!}z^n
\]
tiene radio de convergencia infinito y por ende converge en $\com$. Ahora vemos que
\[f'(z)=\sum_{n=1}^{\infty}\frac{1}{(n-1)!}z^{n-1} =\sum_{n=0}^{\infty}\frac{1}{n!}z^n
\]

Luego $f$ resuelve la simple ecuación diferencial $f'(z)=f(z)$. La misma ecuación es resuelta por $g(z)=e^z$. Además $f(0)=g(0)=1$. Por el Teorema de existencia y unicidad $f(z)=g(z)$ para todo $z$. Hemos probado la importante fórmula.

\boxedeq{e^z=\sum_{n=0}^{\infty}\frac{1}{n!}z^n}{eq:exp}
\end{ejemplo}





\subsection{Funciones analíticas}

\begin{definicion} Una función $f:\Omega\subset\com\to\com$ se dirá analítica si para cada $z_0\in\Omega$, existe un $R$ y $a_n\in\com$, tal que vale la igualdad:

\[ f(z)=\sum_{n=0}^{\infty}a_n(z-z_0)^n,\quad\hbox{para } |z-z_0|<R \]
\end{definicion}


\begin{ejercicio} Si $f$ es analítica tenemos la siguiente fórmula
\[
	a_n=\frac{f^{(n)}(z_0)}{n!}
\]
para los coeficientes $a_n$. 
\end{ejercicio}

 \begin{teorema}\label{teor_oper_series}\textbf{Operaciones entre series de potencias} Supongamos que $f(z)=\sum_{n=0}^{\infty}a_n(z-z_0)^n$ y  $g(z)=\sum_{n=0}^{\infty}b_n(z-z_0)^n$ son series de potencias con radio de convergencia mayor o igual a $R>0$. Entonces $f+g$ y $fg$ son funciones analíticas que tienen por desarrollo en serie
\[
    \begin{split}
      (f+g)(z)&=\sum_{n=0}^{\infty}(a_n+b_n)(z-z_0)^n\\
      (fg)(z)&=\sum_{n=0}^{\infty}\left(\sum_{k=0}^na_kb_{n-k}\right)(z-z_0)^n,\\
    \end{split}
\]
y los radios de convergencia de las series anteriores es, al menos, $R$.
\end{teorema}


\section{Solución de EDO mediante series de potencias. Método coeficientes indeterminados}



\subsection{Método coeficientes indeterminados}

Dada una EDO

\begin{equation} \hbox{(1)}\quad  F(x,y,y',\ldots,y^{(n)})=0\end{equation}
queremos encontrar el desarrollo en series de potencias de la solución general a esta ecuación. El método que estudiaremos se denomina \emph{metodo de los coeficientes indeterminados}. Consiste en proponer el desarrollo en serie de la solución

\[y(x)=a_0+a_1(x-x_0)+a_2(x-x_0)^2+\cdots  \]
remplazar $y(x)$ por este desarrollo en  en la ecuación (1) y tratar de resolver la ecuación resultante para los coeficientes (indeterminados) $a_n$. El método suele funcionar en algunas ecuaciones. Desarrollemos un ejemplo.

 

\begin{ejemplo} Hallar el desarrollo en serie de la solución del siguiente pvi 
\[\left\{\begin{array}{l l} y'&=y\\ y(0)&=1\end{array}\right.\]
La solución, es bien sabido, es $y(x)=e^x$,  pero pretendemos reencontrarla por el método expuesto. Escribimos
\[\begin{split}
   y&=a_0+a_1x+a_2x^2+\cdots+a_nx^n+\cdots\\
   y'&=a_1+2a_2x+3a_3x²+\cdots+(n+1)a_{n+1}x^n+\cdots
  \end{split}
\]
La igualdad $y'=y$ implica que 
\[\begin{split}
   a_1&=a_0\\
   a_2&=\frac{a_1}{2}\\
   a_3&=\frac{a_2}{3}\\
      &\,\,\,\,\vdots \\
   a_{n+1}&=\frac{a_{n}}{n+1}
 \end{split}
\]
Si iteramos la fórmula $a_{n+1}=a_{n}/(n+1)$, obtenemos 
\[a_n=\frac{1}{n}a_{n-1}=\frac{1}{n(n-1)}a_{n-2}=\cdots=\frac{1}{n(n-1)\cdots 1}a_{0}=\frac{a_0}{n!}.\]
Pero $a_0=y(0)=1$. Luego 
\boxedeq{a_n=\frac{1}{n!}}{eq:exp}




  También podemos usar SAGE para obtener el resultado.
\begin{sageblock}
"""
Método de coeficientes indeterminados para
la ecuación y'=y
"""
#La serie la sumamos hasta X^orden
orden=7
# Lista coeficientes
Lista=['a%s'%i for i in range(orden)]
#expresiones polinómicos con los an
A = PolynomialRing(QQ,Lista)
#Series con coeficientes en A
Series.<X> = PowerSeriesRing(A)
#n-upla de los a_n
Coef=A.gens() 
#serie truncada en orden
y=sum(Coef[i]*X^i for i in range(orden)) 
#sustituyo en la ecuación
Ecua=y.derivative()-y
#lista de los a_n convertidos en expresión 
a= [SR(i) for i in A.gens()] 
""" A las ecuaciones le agrego a0=1 y evito usar la
última pues tiene un problema que viene del truncamiento
"""
Ecuaciones=[a[0]-1,list(Ecua)[:-1]] 
#resuelvo las relaciones de recurrencia
Sol_a_n=solve(Ecuaciones,a) 
"""Lista de los coeficientes convertidos en racionales """
L=[QQ(f.rhs()) for f in Sol_a_n[0]] 
# Desarrollo en serie de la solución
y=sum([L[i]*X^i for i in range(orden)])+O(X^orden)
\end{sageblock}

Vemos que la solución es 
\boxedeq{y(x)=\sage{y}}{eq:sol_sage} 
\end{ejemplo}






\subsection{Relaciones de recurrencia}

La expresión $a_{n+1}=\frac{a_{n}}{n+1}$ es un ejemplo de \href{http://es.wikipedia.org/wiki/Relación_de_recurrencia}{relación de recurrencia}.
\begin{definicion} Una  relación de recurrencia para una sucesión $b_n$ de números reales es una sucesión de 
funciones $f_n:\rr^n\to\rr$ que relaciona $b_{n+1}$ con los términos anteriores de la sucesión por medio de 
la expresión
\boxedeq{b_{n+1}=f_n(b_1,\ldots,b_n)}{eq:recu}
 Resolver una relación de recurrencia es encontrar una fórmula explícita de $b_n$ como función de $n$.
 \end{definicion}


Hay técnicas y métodos para resolver relaciones de recurrencia que guardan analogías con técnicas y métodos de resolver ecuaciones diferenciales.  No vamos a desarrollar este tema en este curso, sólo agregamos que  SAGE resuelve relaciones recurrentes. Para ello se necesita de un módulo extra, también basado en \href{https://www.python.org/}{Python}, llamado \href{http://sympy.org/en/index.html}{Sympy}. Este módulo tiene el comando \texttt{rsolve} para resolver relaciones de recurrencia. Sympy tiene sus propios comandos, que son parecidos pero no iguales a SAGE.

\begin{ejemplo}Resolvamos la sucesión de Fibonacci $a_{n+2}=a_{n+1}+a_n$ con SAGE.
\begin{sageblock}
""" Resolvemos la relación de recurrencia 
de los números de Fibonacci"""
#Importamos Sympy 
from sympy import *
#En Sympy se declaran símbolos en lugar de variables
n=Symbol('n',integer=True)
#Y así se declaran funciones
y = Function('y')
#Así una ecuación
f=Equality(y(n),y(n-1)+y(n-2))
#Así la resolvemos
rsolve(f,y(n))

\end{sageblock}  

 El resultado es
 \boxedeq{a_n=\sage{rsolve(f,y(n))}}{}
Las constantes arbitrarias $C_0$ y $C_1$ aparecen porque una relación de recurrencia no tiene una única solución. Se dice que una relación de recurrencia tiene orden $k$ o es de $k$-términos si el coeficiente $a_n$ se expresa en función de $k$ de los anteriores. En general la solución general de una relación de recurrencia de $k$-términos tiene $k$ constantes arbitrarias. Por consiguiente, si queremos una única solución debemos tener $k$ relaciones extras. Usualmente esto se consigue dando los valores de los $k$-primeros términos $a_0,\ldots,a_k$. Por ejemplo, en  la sucesión de Fibonacci si pedimos $a_0=a_1=1$.

\begin{sageblock}
reset()
n=var('n')
C0,C1=var('C0,C1')
A=C0*(1/2+sqrt(5)/2)^n+C1*(1/2-sqrt(5)/2)^n
Cval=solve([A(n=0)-1,A(n=1)-1],[C0,C1],solution_dict=True)
Fib=A.subs(Cval[0])
[Fib(n=i).expand() for i in range(10)]
\end{sageblock}  
Llegamos a la fórmula
\boxedeq{a_n=\sage{Fib}.}{}
Los primeros 20 números de Fibonacci son 
\[\sage{[Fib(n=i).expand() for i in range(20)]}.\]

\end{ejemplo}

\subsection{Serie binomial}

 Se puede utilizar el método de coeficientes indeterminados para encontrar desarrollos en serie de una función  $f$. La técnica consiste en encontrar un pvi que satisfaga $f$ y le aplicamos el método de coeficientes indeterminados a ese pvi.

\begin{ejemplo}  Encontrar el desarrollo en serie de la función
\[y(x)=(1+x)^p\quad p\in\rr\]
La función $y(x)$ resuelve el pvi  $(1+x)y'(x)=py$, $y(0)=1$. apliquemos el método de coeficientes indeterminados a este pvi.
Como
\[\begin{split}
   y&=a_0+a_1x+a_2x^2+\cdots+a_nx^n+\cdots\\
   y'&=a_1+2a_2x+3a_3x²+\cdots+(n+1)a_{n+1}x^n+\cdots
  \end{split}
\]
Tenemos
\[\begin{split}
   py=&pa_0+pa_1x+pa_2x^2+\cdots+pa_nx^n+\cdots\\
  (1+x)y'=&a_1+2a_2x+3a_3x²+\cdots+(n+1)a_{n+1}x^n+\cdots\\
          &+a_1x+2a_2x^2+3a_3x³+\cdots+na_{n}x^n+\cdots\\
-------&----------------------\\
0=(1+x)y'-py =& (a_1-pa_0)+(a_1+2a_2-pa_1)x+\cdots +((n+1)a_{n+1}+na_n-pa_n)x^n+\cdots
  \end{split}
\]
Tenemos la relación
\[ a_{n+1}=\frac{(p-n)}{n+1}a_n.
\]
Que es una relación de reciurrencia de un sólo término. Estas relaciones se resuelven iterando la relación de manera sucesiva de modo de relacionar $a_n$ con $a_0$
\[a_n=\frac{(p-n+1)}{n}a_{n-1}=\frac{(p-n+1)(p-n+2)}{n(n-1)}a_{n-2}=\cdots=\frac{(p-n+1)(p-n+2)\cdots p}{n!}a_0.\]
Como $a_0=y(0)=1$ vemos que
\boxedeq{a_n=\frac{(p-n+1)(p-n+2)\cdots p}{n!}.}{eq:coef_bin}
Si $p\in\mathbb{N}$ entonces $a_n=0$ para $n>p$. Esto es claro, por otro lado, ya que en este caso $(1+x)^p$ es un polinomio. Por fórmula del binomio de Newton los coeficientes cuando $p\in \mathbb{N}$  no son más que los coeficientes binomiales
\[a_n=\binom{p}{n}\]
 Cuando $p\in\mathbb{R}$ aún vamos a seguir denominado a $a_n$, dado por la fórmula \eqref{eq:coef_bin},   el coeficiente binomial. La serie resultante se llama la serie binomial. Cuando $p\in\mathbb{R}-\mathbb{N}$ es una serie infinita y no  un polinomio. Notar que para $p$ no entero positivo
\[\lim\limits_{n\to\infty}\frac{|a_{n+1}|}{|a_n|}=\lim\limits_{n\to\infty}\frac{|p-n|}{|n+1|}=1\]
Luego la serie tiene radio de convergencia 1.  Hemos demostrado asi que vale la siguiente fórmula, que es una generalización de la fórmula binomial de Newton
\boxedeq{(1+x)^p=1+px+\frac{p(p-1)}{2!}x^2  +\cdots=1+\binom{p}{1}x+\binom{p}{2}x^2+\cdots}{}
Esta importante serie se denomina \href{http://en.wikipedia.org/wiki/Binomial_series}{serie binomial}.


\begin{sageblock}
"""
Método de coeficientes indeterminados para
la ecuación (1+x)y'=py , y(0)=1
"""
orden=4
#En la lista de coeficientes debemos poner p
Lista=['a%s'%i for i in range(orden)]+['p']
A = PolynomialRing(QQ,Lista)
Series.<X> = PowerSeriesRing(A)
Coef=A.gens() 
y=sum(Coef[i]*X^i for i in range(orden)) 
Ecua=(1+X)*y.derivative()-A.gens()[-1]*y
a= [SR(i) for i in A.gens()[:-1]] 
Ecuaciones=[a[0]-1,list(Ecua)[:-1]] 
Sol_a_n=solve(Ecuaciones,a) 
L=[f.rhs().factor() for f in Sol_a_n[0]] 
x=var('x')
y=sum([L[i]*x^i for i in range(orden)])
\end{sageblock}

\boxedeq{y(x)=\sage{y}.}{}

\end{ejemplo}

\subsection{Oscilador armónico}

\begin{ejemplo} Consideremos la ecuación 
\[y''+\omega^2y=0.\]
Esta es una ecuación de segundo orden. Veamos si el método de coeficientes indeterminados nos lleva a la solución. Se tiene
\[\begin{split}
    \omega^2y&=\omega^2a_0+\omega^2a_1x+\omega^2a_2x^2+\cdots+\omega^2a_nx^n+\cdots\\
  y''&=2a_2+2\cdot 3a_3x+\cdots+(n+1)(n+2)a_{n+2}x^n+\cdots\\
--&---------------------------\\
0=y''+\omega^2y =& (\omega^2a_0+2a_2)+(\omega^2a_1+2\cdot 3a_3)x+\cdots +(\omega^2a_n+(n+1)(n+2)a_{n+2})x^n+\cdots
  \end{split}
\]
Encontramos la relación de recurrencia de dos términos
\boxedeq{a_{n+2}=-\frac{\omega^2a_n}{(n+1)(n+2)}.}{}
Notar que  en este caso $a_1$ y, obviamente, $a_0$ no se relacionan con ningún coeficiente anterior. Por este motivo es de esperar que podamos elegir de manera arbitraria $a_0$ y $a_1$. Esto está de acuerdo con el hecho que remarcamos antes de que en una relación de recurrencia de dos términos aparecen dos constantes arbitrarias y también está de acuerdo con que la solución general de una ecuación de segundo orden tiene dos constantes arbitrarias. En este caso resolvemos la relación de recurrencia relacionando $a_n$ con $a_0$ cuando $n$ es impar y con $a_1$ cuando es impar. Concretamente si $n=2k$, $k\in\mathbb{N}$,
\[a_{2k}=-\frac{\omega^2}{2k(2k-2)}a_{2k-2}=\cdots=(-1)^k\frac{\omega^{2k}}{(2k)!}a_0.\]
En cambio si $n=2k+1$ es impar
\[a_{2k+1}=-\frac{\omega^2}{2k+1}{2k-1}a_{2k-1}=\cdots=(-1)^k\frac{\omega^{2k}}{(2k+1)!}a_1.\]
Agrupando los términos pares y los impares de la serie de potencias resultante queda
\boxedeq{
   \begin{split}
     y(x) &=a_0\sum_{k=0}^{\infty}(-1)^k\frac{1}{(2k)!}\left(\omega x\right)^{2k}+\frac{a_1}{\omega}\sum_{k=0}^{\infty}(-1)^k\frac{1}{(2k+1)!}\left(\omega x\right)^{2k+1}\\
&=a_0\cos\omega x+\frac{a_1}{\omega}\sen\omega x
   \end{split}
 }{}
La última igualdad es conocida de las asignaturas de análisis. Es facil verificar, lo dejamos de ejercicio, que las series involucradas tienen radio de convergencia infinito.

Podemos hacer los cálculos anteriores con SAGE

\begin{sageblock}
"""
Método de coeficientes indeterminados para
la ecuación y''+omega^2y=0
"""
orden=8
Lista=['a%s'%i for i in range(orden)]+['omega']
A = PolynomialRing(QQ,Lista)
Series.<X> = PowerSeriesRing(A)
Coef=A.gens() 
y=sum(Coef[i]*X^i for i in range(orden)) 
Ecua=y.derivative(2)+A.gens()[-1]^2*y
a= [SR(i) for i in A.gens()[:-1]] 
Ecuaciones=[a[0]-1,a[1],list(Ecua)[:-2]]
Sol_a_n=solve(Ecuaciones,a) 
L=[f.rhs() for f in Sol_a_n[0]] 
x=var('x')
y1=sum([L[i]*x^i for i in range(orden)])
Ecuaciones=[a[0],a[1]-1,list(Ecua)[:-2]]
Sol_a_n=solve(Ecuaciones,a) 
L=[f.rhs() for f in Sol_a_n[0]] 
y2=sum([L[i]*x^i for i in range(orden)])
\end{sageblock}
 Obtenemos las soluciones
\[
    \begin{split}
      y_1(x)&=\cdots\sage{y1}=\cos\omega x\\
      y_2(x)&=\cdots\sage{y2}=\frac{1}{\omega}\sen\omega x\\
    \end{split}
\]
\end{ejemplo}

\subsection{Ecuación de Legendre. Primera aproximación}

\begin{ejemplo} Consideremos la \href{http://es.wikipedia.org/wiki/Polinomios_de_Legendre}{ecuación de Legendre}.
\boxedeq{(1-x^2)y''-2xy'+p(p+1)y=0,}{eq:ecua_lege}
donde $p>0$. Esta ecuación aparece en muchas aplicaciones, por ejemplo en muchos problemas que involucran funciones definidas en esferas, como es el caso de los modos normales de vibración de una esfera y en problemas de potenciales esféricos

\[\begin{split}
   p(p+1)y= p(p+1)&a_0+ p(p+1)a_1x+ p(p+1)a_2x^2+\cdots+ p(p+1)a_nx^n+\cdots\\
  -2xy'=&-2a_1x-4a_2x^2-6a_3x^3+\cdots-2na_{n}x^n+\cdots\\
(1-x^2)y''=& 2a_2+2\cdot 3a_3x+\cdots +(n+1)(n+2)a_{n+2}x^n+\cdots\\
          &-2a_2x^2-2\cdot 3a_3x^3-\cdots -(n-1)na_{n}x^n-\cdots\\
     -------&----------------------\\
0=(1-x^2)y''-2xy+p(p+1)y =& (p(p+1)a_0+2a_2)+(p(p+1)a_1-2a_12\cdot 3a_3)x+\cdots\\
     &+ \left( (p(p+1)-n(n+1)\right)a_n+n(n+1)a_{n+2})x^n+\cdots
  \end{split}
\]
Obtenemos la relación de recurrencia
\boxedeq{a_{n+2}=-\frac{(p-n)(p+n+1)}{(n+1)(n+2)} a_n}{eq:leg_rel_recu}
 Ahora vamos a dividir la serie en los términos pares e impares
\[\sum\limits_{n=0}^{\infty}a_nx^n= \sum\limits_{k=0}^{\infty}a_{2k}x^{2k}+\sum\limits_{k=0}^{\infty}a_{2k+1}x^{2k+1}\]
A cada una de estas series le podemos aplicar el criterio de la razón usando la fórmula de recuerrencia de arriba. Por ejemplo para los términos pares
\[\lim\limits_{k\to\infty}\frac{|a_{2k+2}x^{2k+2}|}{|a_{2k}x^{2k}|}=\lim\limits_{k\to\infty}\frac{|p_0-2k||p_0+2k+1|}{(2k+1)(2k+2)}|x|^2=|x|^2\]
De modo que la serie tiene radio de convergencia 1. La misma situación ocurre con la serie de términos impares. Esto muestra que la serie en su conjunto tambien tiene radio de convergencia igual a 1. Era previsible que el radio de convergencia no fuese mayor a 1, pues la forma explícita de la ecuación de la ecuación de Legendre es
\[y''-\frac{2x}{(1-x^2)}y'+\frac{p(p+1)}{(1-x^2)}y=0.\]
Se observa que $1$ y $-1$ son puntos singulares de la ecuación.

Podemos relacionar cualquier coeficiente de índice par $a_{2k}$ con el $a_0$ y cualquiera con índice impar $a_{2k+1}$ con el $a_1$. Esto lo resolveremos con SAGE

\begin{sageblock}
"""
Método de coeficientes indeterminados para
la ecuación(1-x^2)y''-2xy'+p(p+1)y=0
"""
orden=10
Lista=['a%s'%i for i in range(orden)]+['p']
A = PolynomialRing(QQ,Lista)
Series.<X> = PowerSeriesRing(A)
Coef=A.gens() 
y=sum(Coef[i]*X^i for i in range(orden)) 
p=A.gens()[-1]
Ecua=(1-X^2)*y.derivative(2)-2*X*y.derivative()+p*(p+1)*y
a= [SR(i) for i in A.gens()[:-1]] 
Ecuaciones=list(Ecua)[:-2]
Sol_a_n=solve(Ecuaciones,a) 
Sol_a_n=[i.lhs()==i.rhs().factor() for i in Sol_a_n[0]]
\end{sageblock}
Obtenemos que la solución a la relación de recurrencia de dos términos es
\[\sage{vector(Sol_a_n).column()}\]
En general 
\[a_{2n}=\frac{(p+1)(p_0+3)\cdots (p+2n-1) \times p(p-2)\cdots (p-2n+2)}{(2n)!}a_0\]
y
\[a_{2n+1}=\frac{(p+2)(p+4)\cdots (p+2n) \times (p-1)(p-3)\cdots (p-2n+1)}{(2n+1)!}a_1\]
Podemos elegir $a_0$ y $a_1$ de manera arbitraria (esto está de acuerdo con que en una ecuación de orden 2 aparece 2 constantes de integración) y por las relaciones anteriores deducir el valor de los restantes $a_n$.

Un caso especial se plantea cuando  $p\in\mathbb{N}$. En esa situación vemos que infinitos de los $a_n$ resultan iguales a cero. De hecho una de las series, la de términos impares o la de términos pares acorde a que $p$ sea par o impar respectivamente,  se trunca. Supongamos que esto ocurre con la serie de términos impares, es decir $p$ es entero positivo impar. Si ahora tomamos $a_0=0$ toda la serie de términos pares se hará cero. Para $a_1$ elijo algún valor no nulo y esto me garantiza que los términos impares son no nulos hasta el término $a_{n}$, pero a partir de allí también se hacen cero. Es usual elegir $a_1$ para que $y(1)=1$. Nos queda definida así una función polinómica que se denomina \href{http://es.wikipedia.org/wiki/Polinomios_de_Legendre}{polinomio de Legendre} y se denota por $P_p$. Cuando $p$ es par hacemos una construcción análoga, quedando un polinomio $P_p$, con todas potencias pares, tal que $P_p(1)=1$.

Todos estos cálculos los programamos en la siguiente función de SAGE. Esta función tiene un argumento $n$ que debe ser un número natural y devuelve el correspondiente polinomio de Legendre.

\begin{sageblock}
def legendre(n):
    """
    Método de coeficientes indeterminados para
    la ecuación(1-x^2)y''-2xy'+p(p+1)y=0
    """
    if n not in NN:
        print "n no es entero"
    else:
        orden=n+2
        Lista=['a%s'%i for i in range(orden)]
        A = PolynomialRing(QQ,Lista)
        Series.<X> = PowerSeriesRing(A)
        Coef=A.gens() 
        y=sum(Coef[i]*X^i for i in range(orden)) 
        Ecua=(1-X^2)*y.derivative(2)-2*X*y.derivative()+n*(n+1)*y
        a= [SR(i) for i in A.gens()] 
        Ecuaciones=list(Ecua)[:-2]
        s=var('s')
        if n%2==0:
            Ecuaciones+=[a[0]-s,a[1]]
        else:
            Ecuaciones+=[a[0],a[1]-s]
        Sol_a_n=solve(Ecuaciones,a) 
        L=[f.rhs() for f in Sol_a_n[0]] 
        x=var('x')
        y=sum([L[i]*x^i for i in range(orden)])
        sol=solve(y(x=1)==1,s)
        return y.subs(sol[0])
\end{sageblock}
Con la ayuda de esta función podemos generar rápidamente una tabla de polinomios de Legendre
\begin{sageblock}
L=[legendre(i) for i in range(10)]
\end{sageblock} 
\[\sage{vector(L).column()}\]
y graficarlos
\[\sageplot[scale=.5]{plot(L)}\]
\end{ejemplo}

\section{Teorema fundamental sobre puntos ordinarios}


\begin{definicion} Dada la ecuación diferencial
\[y''(x)+p(x)y'(x)+q(x)y(x)=0\]
donde $p,q$ son funciones   definidas en algún intervalo abierto $I$, diremos que $x_0\in I$ es un \emph{punto ordinario} de la ecuación si $p$ y $q$ son analíticas en $x_0$. Un punto no ordinario se llama \emph{singular}.
\end{definicion}

\begin{ejemplo} En la ecuación del oscilador armónico
\[y''+\omega^2 y=0 \]
todo punto es ordinario. 
\end{ejemplo}

\begin{ejemplo} En la ecuación de Legendre
\[(1-x^2)y''-2xy'+p(p+1)y=0 \]
$1$ y $-1$ son puntos singulares, otros valores de $x$ son puntos ordinarios. 
\end{ejemplo}

Antes de ir al Teorema más importante de esta sección vamos a enunciar un  lema que nos resultará útil.

\begin{lema}\label{lema:comp_recu}\textbf{Comparación.} Supongamos que tenemos una relación de recurrencia de dos términos
 \begin{equation}\label{eq:recu2} a_{n+2}=f_n(a_n,a_{n+1})\end{equation}
donde las funciones $f_n$ son crecientes respecto a sus dos variables. Si la sucesión  $\{a_n\}$ resuelve \eqref{eq:recu2}, la sucesión  $\{b_n\}$ resuelve la desigualdad
 \[b_{n+2}\leq f_n(b_n,b_{n+1})\]
y además vale que  $b_0\leq a_0$ y $b_1\leq a_1$, entonces $b_n\leq a_n$ para todo $n=2,3,\ldots$. En particular la afirmación se satisface cuando $f_n$ es lineal con coeficientes positivos, es decir $f_n(x,y)=\alpha_nx+\beta_ny$, con $\alpha,\beta\geq 0$.
\end{lema}
\begin{demo}
Es muy sencilla y la dejamos de ejercicio (evidentemenete hay que utilizar el principio de inducción). 
\end{demo}


\begin{teorema}\textbf{Teorema Fundamental Sobre Puntos Ordinarios.} Sea $x_0$ un punto ordinario de la ecuación
\[y''(x)+p(x)y'(x)+q(x)y(x)=0\]
y sean $a_0,a_1\in\mathbb{R}$. Existe una solución de la ecuación que es analítica en un entorno de $x_0$ y que satisface $y(x_0)=a_0$ e $y'(x_0)=a_1$. El radio de convergencia del desarrollo en serie de $y$ es al menos tan grande como el mínimo de los radios de convergencia de los desarrollos en serie de $p$ y $q$.
\end{teorema}

\begin{demo}
Supongamos, sin perder generalidad, que $x_0=0$. Consideremos que los desarrollos en serie de potencias de $p$ y $q$.
\begin{equation}\label{eq:series_p_q}p(x)=\sum_{n=0}^{\infty}p_nx^n\quad\text{y}\quad q(x)=\sum_{n=0}^{\infty}q_nx^n.
\end{equation}
Supongamos que ambas series convergen en $|x|<R$, para cierto $R>0$. Vamos a aplicar el método de coeficientes indeterminados. Tenemos
\[
    \begin{split}
      y&=\sum_{n=0}^{\infty}a_nx^n=a_0+a_1x+\cdots+a_nx^n+\cdots\\
      y'&=\sum_{n=0}^{\infty}(n+1)a_{n+1}x^n=a_1+2a_2x+\cdots+(n+1)a_{n+1}x^n+\cdots\\
      y''&=\sum_{n=0}^{\infty}(n+1)(n+2)a_{n+2}x^n= 2a_2+2\cdot 3a_3x+\cdots+(n+1)(n+2)a_{n+2}x^n+\cdots.
    \end{split}
\]
Por el  Teorema \ref{teor_oper_series} 
\[
   \begin{split}
     q(x)y&=\left(\sum_{n=0}^{\infty}a_nx^n\right)\left(\sum_{n=0}^{\infty}q_nx^n\right)=\sum_{n=0}^{\infty}\left(\sum_{k=0}^na_kq_{n-k}\right)x^n,\\
     p(x)y'&=\left(\sum_{n=0}^{\infty}(n+1)a_{n+1}x^n\right)\left(\sum_{n=0}^{\infty}p_nx^n\right)=\sum_{n=0}^{\infty}\left(\sum_{k=0}^n(k+1)a_{k+1}p_{n-k}\right)x^n.
   \end{split}
\]
Sustituyendo estos, y los anteriores, desarrollos en la ecuación, obtenemos
\[\sum_{n=0}^{\infty}\left\{ (n+1)(n+2)a_{n+2}+ \sum_{k=0}^na_kq_{n-k}+ \sum_{k=0}^n(k+1)a_{k+1}p_{n-k} \right\}x^n.\]
De esta forma deducimos la ecuación de recurrencia que se satisface en una ecuación lineal general de segundo orden en un punto ordinario.
\boxedeq{a_{n+2}=-\frac{ \sum_{k=0}^n\left\{ a_kq_{n-k}+ (k+1)a_{k+1}p_{n-k}\right\}}{n(n+1)}}{eq:rel_recu_gral}
Dados los coeficientes $a_0$ y $a_1$ la relación de recurrencia determina los $a_n$, $n\geq 2$. Queda ver que la serie así definida tiene radio de convergencia al menos $R$. 

Tomemos $r$ tal que $0<r<R$. Como las series \eqref{eq:series_p_q} tienen radio de convergencia $R$, estas series convergen absolutamente para $x=r$. El criterio de convergencia de series conocido como criterio del resto implica que $p_nr^n,q_nr^n$ son suceciones que tienden a $0$. En particular estan acotadas, y por ello existe $M>0$ tal que
\[ |p_n|r^n,|q_n|r^n\leq M.\]
Tomando módulo en \eqref{eq:rel_recu_gral} y usando las desigualdades de arriba concluímos
\[
  \begin{split}
    |a_{n+2}|&\leq\frac{M}{(n+1)(n+2)r^n} \sum_{k=0}^n\left(|a_k|+ (k+1)|a_{k+1}|\right)r^k\\
&\leq\frac{M}{(n+1)(n+2)r^n} \sum_{k=0}^n\left(|a_k|+ (k+1)|a_{k+1}|\right)r^k+\frac{M|a_{n+1}|r}{(n+1)(n+2)}.
  \end{split}
  \]
En la última desigualdad se agregó un término en apariencia por capricho, pero este término nos servirá para complementar una expresión en el futuro. Definamos la sucesión $b_n$ como la solución de la siguiente relación de recurrencia
\begin{equation}\label{eq:bn_recu}b_{n+2}=\frac{M}{(n+1)(n+2)r^n} \sum_{k=0}^n\left(b_k+ (k+1)b_{k+1}\right)r^k+\frac{Mb_{n+1}r}{(n+1)(n+2)},
\end{equation}
con las condiciones iniciales $b_0=|a_0|$ y $b_1=|a_1|$. Por el Lema \ref{lema:comp_recu} tenemos que $|a_n|\leq b_n$ para todo $n$. Aplicando \eqref{eq:bn_recu} a $n-1$ en lugar de $n$
\begin{equation}\label{eq:bn_recu_2}b_{n+1}=\frac{M}{n(n+1)r^{n-1}} \sum_{k=0}^{n-1}\left(b_k+ (k+1)b_{k+1}\right)r^k+\frac{Mb_{n}r}{n(n+1)},
\end{equation}
Multiplicando  \eqref{eq:bn_recu} por $r$ y usando \ref{eq:bn_recu_2}
\[
\begin{split}
rb_{n+2}=&\frac{M}{(n+1)(n+2)r^{n-1}} \sum_{k=0}^{n-1}\left(b_k+ (k+1)b_{k+1}\right)r^k\\
&+\frac{Mr\left(b_n+ (n+1)b_{n+1}\right)}{(n+1)(n+2)}+\frac{Mb_{n+1}r²}{(n+1)(n+2)}\\
=&\frac{n(n+1)b_{n+1}-Mb_nr}{(n+1)(n+2)}
+\frac{Mr\left(b_n+ (n+1)b_{n+1}\right)}{(n+1)(n+2)}\\
&+\frac{Mb_{n+1}r²}{(n+1)(n+2)}\\
=&\frac{(n(n+1)+Mr(n+1)+Mr^2)}{(n+1)(n+2)}b_{n+1},
\end{split}
\]
Entonces 
\[\lim_{n\to\infty}\frac{|b_{n+2}x^{n+2}|}{|b_{n+1}x^{n+1}|}=\lim_{n\to\infty}\frac{(n(n+1)+Mr(n+1)+Mr^2)}{(n+1)(n+2)}\frac{|x|}{r}=\frac{|x|}{r}.
\]
Luego la serie $\sum_{n=0}^{\infty}b_nx^n$ converge para $|x|<r$. Como $|a_n|\leq b_n$ la misma afirmación es cierta para $\sum_{n=0}^{\infty}a_nx^n$. Como $r<R$ fue elegido arbitrariamente, tenemos que $\sum_{n=0}^{\infty}a_nx^n$ converge en $|x|<R$.
\end{demo}


\section{Puntos singulares, método de Frobenius}


\begin{definicion}\textbf{ Sigularidades, Polos.} Sea $f$ definida en un intervalo abierto $I$ con valores en $\rr$. Diremos que $f$ posee un \emph{polo de orden $k$} en $x_0\in\rr$, si la función $(x-x_0)^kf(x)$ es analítica en un entorno de $x_0$. Vale decir que $(x-x_0)^kf(x)$ se desarrolla en serie de potencias.
\[(x-x_0)^kf(x)=\sum_{n=0}^{\infty}a_n(x-x_0)^n.\]
En consecuencia
\[f(x)=\sum_{n=0}^{\infty}a_n(x-x_0)^{n-k}=\frac{a_0}{(x-x_0)^k}+\cdots+\frac{a_{k-1}}{(x-x_0)}+a_k+a_{k+1}(x-x_0)+\cdots.\]
Este tipo de desarrollo en serie es un caso particular de serie de Laurent. 

Cuando el orden de un polo es $1$ se lo denomina \emph{polo simple}.
\end{definicion}



\begin{definicion} Un punto singular $x_0$ de la ecuación
\[y''(x)+p(x)y'(x)+q(x)y(x)=0\]
se llama singular regular si $p(x)$ tiene un polo a lo sumo simple en $x_0$ y $q(x)$ tiene un polo a lo sumo de orden $2$ en $x_0$. Es decir
\[(x-x_0)p(x)\quad\hbox{y}\quad (x-x_0)^2q(x)\]
son analíticas en $x_0$.
\end{definicion}
Algunas de las ecuaciones más importantes de la Física-Matemática tienen puntos singulares regulares.

\begin{ejemplo} 1 y -1 son puntos singulares regulares de la ecuación de Legendre de orden $p$
\[y''-\frac{2x}{1-x^2}y'+\frac{p(p+1)}{1-x^2}y=0\]
\end{ejemplo}


\begin{ejemplo} 0 es un punto singular regular de la ecuación de Bessel de orden $p$
\[y''+\frac{1}{x}y'+\left(1-\frac{p^2}{x^2}\right)y=0\]
\end{ejemplo}

El método de coeficientes indeterminados puede fallar en los puntos donde $p$ y $q$ tienen polos. En su lugar vamos a proponer otro tipo de desarrollo en serie. Lo vamos a motivar con un ejemplo.

\begin{ejemplo} Consideremos la ecuación de Euler, para $p,q\in\rr$
\[y''+\frac{p}{x}y'+\frac{q}{x^2}y=0\]
o equivalentemente
\[x^2y''+pxy'+qy=0\]
Aquí es facil verificar que las funciones
\[P(x):=\frac{p}{x}\quad\hbox{ y }\quad Q(x):= \frac{q}{x^2}\]
satisfacen la relación
\[\frac{Q'+2PQ}{Q^{\frac{3}{2}}}\]
es constante. Cuando se daba esta condición el ejercicio 6 de la página 102 del libro de Simmons nos enseña que podemos reducir la ecuación a una ecuación con coeficientes constantes por medio del cambio de la variable independiente
\[z=\int\sqrt{Q}dx\]
En es caso, obviando las constantes, el cambio de variables que debemos hacer, asumamos $x>0$, es
\[z=\ln(x)\]
Seguramente, a esta altura del curso,  el alumno ya hizo los cálculos que muestran que la ecuación de Euler se transforma,  por medio del cambio de variables propuesto, en la ecuación a coeficientes constantes
\[y''+(p-1)y'+qy=0.\]
Cuya ecuación característica es
\[\lambda^2+(p-1)\lambda+q=0\]
Resolviendo esta ecuación con SAGE

\begin{sageblock}
s,p,q=var('s,p,q')
Raices=solve(s^2+(p-1)*s+q==0,s)
\end{sageblock}
Obtenemos las raíces 
\[s_1=\sage{Raices[0].rhs()}\quad\text{y}\quad s_2=\sage{Raices[1].rhs()}.\]
 Si $s_1\neq s_2$ dos soluciones linealmente independientes son:
\[y_1(z)=e^{s_1z}\quad\hbox{y}\quad y_2(z)=e^{s_2z}\]
 Si $s_1=s_2$
\[y_1(z)=e^{s_1z} \quad\hbox{y}\quad y_2(z)=ze^{s_1z}\]
son soluciones linealmente independientes. Asumamos que las raices $s_1$ y $s_2$ son reales, entonces como $z=\ln(x)$ las soluciones en términos de la variable $x$ son 
\[y_1(x)=x^{s_1}\quad\hbox{y}\quad y_2(x)=x^{s_2}\quad\text{para }  s_1\neq s_2\]
y
\[y_1(x)=x^{s_1} \quad\hbox{y}\quad y_2(x)=\ln(x)x^{s_1}\quad\text{para }  s_1= s_2\]



Estas funciones, a menos que $s_1$ y $s_2$ sean enteros positivos, ya no son analíticas en cero, pues una función analítica es derivable infinitas veces y claramente hay derivadas (o las mismas funciones si las raices son negativas) de $y_1$ e $y_2$ que son discontinuas. 
\end{ejemplo}


De modo que, como era de suponer, no podremos encontrar en general en un punto singular una solución analítica.  Vamos a intentar flexibilar nuestro método para incluir otro tipo de desarrollo en serie, que está inspirado en los resultados obtenidos para la ecuación de Euler. 

\begin{definicion} A una expresión de la forma
 \[y(x)=(x-x_0)^m(a_0+a_1(x-x_0)+a_2(x-x_0)^2+\cdots),\]
donde $m\in\rr$, lo llamaremos \emph{Serie de Frobenius}.
\end{definicion}
Las series de Frobenius no son series de potencias ni de Laurent ya que en ellas aparecen potencias no enteras.

El método de Frobenius consiste en proponer como solución de una ecuación diferencial una serie de Frobenius. Este método tiene éxito, por ejemplo, en los puntos sigulares regulares de ecuaciones diferenciales lineales de segundo orden.

\begin{ejemplo} En este ejemplo ilustramos el método de Frobenius. Consideremos la ecuación
\[y''+\left(\frac{1}{2x}+1\right)y'-\left(\frac{1}{2x^2}\right)y=0\]
Notar que $x=0$ es regular singular. Desarrollaremos el método de Frobenius, primero ``a mano'' y por último con SAGE. 

Proponemos como solución

 \[y(x)=x^m\sum_{n=0}^{\infty}a_{n}x^n=\sum_{n=0}^{\infty}a_{n}x^{n+m}\]
y calculamos
\[
    \begin{split}
     -\frac{1}{2x^2}y&=-\sum_{n=0}^{\infty}\frac{a_n}{2}x^{m+n-2}=-\frac{a_0}{2}x^{m-2}
-\frac{a_1}{2}x^{m-1}-\cdots-\frac{a_{n+2}}{2}x^{m+n}-\cdots\\
      y'&=\sum_{n=0}^{\infty}(m+n)a_{n}x^{m+n-1}=ma_0x^{m-1}+(m+1)a_1x^m+\cdots+(m+n+1)a_{n+1}x^{m+n}+\cdots\\
      \frac{1}{2x}y'&=\sum_{n=0}^{\infty}\frac{(m+n)a_{n}}{2}x^{m+n-2}=\frac{ma_0}{2}x^{m-2}+\frac{(m+1)a_1}{2}x^{m-1}+\cdots+\frac{(m+n+2)a_{n+2}}{2}x^{m+n}+\cdots\\
      y''&=\sum_{n=0}^{\infty}(m+n)(m+n-1)a_{n}x^{m+n-2}= m(m-1)a_0x^{m-2}+(m+1)ma_1x^{m-1}+\cdot 3a_3x+\cdots+(n+1)(n+2)a_{n+2}x^n+\cdots.
    \end{split}
\]
Sumando las cuatro igualdades miembro a miembro 

\[y''+\left(\frac{1}{2x}+1\right)y'-\left(\frac{1}{2x^2}\right)y= -\frac{a_0}{2}\frac{1}{x^2}+\left(-\frac{a_1}{2}+\frac{a_1}{2} \right)\frac{1}{x}+\]





e igualando a cero los coeficientes de cada exponente obtenemos las ecuaciones
 
% \[
%     \begin{split}
%       -\frac{1}{2x^}
%     \end{split}
% \]

\end{ejemplo}



\end{document}
