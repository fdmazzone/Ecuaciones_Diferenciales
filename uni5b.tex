\documentclass{article}
%\documentclass[hyperref={colorlinks=true}]{beamer}
%\documentclass[handout,hyperref={colorlinks=true}]{beamer}


%%%%%%%%%%%%%%%%%%%%%%%%%%%%%%Paquetes%%%%%%%%%%%%%%%%%%%%%%%%%%%%%%%%%%%%%%%%%%%%%%%
%%%%%%%%%%%%%%%%%%%%%%%%%%%%%%%%%%%%%%%%%%%%%%%%%%%%%%%%%%%%%%%%%%%%%%%%%%%%%%%%%%%%%

%\usepackage{pgfpages}
%\pgfpagesuselayout{4 on 1}[a4paper,landscape,border shrink=5mm]
\usepackage{empheq}
\usepackage[spanish]{babel}
\usepackage[utf8x]{inputenc}
\usepackage{times}
\usepackage[T1]{fontenc}
\usepackage{amssymb,amsmath}
\usepackage{enumerate}
\usepackage{verbatim}
\usepackage{ esint }
\usepackage{pdfsync}
%\usepackage{pst-all}
%\usepackage{pstricks-add}
\usepackage{array}
%\usepackage[T1]{fontenc}
\usepackage{animate}
%\usepackage{media9}
%\usepackage{movie15}
\usepackage{xparse}
%\usepackage{listings}
\usepackage{ wasysym }
\usepackage{sagetex}
\usepackage{hyperref}
\usepackage{tabularx}
\usepackage{xcolor}
\usepackage{adjustbox}



%%%%%%%%%%%%%%%%%%%%%%%%%%Nuevos comandos entornos%%%%%%%%%%%%%%%%%%%%%%%%%%%%%%%%
%%%%%%%%%%%%%%%%%%%%%%%%%%%%%%%%%%%%%%%%%%%%%%%%%%%%%%%%%%%%%%%%%%%%%%%%
%%%%%%%%%%%%%%%%%%%%%%%%Colores

\definecolor{myblue}{rgb}{.8, .8, 1}
\definecolor{dblackcolor}{rgb}{0.0,0.0,0.0}
\definecolor{dbluecolor}{rgb}{0.01,0.02,0.7}
\definecolor{dgreencolor}{rgb}{0.2,0.4,0.0}
\definecolor{dgraycolor}{rgb}{0.30,0.3,0.30}
\definecolor{color_teor}{HTML}{FFFDA2}
\definecolor{color_ejer}{HTML}{F6BFBF}
\definecolor{color_cor}{HTML}{49E6D8}
\newcommand{\dblue}{\color{dbluecolor}\bf}
\newcommand{\dred}{\color{dredcolor}\bf}
\newcommand{\dblack}{\color{dblackcolor}\bf}

%%%%%%%%%%%%%%%%%%%Teoremas, definiciones , etc

\newenvironment{demo}{\noindent\emph{Dem.}}{{\hspace*{\fill}$\square$} \newline\vspace{5pt}}
\newenvironment{colbox}[2]{%
    \begin{adjustbox}{minipage={\linewidth},margin=1ex,bgcolor=#1,env=center}
        #2}{%
    \end{adjustbox}%
}

\newcounter{defi_cont}\stepcounter{defi_cont}
\newenvironment{definicion}[1]{\begin{colbox}{myblue}{\textbf{Definición \arabic{defi_cont}.} #1}}{\stepcounter{defi_cont}\end{colbox}}

\newcounter{teor_cont}\stepcounter{teor_cont}
\newenvironment{teorema}[1]{\begin{colbox}{color_teor}{\textbf{Teorema \arabic{teor_cont}.} #1}}{\stepcounter{teor_cont}\end{colbox}}

\newenvironment{corolario}[1]{\begin{colbox}{color_cor}{\textbf{Corolario \arabic{teor_cont}.} #1}}{\stepcounter{teor_cont}\end{colbox}}



\newcounter{ejem_cont}\stepcounter{ejem_cont}
\newenvironment{ejemplo}[1]{\vspace{1ex}\noindent\textbf{Ejemplo \arabic{ejem_cont}.} #1}{\stepcounter{ejem_cont}}

\newcounter{ejer_cont}\stepcounter{ejer_cont}
\newenvironment{ejercicio}[1]{\begin{colbox}{color_ejer}{\textbf{Ejercicio \arabic{ejer_cont}.} #1}}{\stepcounter{ejer_cont}\end{colbox}}


%%%%%%%%%%%%%%%%%%%%%%%%%%%%%%%%%%%%Simbolos

\newcommand{\com}{\mathbb{R}}
\newcommand{\dis}{\mathbb{D}}
\newcommand{\rr}{\mathbb{R}}
\newcommand{\oo}{\mathcal{O}}
\renewcommand{\emph}[1]{\textcolor[rgb]{0,0,1}{#1}}
\newcommand{\der}[2]{\frac{\partial #1}{\partial #2}}
\renewcommand{\v}[1]{\overrightarrow{#1}}
\renewcommand{\epsilon}{\varepsilon}
\DeclareMathOperator{\mcd}{mcd}
\DeclareMathOperator{\atan2}{atan2}
\DeclareMathOperator{\sen}{sen}
%%%%%%%%%%%%%%%%%%%%%%%%%%%%%%Formulas en cajas 
\newlength\mytemplen
\newsavebox\mytempbox
\makeatletter
\newcommand\mybluebox{%
    \@ifnextchar[%]
       {\@mybluebox}%
       {\@mybluebox[0pt]}}

\def\@mybluebox[#1]{%
    \@ifnextchar[%]
       {\@@mybluebox[#1]}%
       {\@@mybluebox[#1][0pt]}}

\def\@@mybluebox[#1][#2]#3{
    \sbox\mytempbox{#3}%
    \mytemplen\ht\mytempbox
    \advance\mytemplen #1\relax
    \ht\mytempbox\mytemplen
    \mytemplen\dp\mytempbox
    \advance\mytemplen #2\relax
    \dp\mytempbox\mytemplen
    \colorbox{myblue}{\hspace{1em}\usebox{\mytempbox}\hspace{1em}}}

\makeatother
\DeclareDocumentCommand\boxedeq{ m g }{%
    {\begin{empheq}[box={\mybluebox[2pt][2pt]}]{equation}% #1%
        \IfNoValueF {#2} {\label{#2}}%
       #1
       \end{empheq}
    }%
}

%%%%%%%%%%%%%%%%%%%%%%%%%%%%%%%%%%%%%%%%%%%%%%%%%%%%%%%%%%%%%%%%%%%%%%%%%%%%%%%%%%%%%%%%%%%%%%%%%%%%%%%%%%%%%%%%%%%%%%%%%%%%%%%%%%%%%%%%%%%%%%%%%%%%%%%%%%%%%%%%%%%%%%%%%%%%%%%%%%%%%%%%%%%%%%%




\begin{document}



\section{Series de potencias}

\subsection{Definición} %Primer Frame para información personal.
\begin{definicion} Una serie de potencias es una serie de la forma:
\[
	\sum\limits_{n=0}^{\infty}a_n(z-z_0)^n
\]
donde $a_n$, $n=0,1,\ldots$, $z_0$ y $z$ son elementos de $\com$.
\end{definicion}


Estamos interesados en determinar los valores de $z$ para los cuales   una serie converge. 

\begin{ejemplo} La serie geométrica
\[
	\sum\limits_{n=0}^{\infty}z^n,
\]
es una serie de potencias. Aquí  $a_n=1$, $n=0,1,\ldots$ y $z_0=0$. Esta serie converge para $|z|<1$ a
\[ \frac{1}{1-z}\]
y no converge para cualquier otro valor de $z\in\com$.
\end{ejemplo}

\begin{ejemplo} Supongamos $f:I\to\rr$, donde $I$ es un intervalo abierto $I=(a,b)$ y que $f$ tiene derivadas de todo orden en  $z_0\in I$. Entonces es posible construir la serie de Taylor de $f$ en $z_0$ que es una serie de potencias. Recordemos que esta serie es
\[S(f,z_0,z)=\sum\limits_{n=0}^{\infty}\frac{f^{(n)}(z_0)}{n!}(z-z_0)^n.\]
\end{ejemplo}

 
 

\subsection{Límites superior e inferior}


\begin{definicion} Dada una sucesión de números reales $x_n$, 
 consideramos una nueva sucesión:

\[
	A_n=\sup\{x_n, x_{n+1},\ldots \}
\]

La nueva sucesión de reales $A_n$ es siempre nocreciente ($A_n\geq A_{n+1}$), luego tiene un límite (puede ser $\pm\infty$). A este límite lo llamamos \emph{el límite superior de $x_n$}. Lo denotamos por $\limsup$. Es decir:   


\[
\limsup x_n=\lim_{n\to\infty} A_n=\lim\limits_{n\to \infty} \sup\{x_n, x_{n+1},\ldots \}.
\]


Tomando ínfimo en lugar de supremo conseguimos \emph{el límite inferior} ($\liminf$).  
\end{definicion}
\begin{ejemplo}  Si $x_n=(-1)^n$, entonces
\[
	\{x_n,x_{n+1},\ldots\}=\{\pm1,\mp1,\pm1,\ldots\}.
\]
El supremode este conjunto  es para todo $n$ igual a 1 y el ínfimo igual a -1. Luego $\liminf x_n=-1$ y $\limsup=1$. 
\end{ejemplo}



\begin{ejemplo}  Si $x_n=1/n$, si $n$ es par y $x_n=1$ si $n$ es impar, entonces el conjunto

\[
	\{x_n,x_{n+1},\ldots\}
\]
tiene por supremo 1 y el ínfimo igual a $0$ . Luego $\liminf x_n=0$ y $\limsup=1$. 
\end{ejemplo}

\begin{teorema}\textbf{Propiedades} Sea $x_n$  e $y_n$ dos sucesiones de números reales, entonces:

\begin{enumerate}
	\item El $\limsup$ y el $\liminf$  existen siempre si se permite que $\pm\infty$ sean sus posibles valores.
	\item $\liminf x_n \leq \limsup x_n$.
	\item $\liminf x_n=\limsup x_n$ si y solo si el $\lim x_n$ existe. En este caso todos los límites coinciden.
	\item $\liminf (x_n+y_n)\geq \liminf x_n + \liminf y_n$.
	\item $\limsup (x_n+y_n)\leq \limsup x_n + \limsup y_n$

\end{enumerate}
\end{teorema}



\subsection{Radio de convergencia}


\begin{definicion} Dada la serie de potencias 
\[
	\sum\limits_{n=0}^{\infty}a_n(z-z_0)^n,
\]
definimos el \emph{radio de convergencia} $R$ de la siguiente forma:



\[
\frac{1}{R}=\limsup\limits_{n\to \infty} |a_n|^{1/n}.
\]

\end{definicion}

\begin{ejemplo} La serie  
\[
	\sum\limits_{n=0}^{\infty}z^n,
\]
tiene radio de convergencia:

\[\frac{1}{R}=\limsup\limits_{n\to \infty}1^{1/n}=\lim\limits_{n\to\infty}1^{1/n}=1\]

Luego $R=1$. 
\end{ejemplo}

\begin{ejemplo} La serie  
\[
	\sum\limits_{n=0}^{\infty}\bigg(\frac{1}{M}\bigg)^nz^n,
\]
tiene radio de convergencia:

\[\frac{1}{R}=\limsup\limits_{n\to \infty}\bigg(\bigg(\frac{1}{M}\bigg)^n\bigg)^{1/n}=\lim\limits_{n\to\infty}\bigg(\bigg(\frac{1}{M}\bigg)^n\bigg)^{1/n}=\frac1M\]

Luego $R=M$. 

\end{ejemplo}

\begin{ejemplo} Fijemos $M>0$ y $n$ un natural tal que $[n/2] >M$ (aquí $[x]$ es la parte entera de $x$). Entonces,
como $n-[n/2]\geq [n/2]>M$

\[ 
\begin{split}
n!=n(n-1)\cdots 1 &> n(n-1)\cdots (n-[n/2]) \\
&>\underbrace{M\cdots M }_{[n/2]-\hbox{veces}}\\
&\geq M^{[n/2]}\\
&> M^{n/3}\\
\end{split}
\]
Luego 
\[\frac{1}{R}:=\limsup_{n\to\infty} (1/n!)^{1/n}\leq \lim_{n\to\infty} \bigg(\frac{1}{M^{n/3}}\bigg)^{1/n}=\frac{1}{\sqrt[3]{M}}\]
Como $M$ es arbitrario, haciendo $M\to\infty$ vemos que el radio de convergencia de la serie  $\sum\limits_{n=0}^{\infty}\frac{1}{n!}z^n$ 
es $R=\infty$. 
\end{ejemplo}

\begin{teorema}  Consideremos la serie:

\[
	\sum\limits_{n=0}^{\infty}a_n(z-z_0)^n,
\]

Entonces:

\begin{enumerate}
	\item Si $|z-z_0|<R$, la serie converge absolutamente en $z$. 
	\item Si $|z-z_0|>R$, la serie diverge.
	\item Si $|z-z_0|=R$, no se afirma nada.
	
\end{enumerate}
\end{teorema}
	
 \begin{demo} Se puede suponer sin perdida de generalidad $z_0=0$. Supongamos $0<R<\infty$. Sea $L=1/R$ y tomemos $\epsilon>0$ pequeño. Como
\[\lim_{n\to\infty}\sup\{|a_n|^{1/n},|a_{n+1}|^{1/n+1},\ldots\}=L\]
para $n_0$ suficientemente grande
\[\sup\{|a_n|^{1/n},|a_{n+1}|^{1/n+1},\ldots\}<L+\epsilon.\]
 Así
\[|a_n|^{1/n}<L+\epsilon\quad\hbox{para } n\geq n_0.\]
Elijamos $0<r<1/(L+\epsilon)<1/L=R$. Si $|z|<r$ entonces
\[|a_n||z|^n<(L+\epsilon)^nr^n\quad\hbox{para } n\geq n_0.\]
Pero $r(L+\epsilon)<1$. La desigualdad de arriba y el teorema de comparación (notar que el miembro de la derecha forma una serie geométrica) implican que la serie converge absolutamente para este $z$.  Como $\epsilon$ es arbitrario, dado cualquier $z$, con $|z|<1/L$, tenemos un $\epsilon$ lo suficientemente chico para que $|z|<1/(L+\epsilon)$. \end{demo}

 \begin{ejercicio} Demostrar los casos $R=0$, $R=\infty$ y el segundo inciso.
 \end{ejercicio}
  
\begin{teorema}  La función
\[
f(z)=	\sum\limits_{n=0}^{\infty}a_n(z-z_0)^n,
\]
es diferenciable dentro en $\{z:|z-z_0|<R\}$. Además
\[
f'(z)=g(z):=	\sum\limits_{n=1}^{\infty}na_n(z-z_0)^{n-1},
\]
teniendo esta serie el mismo radio de convergencia que el de $f$.
\end{teorema}

\begin{demo} Nuevamente supondremos $z_0=0$. La afirmación sobre el radio de convergencia es consecuencia de que $\lim_{n\to\infty}n^{1/n}=1$. Como el radio $R'$ de convergencia de $g$ es:

\[
\begin{split}
\frac{1}{R'}&=\limsup_{n\to\infty}|a_{n+1}(n+1)|^{1/(n+1)}\\
&\underbrace{=}_{\hbox{\emph{ Ejercicio}}}\limsup_{n\to\infty}|a_{n+1}|^{1/(n+1)}\lim_{n\to\infty}|(n+1)|^{1/(n+1)}\\
&=\limsup_{n\to\infty}|a_{n+1}|^{1/(n+1)}=\frac{1}{R}
\end{split}\]

Ahora veamos que $f$ es holomorfa y $f'=g$.   Sea $0<r<R$, $|z_0|<r$ y $N\in\mathbb{N}$. Pongamos:
\[f(z)=S_N(z)+E_N(z),\]
\[S_N(z)=\sum_{n=0}^{N}a_nz^n\quad\hbox{y}\quad E_N(z)=\sum_{n=N+1}^{\infty}a_nz^n\]
Tomemos $|h|<r-|z_0|$, así $|z_0+h|<r$.  Tenemos
\[\begin{split}
\frac{f(z_0+h)-f(z_0)}{h}-g(z_0)&= \frac{S_N(z_0+h)-S_N(z_0)}{h}-S_N'(z_0)\\
				&+S_N'(z_0)-g(z_0)\\
				&+\frac{E_N(z_0+h)-E_N(z_0)}{h}
\end{split}\]

Ahora si $\epsilon>0$

\[\begin{split}
				&\bigg|\frac{E_N(z_0+h)-E_N(z_0)}{h}\bigg|\leq\sum_{n=N+1}^{\infty}|a_n|\bigg|\frac{(z_0+h)^n-z_0^n}{h}\bigg|\\
				&=\sum_{n=N+1}^{\infty}|a_n|(|z_0|^{n-1}+|z_0|^{n-2}h+\cdots+h^{n-1})\\
				&\leq2\sum_{n=N+1}^{\infty}|a_n|nr^{n-1}<\epsilon\\
\end{split}\]
Para $N$ suficientemente grande. Además como $S_N'(z)\to g(z)$ cuando $N\to\infty$ podemos elegir, a su vez, $N$ suficientemente grande para que

\[
	|S_N'(z_0)-g(z_0)|<\epsilon
\]

 Fijemos un $N$ que satisfaga las condiciones anteriores. Ahora podemos encontrar $\delta>0$ para que $|h|<\delta$ cumpla que 
\[
\bigg|\frac{S_N(z_0+h)-S_N(z_0)}{h}-S_N'(z_0)\bigg|<\epsilon.
\]

Esto muestra que $f'(z_0)=g(z_0)$ y por consiguiente $f$ es derivavble.  
\end{demo}


 


\begin{corolario}  Una serie de potencias es infinitamente diferenciable. Las sucesivas derivadas se obtienen derivando término a término la serie. El radio de convergencia se conserva.
\end{corolario}

\begin{ejemplo}  Hemos visto que la serie:
\[
	f(z)=\sum_{n=0}^{\infty}\frac{1}{n!}z^n
\]
tiene radio de convergencia infinito y por ende converge en $\com$. Ahora vemos que
\[f'(z)=\sum_{n=1}^{\infty}\frac{1}{(n-1)!}z^{n-1} =\sum_{n=0}^{\infty}\frac{1}{n!}z^n
\]

Luego $f$ resuelve la simple ecuación diferencial $f'(z)=f(z)$. La misma ecuación es resuelta por $g(z)=e^z$. Además $f(0)=g(0)=1$. Por el Teorema de existencia y unicidad $f(z)=g(z)$ para todo $z$. Hemos probado la importante fórmula.

\boxedeq{e^z=\sum_{n=0}^{\infty}\frac{1}{n!}z^n}{eq:exp}
\end{ejemplo}


 



\subsection{Funciones analíticas}

\begin{definicion} Una función $f:\Omega\subset\com\to\com$ se dirá analítica si para cada $z_0\in\Omega$, existe un $R$ y $a_n\in\com$, tal que vale la igualdad:

\[ f(z)=\sum_{n=0}^{\infty}a_n(z-z_0)^n,\quad\hbox{para } |z-z_0|<R \]
\end{definicion}


\begin{ejercicio} Si $f$ es analítica tenemos la siguiente fórmula
\[
	a_n=\frac{f^{(n)}(z_0)}{n!}
\]
para los coeficientes $a_n$. 
\end{ejercicio}



\section{Solución de EDO mediante series de potencias. Puntos Ordinarios}



\subsection{Método coeficientes indeterminados}

Dada una EDO

\begin{equation} \hbox{(1)}\quad  F(x,y,y',\ldots,y^{(n)})=0\end{equation}
queremos encontrar el desarrollo en series de potencias de la solución general a esta ecuación. El método que estudiaremos se denomina \emph{metodo de los coeficientes indeterminados}. Consiste en proponer el desarrollo en serie de la solución

\[y(x)=a_0+a_1(x-x_0)+a_2(x-x_0)^2+\cdots  \]
remplazar $y(x)$ por este desarrollo en  en la ecuación (1) y tratar de resolver la ecuación resultante para los coeficientes (indeterminados) $a_n$. El método suele funcionar en algunas ecuaciones. Desarrollemos un ejemplo.

 

\begin{ejemplo} Hallar el desarrollo en serie de la solución del siguiente pvi 
\[\left\{\begin{array}{l l} y'&=y\\ y(0)&=1\end{array}\right.\]
La solución es bien sabido que es $y(x)=e^x$,  pero pretendemos reencontrarla por el método expuesto. Escribimos
\[\begin{split}
   y&=a_0+a_1x+a_2x^2+\cdots+a_nx^n+\cdots\\
   y'&=a_1+2a_2x+3a_3x²+\cdots+(n+1)a_{n+1}x^n+\cdots
  \end{split}
\]
La igualdad $y'=y$ implica que 
\[\begin{split}
   a_1&=a_0\\
   a_2&=\frac{a_1}{2}\\
   a_3&=\frac{a_2}{3}\\
      &\,\,\,\,\vdots \\
   a_{n+1}&=\frac{a_{n}}{n+1}
 \end{split}
\]
Si iteramos la fórmula $a_{n+1}=a_{n}/(n+1)$, obtenemos 
\[a_n=\frac{1}{n}a_{n-1}=\frac{1}{n(n-1)}a_{n-2}=\cdots=\frac{1}{n(n-1)\cdots 1}a_{0}=\frac{a_0}{n!}.\]
Pero $a_0=y(0)=1$. Luego 
\boxedeq{a_n=\frac{1}{n!}}{eq:exp}


\begin{sageblock}
y=1 
\end{sageblock}

Vemos que la solución es 
\boxedeq{y(x)=\sage{orden}}{eq:sol_sage} 

%  También podemos usar SAGE para obtener el resultado.
% \begin{sageblock}
% """
% Método de coeficientes indeterminados para
% la ecuación y'=y
% """
% #La serie la sumamos hasta X^orden
% orden=7
% # Lista coeficientes
% Lista=['a%s'%i for i in range(orden)]
% #expresiones polinómicos con los an
% A = PolynomialRing(QQ,Lista)
% #Series con coeficientes en A
% Series.<X> = PowerSeriesRing(A)
% #n-upla de los a_n
% Coef=A.gens() 
% #serie truncada en orden
% y=sum(Coef[i]*X^i for i in range(orden)) 
% #sustituyo en la ecuación
% Ecua=y.derivative()-y
% #lista de los a_n convertidos en expresión 
% a= [SR(i) for i in A.gens()] 
% """ A las ecuaciones le agrego a0=1 y evito usar la
% última pues tiene un problema que viene del truncamiento
% """
% Ecuaciones=[a[0]-1,list(Ecua)[:-1]] 
% #resuelvo las relaciones de recurrencia
% Sol_a_n=solve(Ecuaciones,a) 
% """Lista de los coeficientes convertidos en racionales """
% L=[QQ(f.rhs()) for f in Sol_a_n[0]] 
% # Desarrollo en serie de la solución
% y=sum([L[i]*X^i for i in range(orden)])+O(X^orden)
% \end{sageblock}

% Vemos que la solución es 
% \boxedeq{y(x)=\sage{orden}}{eq:sol_sage} 
\end{ejemplo}


Una vez que hemos escrito este código, conviene ``empaquetarlo'' en una función de SAGE de modo que podamos invocarlo con facilidad en otras ecuaciones.

\begin{sageblock}

\end{sageblock}


\subsection{Relaciones de recurrencia}

La expresión $a_{n+1}=\frac{a_{n}}{n+1}$ es un ejemplo de \href{http://es.wikipedia.org/wiki/Relación_de_recurrencia}{relación de recurrencia}.
\begin{definicion} Una  relación de recurrencia para una sucesión $b_n$ de números reales es una sucesión de 
funciones $f_n:\rr^n\to\rr$ que relaciona $b_{n+1}$ con los términos anteriores de la sucesión por medio de 
la expresión
\boxedeq{b_{n+1}=f_n(b_1,\ldots,b_n)}{eq:recu}
 Resolver una relación de recurrencia es encontrar una fórmula explícita de $b_n$ como función de $n$.
 \end{definicion}


Hay técnicas y métodos para resolver relaciones de recurrencia, que son parecidas a las técnicas y métodos de resolver ecuaciones diferenciales.  No vamos a desarrollar este tema en este curso, sólo agregamos que  SAGE resuelve relaciones recurrentes. Para ello se necesita de un módulo extra, también basado en Python, llamado \href{http://sympy.org/en/index.html}{Sympy}. Este módulo tiene el comando \texttt{rsolve} para resolver relaciones de recurrencia. Sympy tiene sus propios comendos que son parecidos pero no iguales a SAGE.

\begin{ejemplo}Resolvamos la sucesión de Fibonacci $a_{n+2}=a_{n+1}+a_n$.
% \begin{sageblock}
% """ Resolvemos la relación de recurrencia 
% de los números de Fibonacci"""
% #Importamos Sympy 
% from sympy import *
% #En Sympy se declaran símbolos en lugar de variables
% n=Symbol('n',integer=True)
% #Y así se declaran funciones
% y = Function('y')
% #Así una ecuación
% f=Equality(y(n),y(n-1)+y(n-2))
% #Así la resolvemos
% rsolve(f,y(n))
% \end{sageblock}  
\end{ejemplo}
% El resultado es
%  \boxedeq{a_n=\sage{rsolve(f,y(n))}}{}
% Las constantes arbitrarias $C_0$ y $C_1$ aparecen porque una relación de recurrencia no tiene una única solución. Sedice que una relación de recurrencia tiene orden $k$ o es de $k$-términos si el coeficiente $a_n$ se expresa en función de $k$ de los anteriores. En general la solución general de una relación de recurrencia de $k$-términos tiene $k$ constantes arbitrarias. Por consiguiente, si queremos una única solución debemos tener $k$ relaciones extras. Usualmente se  tiene el valor de los $k$-primeros términos $a_0,\ldots,a_k$. Por ejemplo, en el ejemplo de la sucesión de Fibonacci si pedimos $a_0=a_1=1$.

% \begin{sageblock}
% reset()
% n=var('n')
% C0,C1=var('C0,C1')
% A=C0*(1/2+sqrt(5)/2)^n+C1*(1/2-sqrt(5)/2)^n
% Cval=solve([A(n=0)-1,A(n=1)-1],[C0,C1],solution_dict=True)
% Fib=A.subs(Cval[0])
% [Fib(n=i).expand() for i in range(10)]
% \end{sageblock}  
% Llegamos a la fórmula
% \boxedeq{a_n=\sage{Fib}.}{}
% Los primeros 20 números de Fibonacci son 
% \[\sage{[Fib(n=i).expand() for i in range(20)]}.\]
% \end{ejemplo}

% \subsection{Serie binomial}

%  Se puede utilizar el método de coeficientes indeterminados para encontrar desarrollos en serie de alguna función dada $f$. La técnica consiste en encontrar un pvi que satisfaga $f$ y le aplicamos el método a ese pvi.

% \begin{ejemplo}  Encontrar el desarrollo en serie de la función
% \[y(x)=(1+x)^p\quad p\in\rr\]
% La función $y(x)$ resuelve el pvi  $(1+x)y'(x)=py$, $y(0)=1$. apliquemos el método de coeficientes indeterminados a este pvi.
% Como
% \[\begin{split}
%    y&=a_0+a_1x+a_2x^2+\cdots+a_nx^n+\cdots\\
%    y'&=a_1+2a_2x+3a_3x²+\cdots+(n+1)a_{n+1}x^n+\cdots
%   \end{split}
% \]
% Tenemos
% \[\begin{split}
%    py=&pa_0+pa_1x+pa_2x^2+\cdots+pa_nx^n+\cdots\\
%   (1+x)y'=&a_1+2a_2x+3a_3x²+\cdots+(n+1)a_{n+1}x^n+\cdots\\
%           &+a_1x+2a_2x^2+3a_3x³+\cdots+na_{n}x^n+\cdots\\
% -------&----------------------\\
% 0=(1+x)y'-py =& (a_1-pa_0)+(a_1+2a_2-pa_1)x+\cdots +((n+1)a_{n+1}+na_n-pa_n)x^n+\cdots
%   \end{split}
% \]
% Tenemos la relación
% \[ a_{n+1}=\frac{(p-n)}{n+1}a_n
% \]
% Que es una relación de reciurrencia de un sólo término. Estas relaciones se resuelven iterando la relación de manera sucesiva de modo de relacionar $a_n$ con $a_0$
% \[a_n=\frac{(p-n+1)}{n}a_n=\frac{(p-n+1)(p-n+2)}{n(n-1)}a_{n-1}=\cdots=\frac{(p-n+1)(p-n+2)\cdots p}{n!}a_0.\]
% Como $a_0=y(0)=1$ vemos que
% \boxedeq{a_n=\frac{(p-n+1)(p-n+2)\cdots p}{n!}.}{}
% Si $p\in\mathbb{N}$ entonces debe ocurrir que $a_n=0$ para $n>p$. Esto es claro por otro lado, ya que en este caso $(1+x)^p$ es un polinomio y por la fórmula del binomio de Newton los coeficientes no son más que los coeficientes binomiales
% \[a_n=\binom{p}{n}\]
%  Cuando $p\in\mathbb{R}$ aún vamos a seguir denominado a $a_n$ el coeficiente binomial. La serie resultante se llama la serie binomial, cuando $p\in\mathbb{R}-\mathbb{N}$ es una serie infinita y no  un polinomio. Notar que para $p$ no entero positivo
% \[\lim\limits_{n\to\infty}\frac{|a_{n+1}|}{|a_n|}=\lim\limits_{n\to\infty}\frac{|p-n|}{|n+1|}=1\]
% Luego la serie tiene radio de convergencia 1.  Hemos demostrado asi que vale la siguiente fórmula, que es una generalización de la forma binomial de Newton
% \boxedeq{(1+x)^p=1+px+\frac{p(p-1)}{2!}x^2  +\cdots=1+\binom{p}{1}x+\binom{p}{2}x^2+\cdots}{}
% Esta importante serie se denomina \href{http://en.wikipedia.org/wiki/Binomial_series}{serie binomial}.


% % \begin{sageblock}
% % """
% % Método de coeficientes indeterminados para
% % la ecuación (1+x)y'=py , y(0)=1
% % """
% % orden=7
% % Lista=['a%s'%i for i in range(orden)]+['p']
% % A = PolynomialRing(QQ,Lista)
% % Series.<X> = PowerSeriesRing(A)
% % Coef=A.gens() 
% % y=sum(Coef[i]*X^i for i in range(orden)) 
% % Ecua=(1+X)*y.derivative()-p*y
% % a= [SR(i) for i in A.gens()[:-1]] 
% % Ecuaciones=[a[0]-1,list(Ecua)[:-1]] 
% % Sol_a_n=solve(Ecuaciones,a) 
% % L=[QQ(f.rhs()) for f in Sol_a_n[0]] 
% % y=sum([L[i]*X^i for i in range(orden)])+O(X^orden)
% % \end{sageblock}

% $\sage{y}$

% \end{ejemplo}




\end{document}
