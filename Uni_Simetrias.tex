%\documentclass{article}
%\documentclass[hyperref={colorlinks=true}]{beamer}
\documentclass[handout,hyperref={colorlinks=true}]{beamer}

\usecolortheme{crane}
\usetheme{Darmstadt}
\usefonttheme{structuresmallcapsserif}
%%%%%%%%%%%%%%%%%%%%%%%%%%%%%%Paquetes%%%%%%%%%%%%%%%%%%%%%%%%%%%%%%%%%%%%%%%%%%%%%%%5
%%%%%%%%%%%%%%%%%%%%%%%%%%%%%%%%%%%%%%%%%%%%%%%%%%%%%%%%%%%%%%%%%%%%%%%%%%%%%%%%%%%%%
%\usepackage{pgfpages}
%\pgfpagesuselayout{2 on 1}[a4paper,border shrink=5mm]
\usepackage{empheq}
\usepackage[spanish]{babel}
\usepackage[utf8x]{inputenc}
\usepackage{times}
%\usepackage[T1]{fontenc}
\usepackage{amssymb,amsmath}
\usepackage{enumerate}
\usepackage{verbatim}
\usepackage{ esint }
%\usepackage{pst-all}
%\usepackage{pstricks-add}
\usepackage{array}
%\usepackage[T1]{fontenc}
\usepackage{animate}
%\usepackage{media9}
\usepackage{xparse}
\usepackage{listings}
\usepackage{ wasysym }
\usepackage{sagetex}
\usepackage{yfonts,mathrsfs,eufrak}
\usepackage{hyperref}

%%%%%%%%%%%%%%%%%%%%%%%%%Configuracion listing
\lstdefinelanguage{Sage}[]{Python}
{morekeywords={False,sage,True},sensitive=true}
\lstset{
frame=none,
showtabs=False,
showspaces=False,
showstringspaces=False,
commentstyle={\ttfamily\color{dgreencolor}},
keywordstyle={\ttfamily\color{dbluecolor}\bfseries},
stringstyle={\ttfamily\color{dgraycolor}\bfseries},
language=Sage,
basicstyle={\fontsize{8pt}{8pt}\ttfamily},
aboveskip=.3em,
belowskip=0.1em,
numbers=none,
numberstyle=\footnotesize
}
%%%%%%%%%%%%%%%%%%%%%%%%Colores
\definecolor{myblue}{rgb}{.8, .8, 1}
\definecolor{dblackcolor}{rgb}{0.0,0.0,0.0}
\definecolor{dbluecolor}{rgb}{0.01,0.02,0.7}
\definecolor{dgreencolor}{rgb}{0.2,0.4,0.0}
\definecolor{dgraycolor}{rgb}{0.30,0.3,0.30}
\newcommand{\dblue}{\color{dbluecolor}\bf}
\newcommand{\dred}{\color{dredcolor}\bf}
\newcommand{\dblack}{\color{dblackcolor}\bf}
%%%%%%%%%%%%%%%%%%%%%%%%%%Nuevos comandos entornos%%%%%%%%%%%%%%%%%%%%%%%%%%%%%%%%
%%%%%%%%%%%%%%%%%%%%%%%%%%%%%%%%%%%%%%%%%%%%%%%%%%%%%%%%%%%%%%%%%%%%%%%%
\newenvironment{demo}{\noindent\emph{Dem.}}{$\square$ \newline\vspace{5pt}}
\newcommand{\com}{\mathbb{C}}
\newcommand{\dis}{\mathbb{D}}
\newcommand{\rr}{\mathbb{R}}
\newcommand{\oo}{\mathcal{O}}
\renewcommand{\emph}[1]{\textcolor[rgb]{1,0,0}{#1}}
\newcommand{\der}[2]{\frac{\partial #1}{\partial #2}}
\renewcommand{\v}[1]{\overrightarrow{#1}}
\renewcommand{\epsilon}{\varepsilon}
\newlength\mytemplen
\newsavebox\mytempbox
\makeatletter
\newcommand\mybluebox{%
\@ifnextchar[%]
{\@mybluebox}%
{\@mybluebox[0pt]}}
\def\@mybluebox[#1]{%
\@ifnextchar[%]
{\@@mybluebox[#1]}%
{\@@mybluebox[#1][0pt]}}
\def\@@mybluebox[#1][#2]#3{
\sbox\mytempbox{#3}%
\mytemplen\ht\mytempbox
\advance\mytemplen #1\relax
\ht\mytempbox\mytemplen
\mytemplen\dp\mytempbox
\advance\mytemplen #2\relax
\dp\mytempbox\mytemplen
\colorbox{myblue}{\hspace{1em}\usebox{\mytempbox}\hspace{1em}}}
\makeatother
\DeclareDocumentCommand\boxedeq{ m g }{%
{\begin{empheq}[box={\mybluebox[2pt][2pt]}]{equation}% #1%
\IfNoValueF {#2} {\label{#2}}%
#1
\end{empheq}
}%
}
\DeclareMathOperator{\atan2}{atan2}
\DeclareMathOperator{\sen}{sen}
\newtheorem{teorema}{Teorema}[section]
\newtheorem{lema}[teorema]{Lema}
\newtheorem{corolario}[teorema]{Corolario}
\newtheorem{proposicion}[teorema]{Proposici\'on}
\newtheorem{definicion}[teorema]{Definici\'on}
%%%%%%%%%%%%%%%%%%%%%%%%%%%%%%%%%%%%%%%%%%%%%%%%%%%%%%%%%%%%%%%%%%%%%%%%%%%%%%%%%%%%%%%%%%%%%%%%%%%%%%%%%%%
%%%%%%%%%%Para escibir en clase articulo o similar
% \usepackage{color}
% \newcommand{\nl}{ }
% \renewenvironment{frame}[1]{}{}
% \newcommand{\qed}{$\square$}
% %\newcommand{\defverbatim}{\def{#1}}
% \newenvironment{block}[1]{\textbf{#1}}{}
% \title{Ecuaciones lineales de segundo orden}
% \author{Fernando Mazzone}
%
%%%%%%%%%%%%%%%%%%%%%%%Para clase beamer
\newcommand{\nl}{\onslide<+-> }

\title[Teoría de Lie y ODE] % (optional, nur bei langen Titeln nötig)
{%
Cambios de variables, Teoría de Lie y EDO
}



\author[] % (optional, nur bei vielen Autoren)
{Fernando Mazzone}

\institute[Depto de Matemática] % (optional, aber oft nötig)
{
 Depto de Matemática\\
Facultad de Ciencias Exactas Físico-Químicas y Naturales\\
Universidad Nacional de Río Cuarto}


\subject{Ecuaciones Diferenciales}

%%%%%%%%%%%%%%%%%%%%%%%%%%%%%%%%%%%%%%%%%%%%%%%%%%%%%%%%%%%%%%%%%%%%%%%%%%%%%%%%%%%%%%


\begin{document}
\begin{frame}
  \maketitle
  \begin{center}
   \includegraphics[scale=0.2]{imagenes/unrc.jpg}
   \end{center}
\end{frame}
\begin{frame}
\tableofcontents

\end{frame}

\section[Historia]{Introducción histórica}

\begin{frame}{Sophus Lie}
\begin{tabular}{m{0.8\linewidth} >{\centering\arraybackslash}m{0.2\linewidth} }
 
Marius Sophus Lie fue un matemático noruego (17 de diciembre de 1842-18 de febrero de 1899) que creó en gran parte la teoría de la simetría continua, y la aplicó al estudio de la geometría y las ecuaciones diferenciales.
La herramienta principal de Lie, y uno de sus logros más grandes fue el descubrimiento de que los grupos continuos de transformación (ahora llamados grupos de Lie), podían ser

 &

\includegraphics[scale=.25]{imagenes/Sophus_Lie.jpg}
\\
\end{tabular}
 entendidos mejor "linealizándolos", y estudiando los correspondientes campos vectoriales generadores (los, así llamados, generadores infinitesimales). 
Los generadores obedecen una versión linealizada de la ley del grupo llamada el corchete o conmutador, y tienen la estructura de lo que hoy, en honor suyo, llamamos un álgebra de Lie. (Wikipedia)
\end{frame}





\section[Formas]{Formas Diferenciales, idea vaga}

\begin{frame}{Distintas formas para una ecuación}

\nl\begin{block}{Ecuación diferencial general de primer orden} 
\boxedeq{\frac{dy}{dx}=f(x,y)}{eq:gral_orden1}
\end{block}

\nl\begin{block}{Forma diferencial}
\boxedeq{M(x,y)dx+N(x,y)dy=0}{eq:gral_orden1_FormDif}
\end{block}
\nl La primera expresión es más asimétrica, entre las variables $x$ e $y$ una de ellas es independiente ($x$) y la otra independiente  ($y$) . 

\nl La segunda expresión es  más simétrica, las dos variables tienen el mismo estatus.

\nl Las expresiones del tipo \eqref{eq:gral_orden1_FormDif} representan un ente matemático importante llamado \href{http://es.wikipedia.org/wiki/Forma_diferencial}{forma diferencial}


\end{frame}



\begin{frame}{Formas diferenciales, idea somera}


\begin{enumerate}
\item<+-> Como los polinomios, las formas diferenciales tienen grado. 
\item<+-> Dadas dos (pueden ser mas) variables $x,y$ una 0-forma diferencial es una función $g(x,y)$ de $x,y$. 
\item<+> La expresión \eqref{eq:gral_orden1_FormDif} es una 1-forma diferencial.
\item<+-> Hay un operador llamado diferencial y denotado por $d$. Si $\omega$ es una $k$-forma diferencial $d\omega$ es una $k+1$ forma diferencial.
\item<+-> En el caso de $0$-forma (función) $g(x,y)$ el diferencial se define
\[dg=\frac{\partial M}{\partial x}dx+ \frac{\partial N}{\partial y}dy.\]
Una $k$-forma diferencial se llama exacta cuando es el diferencial de una $k-1$-forma.
\end{enumerate}


\end{frame}


\section{Cambios de Variables}



\begin{frame}{Cambios de variables}

\begin{block}{Idea básica}
 Suṕongamos la ecuación \eqref{eq:gral_orden1} o \eqref{eq:gral_orden1_FormDif} en las variables $x,y$. La idea es encontrar nuevas variables $\hat{x}=\hat{x}(x,y)$ y $\hat{y}=\hat{y}(x,y)$ tales que la ecuación se transforme en una más sencilla de resolver. 
\end{block}




\end{frame}

\begin{frame}{Cómputos de cambiamos variables}
\textbf{Cambio de la variable dependiente $y=h(x,\hat{y})$ manteniendo la independiente} 
\nl
\boxedeq{\frac{dy}{dx}=\frac{\partial h}{\partial x}+\frac{\partial h}{\partial \hat{y}}\frac{d\hat{y}}{dx}.}{}

La ecuación se convierte

\[\frac{\partial h}{\partial x}+\frac{\partial h}{\partial \hat{y}}\frac{d\hat{y}}{dx}=f(x,h(x,\hat{y})).\]

Que es una expresión sólo en $\hat{y}$ y $x$. Parece más complicada, pero en un ejemplo concreto puede ser más simple.



\end{frame}

\begin{frame}{Cómputos de cambios de variables, ejemplo}

\textbf{Ejemplo 1}\nl Hacer el cambio de variable en la  ecuación 
\boxedeq{y=\frac{e^{\hat{y}}}{x}\quad\text{en}\quad  y'=\left[\ln(xy)\right]^2xy-\frac{y}{x}.}{}
\nl 1) Expresemos $dy/dx$ sólo con $x$, $\hat{y}$ y $d\hat{y}/dx$.
\[\frac{dy}{dx}=-\frac{e^{\hat{y}}}{x^2}+\frac{e^{\hat{y}}}{x}\frac{d\hat{y}}{dx}.\]
\nl 2) Remplacemos $y'$ e $y$ en la ecuación 
\[-\frac{e^{\hat{y}}}{x^2}+\frac{e^{\hat{y}}}{x}\frac{d\hat{y}}{dx}=\left[\ln\left(x \frac{e^{\hat{y}}}{x} \right)\right]^2x\frac{e^{\hat{y}}}{x}-\frac{\frac{e^{\hat{y}}}{x} }{x}.\]
\nl 3) Simplifiquemos
\boxedeq{\frac{d\hat{y}}{dx}=\hat{y}^2x.}{}




\end{frame}










\begin{frame}[fragile]{Cómputos de cambios de variables, ejemplo}
Lo podemos hacer con \texttt{SymPy}

\begin{sageblock}
from sympy import *
x,y,y_n=symbols('x,y,y_n')
y=Function('y')(x)
y_n=Function('y_n')(x)
y=exp(y_n)/x
eq=Eq(y.diff(x)-(ln(x*y))**2*x*y+y/x,0)
simplify(eq)
\end{sageblock}
Obtenemos la ecuación
\[\sage{simplify(eq)}\]
que \texttt{SymPy} no simplifica a nuestro gusto
\end{frame}



\begin{frame}{Cómputos de cambios de variables}
\textbf{Cambio de la variable independiente $\hat{x}=h(x)$ manteniendo la dependiente} 

\boxedeq{\frac{dy}{dx}=\frac{dy}{d\hat{x}}\frac{dh}{dx}=\frac{dy}{d\hat{x}}h'(x).}{}

Suponiendo $h$ biyectiva, la ecuación se convierte

\[\frac{dy}{d\hat{x}}h'(h^{-1}(\hat{x}))=f(h^{-1}(\hat{x}),y).\]

Que es una expresión sólo en $\hat{x}$ e $y$. 



\end{frame}

\begin{frame}{Cómputos de cambios de variables, ejemplo}

\textbf{Ejemplo 2} Hacer el cambio de variable en la  ecuación 
\nl \boxedeq{x=\cos \hat{x}\quad\text{en}\quad  -\frac{dy}{dx}+\frac{1}{\sqrt{1-x^2}}y=0.}{}
\nl 1) $h(x)=\arcsen x$
\[\frac{dy}{dx}=\frac{dy}{d\hat{x}}h'(x)=-\frac{1}{\sqrt{1-x^2}}\frac{dy}{d\hat{x}}.\]
\nl 2) Remplacemos $x$ e $y'$ en la ecuación 
\[\frac{1}{\sqrt{1-x^2}}\frac{dy}{d\hat{x}}+ \frac{1}{\sqrt{1-x^2}}y=0\]
\nl 3) Simplificando
\boxedeq{\frac{dy}{d\hat{x}}+ y=0.}{}



\end{frame}

\begin{frame}[fragile]{Cómputos de cambios de variables, ejemplo}
Lo podemos hacer con \texttt{SymPy}

\begin{sageblock}
from sympy import *
x,x_n=symbols('x,x_n')
x_n=acos(x)
y=Function('y')(x_n)
Ecuacion=-y.diff()+1/(sqrt(1-x**2))*y
\end{sageblock}
Obtenemos la ecuación
\[\sage{simplify(Ecuacion)}=0\]
Nuevamente \texttt{SymPy} no simplifica a nuestro gusto
\end{frame}

\begin{frame}{Cambios de variables}
\textbf{Cambio de variable general $\hat{x}=\hat{x}(x,y)$, $\hat{y}=\hat{y}(x,y)$}

\nl 1) Calculamos $d\hat{y}/d\hat{x}$ en las variables $x,y$
\boxedeq{
\frac{d\hat{y}}{d\hat{x}}=\frac{\frac{d\hat{y}}{dx}}{\frac{d\hat{x}}{dx}}=\frac{\frac{\partial\hat{y}}{\partial x}+\frac{\partial\hat{y}}{\partial y}y'}{\frac{\partial\hat{x}}{\partial x}+\frac{\partial\hat{x}}{\partial y}y'}=\frac{\frac{\partial\hat{y}}{\partial x}+\frac{\partial\hat{y}}{\partial y}f(x,y)}{\frac{\partial\hat{x}}{\partial x}+\frac{\partial\hat{x}}{\partial y}f(x,y)}.
}{eq:subsder}

\nl 2) En la expresión resultante sustituir $x,y$ por las tansformaciones inversas $x=x(\hat{x},\hat{y})$ y  $y=y(\hat{x},\hat{y})$


\end{frame}


\begin{frame}[fragile]{Cambios de variables}
\textbf{Ejemplo 3. Transformar a polares:} $\frac{dy}{dx}=\frac{y^3+x^2y-x-y}{x^3+xy^2-x+y}.$

Dado que el cálculo es extenso lo haremos sólo con \texttt{SymPy}
\begin{sageblock}
from sympy import *
x=symbols('x')
y=Function('y')(x)
r=sqrt(x**2+y**2)
theta=atan(y/x)
Expr2=r.diff(x)/theta.diff(x)
Expr3=Expr2.subs(y.diff(x),\
(y**3+x**2*y-x-y)/(x**3+x*y**2-x+y))
r,theta=symbols('r,theta',positive=True)
Expr4=Expr3.subs([(y,r*sin(theta)),\
(x,r*cos(theta))])
Expr5=simplify(Expr4)
\end{sageblock}
\end{frame}

\begin{frame}[fragile]{Cambios de variables}
\nl Encontramos que en polares la ecuación es mucho más simple
\boxedeq{\frac{dr}{d\theta}=\sage{powsimp(Expr5)}.}{}

\nl Quizás  usar la notación como  forma diferencial sea más efectivo. Como $r$ y $\theta$ son funciones de $x$ e $y$, ellas son 0-formas. Usando las reglas de la diferencial, hay que reemplazar
\boxedeq{
\begin{array}{ll}
x=r\cos\theta; &dx=\cos\theta dr-\sen\theta r d\theta\\
 y= r\sen\theta; & dy=\sen\theta dr+\cos\theta r d\theta\\
\end{array}
}{}
 en la 1-forma:
\boxedeq{ (y^3+x^2y-x-y)dx-(x^3+xy^2-x+y)dy}{}
\end{frame}



\begin{sagesilent}
reset()
\end{sagesilent}


\begin{frame}[fragile]{\texttt{SAGE} y formas diferenciales}
 Encontramos a  \texttt{SAGE} más cómodo para operar con formas diferenciales que \texttt{SymPy}
\begin{sagecommandline}
sage: r,theta=var('r,theta')
sage: U = CoordinatePatch((r,theta))
sage: F = DifferentialForms(U)
sage: x= DifferentialForm(F, 0, r*cos(theta))
sage: y= DifferentialForm(F, 0, r*sin(theta))
sage: w=(x^3+x*y^2-x+y)*y.diff()-(y^3+x^2*y-x-y)*x.diff()
sage: w[0].simplify_full()
sage: w[1].simplify_full()
\end{sagecommandline}
La forma obtenida es $\sage{w[0].simplify_full()}dr+(\sage{w[1].simplify_full()})d\theta$.
\end{frame}


\section[Grupos]{Grupos}

\begin{frame}{Grupos, Repaso}
\begin{block}{Grupos}
Sean $G$ un conjunto y $\alpha$ una función tal que   $\alpha:G\times G\to G$. En el contexto de grupos es más usual la notación  $\alpha(g_1,g_2)=g_1g_2$. El par $(G,\alpha)$ se llama un grupo si se satisface
\begin{enumerate}
\item $(g_1g_2)g_3=g_1(g_2g_3)$, para todos $g_1,g_2,g_3\in G$,
\item Existe $e\in G$ tal que $eg=ge=g$,  para todo $g\in G$.
\item Para todo $g\in G$ existe $h\in G$ tal que $gh=hg=e$. Se acostumbra denotar $h=g^{-1}$.
\end{enumerate}
\end{block}


\end{frame}



\begin{frame}{Ejemplos de grupos}
\nl\textbf{Ejemplo 1} Sea $\Pi$ un plano euclideano y $G$ el conjunto de todas las transformaciones rígidas de $\Pi$ en si mismo. Entonces $G$ es un grupo con la operación de composición. Se llama el \emph{grupo de transformaciones rígidas}

\nl\textbf{Ejemplo 2} Sea $X=\{x_1,\ldots,x_n\}$ un conjunto de $n$ elementos y $S_n$ definido por 
\[S_n=\{\sigma|\sigma:X\to X\hbox{ y }\sigma \hbox{ es biyectiva }\}\]
Entonces $S_n$ es un grupo  con la operación de composición. Se denomina \href{http://es.wikipedia.org/wiki/Grupo_simétrico}{\emph{grupo simétrico}} 
\end{frame}

\begin{frame}{Ejemplos de grupos}
\nl\textbf{Ejemplo 3} Sea $\Delta$ un polígono regular de $n$ lados  en un plano euclideano $\Pi$ y $D_{2n}$ el conjunto de todas las transformaciones rígidas de $\Pi$ en si mismo que llevan $\Delta$ en si mismo. $D_{2n}$ se llama el \href{http://es.wikipedia.org/wiki/Grupo_diedral}{\emph{grupo diedral}}  de orden $2n$.
Para un triángulo equilatero:
\begin{center}
\includegraphics[scale=.4]{imagenes/SimTria.jpg}
\end{center}

\end{frame}



\begin{frame}{Teoría de grupos computacional: SAGE y GAP}

\href{http://www.gap-system.org/}{GAP - Groups, Algorithms, Programming} Lenguaje de programación para algebra discreta

\href{http://www.sagemath.org/}{SAGE:}  es un sistema de software de matemáticas libre de código abierto bajo la licencia GPL. Se basa en  muchos paquetes de código abierto existentes: NumPy, SciPy, matplotlib, SymPy, Maxima, GAP, FLINT, R y muchos más. Acceda a su poder combinado a través de un lenguaje común, basado en Python 
Misión: Creación de una alternativa libre de código abierto viable a Magma, Maple, Mathematica y Matlab.


\end{frame}

\begin{sagesilent}
reset()
\end{sagesilent}

\begin{frame}[fragile]{Teoría de grupos computacional: SAGE y GAP}
\begin{sagecommandline}
sage: G=SymmetricGroup(5)
sage: sigma=G([(1,2,3),(4,5)])
sage: sigma^2
sage: sigma^3
sage: sigma^6
sage: G.order()
sage: H=G.subgroup([sigma])
sage: H.order()
\end{sagecommandline}
\end{frame}


\begin{frame}[fragile]{Teoría de grupos computacional: SAGE y GAP}
\begin{sagecommandline}
sage: H.list()
sage: H.is_normal()
sage: G1=DihedralGroup(3)
sage: G1[-2]
sage: H1=G1.subgroup(G1[-2])
sage: H1.is_normal()
sage: G1.quotient(H1)
\end{sagecommandline}
\end{frame}


\section[Simetrías]{Grupos continuos de simetrías}

\begin{frame}{Grupos de simetrías}
\nl\begin{block}{Grupos de simetrías}
Los cambios de variables de un conjunto de dos variables, digamos $x$ e $y$, son funciones $\Gamma$, invertibles,  de clase $C^{\infty}$, donde $\Gamma:\Omega_1\to\Omega_2$, con $\Omega_1,\Omega_2$ abiertos de $\rr^2$. \nl Acostumbraremos escribir $(\hat{x},\hat{y})=\Gamma(x,y)$ y diremos que $(\hat{x},\hat{y})$ son la variables nuevas e $(x,y)$ las viejas.

\nl Llamaremos $\mathscr{T}$ al conjunto de todas los cambios de variables $\Gamma$. 
El conjunto  $\mathscr{T}$ tiene una estructura de grupo con la operación de composición.

\nl El grupo de las transformaciones rígidas, los grupos diedrales $D_{2n}$, el grupo de todas las rotaciones alrededor del orígen son subgrupos de  $\mathscr{T}$.

\end{block}


\end{frame}


\begin{frame}{Grupos de simetrías, ejemplos}
\textbf{Ejemplo, polares:} Es más facil describir la transformación que lleva coordenadas polares en cartesinas. En este caso $(x,y)=\Gamma(r,\theta)$ y
\[
\begin{array}{ll}
\Gamma(r,\theta)&=(r\cos(\theta),r\sen(\theta)),\\
\Omega_1&=(0,\infty)\times (-\pi,\pi),\\
\Omega_2&=\rr^2-\{(x,y)|y=0,x\leq 0\}\\
\end{array}
\]

\end{frame}

\begin{frame}{Grupos de Lie uniparamétricos}
\nl\begin{block}{Grupos uniparamétricos de simetrías}
Sea $\mathscr{T}$ el grupo de cambios de variables. Supongamos dado un homomorfismo de grupos $\Gamma:(\rr,+)\to (\mathscr{T},\circ)$.  

\nl\textbf{Notación:} 
\begin{enumerate}
\item<+-> Para $\epsilon\in\rr$ escribiremos $\Gamma_{\epsilon}=\Gamma(\epsilon)$ 
\item<+> Si $(x,y)\in\rr^2$ escribimos $(\hat{x},\hat{y})=\Gamma_{\epsilon}(x,y)$. Notar que $\hat{x}$, $\hat{y}$ son funciones de $x,y$ y $\epsilon$.
\end{enumerate}
\nl Si $\Gamma_{\epsilon}(x,y)$ es diferenciable, con inversa diferenciable, respecto a $(x,y)$ y analítica respecto a $\epsilon$ diremos que $\{\Gamma_{\epsilon}|\epsilon\in\rr\}$ es un \href{http://es.wikipedia.org/wiki/Grupo_uniparamétrico}{\emph{grupo de Lie uniparamétrico de simetrías}}.
\end{block}


\end{frame}






\begin{frame}{Grupos de Lie uniparamétricos}
\nl\begin{block}{Propiedades de $\Gamma_{\epsilon}$}
\begin{enumerate}
\item<+->$\Gamma_{\epsilon}$ es biyectiva y diferenciable sobre su dominio de definición.
 \item<+-> $\Gamma_{\epsilon_1}\circ \Gamma_{\epsilon_2}=\Gamma_{\epsilon_1+\epsilon_2}$.

\item<+-> $\Gamma_0=I$.

\item<+-> $\left(\Gamma_{\epsilon}\right)^{-1}=\Gamma_{-\epsilon}$

\item<+-> Si $\Gamma_{\epsilon}(x,y)=(\hat{x},\hat{y})$, entonces  $\hat{x}(x,y,\epsilon)$ y $\hat{y}(x,y,\epsilon)$ son  diferenciables respecto $(x,y)$ y se desarrollan en serie de potencias respecto a $\epsilon$. Es decir para todo $\epsilon_0\in\rr$ 
\[
\begin{array}{cc}
\hat{x}(x,y,\epsilon)&=a_0(x,y)+a_1(x,y)(\epsilon- \epsilon_0)+\cdots\\
\hat{y}(x,y,\epsilon)&=b_0(x,y)+b_1(x,y)(\epsilon- \epsilon_0)+\cdots\\
\end{array}
\]

\end{enumerate}

\end{block}


\end{frame}


\begin{frame}{Ejemplos grupos de Lie uniparamétricos}
\textbf{Ejercicio:} Demostrar que las siguientes aplicaciones inducen grupos de Lie uniparamétricos
\begin{enumerate}
\item<+-> $\Gamma_{\epsilon}(x,y)=(x+\epsilon,y)$ y $\Gamma_{\epsilon}(x,y)=(x,y+\epsilon)$.
\item<+-> $\Gamma_{\epsilon}(x,y)=(e^{\epsilon}x,y)$
\item<+->$\Gamma_{\epsilon}(x,y)=\left(\frac{x}{1-\epsilon x},\frac{y}{1-\epsilon x} \right)$
\item<+->$\Gamma_{\epsilon}(x,y)=\begin{pmatrix} \cos(\epsilon) & -\sen(\epsilon)
\\ \sen(\epsilon) & \cos(\epsilon)
\end{pmatrix} \begin{pmatrix} x\\ y
\end{pmatrix}
$
\end{enumerate}

\end{frame}

\begin{frame}[fragile]{Ejemplos grupos de Lie uniparamétricos}
Podemos usar \texttt{SymPy} para la tarea
\begin{sageblock}
from sympy import *
T=lambda x,y,epsilon:  Matrix([x+epsilon,y])
x,y,epsilon1,epsilon2=\
symbols('x,y,epsilon1,epsilon2')
PropGrupo=T(T(x,y,epsilon1)[0],\
T(x,y,epsilon1)[1],epsilon2)-\
T(x,y,epsilon1+epsilon2)
\end{sageblock}

El resultado es $\sage{PropGrupo}$.
\end{frame}

\begin{frame}[fragile]{Ejemplos grupos de Lie uniparamétricos}
El mismo ejemplo lo podemos desarrollar usando expresiones en lugar del operador \texttt{lambda}.
\begin{sageblock}
from sympy import *
x,y,epsilon,epsilon1,epsilon2=\
symbols('x,y,epsilon,epsilon1,epsilon2')
T=Matrix([x+epsilon,y])
PropGrupo=T.subs([(\
x,T.subs(epsilon,epsilon1)[0]),\
(1,T.subs(epsilon,epsilon1)[1]),\
(epsilon,epsilon2)\
])\
-T.subs(epsilon,epsilon1+epsilon2)

\end{sageblock}

El resultado es $\sage{PropGrupo}$.
\end{frame}



\begin{frame}{Grupo de simetrías de una ecuación}

\begin{block}{Definición}
\nl Consideremos una ecuación
\boxedeq{y'=f(x,y).}{eq:principi}
Una transformación $\Gamma\in \mathscr{T}$ se denomina una simetría de la ecuación si el cambio de variables dado por $(\hat{x},\hat{y})=\Gamma(x,y)$ deja invariante  la ecuación. 

\nl El conjunto de todas las simetría de una ecuación es un subgrupo de  $( \mathscr{T},\circ)$. Lo llamaremos \emph{grupo de simetrías} de la ecuación.

 

\end{block}


\end{frame}


\begin{frame}{Grupo de simetrías de una ecuación}

\nl De acuerdo con \eqref{eq:subsder} para que $(\hat{x},\hat{y})=\Gamma(x,y)$ sea una simetría de \eqref{eq:principi} se debe cumplir que
 \boxedeq{\frac{\frac{\partial\hat{y}}{\partial x}+\frac{\partial\hat{y}}{\partial y}f(x,y)}{\frac{\partial\hat{x}}{\partial x}+\frac{\partial\hat{x}}{\partial y}f(x,y)}=f(\hat{x},\hat{y})}{eq:cond_sim}

\nl Esta ecuación se llama \emph{condición de simetría}. Es una ecuación en derivadas parciales, en principio más compleja que la ecuación original. Tiene varios grados de libertad, por lo que suele haber muchas simetrías.  Es común que encontremos soluciones a  traves de un  \href{http://es.wikipedia.org/wiki/Ansatz}{ansatz}.

\end{frame}



\begin{frame}{Ejemplos simetrías ecuaciones }
Consideremos la ecuación
\boxedeq{y'=0}{eq:trivial}

La condición de simetría se reduce a 

\[
\frac{\frac{\partial\hat{y}}{\partial x}}{\frac{\partial\hat{x}}{\partial x}}=0
\]

Debemos tener que $\frac{\partial\hat{y}}{\partial x}=0$. Vale decir $\hat{y}$ es independiente de $x$. La forma general de una simetría es
\boxedeq{\hat{x}=\hat{x}(x,y)\quad \hat{y}=\hat{y}(y).}{}

\end{frame}



\begin{frame}{Ejemplos simetrías ecuaciones }


\nl \begin{tabular}{m{5cm}m{5cm}}
\nl Hay muchas simetrías. Las traslaciones en cualquier dirección $(x,y)\mapsto (x+\alpha ,y+\beta)$ ($\alpha,\beta\in\rr$). Cambios de escala en ambos ejes  $(x,y)\mapsto (e^{\epsilon}x,y)$, $(x,y)\mapsto (x,e^{\epsilon}y)$. Reflexiones respecto ambos ejes  $(x,y)\mapsto (-x,y)$, $(x,y)\mapsto (x,-y)$. \nl Observar que el gráfico de las soluciones posee las mismas simetrías, pues en general \emph{las simetrías de una ecuación llevan soluciones en soluciones}.
& 
\includegraphics[scale=.3]{imagenes/sol_trivial.png}\\
\end{tabular}
\end{frame}


\begin{frame}{Simetrías triviales}
\nl De todas las simetrías encontradas $\Gamma_{\epsilon}(x,y)=(e^{\epsilon}x,y)$, $\Gamma_{\epsilon}(x,y)=(x+\epsilon,y)$ y  $\Gamma_{\epsilon}(x,y)=(x+\epsilon,y)$ se llaman \emph{triviales} pues llevan una curva solución en si misma. 

\nl Cualquier cambio de la forma $\hat{x}=\hat{x}(x,y)\quad \hat{y}=y$ es trivial.

\nl \fbox{\begin{minipage}{\linewidth}\emph{Estamos interesados en hallar grupos de Lie uniparamétricos de simetrías no triviales.}
    \end{minipage}
  }
\end{frame}


\begin{frame}{Simetrías triviales}


\nl Las reflexiones $\Gamma(x,y)=(-x,y)$  no pertenecen a tal tipo de  grupo. Para demostrar esto supongamos que  $\Gamma$ es alguna instancia de un tal grupo $\Gamma{\epsilon}$, supongamos por ejemplo que $\Gamma=\Gamma{\epsilon_0}$.  Consideramos el jacobiano de la transformación:

\[J(\epsilon):=\det\begin{pmatrix} \frac{\partial\hat{x}}{\partial x}&  \frac{\partial\hat{x}}{\partial y}\\
 \frac{\partial\hat{y}}{\partial x} &  \frac{\partial\hat{y}}{\partial y}\\
\end{pmatrix}
\]
\nl Tenemos que $J(0)=1$ y $J(\epsilon_0)=-1$. Y $J(\epsilon)$ es continua respecto a $\epsilon$. Por ende existiría $\epsilon'$ con $J(\epsilon')=0$. Esto implica que la matriz jacobiana $D\Gamma$ es singular y esto contradice que $\Gamma:\rr^2\to \rr^2$ es difeomorfismo ($D\Gamma D\Gamma^{-1}=I$).



\nl $\Gamma$  genera un grupo discreto, ya que $\Gamma^2=\Gamma\circ \Gamma=I$. Luego $\Gamma$ genera el grupo $G=\{I,\Gamma\}$ que es isomorfo a $\mathbb{Z}_2$. En este caso diremos que    $\{I,\Gamma\}$ es un \emph{grupo discreto} de simetrías. 
\end{frame}



 \begin{frame}{Ejemplo de Simetrías}
 \textbf{Ejemplo: hallar simetrías de }
\[\frac{dy}{dx}=f(x).\]

De acuerdo con \eqref{eq:subsder} se debe cumplir que 
 \[\frac{\frac{\partial\hat{y}}{\partial x}+\frac{\partial\hat{y}}{\partial y}f(x)}{\frac{\partial\hat{x}}{\partial x}+\frac{\partial\hat{x}}{\partial y}f(x)}=f(\hat{x})\]

La forma de la ecuación sugiere el  \href{http://es.wikipedia.org/wiki/Ansatz}{ansatz}
   \[\boxed{\hat{x}=x},\quad \frac{\partial\hat{y}}{\partial x}=\frac{\partial\hat{x}}{\partial y}=0,\quad
   \frac{\partial\hat{y}}{\partial y}=\frac{\partial\hat{x}}{\partial x}. \]
 Luego 
\[\frac{\partial\hat{y}}{\partial y}=1\Rightarrow \boxed{\hat{y}=y+\epsilon} \]
con $\epsilon$ constante arbitraria.
 \end{frame}

\begin{frame}{Ejemplo de Simetrías}
Hallamos que 
\[\Gamma_{\epsilon}(x,y)=(x,y+\epsilon)\]
es un grupo de Lie uniparamétrico de simetrías. De manera similar
\[\Gamma_{\epsilon}(x,y)=(x+\epsilon,y)\]
es un grupo uniparamétrico de simetrías para 
\[\frac{dy}{dx}=f(y).\]
Geométricamente en el primer caso todas las soluciones se obtienen trasladando una cualquiera verticalmente y en el segundo caso horizontalmente.

\end{frame}

\begin{frame}{Ejemplo de Simetrías}

\begin{tabular}{cc}
\includegraphics[scale=.3]{imagenes/sol_paralelas.png} &\hspace{-1.5cm}\includegraphics[scale=.3]{imagenes/sol_paralelas2.png} \\
Soluciones de $y'=x^3-x$ &\hspace{-1.5cm}Soluciones de $y'=y^3-y$
\end{tabular}

\end{frame}



\begin{frame}{Ejemplo de Simetrías}
\textbf{Demostrar que las rotaciones alrededor del origen es un grupo de Lie uniparamétrico de simetrías de}
 \[\frac{dy}{dx}=\frac{y^3+x^2y-x-y}{x^3+xy^2-x+y}.\]
Sea $\Gamma_{\epsilon}$ la transformación que rota un ángulo $\epsilon$ alrededor del origen. Era un ejercicio demostrar que  $\{\Gamma_{\epsilon}|\epsilon\in\rr\}$ es un grupo uniparamétrico de simetrías. Se tiene la representación matricial
\[
\Gamma_{\epsilon}(x,y)= \begin{pmatrix} \hat{x}\\ \hat{y}
\end{pmatrix}=\begin{pmatrix} \cos(\epsilon) & -\sen(\epsilon)
\\ \sen(\epsilon) & \cos(\epsilon)
\end{pmatrix} \begin{pmatrix} x\\ y
\end{pmatrix}
\]

\[
\Gamma^{-1}_{\epsilon}(\hat{x},\hat{y})= \begin{pmatrix} x\\ y
\end{pmatrix}=\begin{pmatrix} \cos(\epsilon) & \sen(\epsilon)
\\ -sen(\epsilon) & \cos(\epsilon)
\end{pmatrix} \begin{pmatrix} \hat{x}\\ \hat{y}
\end{pmatrix}
\]



\end{frame}


\begin{frame}[fragile]{Ejemplo de Simetrías}
 Para el cálculo recurrimos a \texttt{SymPy} (usamos $x_n$ en lugar de $\hat{x}$)
\begin{sageblock}
from sympy import *
x,theta=symbols('x,theta')
y=Function('y')(x)
x_n=cos(theta)*x-sin(theta)*y
y_n=sin(theta)*x+cos(theta)*y
Expr2=y_n.diff(x)/x_n.diff(x)
Expr3=Expr2.subs(y.diff(),\
(y**3+x**2*y-x-y)/(x**3+x*y**2-x+y))
x_n,y_n=symbols('x_n,y_n')
Expr4=Expr3.subs([(y, -sin(theta)*x_n+cos(theta)*y_n),\
(x,cos(theta)*x_n+sin(theta)*y_n)])
Expr5=simplify(Expr4)
\end{sageblock}
\end{frame}

\begin{frame}[fragile]{Ejemplo de Simetrías}
Tiene problemas para simplificar, lo tenemos que ayudar
\begin{sageblock}

Expr6=Expr5.subs(sin(2*theta + pi/4),\
sin(2*theta)*sqrt(2)+cos(2*theta)*sqrt(2))
Expr7=Expr6.subs(cos(2*theta),\
1+(cos(theta))**2   )
Expr8=Expr7.subs(cos(2*theta),\
1+(cos(theta))**2)
Expr9=Expr8.subs(sin(2*theta),\
1-(cos(theta))**2)
\end{sageblock}
La ecuación resultante es \emph{la misma}
\boxedeq{\frac{dy_n}{dx_n}=\sage{simplify(Expr9)}.}{}
\end{frame}

\begin{frame}{Ejemplo de Simetrías}
A la misma conclusión arribábamos si recordabamos que en en coordenadas polares la ecuación se escribe
\[\frac{dr}{d\theta}=r-r^3,\]
y que esta ecuación tiene las simetrías $\Gamma_{\epsilon}:(r,\theta)\mapsto (r,\theta+\epsilon)$. 

\begin{tabular}{m{6cm} m{4cm}}
\includegraphics[scale=.4]{imagenes/sol_rotadas.png}&
 Si rotamos un ángulo fijo el gráfico de una solución obtenemos el gráfico de otra solución.

  \\
\end{tabular} 
\end{frame}










\begin{frame}{Simetrías resuelven ecuaciones}\label{pag:sim_tras}
\textbf{Ejemplo:} Supongamos que $y'=f(x,y)$ tiene  el grupo de Lie uniparamétrico de simetrías
\boxedeq{(\hat{x},\hat{y})=\Gamma_{\epsilon}(x,y)=(x,y+\epsilon)}{eq:tras_lie}
Usando la condición de simetrías \eqref{eq:cond_sim} tenemos
\[f(x,y)=f(\hat{x},\hat{y})=f(x,y+\epsilon).\]
Luego
\[\frac{\partial f(x,y)}{\partial y}=\lim_{\epsilon\to 0}\frac{f(x,y+\epsilon)-f(x,y)}{\epsilon}=0\]
Asi $f$ es independiente de $y$: $f(x,y)=f(x)$ y la ecuación
\[y'=f(x),\]
se resuelve simplemente integrando.

\end{frame}

\section[Órbitas]{Órbitas, tangentes y curvas invariantes}
\begin{frame}{Órbitas}

\nl\begin{block}{Definición}
Dado un grupo uniparamétrico de simetrías $G=\{\Gamma_{\epsilon}|\epsilon\in\rr\}$, y $(x_0,y_0)\in\rr^2$ llamamos \emph{órbita $(x_0,y_0)$ bajo la acción de  $G$} (simplemente órbita si es claro quien es $G$) a la curva
\[\{\Gamma_{\epsilon}(x_0,y_0)|\epsilon\in\rr\} \]
\end{block}

\nl 
\begin{tabular}{m{5cm} m{4cm}}
\includegraphics[scale=.3]{imagenes/sol_trivialB.png} &
Si $G$ es un grupo de simetrías no trivial, entonces es de esperar que la órbita de  $(x_0,y_0)$ cruza transversalmente las curvas solución. La órbita se usará como una nueva coordenada.   
\end{tabular}
\end{frame}
\begin{frame}{Órbitas}
\nl La órbita atraves de $(x,y)$ es el conjunto de puntos de coordenadas
\begin{equation}\label{eq:orb} (\hat{x}(x,y,\epsilon),\hat{y}(x,y,\epsilon))=\Gamma_{\epsilon}(x,y),
\end{equation}
donde
\[(\hat{x}(x,y,0),\hat{y}(x,y,0))=(x,y).\]

\eqref{eq:orb} son ecuaciones parámetricas (parámetro $\epsilon$) de una  curva en el plano.

\nl \begin{block}{Puntos invariantes}
Un punto $(x,y)$ se llama invariante si su órbita se reduce a $\{(x,y)\}$, vale decir
\[(x,y)=\Gamma_{\epsilon}(x,y),\quad\forall \epsilon>0\]

\end{block}


\end{frame}


 \begin{frame}{Órbitas}
 \textbf{Ejemplo:} La órbita de $(x,y)$ bajo la acción del grupo  de Lie uniparamétrico 
\[
\Gamma_{\epsilon}(x,y)= \begin{pmatrix} \hat{x}\\ \hat{y}
\end{pmatrix}=\begin{pmatrix} \cos(\epsilon) & -\sen(\epsilon)
\\ \sen(\epsilon) & \cos(\epsilon)
\end{pmatrix} \begin{pmatrix} x\\ y
\end{pmatrix}
\]

Son circunsferencias con centro en el origen. El punto $(0,0)$ es invariante.

\end{frame}


 \begin{frame}{Campo vectorial de tangentes}

\nl\begin{block}{Definición}
 Dado un grupo de Lie uniparamétrico $(\hat{x}(x,y,\epsilon),\hat{y}(x,y,\epsilon))=\Gamma_{\epsilon}(x,y)$  definimos el campo vectorial
\[(\xi(\hat{x},\hat{y}),\eta(\hat{x}, \hat{y}))=\left(\left.\frac{d\hat{x}}{d\epsilon}\right|_{\epsilon=0}, \left.\frac{d\hat{y}}{d\epsilon}\right|_{\epsilon=0}   \right)\]

 $\xi$ y $\eta$ se llaman \emph{símbolos infinitesimales}.
 \end{block}

\nl Como $\hat{x},\hat{y}$ eran analíticas respecto a $\epsilon$ :
\[
\begin{array}{cc}
\hat{x}&=x+\epsilon\xi(x,y)+O(\epsilon^2)\\
\hat{y}&=y+\epsilon\eta(x,y)+O(\epsilon^2)\\
\end{array}
\]
 En un punto invariante \fbox{$\xi(x,y)=\eta(x,y)=0$}.
 \end{frame}


\begin{sagesilent}
reset()
\end{sagesilent}

\begin{frame}[fragile]{Campo vectorial de tangentes}
\begin{sageblock}
x,y,epsilon = var('x,y,epsilon')
x_n=cos(epsilon)*x-sin(epsilon)*y
y_n=sin(epsilon)*x+cos(epsilon)*y
xi=x_n.diff(epsilon)(epsilon=0)
eta=y_n.diff(epsilon)(epsilon=0)
p=plot([])
for x_abs in srange(0,1,.2):
    p+=parametric_plot([x_n(x=x_abs,y=0),\
y_n(x=x_abs,y=0)], (epsilon,0,2*pi))
p+= plot_vector_field((xi,eta),(x,-1,1),\
(y,-1,1))
\end{sageblock}


\end{frame}

\begin{frame}[fragile]{Campo vectorial de tangentes}
\sageplot[scale=.5]{p}


\end{frame}

\begin{frame}{Curvas invariantes}
\begin{block}{Definición}
Una curva plana $C$ se dice invariante por un grupo uniparamétrico de simetrías de Lie si y sólo si la tangente a  $C$ en cada punto $(x,y)$ es paralela a $(\xi(x,y),\eta(x,y))$. Si $C$ es el gráfico de una función $x\mapsto y(x)$, como $(1,y'(x))$ es un vector tangente a la gráfica, la condición que $C$ es invariante se escribe
\boxedeq{Q(x,y,y')\overset{\hbox{\small def}}{=}\eta(x,y)-y'\xi(x,y)\equiv 0}{eq:inv}
Esta se llama \emph{ecuación característica}.
\end{block} 


\end{frame}


\begin{frame}{Curvas invariantes}
\nl \begin{block}{Soluciones invariantes}
Si $\Gamma_{\epsilon}$ es un grupo de Lie de simetrías de $y'=f(x,y)$ entonces una solución $y(x)$ es una curva invariantes si y sólo si
\boxedeq{\overline{Q}(x,y)=\eta(x,y)-f(x,y)\xi(x,y)\equiv 0}{eq:sol_inv}

Se llaman \emph{ecuación característica reducida}.
\end{block} 

\nl \emph{Nos conviene tener soluciones no invariantes}

\end{frame}


\begin{frame}{Ejemplos}
\textbf{Ejemplo:} La EDO
\[y'=y\]
Tiene simetrías de escala
\[(\hat{x},\hat{y})=\Gamma_{\epsilon}(x,y)=(x,e^{\epsilon}y).\]
Luego 
\[(\xi,\eta)=\left(\left.\frac{d\hat{x}}{d\epsilon}\right|_{\epsilon=0}, \left.\frac{d\hat{y}}{d\epsilon}\right|_{\epsilon=0}   \right)=(0,y)\]
Cualquier punto en el conjunto $\{(x,0)|x\in\rr\}$  es invariante. La ecuación característica reducida es.
\[\overline{Q}(x,y)=0\Rightarrow y=0\]
Esta formada enteramente por puntos invariantes.

\end{frame}


\begin{frame}{Ejemplos}

\textbf{Ejercicio:} demostrar que la siguiente expresión es un grupo de Lie uniparamétrico de simetrías
\[(\hat{x},\hat{y})=\Gamma_{\epsilon}(x,y)=(e^{\epsilon}x,e^{(e^{\epsilon}-1)x}y).\]
Para este grupo tenemos
\[(\xi,\eta)=(x,xy)\]
Todo punto en $x=0$ es invariante. La ecuación característica reducida es
\[\overline{Q}(x,y)=0\Rightarrow xy-xy=0\]
De modo que estas simetrías actuan trivialmente sobre las soluciones. Llevan una solución en si misma. Chequeemos esta afirmación de manera directa. 


\end{frame}


\begin{frame}{Ejemplos}
Buscamos el cambio de variables inverso
\[
\left.
\begin{array}{ll} 
  \hat{x}&= e^{\epsilon}x\\
  \hat{y}&=e^{(e^{\epsilon}-1)x}y\\
\end{array}
\right\} 
\Rightarrow  
    \left.
\begin{array}{ll} 
  x&= e^{-\epsilon}\hat{x}\\
  y&=e^{(e^{-\epsilon}-1)\hat{x}}\hat{y}\\
\end{array}
\right\} 
\]    

Las soluciones son $y=ke^x$, sustituímos en esta expresión, luego de unas operaciones, llegamos a $\hat{y}=ke^{\hat{x}}$.
\end{frame}



\begin{frame}{Ejemplos}

\textbf{Ejemplo.} La ecuación de Riccati
\[y'=xy^2-\frac{2y}{x}-\frac{1}{x^3},\quad x\neq 0\] 
 Tiene el grupo de Lie de simetrías
\[(\hat{x},\hat{y})=\Gamma_{\epsilon}(x,y)=(e^{\epsilon}x,e^{-2\epsilon}y).\]
Tenemos
\[(\xi,\eta)=(x,-2y).\]
La característica reducida
\[\overline{Q}(x,y)=\frac{1}{x^2}-x^2y^2=0.\]
Tenemos dos soluciones invariantes
\[y=\pm\frac{1}{x^2}.\] 


\end{frame}

\begin{frame}{Una observación}

\nl La mayoría de los métodos de simetría usan los infinitesimales $(\xi,\eta)$ en lugar de las simetrías en si mismas.  
No obstante  $(\xi(\hat{x},\hat{y}),\eta(\hat{x},\hat{y}))$ determinan las simetrías a traves de las ecuaciones

\[
\left\{
\begin{array}{ll}
\frac{d\hat{x}}{d\epsilon}&=\xi(\hat{x},\hat{y})\\
\frac{d\hat{y}}{d\epsilon}&=\eta(\hat{x},\hat{y})\\
\hat{x}(x,y,0)&=x\\
\hat{y}(x,y,0)&=y\\
\end{array}
\right.
\] 

\nl \textbf{Ejemplo:} Estas ecuaciones pueden ser dificiles de resolver. Pero en algunos casos sencillos puede ser fácil. Supongamos dado  $(\xi,\eta)=(0,\hat{y})$. Las ecuaciones implican $\hat{x}$ independiente de $\epsilon$.  Luego por la condición inicial $\hat{x}=x0$
 es función de $x,y$ solo. Además para $\hat{y}$ tenemos $\hat{y}=B(x,y)e^{\epsilon}$. Las condiciones iniciales implican $A(x,y)=x$ y $B(x,y)=y$.  

\end{frame}

\section{Coordenadas canónicas}

\begin{frame}{Coordenadas canónicas}
\nl \begin{block}{Definición}   Diremos que las coordenadas $(r,s)$ son canónicas respecto a el grupo de Lie de simetrías $\Gamma{\epsilon}$  si en las coordenadas $(r,s)$ la acción de grupo es la traslación 

\boxedeq{(\hat{r},\hat{s})\overset{\hbox{def}}{=} (r(\hat{x},\hat{y}),s(\hat{x},\hat{y}))=(r(x,y),s(x,y)+\epsilon).}{eq:coorcan}
\end{block}
\nl \textbf{Ejemplo:} Las coordenadas polares son canónicas respecto al grupo de Lie de rotaciones. Las rotaciones en coordenadas cartesianas y polares se escriben
\[
 \begin{pmatrix} \hat{x}\\ \hat{y}
\end{pmatrix}=\begin{pmatrix} \cos(\epsilon) & -\sen(\epsilon)
\\ \sen(\epsilon) & \cos(\epsilon)
\end{pmatrix} \begin{pmatrix} x\\ y
\end{pmatrix},\quad \begin{array}{ll} \hat{r}&=r\\ \hat{\theta}&=\theta+\epsilon\\ 
\end{array}
\]
\end{frame}

\begin{frame}{Coordenadas canónicas}
\nl Derivando las ecuaciones \eqref{eq:coorcan} respecto a $\epsilon$ obtenemos
\boxedeq{
\begin{array}{ll}
\xi(x,y)\frac{\partial r}{\partial x}+\eta(x,y)\frac{\partial r}{\partial y}&=0\\
\xi(x,y)\frac{\partial s}{\partial x}+\eta(x,y)\frac{\partial s}{\partial y}&=1
\end{array}
}{eq:coor_can_dif}
\nl Los cambios de coordenadas deben ser invertibles, de modo que pediremos la condición de no degeneración
\boxedeq{\frac{\partial r}{\partial x}\frac{\partial s}{\partial y}-\frac{\partial r}{\partial y}\frac{\partial s}{\partial x}\neq 0}{eq:no_deg}

\nl \begin{block}{Extensión definición}
Cualquier par de coordenadas que satisfacen \eqref{eq:coor_can_dif} y \eqref{eq:no_deg} se llaman canónicas.
\end{block}



\end{frame}


\begin{frame}{Coordenadas canónicas}

\nl \begin{block}{Observaciones}
\begin{enumerate}
\item<+->El vector tangente en cualquier punto no invariante es paralelo a la curva $r=\hbox{cte}$ que pasa por ese punto. Luego esa curva es una órbita. Las órbitas son invariantes, así  $r$ se llama la \emph{coordenada invariante}. Las curvas $s=\hbox{cte}$ son transversales a las órbitas.
\item<+-> Las coordenadas canónicas no estan definidas en un punto $(x,y)$ invariante pues en esos puntos $\xi(x,y)=\eta(x,y)=0$.

\item<+-> Las coordenadas canónicas estan definidas en un entorno de cualquier   punto no invariante.

\item<+-> Las coordenadas canónicas no son únicas. De hecho si $(r,s)$ son canónicas $(\tilde{r},\tilde{s})=(F(r),G(r)+s)$ lo son para cualquier $F$ y $G$ con $F'(r)\neq 0$ para la no degeneración.

\end{enumerate}
\end{block}



\end{frame}


\begin{frame}{Encontrando coordenadas canónicas}

\nl \begin{block}{Integrales primeras}
Una integral primera de la EDO $y'=f(x,y)$ es una función $\phi(x,y)$ que es constante a lo largo de una curva solución de la EDO. 
\end{block}

\nl \begin{block}{Teorema} Si $(r,s)$ son coordenadas canónicas de una grupo de Lie de simetrías entonces $r$ es una integral primera de la ecuación
\begin{equation}\label{eq:int_pri} \frac{dy}{dx}=\frac{\eta(x,y)}{\xi(x,y)}.
\end{equation}
\end{block}


\end{frame}


\begin{frame}{Encontrando coordenadas canónicas}

\textbf{Dem.} Supongamos $y(x)$ solución de la EDO, es suficiente demostrar que $\frac{d}{dx}r(x,y(x))=0$. En efecto
\[
\begin{array}{lll}
 \frac{d}{dx}r(x,y(x)) &=\frac{\partial r}{\partial x}+\frac{\partial s}{\partial y}y' &\text{(regla cadena)}\\
&=\frac{\partial r}{\partial x}+\frac{\partial s}{\partial y}\frac{\eta(x,y)}{\xi(x,y)}& \text{(Ec. \eqref{eq:int_pri})}\\
&=0 & \text{(Ec. \eqref{eq:coor_can_dif})}\\
\end{array}
\]
\end{frame}




\begin{frame}{Encontrando coordenadas canónicas}
Hallar una integral primera implica resolver la ecuación.
\textbf{Ejemplo} Ya conocemos las coordenadas canónicas de las rotaciones, 
\[
 \begin{pmatrix} \hat{x}\\ \hat{y}
\end{pmatrix}=\begin{pmatrix} \cos(\epsilon) & -\sen(\epsilon)
\\ \sen(\epsilon) & \cos(\epsilon)
\end{pmatrix} \begin{pmatrix} x\\ y
\end{pmatrix}\Rightarrow  \begin{pmatrix} \xi\\ \eta
\end{pmatrix}= \begin{pmatrix} -y\\ x
\end{pmatrix}
\]
hallemosla por el método propuesto. Hay que resolver 
\[\frac{dy}{dx}=-\frac{x}{y}\Rightarrow ydy=-xdx\Rightarrow y^2+x^2=C\]
Luego $r=\sqrt{x^2+y^2}$ es una integral primera. 


\end{frame}



\begin{frame}{Encontrando coordenadas canónicas}
La coordenada $r$ es constante sobre los puntos en la gráfica de una solución de \eqref{eq:int_pri}. Sobre esos puntos $(x,y(x))$, la coordenada $s$ satisface:

\begin{equation} \label{eq:coor_can_s}
\begin{array}{lll}
\frac{ds}{dx}&=\frac{\partial s}{\partial x}+\frac{\partial s}{\partial y} y'(x) & \hbox{(Regla cadena)}\\
&=\frac{\partial s}{\partial x}+\frac{\partial s}{\partial y} \frac{\eta}{\xi} &
  \eqref{eq:int_pri}\\
&=\frac{1}{\xi} &\hbox{\eqref{eq:coor_can_dif}}.
\end{array}
\end{equation}

Ahora podemos aprovechar que ya conocemos $r$ y  expresar $y$ como función de $r,x$. Luego

\begin{block}{Teorema: Expresión para $s$}
\boxedeq{s=\int\frac{ds}{dx}dx=\int\frac{dx}{\xi(x,y(r,x))}.}{eq:expr_coor_s}
 
\end{block}

 






\end{frame}




\begin{frame}{Encontrando coordenadas canónicas}
En la igualdad resultante puede ser necesario reemplazar $r$ por su expresión en las variables $x,y$.


Si ocurriese que $\xi=0$ y $\eta\neq 0$. Entonces por \eqref{eq:int_pri} $r_y=0$, de modo que $r$ es sólo función de $x$. Se puede asumir $r=x$. Además $\eta s_y=1$, entonces
\boxedeq{s=\int\frac{dy}{\eta(r,y)} .}{eq:s_xi_0}


\textbf{Ejemplo} Retornando al ejemplo de las rotaciones, donde hallamos que $r=\sqrt(x^2+y²)$, vemos que

\[s=\int\frac{dx}{-y}=-\int\frac{dx}{\sqrt{r^2-x^2}}=\arccos\left(\frac{x}{r}\right).\]
Por consiguiente  $s$ es el ángulo polar.

\end{frame}



\begin{frame}{Encontrando coordenadas canónicas}\label{pag_ejem_canon1}
\textbf{Ejemplo} Encontrar coordenadas canónicas para el grupo de Lie de simetrías
\[(\hat{x},\hat{y})=(e^{\epsilon}x,e^{k\epsilon}y)\quad k>0.\]
El vector tangente es
\[(\xi,\eta)=\left(\left.\frac{d\hat{x}}{d\epsilon}\right|_{\epsilon=0},\left.\frac{d\hat{y}}{d\epsilon}\right|_{\epsilon=0}\right)=(x,ky).\]
Resolvamos la ecuación  \eqref{eq:int_pri} 
\[\frac{dy}{dx}=\frac{ky}{x}\Rightarrow y=Cx^k.\]
Luego $\Phi=y/x^k$ es integral primera. Entonces podemos tomar $r=y/x^k$. 




\end{frame}




\begin{frame}{Encontrando coordenadas canónicas}\label{pag_ejem_canon2}
 Para $s$
\[s=\int\frac{dx}{\xi}=\frac{dx}{x}=\ln|x|.\]
Entonces $(r,s)=(yx^{-k},\ln|x|)$ son coordenadas canónicas. No estan definidas en $x=0$.
Podemos encontrar coordenadas canónicas definidas en $x=0$ del siguiente modo. Recordamos que para todas $F$ y $G$
\[(\tilde{r},\tilde{s})=(F(r),G(r)+s)=(F(x^{-k}y),G(x^{-k}y)+\ln|x|).\]
So canónicas también. Si tomamos $F(r)=1/r$ y $G(r)=\frac{1}{k}\ln|x|$, evitamos la singularidad. Luego

\[(\tilde{r},\tilde{s})=(x^ky^{-1},\frac{1}{k}\ln|y|).\]
Son canónicas, están definidas en $x=0$ pero no en $y=0$.  






\end{frame}


\begin{frame}{Encontrando coordenadas canónicas}\label{pag:can_ejem_p}
\textbf{Ejemplo} Encontrar coordenadas canónicas para
\[(\hat{x},\hat{y})=\left( \frac{x}{1-\epsilon x},\frac{y}{1-\epsilon x}\right).\]

\[(\xi,\eta)=\left(\left.\frac{d\hat{x}}{d\epsilon}\right|_{\epsilon=0},\left.\frac{d\hat{y}}{d\epsilon}\right|_{\epsilon=0}\right)=(x^2,xy).\]

\[\frac{dy}{dx}=\frac{eta}{\xi}=\frac{y}{x}\Rightarrow \frac{y}{x}=\hbox{cte}.\]

Podemos tomar $r=y/x$. Para $s$
\[s=\int\frac{dx}{x^2}=-\frac{1}{x}.\]
Luego $(r,s)=(\frac{y}{x},-\frac{1}{x})$ son canónicas.



En este caso los puntos sobre $x=0$ son invariantes, no podemos definir coordenadas canónicas allí.


 
\end{frame}



\begin{frame}[fragile]{Encontrando coordenadas canónicas}
Lo podemos desarrollar con \texttt{SymPy}. 
\begin{sagecommandline}
sage: from sympy import *
sage: x,y,epsilon=symbols('x,y,epsilon')
sage: T=Matrix([x/(1-epsilon*x),y/(1-epsilon*x)])
sage: xi=T[0].diff(epsilon).subs(epsilon,0)
sage: xi
sage: eta=T[1].diff(epsilon).subs(epsilon,0)
sage: eta/xi
sage: y=Function('y')(x)
sage: dsolve(y.diff(x)-y/x,y)


\end{sagecommandline}
\end{frame}

\begin{frame}[fragile]{Encontrando coordenadas canónicas}
 
\begin{sagecommandline}
sage: Integral(1/xi,x).doit()

\end{sagecommandline}
\end{frame}









\section[Resolviendo EDO]{Resolviendo EDO con grupos de Lie de simetrías}

\begin{frame}{Resolviendo EDO con grupos de Lie de simetrías}

\onslide<+-> Supongamos dado  un grupo de Lie de simetrías $(\hat{x},\hat{y})$ de la ecuación
\boxedeq{y'=f(x,y)}{eq:eq_princi}

\onslide<+-> Supongamos  que las simetrías son no triviales. Según \eqref{eq:sol_inv} debemos tener
 \[\eta(x,y)\not\equiv f(x,y)\xi(x,y)\]
La razón de esta condición es que si fuese falsa entonces la ecuación \eqref{eq:int_pri} es la misma que la ecuación \eqref{eq:eq_princi} y el método es inútil.

\onslide<+->Supongamos $(r,s)$ coordenadas canónicas. La ecuación en las coordenadas $(r,s)$, según \eqref{eq:subsder}, se escribirá
\boxedeq{\frac{ds}{dr}=\hat{f}(r,s):=\frac{s_x+f(x,y)s_y}{r_x+f(x,y)r_y}.}{eq:prin_en_canon}
 
\end{frame}


\begin{frame}{Resolviendo EDO con grupos de Lie de simetrías}
\onslide<+->Las coordenadas canónicas se definen por \eqref{eq:coorcan} de modo que el grupo de simetrías actúe por traslación $(\hat{r},\hat{s})=(r,s+\epsilon)$. 
\onslide<+->Por los resultados de la página  \ref{pag:sim_tras}, $\hat{f}$ es independiente de $s$. Y la ecuación se reduce a
\boxedeq{\frac{ds}{dr}=\hat{f}(r)}{eq:ecu_prin_canonica}

que se resuelve integrando.


\onslide<+->\textbf{Ejemplo:} Resolver 
\[y'=xy^2-\frac{2y}{x}-\frac{1}{x^3},\quad x\neq 0,\]
sabiendo que la ecuación es invariante para el grupo de simetrías
\[(\hat{x},\hat{y})=(e^{\epsilon}x,e^{-2\epsilon}y).\]
\end{frame}


\begin{frame}{Resolviendo EDO con grupos de Lie de simetrías}
\onslide<+->Por los resultados de las páginas \ref{pag_ejem_canon1} y \ref{pag_ejem_canon2}:
\[(r,s)=(x^2y,\ln|x|)\]
son canónicas.

\onslide<+-> Según \eqref{eq:prin_en_canon} la ecuación en $(r,s)$ es:
\[\frac{dr}{ds}=\frac{\frac{1}{x}}{2xy+x^2\left(  xy^2-\frac{2y}{x}-\frac{1}{x^3}     \right)}=\frac{1}{x^4y^2-1}=\frac{1}{r^2-1}\]
Como sabíamos que debía suceder el resultado del segundo miembro ndepende sólo de $r$. Integrando
\[s=\frac12\ln\left( \frac{r-1}{r+1}  \right)+C.\]
Sustituyendo
\[\ln|x|=\frac12\ln\left( \frac{x^2y-1}{x²y+1}  \right)+C.\]



\end{frame}


\begin{frame}{Resolviendo EDO con grupos de Lie de simetrías}
Despejando
\boxedeq{y =-\frac{x^2 + C}{x^2(x^2 - C)}}{eq:sol_ejem}
La ecuación característica reducida \eqref{eq:sol_inv} es para la ecuación de este ejemplo:

\[0=\overline{Q}=-2y- \left(xy^2-\frac{2y}{x}-\frac{1}{x^3}  \right)x=-x^2y^2+\frac{1}{x^2}\]
Cuyas soluciones son
\[y=\pm\frac{1}{x^2}.\]
Que además son solución de la ecuación diferencial. La curva $y=-1/x^2$ se obtiene de \eqref{eq:sol_ejem} con $C=0$. La  curva $y=-1/x^2$

\end{frame}


\begin{frame}{Resolviendo EDO con grupos de Lie de simetrías}

\begin{tabular}{m{0.3\linewidth} >{\centering\arraybackslash}m{0.7\linewidth} }
Las curvas azules y naranjas se corresponden con las gráficas de \eqref{eq:sol_ejem} con $c>0$ y $c<0$ respectivamente. La verde es la de $y=1/x^2$ y la roja de $y=-1/x^2$.
&
\includegraphics[scale=.4]{imagenes/SolGrup.png}
\end{tabular}
\end{frame}



\begin{frame}{Resolviendo EDO con grupos de Lie de simetrías}



\onslide<+->\textbf{Ejemplo:} Resolver 
\[y'=\frac{y+1}{x}+\frac{y^2}{x^3},\quad x\neq 0,\]
sabiendo que la ecuación es invariante para el grupo de simetrías
\[(\hat{x},\hat{y})=(\frac{x}{1-\epsilon x},\frac{y}{1-\epsilon x}   ).\]
Ya hemos computado las coordenadas canónicas en página \ref{pag:can_ejem_p}:
\[(r,s)=\left(\frac{y}{x},\frac{1}{x}\right).\]
Por \eqref{eq:prin_en_canon} la ecuación se escribe
\[\frac{dr}{ds}=\frac{-\frac{1}{x^2} }{-\frac{y}{x^2}+\frac{1}{x}\left(
\frac{y+1}{x}+\frac{y^2}{x^3}\right)}=\frac{1}{1+r^2}\]

\end{frame}



\begin{frame}{Resolviendo EDO con grupos de Lie de simetrías}
Cuya solución es 
\[s=\arctan(r)+C\Rightarrow y=-x\tan\left(\frac{1}{x}+C\right).\]
\end{frame}


\begin{frame}{Ecuaciones homogéneas}
\textbf{Ejemplo:} Resolver la ecuación 
\boxedeq{y'=F\left(\frac{y}{x}\right).}{eq:homogenea}
Aquí tenemos el Grupo de Lie de simetrías de cambio de escalas
\[(\hat{x},\hat{y})=(e^{\epsilon}x,e^{\epsilon}y).\] 

Por los resultados de páginas \ref{pag_ejem_canon1} y \ref{pag_ejem_canon2}, $(r,s)=(y/x,\ln|x|)$ son canónicas y la ecuación se escribe
\[\frac{ds}{dr}=\frac{\frac{1}{x}}{-\frac{y}{x^2}+\frac{F\left(\frac{y}{x}\right)}{x}}=\frac{1}{F(r)-r}.\]
La solución general es 
\[\ln|x|=\int^{y/x}\frac{dr}{F(r)-r}+c.\]

\end{frame}


\begin{frame}{Método de Lie y \texttt{SymPy}}
\text{SymPy} Incorpora distintas estrategías para resolver ecuaciones por el método de Lie. Hay mucho por indagar al respecto, pero sólo vamos a mencionar una función para calcular los infinitesimales $(\xi,\eta)$.  Aprovechamos para mostrar como luce una consola de \texttt{ipython}, otra manera de usar \texttt{Python} y \texttt{SymPy}.

\includegraphics[scale=.45]{imagenes/ipython.png}



\end{frame}

\end{document}

