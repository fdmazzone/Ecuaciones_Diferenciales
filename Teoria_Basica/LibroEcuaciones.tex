\documentclass[oneside]{book}

%%%%%%%%%%%%%%%%%%%%%%%%%%%%%%Paquetes%%%%%%%%%%%%%%%%%%%%%%%%%%%%%%%%%%%%%%%%%%%%%%%5
%%%%%%%%%%%%%%%%%%%%%%%%%%%%%%%%%%%%%%%%%%%%%%%%%%%%%%%%%%%%%%%%%%%%%%%%%%%%%%%%%%%%%
\usepackage{empheq}
\usepackage[spanish]{babel}
%\usepackage[utf8]{inputenc}
\usepackage[T1]{fontenc}
\usepackage{amssymb,amsmath,amsthm}
\usepackage{enumerate}
\usepackage{verbatim}
\usepackage{ esint }
%\usepackage{pst-all}
%\usepackage{pstricks-add}
\usepackage{array}
\usepackage{animate}
\usepackage{xparse}
\usepackage{listings}
\usepackage{ wasysym }
\usepackage{yfonts,mathrsfs,eufrak}
\usepackage{hyperref}
\usepackage{color}
\usepackage{url}
\usepackage{theorem}
\usepackage{boiboites}
\usepackage{wrapfig}
\usepackage{ esint }
\usepackage[spanish]{varioref}
\usepackage{fontspec}
\usepackage[a4paper,driver=xetex,top=2.5cm, bottom=2.7cm,%
layouthoffset=10mm, left=2.3cm, right=6.2cm]{geometry}
\usepackage{pst-barcode}
\usepackage{fancyhdr}
\usepackage{marginnote}
 
%%%%%%%%%%%%%%  Configuración página

\fancyfoot{}
\fancyhead[RO,LE]{\thepage}
\fancyhead[LO]{\leftmark}
\fancyhead[RE]{\rightmark}




%%%%%%%%%%%%%%%%%%%%%%%%Colores
\definecolor{color1}{rgb}{0.24,0.45,0.42}
\definecolor{color2}{rgb}{0.44,0.62,0.42}
\definecolor{color3}{rgb}{0.28, 0.51, .68}
\definecolor{color4}{rgb}{0.29,0.3,0.57}
\definecolor{color5}{rgb}{0.7,0.24,0.24}
\definecolor{color6}{rgb}{0.72,0.4,0.28}
\definecolor{color7}{rgb}{0.84,0.66,0.21}
\definecolor{color8}{rgb}{0.3,0.29,0.28}


%%%%%%%%%%%%%%%%%%%%%% Configuracion listing

\lstset{ %
  backgroundcolor=\color{white},   % choose the background color; you must add \usepackage{color} or \usepackage{xcolor}
  basicstyle=\footnotesize,        % the size of the fonts that are used for the code
  breakatwhitespace=false,         % sets if automatic breaks should only happen at whitespace
  breaklines=true,                 % sets automatic line breaking
  captionpos=b,                    % sets the caption-position to bottom
  commentstyle=\color{color2},    % comment style
  deletekeywords={...},            % if you want to delete keywords from the given language
  escapeinside={\%*}{*)},          % if you want to add LaTeX within your code
  extendedchars=true,              % lets you use non-ASCII characters; for 8-bits encodings only, does not work with UTF-8
  frame=single,	                   % adds a frame around the code
  keepspaces=true,                 % keeps spaces in text, useful for keeping indentation of code (possibly needs columns=flexible)
  keywordstyle=\color{blue},       % keyword style
  language=Python,                 % the language of the code
  otherkeywords={symbols,dsolve,solve,Eq, simplify, subs, plot,Function},           % if you want to add more keywords to the set
  numbers=left,                    % where to put the line-numbers; possible values are (none, left, right)
  numbersep=5pt,                   % how far the line-numbers are from the code
  numberstyle=\tiny\color{color3}, % the style that is used for the line-numbers
  rulecolor=\color{black},         % if not set, the frame-color may be changed on line-breaks within not-black text (e.g. comments (green here))
  showspaces=false,                % show spaces everywhere adding particular underscores; it overrides 'showstringspaces'
  showstringspaces=false,          % underline spaces within strings only
  showtabs=false,                  % show tabs within strings adding particular underscores
  stepnumber=2,                    % the step between two line-numbers. If it's 1, each line will be numbered
  stringstyle=\color{color4},     % string literal style
  tabsize=2,	                   % sets default tabsize to 2 spaces
  title=\lstname                   % show the filename of files included with \lstinputlisting; also try caption instead of title
}


%%%%%%%%%%%%%Configuración de fuente para XeLaTeX

\setsansfont{Roboto Condensed}
\renewcommand{\familydefault}{\sfdefault}


%%%%%%%%%%%%%%%%%%%%%%%%%%Nuevos comandos entornos%%%%%%%%%%%%%%%%%%%%%%%%%%%%%%%%
%%%%%%%%%%%%%%%%%%%%%%%%%%%%%%%%%%%%%%%%%%%%%%%%%%%%%%%%%%%%%%%%%%%%%%%%
\newenvironment{demo}{\noindent\emph{Dem.}}{$\square$ \newline\vspace{5pt}}

\newcommand{\com}{\mathbb{C}}
\newcommand{\dis}{\mathbb{D}}
\newcommand{\rr}{\mathbb{R}}
\newcommand{\oo}{\mathcal{O}}
\newcommand{\der}[2]{\frac{\partial #1}{\partial #2}}
\renewcommand{\v}[1]{\overrightarrow{#1}}
\renewcommand{\epsilon}{\varepsilon}
\renewcommand{\b}[1]{\boldsymbol{#1}}
\renewenvironment{frame}[1]{}{}
\DeclareMathOperator{\atan2}{atan2}
\DeclareMathOperator{\sen}{sen}




%%%%%%%%%%Definimos una caja con color
\newlength\mytemplen
\newsavebox\mytempbox
\makeatletter
\newcommand\mybluebox{%
\@ifnextchar[%]
{\@mybluebox}%
{\@mybluebox[0pt]}}
\def\@mybluebox[#1]{%
\@ifnextchar[%]
{\@@mybluebox[#1]}%
{\@@mybluebox[#1][0pt]}}
\def\@@mybluebox[#1][#2]#3{
\sbox\mytempbox{#3}%
\mytemplen\ht\mytempbox
\advance\mytemplen #1\relax
\ht\mytempbox\mytemplen
\mytemplen\dp\mytempbox
\advance\mytemplen #2\relax
\dp\mytempbox\mytemplen
\colorbox{color1}{\hspace{1em}\usebox{\mytempbox}\hspace{1em}}}
\makeatother
\DeclareDocumentCommand\boxedeq{ m g }{%
{\begin{empheq}[box={\mybluebox[2pt][2pt]}]{equation}% #1%
\IfNoValueF {#2} {\label{#2}}%
#1
\end{empheq}
}%
}


%%%%%%%%%%%%%%%%%%
\newboxedtheorem[boxcolor=color1, background=color2, titlebackground=color1,
titleboxcolor = black,thcounter=section]{problema}{Problema}{thcounter1}

\newboxedtheorem[boxcolor=color1, background=color2, titlebackground=color1,
titleboxcolor = black,thcounter=section]{teorema}{Teorema}{thcounter2}

\newboxedtheorem[boxcolor=color1, background=color2, titlebackground=color1,
titleboxcolor = black,thcounter=section]{definicion}{Definici\'on}{thcounter3}

\newboxedtheorem[boxcolor=color1, background=color2, titlebackground=color1,
titleboxcolor = black,thcounter=section]{lema}{Lema}{thcounter4}

\newboxedtheorem[boxcolor=color1, background=color2, titlebackground=color1,
titleboxcolor = black,thcounter=section]{corolario}{Corolario}{thcounter5}

\newboxedtheorem[boxcolor=color1, background=color2, titlebackground=color1,
titleboxcolor = black,thcounter=section]{proposicion}{Proposici\'on}{thcounter6}

\newboxedtheorem[boxcolor=color1, background=color2, titlebackground=color1,
titleboxcolor = black,thcounter=section]{codigo}{Función SymPy}{}

\newboxedtheorem[boxcolor=color1, background=color2, titlebackground=color1,
titleboxcolor = black,thcounter=section]{ejercicio}{Ejercicio}{}


\newcounter{ejemplo_cont}
\setcounter{ejemplo_cont}{1}

\newenvironment{ejemplo}{\noindent\textbf{Ejemplo  \arabic{ejemplo_cont}.} }{\addtocounter{ejemplo_cont}{1}}
%%%%%%%%%%%%%%%%%%%%%%%%%%%%%%%%%%%%%%%%%%%%%%%%%%%%%%%%%%%%%%%%%%%%%%%%%%%%%%%%%%%%%%%%%%%%%%%%%%%%%%%%%%%
%%%%%%%%%%Para escibir en clase articulo o similar









\title{Ecuaciones Diferenciales}
%\subtitle{Una Aproximación Teórica y Computacinal }
\author{Fernando Mazzone}

%%%%%%%%%%%%%%%%%%%%%%%%%%%%%%%%%%%%%%%%%%%%%%%%%%%%%%%%%%%%%%%%%%%%%%%%%%%%%%%%%%%%%%


\begin{document}
\fontsize{9pt}{10pt}\selectfont

\pagestyle{fancy}
 
 
 \maketitle
 \tableofcontents
%   

\chapter*{Prólogo}
\addcontentsline{toc}{chapter}{Prólogo}



\setlength{\unitlength}{1cm}
%
%\setlength{\extrarowheight}{5mm}
%





 \textbf{CARRERA:}  Lic en Matemática.

 \textbf{PLAN DE ESTUDIOS:} 2008 versión 1

 \textbf{ASIGNATURA:}  Ecuaciones Diferenciales\quad\textbf{CÓDIGO:} 1913

\textbf{DOCENTES RESPONSABLES:} Fernando Mazzone

\textbf{EQUIPO DOCENTE:} Fernando Mazzone

\textbf{AÑO ACADÉMICO:} 2017

\textbf{REGIMEN DE LA ASIGNATURA:} Cuatrimestral

\textbf{RÉGIMEN DE CORRELATIVIDADES:}

\begin{table}[h]
\begin{tabular}{|l|l|}\hline
 Aprobada & Regular\\\hline
 &
 Álgebra Lineal Aplicada (2261)\\\hline
 &
Topología (1917)\\ \hline
\end{tabular}
 \end{table}






\textbf{CARGA HORARIA TOTAL:}  80 hs.

\hspace{1cm}\textbf{Teóricas:} 40 hs.,      \textbf{Prácticas:}  40 hs..

\textbf{CARÁCTER DE LA ASIGNATURA:} Obligatoria
\renewcommand{\theenumi}{\Alph{enumi}}
\renewcommand{\theenumii}{\theenumi .\arabic{enumii}}
\begin{enumerate}
 \item \textbf{CONTEXTUALIZACIÓN DE LA ASIGNATURA}

 	Primer cuatrimestre de cuarto año

 \item \textbf{OBJETIVOS PROPUESTOS}
      %\begin{description}

          \begin{itemize}
	    \item  Presentar la teoría de las ecuaciones diferenciales desde un perspectiva rigurosa.

	    \item  Poner en evidencia la retroalimentación entre teoría matemática y modelos físicos. En este sentido se desarrollan aplicaciones a la caída de cuerpos a lo largo de guías (y su relación con la óptica), problema de la braquistócrona, vibraciones de sistemas mecánicos, membranas,  potencial sobre una esfera, movimiento planetario, etc.

	    \item  Integrar la asignatura a otras asignaturas del plan de estudios de la Lic. en Matemática. En este sentido se desarrolla la teoría de Lie de solución de ecuaciones por medios de grupos continuos. No es costumbre que esta teoría se desarrolle en cursos introductorios.

\item  Incorporar el uso de sistemas algebraicos computacionales en la práctica del  alumno. Se utilizaran recursos de código abierto que derivan del lenguaje Python, en particular SymPy y SAGE.


       \end{itemize}


 % \end{description}





 \item\textbf{CONTENIDOS BÁSICOS DEL PROGRAMA A DESARROLLAR}
    Ecuaciones de primer orden, métodos de solución. Métodos de Lie. Teorema de existencia y unicidad. Ecuaciones lineales de orden superior. Osciladores armónicos.   Método de desarrollo en serie. Método de Frobenius.     Sistemas de Ecuaciones.

 \item\textbf{ FUNDAMENTACIÓN DE LOS CONTENIDOS}

	La gran parte del curso versa sobre temas que ya son estándar en las ecuaciones diferenciales y se consideran básicos en el desarrollo de esta área. No creemos necesario abundar en fundamentos sobre la incorporación de ellos. Si merece fundamentarse aquellos que no son del todo habituales.

	Con frecuencia las leyes de la física o modelos matemáticos de sistemas biológicos, sociales, económicos, etc, se expresan por medio de ecuaciones y particularmente con ecuaciones diferenciales. Con igual frecuencia en nuestras aulas la enseñanza de esta, como de otras  ramas de la matemática, omite la consideración de las relaciones entre los conceptos matemáticos y otras ciencias. A lo sumo se suele presentar alguna aplicación de la teoría como medio de justificar la relevancia de ella. Se hace extensivo al quehacer pedagógico  el postulado formalista que la matemática se valida por si misma y es independiente de cualquier  realidad ajena a ella.

	Nuestro punto de vista es que las relaciones de la teoría matemática con su entorno constituye un ingrediente insoslayable en la enseñanza de la teoría.  Fundamentamos este punto de vista, en que es frecuente que el sistema físico  que es modelizado por cierta teoría, ilumine el entendiemiento de la teoría misma. Por ejemplo, el principio de máximo, que afirma que una solución de la ecuación $\triangle u=0$ en un abierto y acotado $\Omega$ alcanza su máximo en la frontera de $\Omega$, no es evidente de la ecuación en si misma, pero si lo es en gran medida de una de las situaciones físicas que modeliza:  la temperatura en estado estacionario de un cuerpo sobre el cual el calor fluye por difusión. No puede desaprovecharse el recurso de pensar una solución de la ecuación aludida en estos dos sentidos.

	También ocurre el camino inverso, esto es que el desarrollo de la teoría matemática ilumine el entendimiento del sistema físico. Al fin y al cabo, ese es el propósito primario de la modelización matemática. Por ejemplo, para quien escribe no resulta físicamente evidente que un sistema de resortes acoplados tenga esencialmente sólo dos modos normales de vibración, cosa que se demuestra en el actual curso. Por estas consideraciones, entre otras, también estamos convencidos que la demostración matemática rigurosa también constituye un ingrediente insoslayable para el entendimiento de la teoría.

	Se incorpora activamente el uso de sistemas algebraicos computacionales SAC.  Una causa es contar con asistencia para el desarrollo de cálculos que son engorrosos. Pero la causa fundamental de la introducción de SAC es que ponen al alumno en la situación de hacer un programa que implemente procedimientos de la teoría. Esto suele ser una tarea no trivial para el recién iniciado y obliga a desarrollar aptitudes de programación, pero más importante, obliga a repensar la teoría matemática para adaptarla al nuevo contexto.

	En el mismo orden de ideas, esto es poner los conocimientos de la asignatura en diversos contextos,  se buscó una integración con otras materias del plan de  estudios. Por supuesto que hay algunas de ellas que son absolutamente necesarias  para desarrollar la teoría de las ecuaciones diferenciales, pero no es costumbre en los cursos elementales sobre ecuaciones diferenciales recurrir a algunas ramas, por ejemplo teoría de grupos. Sin embargo la teoría de grupos tiene cosas importantes para decir sobre las ecuaciones. Se buscó establecer estas vinculaciones menos tradicionales, por ejemplo se desarrollo una unidad sobre la utilización de grupos de Lie de simetrías para resolver EDO.  Utilizamos un concepto particular de grupo de Lie, para evitar las complicaciones técnicas en la definición de este concepto en general.  La consideración de simetrías es una técnica matemática básica y las simetrías están indisolublemente ligadas al concepto de grupo.


 \item\textbf{  ACTIVIDADES A DESARROLLAR}

		  \textbf{ CLASES TEÓRICAS:} Presencial, 4 horas semanales. La metodología que se desarrollará es la exposición por parte del docente de los fundamentos teóricos de los contenidos impartidos. Se incentivará la participación de los alumnos durante la clase, requiriendo que ellos aporten, por ejemplo, demostraciones de determinados hechos o, en general, soluciones a determinadas situaciones problemáticas que plantea el desarrollo teórico de la materia.

		  \textbf{ CLASES PRÁCTICAS:} Presencial 4 horas semanales. Se espera que los alumnos trabajen sobre los ejercicios de la práctica en forma independiente fuera de los horarios de la asignatura. Posteriormente estos ejercicios se discutirán durante la clase,  el profesor tratará de limitar su participación de modo tal de favorecer que los alumnos autogestionen su aprendizaje.


		  Internet:  Se utilizaron diversos recursos de internet, que estan compendiados en una \href{http://fdmazzone.github.io/Ecuaciones_Diferenciales/}{página de la asignatura} y en un \href{https://github.com/fdmazzone/Ecuaciones_Diferenciales}{repositorio de Git Hub}.  En la red  hay excelentes recursos, videos, páginas web, wikis  y, en general, distintos materiales multimedia especialmente útiles para visualizar algunos conceptos, métodos, etc.

 \item\textbf{  NÓMINA DE TRABAJOS PRÁCTICOS}
		Hay un trabajo práctico por cada unidad de la materia.

\item\textbf{   HORARIOS DE CLASES:} martes y jueves de 14 a 18.


 \textbf{  Horario de clases de consultas:}
	Se convendrá con los alumnos, durante el desarrollo de la materia,
  los horarios de consultas.


\item\textbf{   MODALIDAD DE EVALUACIÓN:}

\textbf{Evaluaciones Parciales:}  Se le presentará al alumno una serie de problemas que deberá resolver.

\textbf{Evaluación Final:} Será oral, el alumno deberá desarrollar los ejes conceptuales y fundamentos teóricos de la materia.

\textbf{  Condiciones de regularidad:}
Aprobar los exámenes parciales o sus respectivos recuperatorios.

\textbf{  Condiciones de promoción:} no se prevé

\item\textbf{ CONTENIDOS:}

\emph{
El asterísco al lado de las referencias citadas más abajo indica la bibliografía considerada principal. Las citas sin el asterísco corresponden a bibliografía suplementaria.}

\begin{description}
 \item[Unidad 0. Python, SymPy, SAGE.] Panorama de instalación, distribuciones y recursos online. Tipos de datos en Python, programación elemental.

\item[Unidad 1. Ecuaciones de Primer Orden.]  Noción de ecuación diferencial y clasificación. Ecuaciones diferenciales ordinarias (EDO).   Problemas de valores iniciales. Familia de curvas y la famila ortogonal. Método de separación de variables. Aplicaciones: dinámica de mezclas, cuerpos en caída a lo largo de guías, curvas braquistócrona y tautócrona. Ecuaciones homogéneas. Ecuaciones exactas. Factores integrantes. Ecuaciones lineales de primer orden. Métodos de reducción de orden. Curvas de persecución. Velocidad de escape. Problema del resorte. Cambios de variables. Usando SymPy para cambiar variables.  Grupos de Lie uniparamétricos. Grupos de simetrías. Generadores infinitesimales. Variables canónicas. Solución de EDO por medio de sus grupos de Lie de simetrías. \cite{simmons_esp, WilliamE.Boyce496,PeterE.Hydon455,S.V.DuzhinB.D.Chebotarevsky465}

\item[Unidad 2. Teorema de Existencia y Unicidad.] Presentación filosófica: determinismo científico. Funciones Lipschitzianas. Teorema de punto fijo de Banach. Método iterativo de Picard. Teorema de existencia y unicidad para sistemas de EDO de primer orden.\cite{simmons_esp, WilliamE.Boyce496, VladimirI.Arnold544, JorgeSotomayor513}

\item[Unidad 3. Ecuaciones Lineales de Segundo Orden.] Ecuaciones lineales. Reducción de orden. Ecuaciones homogéneas a coeficientes constantes. El problema no homogéneo. Independencia lineal. Bases de soluciones. Polinomio característico. Ecuaciones no homogéneas. Coeficientes indeterminados y variación de los parámetros. Vibraciones mecánicas. Solución del problema Kepleriano de los dos cuerpos. Osciladores armónicos acoplados. \cite{simmons_esp, WilliamE.Boyce496}

\item[Unidad 4. Métodos cualitativos.] Teoremas de separación y de comparación de Sturn. Aplicaciones, ceros de las funciones de Bessel.\cite{simmons_esp,JorgeSotomayor513}

\item[Unidad 5. Desarrollo en serie de potencias.] Repaso de series de potencias. Método de coeficientes indeterminados.  Resolución de problemas de desarrollo en serie con  SymPy.   Ecuaciones lineales de segundo orden: puntos regulares. Puntos singulares regulares. Series de Frobenius. Teoremas fundamentales.\cite{simmons_esp, WilliamE.Boyce496}

\item[Unidad 6.  Sistemas lineales.]  Base de soluciones. Matriz fundamental. Sistemas lineales a coeficientes constantes. Solución del problema homogéneo con formas de Jordan. Problema no homogéneo. Sistemas no-lineales. \cite{ WilliamE.Boyce496,JorgeSotomayor513}

\end{description}

\item\textbf{CÓDIGO QR}
Algunos bolques de código Python que expondremos en el texto vienen acompañados de un código QR
como note al margen. La finalidad es que el alumno pueda leer este código con algún dispositivo
movil y ejecutar el código dentro del dispositivo


\item\textbf{MARGENES}
El significado de las imágenes en los margenes es el siguiente:


\begin{center}
 \begin{tabular}{|l|l|}\hline
{\fontsize{26}{26}\selectfont \faLink} &  Enlace de Internet\\ \hline
{\fontsize{26}{26}\selectfont \faBolt}  & Prestar atención\\ \hline
{\fontsize{26}{26}\selectfont \faBook}  & Lectura adicional\\ \hline
 \end{tabular} 
\end{center}



\end{enumerate}
















%
%
  \bibliographystyle{apalike-url}
  \bibliography{diferenciales_ecuaciones,diferenciales_ecuaciones_sim}
 
 



\documentclass{article}
%\documentclass[hyperref={colorlinks=true}]{beamer}
%\documentclass[handout,hyperref={colorlinks=true}]{beamer}


%%%%%%%%%%%%%%%%%%%%%%%%%%%%%%Paquetes%%%%%%%%%%%%%%%%%%%%%%%%%%%%%%%%%%%%%%%%%%%%%%%5
%%%%%%%%%%%%%%%%%%%%%%%%%%%%%%%%%%%%%%%%%%%%%%%%%%%%%%%%%%%%%%%%%%%%%%%%%%%%%%%%%%%%%
\usepackage{empheq}
\usepackage[spanish]{babel}
\usepackage[utf8x]{inputenc}
\usepackage{times}
%\usepackage[T1]{fontenc}
\usepackage{amssymb,amsmath}
\usepackage{enumerate}
\usepackage{verbatim}
\usepackage{ esint }
%\usepackage{pst-all}
%\usepackage{pstricks-add}
\usepackage{array}
%\usepackage[T1]{fontenc}
\usepackage{animate}
%\usepackage{media9}
\usepackage{xparse}
\usepackage{listings}
\usepackage{ wasysym }
\usepackage{sagetex}
\usepackage{yfonts,mathrsfs,eufrak}
\usepackage{hyperref}
\usepackage{color}
\usepackage{url}


\definecolor{mygreen}{rgb}{0,0.6,0}
\definecolor{mygray}{rgb}{0.5,0.5,0.5}
\definecolor{mymauve}{rgb}{0.58,0,0.82}

\lstset{ %
  backgroundcolor=\color{white},   % choose the background color; you must add \usepackage{color} or \usepackage{xcolor}
  basicstyle=\footnotesize,        % the size of the fonts that are used for the code
  breakatwhitespace=false,         % sets if automatic breaks should only happen at whitespace
  breaklines=true,                 % sets automatic line breaking
  captionpos=b,                    % sets the caption-position to bottom
  commentstyle=\color{mygreen},    % comment style
  deletekeywords={...},            % if you want to delete keywords from the given language
  escapeinside={\%*}{*)},          % if you want to add LaTeX within your code
  extendedchars=true,              % lets you use non-ASCII characters; for 8-bits encodings only, does not work with UTF-8
  frame=single,	                   % adds a frame around the code
  keepspaces=true,                 % keeps spaces in text, useful for keeping indentation of code (possibly needs columns=flexible)
  keywordstyle=\color{blue},       % keyword style
  language=Python,                 % the language of the code
  otherkeywords={*,...},           % if you want to add more keywords to the set
  numbers=left,                    % where to put the line-numbers; possible values are (none, left, right)
  numbersep=5pt,                   % how far the line-numbers are from the code
  numberstyle=\tiny\color{mygray}, % the style that is used for the line-numbers
  rulecolor=\color{black},         % if not set, the frame-color may be changed on line-breaks within not-black text (e.g. comments (green here))
  showspaces=false,                % show spaces everywhere adding particular underscores; it overrides 'showstringspaces'
  showstringspaces=false,          % underline spaces within strings only
  showtabs=false,                  % show tabs within strings adding particular underscores
  stepnumber=2,                    % the step between two line-numbers. If it's 1, each line will be numbered
  stringstyle=\color{mymauve},     % string literal style
  tabsize=2,	                   % sets default tabsize to 2 spaces
  title=\lstname                   % show the filename of files included with \lstinputlisting; also try caption instead of title
}


%%%%%%%%%%%%%%%%%%%%%%%%%%Nuevos comandos entornos%%%%%%%%%%%%%%%%%%%%%%%%%%%%%%%%
%%%%%%%%%%%%%%%%%%%%%%%%%%%%%%%%%%%%%%%%%%%%%%%%%%%%%%%%%%%%%%%%%%%%%%%%
\newenvironment{demo}{\noindent\emph{Dem.}}{$\square$ \newline\vspace{5pt}}
\newenvironment{block}[1]{\textbf{#1}}{}
\newcommand{\com}{\mathbb{C}}
\newcommand{\dis}{\mathbb{D}}
\newcommand{\rr}{\mathbb{R}}
\newcommand{\oo}{\mathcal{O}}
\renewcommand{\emph}[1]{\textcolor[rgb]{1,0,0}{#1}}
\newcommand{\der}[2]{\frac{\partial #1}{\partial #2}}
\renewcommand{\v}[1]{\overrightarrow{#1}}
\renewcommand{\epsilon}{\varepsilon}
%\newcommand{\defverbatim}{\def{#1}}
\newcommand{\nl}{ }
\renewenvironment{frame}[1]{}{}
\newcommand{\qed}{$\square$}
\DeclareMathOperator{\atan2}{atan2}
\DeclareMathOperator{\sen}{sen}


% %%%%%%%%%%Definimos una caja con color
% \newlength\mytemplen
% \newsavebox\mytempbox
% \makeatletter
% \newcommand\mybluebox{%
% \@ifnextchar[%]
% {\@mybluebox}%
% {\@mybluebox[0pt]}}
% \def\@mybluebox[#1]{%
% \@ifnextchar[%]
% {\@@mybluebox[#1]}%
% {\@@mybluebox[#1][0pt]}}
% \def\@@mybluebox[#1][#2]#3{
% \sbox\mytempbox{#3}%
% \mytemplen\ht\mytempbox
% \advance\mytemplen #1\relax
% \ht\mytempbox\mytemplen
% \mytemplen\dp\mytempbox
% \advance\mytemplen #2\relax
% \dp\mytempbox\mytemplen
% \colorbox{myblue}{\hspace{1em}\usebox{\mytempbox}\hspace{1em}}}
% \makeatother
% \DeclareDocumentCommand\boxedeq{ m g }{%
% {\begin{empheq}[box={\mybluebox[2pt][2pt]}]{equation}% #1%
% \IfNoValueF {#2} {\label{#2}}%
% #1
% \end{empheq}
% }%
% }


%%%%%%%%%%%%%%%%%%
\newtheorem{teorema}{Teorema}[section]
\newtheorem{lema}[teorema]{Lema}
\newtheorem{corolario}[teorema]{Corolario}
\newtheorem{proposicion}[teorema]{Proposici\'on}
\newtheorem{definicion}[teorema]{Definici\'on}
%%%%%%%%%%%%%%%%%%%%%%%%%%%%%%%%%%%%%%%%%%%%%%%%%%%%%%%%%%%%%%%%%%%%%%%%%%%%%%%%%%%%%%%%%%%%%%%%%%%%%%%%%%%
%%%%%%%%%%Para escibir en clase articulo o similar






\title{Breve introducción a Python y SymPy}
\author{Fernando Mazzone}

%%%%%%%%%%%%%%%%%%%%%%%%%%%%%%%%%%%%%%%%%%%%%%%%%%%%%%%%%%%%%%%%%%%%%%%%%%%%%%%%%%%%%%


\begin{document}

  \maketitle




\section[Descripción]{Descripción}

\href{https://www.python.org/}{Python} es un lenguaje de programación interpretado, abierto, facil de aprender, potente y portátil. Es utilizado en proyectos de todo tipo, no sólo aplicaciones científicas. \marginpar{\includegraphics[scale=.08]{imagenes/python-logo.png} }




\href{http://www.scipy.org/}{SciPy}, Python científico, es un conjunto de módulos de python para distintos tipos de cálculos. Está integrado por los módulos, SymPy (para cálculos simbólicos), numpy (cálculos numéricos), matplotlib (gráficos) entre otros.  En este curso sólo usaremos SymPy.\marginpar{\includegraphics[scale=.6]{imagenes/scipy_logo.png} }




\href{http://www.sympy.org/}{SymPy}
es una biblioteca de Python para matemática simbólica. Su objetivo es convertirse en un sistema de álgebra computacional (SAC) completo, manteniendo el código lo más simple posible para que sea comprensible y fácilmente extensible. SymPy está escrito enteramente en Python y no requiere de ninguna biblioteca externa.\marginpar{\includegraphics[scale=.25]{imagenes/sympy_logo.png}}

\href{http://www.sagemath.org/}{SageMath}  es un sistema de software de matemáticas, libre, de código abierto bajo la licencia GPL. Es construído sobre  muchos paquetes de código abierto existentes: NumPy, SciPy, matplotlib, SymPy, Maxima, GAP, FLINT, R y muchos más. Se acceda a su poder combinado a través de un lenguaje común, basado en Python.\marginpar{\includegraphics[scale=.13]{imagenes/sage_logo.png} }




\section{Local y online} 

Se pueden usar todos los recursos anteriores de dos formas
\begin{enumerate}
\item Instalando el software necesario en una computadora. Nos referiremos a este modo como de acceso local.

\item A traves de trasacciones en línea que permiten usar una computadora remota que ejecuta las instrucciones y programas que se tipean en una página web con la que se interactúa usualmente por medio de un navegador.  Hay varios sitios que ofrecen este servicio. Sugerimos la \href{https://cloud.sagemath.com/}{SageMathCloud}. El usuario debe registrase.
\end{enumerate}



\section{Instalación local}

Hay mucho software dedicado a gestionar el uso de python, recomendamos las siguientes por la sencillez de la instalación.  

\subsection{Windows} La distribución  \href{https://code.google.com/p/pythonxy/}{python(x,y)}  instala el interprete de python y todos los módulos de scipy. Además el entorno de desarrollo integrado (IDE) spyder.

Lamentablemente no es posible instalar SageMath en windows, sólo se instala bajo linux.

\subsection{linux} Aquí todo es más sencillo, el interprete de python suele venir con la distribución del SO y se puede instalar los módulos, SymPy, NumPy, etc, recurriendo al administrador de paquetes o tipeando la sentencia adecuada en la línea de comandos.  Para instalar SAGE se lo descarga de la página oficial y se descomprime.

\subsection{Android} \href{http://qpython.com/}{Qpython} es una aplicación que permite ejecutar código python y una versión básica de sympy desde tablets y smartphones. Se descarga desde la plataforma \href{https://play.google.com/store/apps/details?id=com.hipipal.qpyplus}{google play}.

\subsection{Otros recursos de utilidad:} Tanto para linux o windows \href{http://ipython.org}{ipython}, \href{http://continuum.io/downloads}{Anaconda}, \href{http://www.gnu.org/software/emacs/}{emacs}.


\section{Forma de trabajo: por medio de scripts e interactiva}


Se puede trabajar de dos formas

\begin{enumerate}
\item Interactivamente, ingresando sentencias, de a una por vez, en la línea de comandos y obteniendo respuestas.

\item Haciendo un script (programa) donde se guardan todas las sentencias que se desea ejecutar. Posteriormente este script se puede ejecutar, ya sea desde la línea de comandos o en desde un IDE (spyder) oprimiendo un botón de ejecución.

\end{enumerate}





\section{Características del Lenguaje}

Seguiremos en esta exposición a \cite{wiki_python} de manera cercana. Las principales características del lenguaje son:

\begin{itemize}
\item Interpretado. Es necesario un conjutno de programas, el interprete, que entienda el código python y ejecute las acciones contenidas en él.
\item implementa  tipos dinámicos
\item  Multiparadigma, ya que soporta orientación a objetos, programación imperativa y, en menor medida, programación funcional.
\item Multiplataforma.

\item Es comprendido  con facilidad. Usa  palabras donde otros lenguajes utilizarían símbolos. Por ejemplo, los operadores lógicos \verb~!, || y \&\&~ en Python se escriben not, or y and, respectivamente.


\item  El contenido de los bloques de código (bucles, funciones, clases, etc.) es delimitado mediante espacios o tabuladores.

\item Empieza a contar desde cero (elementos en listas, vectores, etc).



\end{itemize}




\section{Elementos del Lenguaje}

\subsection{Comentarios}

Hay dos formas de producir comentarios, texto que el interprete  no ejecuta y que sirve para entender un programa.

La primera, para comentarios largos es utilizanda la notación \linebreak\verb~''' comentario '''~ .


 La segunda notación utiliza el símbolo \verb~#~, no necesita símbolo de finalización pues se extienden hasta el final de la línea.

 \begin{lstlisting}
'''
Comentario  largo en un script de Python
'''
print "Hola mundo" # Comentario corto
\end{lstlisting}

El intérprete no tiene en cuenta los comentarios, lo cual es útil si deseamos poner información adicional en nuestro código como, por ejemplo, una explicación sobre el comportamiento de una sección del programa.






\begin{lstlisting}
x = 1
x = "texto" # Esto es posible porque los tipos son asignados \
dinamicamente
\end{lstlisting}



\subsection{Variables}
Las variables se definen de forma dinámica, lo que significa que no se tiene que especificar cuál es su tipo de antemano y que una variable puede tomar distintos valores en distintos momentos de un programa, incluso puede tomar
 un tipo diferente al que tenía previamente. Se usa el símbolo = para asignar valores a variables. Es importante distinguir este = (de asignación) con el igual que es utilizado para definir igualdades en sympy, para ecuaciones por ejemplo.




\subsection{Tipo de datos}

\includegraphics[scale=.4]{imagenes/tipo_datos.jpg}

Se clasifican en:
\begin{description}
 \item[Mutable] si su contenido puede cambiarse.
 \item[Inmutable] si su contenido no puede cambiarse.
\end{description}

Se usa el comando \verb~\type~ para averiguar que tipo de dato contiene una variable

 \begin{lstlisting}
>>> x=1
>>> type(x)
<type 'int'>
>>> x='Ecuaciones'
>>> type(x)
<type 'str'>
\end{lstlisting}





\subsection{Listas y tuples}


\begin{itemize}

\item Es una estructura de dato, que contiene, como su nombre lo indica, listas de otros datos en cierto orden. Listas y tuplas son muy similares.

\item Para declarar una lista se usan los corchetes [], en cambio, para declarar una tupla se usan los paréntesis (). En ambos casos los elementos se separan por comas, y en el caso de las tuplas es necesario que tengan como mínimo una coma.

\item    Tanto las listas como las tuplas pueden contener elementos de diferentes tipos. No obstante las listas suelen usarse para elementos del mismo tipo en cantidad variable mientras que las tuplas se reservan para elementos distintos en cantidad fija.
    
\item Para acceder a los elementos de una lista o tupla se utiliza un índice entero (empezando por "0", no por "1"). Se pueden utilizar índices negativos para acceder elementos a partir del final.


\item Las listas se caracterizan por ser mutables, mientras que las tuplas son inmutables.

\end{itemize}






\begin{lstlisting}
>>> lista = ["abc", 42, 3.1415]
>>> lista[0] # Acceder a un elemento por su indice
'abc'
>>> lista[-1] # Acceder a un elemento usando un indice negativo
3.1415
>>> lista.append(True) # Agregar un elemento al final de la lista
>>> lista
['abc', 42, 3.1415, True]
>>> del lista[3] # Borra un elemento de la lista usando un indice
>>> lista[0] = "xyz" # Re-asignar el valor del primer elemento
>>> lista[0:2] # elementos del indice "0" al "2" (sin incluir  ultimo)
['xyz', 42]
>>> lista_anidada = [lista, [True, 42L]] # Es posible anidar listas
>>> lista_anidada
[['xyz', 42, 3.1415], [True, 42L]]
>>> lista_anidada[1][0] # Acceder a un elemento de una lista dentro de otra lista
True
\end{lstlisting}

\begin{lstlisting}
>>> tupla = ("abc", 42, 3.1415)
>>> tupla[0] # Acceder a un elemento por su indice
'abc'
>>> del tupla[0] # No es posible borrar ni agregar
( Excepcion )
>>> tupla[0] = "xyz" # Tampoco es posible re-asignar
( Excepcion )
>>> tupla[0:2] # elementos del indice "0" al "2" sin incluir
('abc', 42)
>>> tupla_anidada = (tupla, (True, 3.1415)) # es posible anidar
>>> 1, 2, 3, "abc" # Esto tambien es una tupla
(1, 2, 3, 'abc')
>>> (1) #  no es una tupla, ya que no posee al menos una coma
1
>>> (1,) # si es una tupla
(1,)
>>> (1, 2) # Con mas de un elemento no es necesaria la coma final
(1, 2)
>>> (1, 2,) # Aunque agregarla no modifica el resultado
(1, 2)
\end{lstlisting}

\subsection{Diccionarios}
\begin{itemize}
\item    Para declarar un diccionario se usan las llaves \verb~\{\}~. Contienen elementos separados por comas, donde cada elemento está formado por un par clave:valor (el símbolo : separa la clave de su valor correspondiente).
 \item   Los diccionarios son mutables, es decir, se puede cambiar el contenido de un valor en tiempo de ejecución.
\item    En cambio, las claves de un diccionario deben ser inmutables. Esto quiere decir, por ejemplo, que no podremos usar ni listas ni diccionarios como claves.
\item    El valor asociado a una clave puede ser de cualquier tipo de dato, incluso un diccionario.

\end{itemize}






\begin{lstlisting}
>>> dicci = {"cadena": "abc", "numero": 42, "lista": [True, 42L]}
>>> dicci["cadena"] # Usando una clave, se accede a su valor
'abc'
>>> dicci["lista"][0]
True
>>> dicci["cadena"] = "xyz" # Re-asignar el valor de una clave
>>> dicci["cadena"]
'xyz'
>>> dicci["decimal"] = 3.1415927 # nuevo elemento clave:valor
>>> dicci["decimal"]
3.1415927
>>> dicci_mixto = {"tupla": (True, 3.1415), "diccionario": dicci}
>>> dicci_mixto["diccionario"]["lista"][1]
42L
>>> dicci = {("abc",): 42} # tupla puede ser clave pues es inmutable
>>> dicci = {["abc"]: 42} # No es posible que una clave sea una lista
( Excepcion )
\end{lstlisting}



\subsection{Listas por comprensión}
Una lista por comprensión es una expresión compacta para definir listas. Al igual que el operador lambda, aparece en lenguajes funcionales. Ejemplos:


\begin{lstlisting}
>>> range(5) #  "range" devuelve una lista, empezando en 0 \
y terminando con el numero indicado menos uno
[0, 1, 2, 3, 4]
>>> [i*i for i in range(5)]
[0, 1, 4, 9, 16]
>>> lista = [(i, i + 2) for i in range(5)]
>>> lista
[(0, 2), (1, 3), (2, 4), (3, 5), (4, 6)]
\end{lstlisting}



\subsection{Funciones}
\begin{itemize}

  \item  Las funciones se definen con la palabra clave \verb~def~, seguida del nombre de la función y sus parámetros. Otra forma de escribir funciones, aunque menos utilizada, es con la palabra clave \verb~lambda~ (que aparece en lenguajes funcionales como Lisp). Generalemente esta forma es apropiada para funciones que es posible definir en una sola línea.

  \item  El valor devuelto en las funciones con \verb~def~ será el dado con la instrucción \verb~return~.
  \end{itemize}


\begin{lstlisting}
>>> def suma(x, y = 2): # el argumento y tiene un valor por defecto
...     return x + y # Retornar la suma
...
>>> suma(4) # La variable "y" no se modifica, siendo su valor: 2
6
>>> suma(4, 10) # La variable "y" si se modifica
14
\end{lstlisting}


\begin{lstlisting}
>>> suma = lambda x, y = 2: x + y
>>> suma(4) # La variable "y" no se modifica
6
>>> suma(4, 10) # La variable "y" si se modifica
14
\end{lstlisting}

\subsection{Condicionales}
 Una sentencia condicional (\verb~if condicion~) ejecuta su bloque de código interno sólo si \verb~condicion~ tiene el valor bolleano \verb~True~.  Condiciones adicionales, si las hay, se introducen usando \verb~elif~ seguida de la condición y su bloque de código. Todas las condiciones se evalúan secuencialmente hasta encontrar la primera que sea verdadera, y su bloque de código asociado es el único que se ejecuta. Opcionalmente, puede haber un bloque final (la palabra clave \verb~else~ seguida de un bloque de código) que se ejecuta sólo cuando todas las condiciones fueron falsas.



\begin{lstlisting}
>>> verdadero = True
>>> if verdadero: # No es necesario poner "verdadero == True"
...     print "Verdadero"
... else:
...     print "Falso"
...
Verdadero
>>> lenguaje = "Python"
>>> if lenguaje == "C": 
...     print "Lenguaje de programacion: C"
... elif lenguaje == "Python": # Se pueden agregar "elif" como se quiera
...     print "Lenguaje de programacion: Python"
... else: 
...     print "Lenguaje de programacion: indefinido"
...
Lenguaje de programacion: Python
>>> if verdadero and lenguaje == "Python": 
...     print "Verdadero y Lenguaje de programacion: Python"
...
Verdadero y Lenguaje de programacion: Python
\end{lstlisting}






\subsection{Bucles}
El bucle \verb~for~ es similar a  otros lenguajes. Recorre un objeto iterable, esto es  una lista o una tupla, y por cada elemento del iterable ejecuta el bloque de código interno. Se define con la palabra clave \verb~for~ seguida de un nombre de variable, seguido de \verb~in,~ seguido del iterable, y finalmente el bloque de código interno. En cada iteración, el elemento siguiente del iterable se asigna al nombre de variable especificado:

\begin{lstlisting}
>>> lista = ["a", "b", "c"]
>>> for i in lista: # Iteramos sobre una lista, que es iterable
...     print i
...
a
b
c
>>> cadena = "abcdef"
>>> for i in cadena: # Iteramos sobre una cadena, que es iterable
...     print i, # una coma al final evita un salto de linea
...
a b c d e f
\end{lstlisting}



%El bucle \verb~while~ evalúa una condición y, si es verdadera, ejecuta el bloque de código interno. Continúa evaluando y ejecutando mientras la condición sea verdadera. Se define con la palabra clave }verb~while~ seguida de la condición, y a continuación el bloque de código interno:
\begin{lstlisting}
>>> numero = 0
>>> while numero < 10:
...     print numero
...     numero += 1,  #un buen programador modificara las variables de control al finalizar el ciclo while
...
0 1 2 3 4 5 6 7 8 9
\end{lstlisting}



 \bibliographystyle{plain}
 \bibliography{biblio}


\end{document}






%


\chapter{Generalidades y ecuaciones de primer orden}

\section{¿Que son las ecuaciones diferenciales?}


\begin{definicion}[Ecuación diferencial, definición informal]{}
 Es una o varias relaciones entre una o varias variables dependientes y sus tasas de cambio respecto a ciertas variables independientes.
 \end{definicion}

 El problema básico asociado
 a las ecuaciones diferenciales es hallar las variables dependientes que las resuelven.

 Las ecuaciones diferenciales son usadas muy a menudo en matemática aplicada, puesto que muchas
 leyes (de la física por ejemplo) se expresan atraves de este tipo de ecuaciones.








 \begin{ejemplo}[\href{http://es.wikipedia.org/wiki/Caída_libre}{Caída libre}]{} Modelizar matemáticamente el movimiento de un cuerpo de masa $m$ en las proximidades de la superficie
terrestre, asumiendo que su movimiento es
sobre la vertical y que las fuerzas que sobre él actúan son la gravedad y el rozamiento con el aire.
\end{ejemplo}

\begin{wrapfigure}[20]{l}{6cm}
  \begin{center}
  \setlength{\unitlength}{1cm}
    \begin{picture}(5, 5)(0,1)
      \put(1,2){\vector(0,1){3.5}}
      \put(1,2){\vector(0,-1){1}}
      \put(0,2){\vector(1,0){5}}
      \multiput(0,2)(.25,0){20}{\line(-1,-1){0.5}}
      \put(2,4){\circle*{.5}}
      \put(2,4){\vector(0,-1){1.5}}
      \put(2.3,3){$F_{\hbox{grav}}=mg\approx 9.8 m/s^2$}
      \put(2,4){\vector(0,1){1.5}}
      \put(2.3,5){$F_{\hbox{roz}}=-cv$}
      \multiput(1,4)(.1,0){10}{\line(1,0){0.05}}
      \put(0,4){$x(t)$}
      \put(0,1){$x>0$}
      \put(0,5){$x<0$}

    \end{picture}\caption{Caída libre}\label{fig:caída}
  \end{center}
\end{wrapfigure}
\noindent $x(t)=$ posición a lo largo de la vertical, relativo a un eje de coordenadas\newline
$v(t)=\frac{dx}{dt}=$ velocidad\newline
$F_{grav}=$ fuerza debida a la gravedad $=mg$ donde $g=9.8m/s^2$
$F_{roz}=$ fuerza de rozamiento, proporcional a la velocidad y de sentido
contrario $=-cv$, $c>0$.




 Usamos la \href{http://es.wikipedia.org/wiki/Leyes_de_Newton}{Segunda Ley de Newton}, esto es la suma de las fuerzas totales que actuan sobre un cuerpo de masa $m$
es igual al producto de la masa $m$ y la aceleración $a(t)$.
Recordemos que la aceleración es la tasa de cambio de la velocidad, es decir la derivada 
segunda de la posición.   

  Usando todas las relaciones mencionadas

\[\begin{split} ma(t)&=mv'(t)=F_{\hbox{total}}\\
  &=F_{\hbox{grav}}+F_{\hbox{roz}}\\
  &=mg-cv
\end{split}
\]
vale decir
\begin{equation}\label{caidaLibre1}
 \boxed{ x''(t)+\frac{c}{m} x'(t)=g.}
\end{equation}
\begin{equation}\label{caidaLibre2}
\boxed{ v'(t)+\frac{c}{m} v=g.}
\end{equation}





\section{Algunos conceptos relacionados con ecuaciones diferenciales}

\begin{definicion}[Orden]{} El índice de la mayor derivada interviniente en la ecuación.
 \end{definicion}

 Por ejemplo la ecuación \eqref{caidaLibre1} es de orden 2 y la
ecuación \eqref{caidaLibre2}, si bien está estrechamente relacionada con la anterior, es de orden 1. Como regla casi general, cuanto menor es el orden,  más fáciles de estudiar y/o resolver las ecuaciones son.

  \begin{definicion}[Solución]{} Una función que satisface la relación que indica la ecuación.

   \end{definicion}

   Por ejemplo
\begin{equation}\label{SolGencaidaLibre2} v(t)=\frac{m}{c}g+ke^{-\frac{c}{m}t},\end{equation}
resuelve \eqref{caidaLibre2}, para todo $C\in\rr$. No deberíamos perder tiempo en chequear una cuestión tan sencilla, pero aprovechemos la ocasión para usar  \href{http://www.sympy.org}{SymPy}.





 \begin{pyverbatim}
>>>from sympy import *
>>>m,g,c,k,t=symbols('m,g,c,k,t')
>>>v=m/c*g+k*exp(-c/m*t)
>>>simplify(v.diff(t)+c/m*v)
g
 \end{pyverbatim}
Notar que las líneas que comienzan con el signo del prompt \verb~>>>~ indican entradas por línea de comandos y las que comienzan sin este signo son las respuestas del interprete.

Rara vez utilizaremos las siguientes funcionalidades de sympy, pero es oportuno decir que
 \texttt{SymPy} puede encontrar la solución a una ecuación diferencial

 \begin{pyverbatim}
>>>v=symbols('v',cls=Function)
>>>EqCaida=Eq(v(t).diff(t)+c/m*v(t),g)
>>>Vel=dsolve(EqCaida,v(t))
>>> Vel
v(t) == (g*m + exp(c*(C1 - t/m)))/c
 \end{pyverbatim}

La solución obtenida es la ya conocida. Es instructivo averiguar que tipo de dato tiene la variable \verb~Vel~

 \begin{pyverbatim}
>>> type(Vel)
<class 'sympy.core.relational.Equality'>
 \end{pyverbatim}
A este tipo de cuestiones hay que prestar atención cuando se trabaja con sympy, 
pues existe la tendencia a confudir los conceptos matemáticos con los propios del lenguaje. 
Por ejemplo, matemáticamente una solución es una función. Sin embargo, en este caso, 
cuando le solicitamos una solución a sympy nos entrega un objeto de tipo  
``relación de igualdad''.





\begin{definicion}[Ecuación diferencial ordinaria (EDO)]{} Es una ecuación donde las variables dependientes sólo dependen de una única variable independiente.
\end{definicion}

Las
ecuaciones \eqref{caidaLibre1} y \eqref{caidaLibre2} son ejemplo de ello, la variable independiente es el tiempo.
  \begin{definicion}[Ecuación en derivadas parciales (EDP)]{} Es una ecuación donde las variables dependientes dependen de más de una variable independiente.
   \end{definicion}

Ejemplo de este tipo de ecuación es la ecuación de Laplace

\[\frac{\partial^2 u}{\partial x^2}+\frac{\partial^2 u}{\partial y^2}=0\]

Puede ocurrir también que dispongamos de varias ecuaciones diferenciales
que se deben satisfacer simultaneamente. 
En estos casos, el conjuntos de ecuaciones se suele escribir como una única 
ecuación vectorial. Por este motivo, en algunas ocasiones cuando dispongamos de una
única ecuación diremos que tenemos una \emph{ecuación  escalar}.

\begin{definicion}[Sistema de ecuaciones]{} Es un  conjunto de ecuaciones diferenciales que se deben satisfacer simultaneamente.
 \end{definicion}

En ese caso es  de esperar que tengamos varias incognitas en
nuestro problema. En general una ecuación escalar determina sólo una incognita. De hecho aquí ocurre, a semejanza con ecuaciones algebraicas, que es frecuente necesitar tantas ecuaciones como incognitas.

\begin{ejemplo}[Ecuación del péndulo]{} El sistema de ecuaciones:
\[\left\{ \begin{array}{l l} x'&=y \\y'&=-\sen(x) \end{array}\right.\]
es muy conocido pues modeliza el movimiento de un péndulo.
\end{ejemplo}



Muy a menudo hablaremos de resolver una ecuación, pero es oportuno discutir que queremos significar con esto.

\begin{definicion}[Resolver una ecuación]{} Es expresar la solución como combinaciones algebraicas y composiciones de funciones que consideramos elementales.
\end{definicion}

Esta definición contiene una vaga apelación a ciertas ``funciones elementales''. El universo de funciones que se considera elemental es una cuestión política, no matemática. En  principio, consideraremos elementales a las potencias, exponenciales, logarítmos, trigonométricas y trigonométricas inversas. No obstante esta lista se puede expandir con muchas funciones especiales.  Las operaciones permitidas para combinar estas funciones
también están sujetas a convenciones. Por ejemplo, admitiremos como válida una expresión que contenga una integral, al menos en el caso que no sea claro como resolver esta integral.

 Casi todo este curso trata con la discusión de métodos para resolver ecuaciones.  
 Sin embargo resolver ecuaciones no es quizás el problema principal relacionado 
 con las ecuaciones diferenciales. No importa tanto lograr una expresión 
 formal de la solución, como, por ejemplo, conocer las propiedades que poseen las soluciones. 
 Al fin y al cabo, uno conoce una función a través de sus propiedades.
 





\begin{definicion}[Solución general]{} Usualmente una ecuación presenta infinitas soluciones. Una solución general  es una expresión que representa todas estas soluciones. Es habitual que una solución general contenga parámetros. Cada elección de estos parámetros determina una solución distinta.
 \end{definicion}

 Por ejemplo \eqref{SolGencaidaLibre2} es la solución general de \eqref{caidaLibre2}.
  La afirmación anterior requiere una demostración puesto que sólo hemos mostrado que \eqref{SolGencaidaLibre2} es solución, pero no
que toda solución se expresa con \eqref{SolGencaidaLibre2}. Para demostrar la afirmación, hay que multiplicar ambos miembros de \eqref{caidaLibre2} 
por $e^{\frac{c}{m}t}$ y luego integrar respecto a $t$ 
\[\frac{mg}{c}e^{\frac{c}{m}t}=\int ge^{\frac{c}{m}t}dt=\int v'(t)e^{\frac{c}{m}t}+\frac{c}{m}e^{\frac{c}{m}t} vdt=e^{\frac{c}{m}t}v+C.\]
 Despejando $v$ del primer y último miembro obtenemos \eqref{SolGencaidaLibre2} con $k=-C$.

 




\begin{ejemplo}{} En algunas ocasiones sólo podemos dejar una relación implícita entre las variable dependientes e independientes. Por ejemplo
\begin{equation}\label{solImpl}x=e^y+y+C\quad\text{ para } C\in\rr\end{equation}
es solución general de
\[y'(e^y+1)=1.\]
Vamos a chequear sólo que \eqref{solImpl} es solución, dejando la justificación que toda solución tiene esa forma para más adelante. Derivando \eqref{solImpl}
\[ 1=e^yy'+y'\]
Luego $y'=1/(1+e^{y})$. Reemplazando esta relación  en la ecuación diferencial corroboramos que es solución.
\end{ejemplo}



\section{Definición formal}

\begin{definicion}[Ecuación diferencial]{def:eq} Una ecuación diferencial
ordinaria de orden $n$ es una relación de la forma
\[\boxed{F(x,y(x),y'(x),\ldots,y^{(n)}(x))=0}.\]
donde $F:(a,b)\times \Omega\to\rr$, $\Omega$ es abierto de $\rr^{n+1}$ y $(a,b)$ un intervalo de $\rr$.
  \end{definicion}





\begin{ejemplo}{} El problema de hallar una primitiva de una función es una ecuación
diferencial que, como ya se ha visto en cursos iniciales de análisis, se relaciona con el concepto de integral. Supongamos $f:(a,b)\subset \rr\to\rr$ una función continua. Consideremos la ecuación diferencial
\[y'(x)=f(x).\]
Sea $x_0\in(a,b)$, integrando respecto a $x$ entre $x_0$ y $x$
\[y(x)=y(x_0)+\int_{x_0}^xf(t)dt\]
Que es una solución general. Quedaría determinada una única solución si, por ejemplo, conociecemos $y(x_0)$.
\end{ejemplo}



\begin{ejemplo}{} Supongamos $f:(a,b)\subset \rr\to\rr$ como antes. Consideremos la ecuación diferencial
\[y''(x)=f(x).\]
Tomemos una integral indefinida respecto a $x$ 
\[y'(x)=C_1+\int f(t)dt\]
Ahora deberemos tomar una integral indefinida más
\[y(x)=C_2+C_1t +\int\left(\int f(x)dx\right)dx.\]
Ahora quedan dos constantes $C_1$ y $C_2$. 
\end{ejemplo}

En el caso de un sistema de ecuaciones diferenciales, la forma de describirlo es esencialmente 
 igual a la de la Definición \ref{def:eq}. La nueva definición contiene a la Definición \ref{def:eq}
 como caso particular.

 \begin{definicion}[Sistema de ecuación diferenciales]{def:sist_eq} 
 Una sistema de $k$-ecuaciones diferenciales ordinarias de orden $n$ 
 es una relación de la forma
\[\boxed{F(x,y(x),y'(x),\ldots,y^{(n)}(x))=0}.\]
donde $F:(a,b)\times \Omega\to\rr^k$, 
\[\Omega\subset\underbrace{\rr^k\times\cdots\times\rr^k}_{n+1-\hbox{veces}}\]
es abierto y $(a,b)$ un intervalo de $\rr$. Notar que el $=$ en la derecha de la ecuación está
en $\rr^k$ y que tenemos $k$ ecuaciones e incognitas, las $k$ componentes de
$y=(y_1,\ldots y_k)$.
  \end{definicion}
  
\begin{ejemplo}{}
 En el caso del péndulo  $k=2$, $n=1$, mientras que $(a,b)=\rr$, $\Omega=\rr\times\rr$ y
 $F:\rr^2\times \rr^2\to\rr^2$ viene dada por
 \[F\left(\begin{pmatrix}
      x \\ v
     \end{pmatrix}
,\begin{pmatrix}
  \xi \\ \eta 
  \end{pmatrix}
  \right)
  =\begin{pmatrix} \xi-v \\ \eta+\sen(x)  \end{pmatrix}
\]
\end{ejemplo}
Más adelante veremos que toda ecuación, o sistemas de ecuaciones, ordinaria de orden $n$ 
se puede reducir a un sistema de orden $1$. De modo tal que es más frecuente considerar
este tipo de sistemas. También es común considerar, se trate ya de ecuaciones 
escalares o sistemas,
 ecuaciones donde es posible despejar 
de manera explícita la derivada de mayor orden. A esta forma la denominaremos 
\emph{forma explícita} de la ecuación

\boxedeq{ y^{(n)}(x)=f(x,y(x),y'(x),\ldots,y^{(n-1)}(x)).}{eq:eq_expl}


\section{Problemas bien planteados, condiciones iniciales}

La siguiente definición expresa lo que entenderemos cuando digamos que un problema está bien planteado.


 \begin{definicion}[Principio de Hadamard]{defi:had}
 Un problema se dice  \href{http://es.wikipedia.org/wiki/Problema_bien_definido}{bien planteado} segun Hadamard si satisface que

 \begin{enumerate}
  \item\label{it:had1} El problema admite solución
  \item\label{it:had2} La solución es única
  \item\label{it:had3} La solución depende de manera continua de los datos numéricos del problema.
 \end{enumerate}
\end{definicion}
 \marginpar{
    \includegraphics[scale=.3]{imagenes/hadamard.jpg}\\
    Jacques Hadamard (1865-1963)
}
 Como hemos visto, una ecuación diferencial no determina una única solución, por consiguiente no sería un problema bien planteado. Debemos agregar relaciones
a nuestro problema para que sea bien planteado. Es así que aparecen condiciones iniciales, problemas de contorno, etc.




\begin{definicion}[Problemas de valores iniciales (PVI)]{} Sea $x_0\in(a,b)$, $F:(a,b)\times \Omega\to\rr$ e $y_0,y_0^1,\ldots,y_0^{n-1}\in\rr$. Las siguientes relaciones  se denominan problema de
valores iniciales
\begin{equation}\label{eq:pvi}
 \left\{\begin{array}{l l l}
         F(x,y,y',&\ldots,y^{(n)})=0& x\in(a,b)\\
         y(x_0)&=y_0&\\
         y'(x_0)&=y_0^1&\\
          & \vdots &\\
          y^{(n-1)}(x_0)&=y_0^{n-1}&\\
        \end{array}
   \right.
\end{equation}

\end{definicion}

Más delante demostraremos el teorema de existencia y unicidad para ecuaciones y sistemas de ecuaciones. Este teorema nos dice que, para ciertas $F$, el problema \eqref{eq:pvi} tiene una única solución alrededor de un entorno de $x_0$.

Es común en las aplicaciones físicas de las EDO, que la variable $x$ represente el tiempo e $y=y(x)$ represente alguna cantidad física (posición, velocidad, temperatura, etc) caracterizando a cierto sistema físico. Es muy frecuente que aparezcan ecuaciones de orden $2$  ($n=2$). En esta situación, al par $(y,y')$ se lo denomina \emph{estado} del sistema y las condiciones \ref{it:had1} y \ref{it:had2} en la Definición \ref{defi:had} se leen como que el estado del sistema en un momento $x_0$ dado determina el estado del mismo en cualquier otro momento. Este es el postulado principal de la corriente  filosófica conocida por   \href{https://es.wikipedia.org/wiki/Determinismo}{determinismo} .

 Es común que una ecuación que modela un sistema físico dependa de parámetros. Pensar por ejemplo del papel de la masa en la ecuación \eqref{caidaLibre1}. Estos parámetros o incluso los valores iniciales del problema \eqref{eq:pvi} dependen muchas veces de resultados de mediciones y sólo es posible tener de ellos un valor aproximado. Por consiguiente,casi nunca se resuelve el PVI que se debería resolver, sino otro, con parámetros y valores iniciales aproximadamente iguales a los correctos\footnote{Estamos usando una noción bastante ingenua de corrección}.
 La condición \ref{it:had3} en la Definición \ref{defi:had} expresa el hecho de que la soluciones a PVIs aproximadamente iguales son aproximadamente iguales entre si.


 Vamos a destacar la siguiente instancia particular de la Definición \ref{eq:pvi}.


\begin{definicion}[PVI para ecuaciones de primer orden]{} La ecuación general de primer orden tiene la forma
\[
F(x,y(x),y'(x))=0,
\]
donde $F:(a,b)\times \Omega\to\rr$ y $\Omega$ abierto de $\rr^2$. Con frecuencia asumiremos que $y'$ se despeja de la relación anterior, es decir que existe $f:\Omega'\to\rr$,
$\Omega'$ abierto de $\rr^2$, tal que
\[y'=f(x,y).\]
Bajo esta suposición, si $(x_0,y_0)\in\Omega'$ el problema de valores iniciales se escribe
\[\left\{\begin{array}{l l}
	    y'&=f(x,y)\\
	    y(x_0)&=y_0
         \end{array}\right.
\]
\end{definicion}



\section{Familias paramétricas de funciones}

Hemos dicho que en el común de los casos es de esperar que una
ecuación tenga asociado un conjunto de soluciones. En esta sección vamos a preguntarnos sobre el problema inverso, dao un conjunto de funciones encontrar una ecuación que lo tenga por solución.



\begin{mdframed}[style=MiEstilo]\relax%
\textbf{Problema. }
Dada una familia paramétrica de funciones
 \begin{equation}\label{flia_param}y=y(x,c),\end{equation}
 dependiente del parámetro $c\in\rr$, ¿Será posible hallar una ecuación para la cual la familia sea la solución general?
\end{mdframed}

En líneas generales la respuesta es si. Nos conviene expresar \eqref{flia_param} como una ecuación implícita
\begin{equation}\label{flia_impl}f(x,y,c)=0.\end{equation}
Derivando esta ecuación respecto a $x$
\begin{equation}\label{der_impl}
 \frac{\partial f}{\partial x}+\frac{\partial f}{\partial y}y'(x,c)=0.
\end{equation}
Ahora es posible eliminar $c$ de \eqref{flia_impl} y \eqref{der_impl} al costo de quedarnos con una sola ecuación.




\begin{ejemplo}{} Encontrar la ecuación que satisface la familia paramétrica
\[x^2+y^2=c^2.\]
Derivamos
\[2x+2yy'=0.\]
Ya está!!!
\end{ejemplo}

\begin{ejemplo}{} Idem $x^2+y^2=2cx$.  Derivando
\[2x+2yy'=2c.\]
Eliminamos $c$ de las dos relaciones
\[\frac{x^2+y^2}{x}=2x+2yy'\Rightarrow \boxed{y'=\frac{y^2-x^2}{2xy}}.\]
\end{ejemplo}

\begin{mdframed}[style=MiEstilo]\relax%
 Dada una familia paramétrica
 \[f(x,y,c)=0,\]
 encontrar otra
 \[g(x,y,d)=0\]
 tal que los ángulos que forman los gráficos entre las funciones de una y de otra familia sean rectos  en cada punto de corte entre ellos.
\end{mdframed}




Para resolver este problema se completan estos pasos
\begin{itemize}
 \item Se encuentra la ecuación diferencial que satisface la familia dada, digamos
 \[y'=h(x,y).\]
 \item Se resuelve
 \[y'=-\frac{1}{h(x,y)}.\]
\end{itemize}




\begin{ejemplo}{} Encontrar la familia de curvas ortogonales a la flia de circunferencias

\[x^2+y^2=c^2\]

Hallamos antes que la ecuación que satisfacen estas curvas es
\[y'=-\frac{x}{y}.\]
Luego deberíamos resolver
\[y'=\frac{y}{x}.\]
Que aún no sabemos resolver,  pero SymPy si!!!

\end{ejemplo}


\begin{mdframed}[style=MiEstilo]\relax%
\textbf{\texttt{dsolve}}

\textbf{Sintaxis} (ampliar en la \href{http://docs.sympy.org/latest/modules/solvers/ode.html#}{documentación \texttt{SymPy}})

\texttt{dsolve(eq, f(x), hint)}

\texttt{eq:} Ecuación (posicional)

\texttt{f(x):} función incognita (posicional)

\texttt{hint: } Método a emplear. Argumento con nombre \texttt{hint='cadena'}.

\end{mdframed}




\begin{pyverbatim}
x=symbols('x')
y=Function('y')(x)
MiEcua=Eq(y.diff(x),y/x)
f=dsolve(MiEcua,y)
\end{pyverbatim}

\noindent\textbf{Resultado:}
Flía rectas  por el origen. \\
La instrucción\\
\texttt{f=dsolve(MiEcua,y,hint='separable')}
\\produce el mismo resultado.





\begin{mdframed}[style=MiEstilo]\relax%
\textbf{\texttt{plot}}

\textbf{Sintaxis} (Ampliar en la \href{http://docs.sympy.org/latest/modules/plotting.html}{documentación \texttt{SymPy}})

Para un gráfico simple

\texttt{plot(expr,rango,opcionales(claves))}

\texttt{expr:} Expresión a graficar 

\texttt{rango:} Conjunto donde varia la variable independiente 

\texttt{opcionales } Argumentos que modifican la apariencia del gráfico. Generalemente de la forma de clave=valor

 
\end{mdframed}

\begin{ejemplo}{}

\end{ejemplo}


\begin{pyverbatim}
x=symbols('x')
f=plot(1/x,(x,-3,3),ylim=(-3,3))
\end{pyverbatim}

\noindent\textbf{Resultado:}
\begin{center}
\includegraphics[scale=.35]{imagenes/ejemplo_plot.png}
\end{center}












\begin{ejemplo}{}Grafiquemos un familias paramétrica de funciones.

\end{ejemplo}



\begin{pyverbatim}
from sympy import *
x,y=symbols('x,y')
Rango=range(21)
L=[tan(pi*k/21.0) for k in Rango] 
p=plot(L[0]*x,(x,-2,2),show=False,xlim=(-2,2),\
ylim=(-2,2),aspect_ratio=(1,1))
for pend in L[1:]:
    p1=plot(pend*x,(x,-2,2),show=False,\
xlim=(-2,2),ylim=(-2,2),aspect_ratio=(1,1))
    p.append(p1[0])
for r in range(1,10):
    p1=plot_implicit(Eq(x**2 + y**2, 0.2*r),\
show=False,aspect_ratio=(1,1),xlim=(-2,2),ylim=(-2,2))
    p.append(p1[0])
p.show()
\end{pyverbatim}


\noindent\textbf{Resultado:}
\begin{figure}[h]
\begin{center}
\includegraphics[scale=.2]{imagenes/flia_curvas_ortogonales.png}
\end{center}
\caption{Familia curvas ortogonales}\label{fig:ortogonales}
\end{figure}


\begin{mdframed}[style=MiEstilo]\relax% 
\textbf{Familias paramétricas de funciones en coordenadas polares.}
En ocasiones la ecuación de la familia de curvas esta dada en otras coordenadas. Por ejemplo supongamos que tenemos la flia de curvas dadas por una EDO en coordenadas polares
\[\frac{dr}{d\theta}=f(r,\theta)\label{eq:flia_curvas_polar},\]
y queremos hallar su flia ortogonal.
 
\end{mdframed}


\noindent\textbf{Solución:} Calculemos $dy/dx$ para las curvas en la familia dada.
\[\frac{dy}{dx}=\frac{dy/d\theta}{dx/d\theta}=\frac{r_{\theta}\sen\theta+r\cos\theta}{r_{\theta}\cos\theta-r\sen\theta}=\frac{f\sen\theta+r\cos\theta}{f\cos\theta-r\sen\theta},\]

donde $r_{\theta}=dr/d\theta$. La flia ortogonal tiene que satisfacer

\[\frac{dy}{dx}=-\frac{f\cos\theta-r\sen\theta}{f\sen\theta+r\cos\theta}\]
Luego
\[\frac{r_{\theta}\sen\theta+r\cos\theta}{r_{\theta}\cos\theta-r\sen\theta}=-\frac{f\cos\theta-r\sen\theta}{f\sen\theta+r\cos\theta}.\]
Si despejamos $r_{\theta}$ llegamos a la ecuación de la flia de curvas ortogonales en coordenadas polares

\boxedeq{\frac{dr}{d\theta}=-\frac{r^2}{f}.}{eq_flia_ortogonal_polares}


Aconsejamos ampliar este tema de \cite{simmons_esp}.  



\section{Separación de variables}
En esta unidad presentaremos algunos métodos de solución para ecuaciones de primer orden. Empezaremos con el, quizás, más popular de todos, el método de \emph{separación de variables}.

\begin{definicion}[Ecuaciones en variables separadas]{} Se dice que en una ecuación de primer orden
 \[y'=f(x,y)\]
 separan las variables, si es posible la factorización $f(x,y)=g(x)h(y)$.
\end{definicion}



Una ecuación  en la que se separan variables se puede resolver siguiendo los siguientes pasos. Hallamos los valores $y_0$ tales que cuando $h(y_0)=0$. Notar que para cualquiera de esos valores la función constante $y(x)\equiv y_0$  es solución. De esta manera podemos encontrar algunas soluciones por esta consideración. 

Para hallar otras, asumamos $h(y)\neq 0$. 
\[\frac{y'}{h(y)}=g(x),\]
Si $H$ y $G$ son primitivas de $1/h$ y $g$ respectivamente por la regla de la cadena tenemos
\[\frac{d}{dx}H(y)=\frac{d}{dx}G(x)\]
Luego 
\[H(y)=G(x)+C,\quad C\in\rr.\]
Por último, si podemos encontrar la inversa de $H$

\boxedeq{ y(x)=H^{-1}\left(G(x)+C\right). }{eq:var_sep}
 
será candidata a solución general. No podemos estar seguros de esta afirmación, sobre todo porque la deducción de esta fórmula estuvo sujeta a suposiciones, como $h(y)\neq 0$.



\begin{ejemplo}{} Resolver
\[y'=\frac{y}{x}.\]
Como queremos agrupar todos los factores que contienen la variable $y$ en uno de los miembros, debemos ``pasar dividiendo'' $y$. Esto es imposible si $y=0$. Reemplazando $y\equiv 0$ en la ecuación vemos que la resuelve. Así hemos encontrado nuestra primera solución $y\equiv 0$. El resto del método, es común llevarlo adelante de la siguiente forma
\[\begin{split}
   \frac{dy}{dx}=\frac{y}{x} &\Longrightarrow \frac{dy}{y}=\frac{dx}{x} \Longrightarrow \int \frac{dy}{y}=\int \frac{dx}{x}\\
   &\Longrightarrow\ln|y|=\ln|x|+C \Longrightarrow |y|= k|x|, \hbox{ con }k>0\\
   &\Longrightarrow y= kx, \hbox{ con }k\neq 0
  \end{split}
\]
\end{ejemplo}
Notar que, al fin y al cabo, la solución general es $y=kx$ con $k\in\rr$. La solución ``díscola'' $y\equiv 0$ también se escribe $y=kx$, cuando $k=0$.


\begin{mdframed}[style=MiEstilo]\relax%
\textbf{\texttt{classify\_ode}}: clasificación de ecuaciones
\textbf{Sintaxis}\\
\texttt{classify\_ode(eq, f(x))}\\
\end{mdframed}


Sympy contiene un comando que nos dice que metodo es posible aplicar a una ecuaqción dada.
\begin{pyverbatim}
x=symbols('x')
y=Function('y')(x)
MiEcua=Eq(y.diff(x),y/x)
tipo=classify_ode(MiEcua,y)
\end{pyverbatim}

\textbf{Resultado:}\\
\begin{verbatim}
('separable', '1st_exact', '1st_linear', 
'almost_linear', 'lie_group',  etc)
\end{verbatim}


\section{Aplicaciones}

Vamos a describir algunos ejemplos. Algunos de ellos llevan a problemas matemáticos muy simples. No obstante es oportuno discutirlos por dos motivos, habituarnos a la utilización de la matemática para resolver problemas de otras ciencias y sentar las bases para discutir problemas más relevantes desde una óptica matemática.

\subsection{Ley de reproducción normal}
  En muchos ejemplos de biología, química-física, etc, hay magnitudes que crecen(decrecen) siguiendo una ley que denominaremos
\href{http://es.wikipedia.org/wiki/Crecimiento_exponencial}{Ley de reproducción  normal}. Según esta ley la cantidad de individuos, sustancia, materia,
energía, etc, que se agrega o elimina de una población, cuerpo, etc por unidad de tiempo es proporcional a la cantidad de individuos, sustancia, etc que hay presente.   Una población de seres vivos puede reproducirse de esta manera bajo algunas circunstancias
especiales, por ejemplo si cuenta con fuente ilimitada de alimentos.

  Si $P(t)$ es la cantidad de individuos en el momento $t$, la ley de reproducción normal establece
en este caso la siguiente ecuación diferencial de primer orden
\[P'(t)=kP(t),\quad\text{ con } k>0.\]
La solución es hallada con suma facilidad, por ejemplo usando separación de variables, siendo ella
\[\boxed{P(t)=Ce^{kt},\quad C\in\rr.}\]
La constante  $C$ se puede determinar si tenemos un problema a valores iniciales (pvi), por ejemplo $P(0)=P_0$, siendo $P_0\in\rr$ dado. En ese caso
$C=P_0$.



  Otro ejemplo de comportamiento similar es la \href{http://es.wikipedia.org/wiki/Radiactividad}{desintegración radiactiva}. Algunos átomos de
ciertas sustancias, pueden ``desarmarse'' en átomos de otras sustancias. En el proceso suelen emitir radiaciones.  La velocidad de desintegración sigue una ley
de reproducción normal pero hay que tener en cuenta que la materia radiactiva, es decir la ``población''  en este caso, se pierde. Si $x(t)$ es la masa de materia radiactiva en el momento $t$, evolucionará
acorde a la ley
\[\boxed{x'(t)=-kx(t),\quad\text{con } k>0.}\]



\subsection{Soluciones}

\begin{mdframed}[style=MiEstilo]\relax%
 \textbf{Problema}
\begin{tabular}{m{5cm} m{4.5cm}}
 Un tanque contiene inicialmente $N$ $\hbox{m}^3$ de $H_2O$ entre los cuales hay disueltos $C$ kg de sal común
 $NaCl$. A través de una boca de entrada y una de salida empieza circular la solución, entrando y saliendo con el mismo caudal $q\frac{\hbox{m}^3}{s}$. Se supone que
 la solución entrante tiene una concentración conocida $r$. Encontrar la cantidad de sal en el momento $t$. &
\scalebox{.6} % Change this value to rescale the drawing.
{
\begin{pspicture}(0,-5.26)(8.346,5.246)
\definecolor{color407b}{rgb}{0.2823529411764706,0.22745098039215686,0.8745098039215686}
\definecolor{color830b}{rgb}{0.12549019607843137,0.23529411764705882,0.9411764705882353}
\psbezier[linewidth=0.012,fillstyle=solid,fillcolor=color830b](1.96,2.58)(1.9916879,3.1224957)(5.9146633,3.2288723)(5.96,2.56)(6.0053368,1.8911277)(5.9,-1.74)(5.96,-2.4)(6.02,-3.06)(1.8660278,-2.908184)(1.94,-2.44)(2.013972,-1.971816)(1.9283121,2.0375042)(1.96,2.58)
\psbezier[linewidth=0.012,fillstyle=gradient,gradlines=2000,gradmidpoint=1.0,fillcolor=color407b](5.06,-1.3)(4.6,-1.18)(4.76,-0.68)(5.08,-0.68)(5.4,-0.68)(6.86,-0.66)(7.6,-0.66)(8.34,-0.66)(8.160402,-3.7864435)(8.14,-3.78)(8.119597,-3.7735565)(7.6556315,-3.938399)(7.52,-3.66)(7.384369,-3.381601)(7.551398,-1.3824744)(7.24,-1.4)(6.928602,-1.4175255)(5.52,-1.42)(5.06,-1.3)
\psbezier[linewidth=0.012,fillcolor=color407b](1.98,2.5611498)(2.22,2.4229145)(2.8400288,2.32)(3.64,2.3223798)(4.4399714,2.3247597)(5.08,2.3412302)(5.98,2.58)
\psline[linewidth=0.012cm,fillcolor=color407b](2.0,2.58)(2.0,4.7)
\psline[linewidth=0.012cm,fillcolor=color407b](5.96,2.62)(5.96,4.72)
\psbezier[linewidth=0.012,fillcolor=color407b](1.98,4.66)(2.06,4.12)(6.02,4.2)(5.94,4.7)
\psbezier[linewidth=0.012,fillcolor=color407b](2.06,4.7165217)(2.2734654,5.24)(5.98,5.130435)(5.9411883,4.68)
\psframe[linewidth=0.012,dimen=outer,fillstyle=solid,fillcolor=color407b](1.96,1.62)(0.9,1.1)
\psline[linewidth=0.012cm,fillcolor=color407b,linestyle=dashed,dash=0.16cm 0.16cm](1.98,1.6)(2.94,1.58)
\psline[linewidth=0.012cm,fillcolor=color407b,linestyle=dashed,dash=0.16cm 0.16cm](2.0,1.08)(2.76,1.08)
\psellipse[linewidth=0.012,linestyle=dashed,dash=0.16cm 0.16cm,dimen=outer](2.78,1.28)(0.1,0.26)
\psline[linewidth=0.04cm,fillcolor=color407b,arrowsize=0.05291667cm 2.0,arrowlength=1.4,arrowinset=0.4]{->}(7.82,-3.94)(7.84,-5.24)
\psline[linewidth=0.04cm,fillcolor=color407b,arrowsize=0.05291667cm 2.0,arrowlength=1.4,arrowinset=0.4]{->}(0.0,1.38)(0.76,1.34)
\rput(0.26,1.92){$q$}
\rput(8.4,-4.66){$q$}
\end{pspicture} 
}
\\
\end{tabular}

\end{mdframed}




 Sea $x(t)$ la cantidad de $NaCl$ en el tanque en el momento $t$. entonces

\[\begin{split}
   x'(t)&=\text{cantidad que entra }-\text{cantidad que sale }\\
          &=qr-q\frac{x(t)}{N}
  \end{split}
\]




\subsection{Dinámica del punto}

\subsubsection{Discusión Teórica}
Vamos a recordar algunas temas de la asignatura física. En particular, el tema del movimiento de un cuerpo de masa $m$,\marginpar{En mecánica newtoniana, un sistema de referencia inercial es un sistema de referencia en el que las leyes del movimiento cumplen las leyes de Newton} al que podemos suponer puntual, dentro del espacio euclideano tridimensional. A un cuerpo de estas características lo llamaremos punto masa. Denotamos por $x(t)\in \rr^3$ su posición en algún marco de referencia cartesiano ortogonal e \href{https://es.wikipedia.org/wiki/Sistema_de_referencia_inercial}{inercial}. Suponemos
que sobre él actúa una fuerza $F$. Recordemos que $x'(t)$ es la velocidad $v(t)$ y que $x''(t)$ es la aceleración $a(t)$.
\href{http://es.wikipedia.org/wiki/Leyes_de_Newton\#Segunda_ley_de_Newton_o_ley_de_fuerza}{La segunda ley de Newton} implica que
\begin{equation}\label{2leyR}\boxed{mx''(t)=F}\end{equation}


Supongamos que el movimiento del punto masa se realiza entre los momentos $t_0$ y $t_1$. Como has visto en Cálculo III la lóngitud de la curva recorrida 
$s(t)$ se puede calcular por
\begin{equation}\label{2ley}s=\int_{t_0}^{t_1}|x'(t)|dt=\int_{t_0}^{t_1}|v(t)|dt.\end{equation}
A $s$ se lo suele denominar \href{http://es.wikipedia.org/wiki/Longitud_de_arco}{elemento de arco}. Es comun querer utilizar a $s$ como variable independiente en
lugar de $t$, puesto que algunas fórmulas se simplifican de esta forma. Por ejemplo

\begin{equation}\label{pre_trab} F\cdot v(t)dt=F\cdot\frac{v(t)}{|v(t)|}|v(t)|dt=F_tds,\end{equation}
donde $F_t$ denota la proyección de la fuerza $F$ sobre la dirección tangente a la trayectoria.





Si integramos \eqref{pre_trab} entre $t_0$ y $t_1$ y usamos \eqref{2leyR} obtenemos
\begin{equation}\label{eq:trabajo}\begin{split} W:=\int_{s_0}^{s_1}F_tds&=\int_{t_0}^{t_1} F\cdot v(t)dt\\
   & =m \int_{t_0}^{t_1} v'(t)\cdot v(t)dt\\
   &=\frac{m}{2} \int_{t_0}^{t_1} \frac{d|v|^2}{dt}dt\\
   &=\frac{m}{2}|v(t_1)|^2-\frac{m}{2}|v(t_0)|^2.
\end{split}
   \end{equation}



  A la cantidad $\frac{m}{2}|v|^2$ se la denomina \emph{energía cinética} $E_c$ y a $W$ se lo denomina \emph{trabajo}.
Las relaciones obtenidas dicen que la variación de la energía cinética es igual al trabajo realizado $W=\Delta E_c$ por la fuerza $F$.
El trabajo realizado depende de la proyección tangencial de la fuerza $F_t$.  Llamaremos a esta relación
\href{https://es.wikipedia.org/wiki/Conservaci%C3%B3n_de_la_energ%C3%ADa}{Principio de Conservación de la Energía Mecánica.}

  Vamos a referirnos por \emph{rapidez} al módulo de la velocidad.
Si uno quiere incrementar o reducir la rapidez final $|v(t_1)|$ entonces deberá tener fuerzas con una componente tangencial no nula. 
Dicho de otra forma, si la fuerza es perpendicilar al movimiento, no hay cambio de rapidéz. 

  Hay fuerzas que siempre actuan en la dirección del movimiento. El ejemplo más conocido son las fuerzas de fricción, resistencia del aire,
resistencia a la rodadura, etc. Estás fuerzas, actúan sólo en la dirección del movimiento y se oponen a él.


 

  Por el contrario hay otras que actúan perpendiculares al movimiento $F_t=0$. Ejemplo de ello son las fuerzas que mantienen a un cuerpo moviéndosé a lo largo
de una guía. Por ejemplo un niño cayendo por un tobogán. Que el niño no se despegue de la guía (tobogán) se explica por la aparición de una fuerza que se denomina
reacción de vínculo que actúa en la dirección perpendicular al movimiento esto es decir a la guía. Está fuerza debe compensar a toda otra fuerza que trata de apartar
al cuerpo de la guía.  En el caso del tobogán la gravedad trata de apartar al niño de aquel.





\subsubsection{Cuerpos cayendo por guías}
Analicemos más en detalle el movimiento de un cuerpo cayendo a lo largo de una guía estando además influído  por la acción de la gravedad. Supondremos el movimiento en las proximidades de la superficie de la Tierra y por ello, supondremos que la magnitud de la fuerza de la gravedad es la constante $mg$.




 \begin{wrapfigure}[16]{l}{6cm}
  \begin{center}
  \setlength{\unitlength}{1.2cm}
    \begin{picture}(5, 4)(0,1)
      \put(1,0.75){\vector(0,1){4.5}}
      \put(0,1){\vector(1,0){5}}
      \qbezier(1.3,5)(2,2)(5,1.3)
      \put(2.6,2.6){\circle*{.3}}
      \put(2.6,2.6){\vector(0,-1){1.5}}
      \put(1.3,1.2){$(0,-mg)$}
      \put(2.6,.75){$x$}
      \multiput(2.6,2.6)(-.1,0){17}{\line(1,0){0.05}}
      \put(.75,2.6){$y$}
      \put(4.17,.97){\line(-1,1){4}}
      \put(2.67,2.1){$\alpha$}
      \put(4.08,1.15){$\frac{\pi}{2}+\alpha$}
    \end{picture}\caption{Caída por una guía}\label{fig:caída}
  \end{center}
\end{wrapfigure}
 Supongamos que la guía esta
confinada a un plano. Introducimos un sistema de coordenadas ortogonales en dicho plano, con el suelo paralelo al eje $x$


El elemento longitud de arco $s$ de una curva que es el gráfico de una función $y(x)$ para $x$ en $[x_0,x_1]$ viene dado por 
\[s=\int_{x_0}^{x_1}\sqrt{1+y'(x)^2}dx\]
La fuerza de vínculo de la guía tiene componente tangencial nula,  la gravedad tiene una componente tangencial no nula. 
Su magnitud es $mg\cos\alpha$ (ver dibujo). Vamos a tratar de expresar $\cos\alpha$  en términos de   $y'(x)$. Supongamos  $\cos\alpha>0$ e $y'(x)<0$.
Los demás casos quedan como \textbf{ejercicio}. 
\[ \tan^2\alpha=\frac{\sen^2\alpha}{\cos^2\alpha}=\frac{1-\cos^2\alpha}{\cos^2\alpha}=\frac{1}{\cos^2\alpha}-1\]
e
\[y'(x)=\tan \left(\frac{\pi}{2}+\alpha\right)=-\frac{1}{\tan\alpha}\]
Podemos usar las relaciones anteriores para escribir $\cos\alpha$ en función de $y'(x)$
\begin{equation}\label{cos_alpha}\cos\alpha=-\frac{y'(x)}{\sqrt{1+y'(x)^2}}\end{equation}
Entonces
\begin{equation}\label{cons_ener}
 \begin{split} \frac{m}{2}|v(t_1)|^2-\frac{m}{2}|v(t_0)|^2&=\int_{s_0}^{s_1}F_tds =\int_{x_0}^{x_1}F_t\frac{ds}{dx}dx\\
&= -mg\int_{x_0}^{x_1}\frac{y'(x)}{\sqrt{1+y'(x)^2}}\sqrt{1+y'(x)^2}dx\\
&=-mg\left(y_1-y_0\right)
    \end{split}\end{equation}
Esto nos permite escribir la rapidez en función de la altura repecto al piso.



\subsection{El péndulo}


 \begin{wrapfigure}[20]{l}{6cm}
  \begin{center}
  \setlength{\unitlength}{1cm}
    \begin{picture}(5, 6)(0,1)
      \put(0,6){\vector(1,0){5}}
      \put(2.5,1){\vector(0,1){6}}
      \put(5,6){$x$}
      \put(2.6,7){$y$}
      \put(2.5,6){\line(-1,-2){2}}
      \put(0.45,1.9){\circle{.25}}
      \multiput(.6,1.9)(.1,0){19}{\line(1,0){0.05}}
      \put(2.6,1.9){$-l\cos\alpha$}
      \put(2.2,5.2){$\alpha$}
      \put(1.2,4){$l$}


    \end{picture}\caption{Péndulo}\label{fig:caída}
  \end{center}
\end{wrapfigure}

  Se trata de una masa puntual $m$ suspendida de un punto por medio de una barra de longitud $l$
 a la que suponemos sin masa. Equivale al movimiento sobre una guía circular.  Usaremos el ángulo $\alpha$ marcado en la figura, como variable dependiente.



Supondremos que el origen del sistema de de coordenadas está sobre el punto de amarre de la barra. Entonces de \eqref{cons_ener} con $t_1=t$ deducimos 
\[\begin{split}\frac{m|v(t)|^2}{2}&-\frac{m|v(t_0)|^2}{2}\\&=-mg\left(y(t)-y(t_0)\right)\\
  &=mg\cos\alpha(t)-mg\cos\alpha(t_0).
   \end{split}
\]
Ahora la posición de la masa es $x(t)=l(\sen\alpha,-\cos\alpha)$ luego 
\[v(t)=l\alpha'(t)(\cos\alpha,\sen\alpha).\]
Tomando módulo
\[ |v(t)|^2=l^2\alpha'(t)^2.\]
Entonces
\[\frac{ml^2\alpha'(t)^2}{2}= mgl\cos\alpha(t)-mgl\cos\alpha(t_0) +\frac{mv(t_0)^2}{2}.\]
Derivando esta relación
\[ml^2\alpha'(t)\alpha''(t)=-mgl\alpha'(t)\sen\alpha(t).\]
De esto deducimos la ecuación del \href{http://es.wikipedia.org/wiki/Péndulo}{péndulo}
\[\boxed{\alpha''(t)=-\frac{g}{l}\sen\alpha(t)}.\]



\section{Ecuaciones homogéneas}

Volvemos al problema de resolver ecuaciones de primer orden. Abordamos un nuevo método.

\begin{definicion}[Funciones homogéneas]{}
 Una función $F:\rr\times\rr\to\rr$ se dice homogénea de grado $\alpha$ si
 \[f(rx,ry)=r^{\alpha}f(x,y).\]
\end{definicion}

\begin{ejemplo}{}

\begin{itemize}
                 \item $f(x,y)=\tfrac{y}{x}$ es homogénea de grado $0$.
                 \item Más generalmente, cualquier función $f(x,y)$ que dependa sólo de $x/y$, esto es que se escriba de la forma $f(x,y)=g(y/x)$
                 es homogénea de grado  $0$. Así $f(x,y)=\tfrac{x-y}{x+y}$ es homogénea de grado $0$ pues $\tfrac{x-y}{x+y}= \tfrac{1-x/y}{1+x/y}$
                 \item $f(x,y)=\sum_{k=0}^na_kx^ky^{n-k}$ es homogénea de grado $n$.
\end{itemize}
\end{ejemplo}



\begin{definicion}[Ecuaciones homogéneas]{}
 Una ecuación
 \begin{equation}\label{ecua_prin}y'=f(x,y)\end{equation}
 tal que $f$ es homogénea de grado $0$ se llamará ecuación homogénea.
\end{definicion}

Si una ecuación del tipo \eqref{ecua_prin} es homogénea entonces se transforma en una ecuación separable mediante el cambio de variable dependiente $\boxed{z=y/x}$. En efecto, para $x\neq 0$

\[f(x,y)=x^0f\left(1,\frac{y}{x}\right)=f(1,z)\]
y
\[y'=z'x+z\]
Como $y'=f(x,y)$ tenemos
\begin{equation}\label{cambio_hom}z'x+z=f(1,z)\Longrightarrow \frac{dz}{f(1,z)-z}=\frac{dx}{x}.\end{equation}



\begin{ejemplo}{} Resolver $y'=\frac{x+y}{x-y}$.

La ecuación \eqref{cambio_hom} queda
\[ \begin{array}{c} \frac{dx}{x}=\frac{dz}{\frac{1+z}{1-z}-z}=\frac{(1-z)dz}{1+z^2}\\
 \Downarrow\\
\ln|x|+C=\arctan(z)-\frac{1}{2}\ln|1+z^2|
    \end{array}.
\]

\end{ejemplo}



\section{Ecuaciones exactas}



\begin{definicion}[Diferencial]{}
 Dada una función $f$ de $n$ variables independientes $x_1,\ldots,x_n$ definimos su \href{http://es.wikipedia.org/wiki/Diferencial_de_una_función}{diferencial}
 por
 \begin{equation}\label{eq:formas} df=\frac{\partial f}{\partial x_1}dx_1+\cdots +\frac{\partial f}{\partial x_n}dx_n.
   \end{equation}

 \end{definicion}

 Esta definición obviamente carece de rigor pues el miembro de la derecha en \eqref{eq:formas} contiene las expresiones indefinidas $dx_j$.
La diferencial puede ser definida con toda corrección, es un ejemplo de \href{https://es.wikipedia.org/wiki/Forma_diferencial}{forma diferencial}, en particular  es una 1-forma. En la unidad que sigue hablaremos un poco más del concepto de forma diferencial. El uso que haremos de las formas diferenciales es muy elemental, podríamos evitar por completo su uso al costo de usar una notación ligeramente menos compacta y simétrica.

 Es costumbre escribir una ecuación diferencial  como la 1-forma diferencial
 \begin{empheq}[box=\tcbhighmath]{equation}\label{eq:ecua_forma}
  M(x,y)dx+N(x,y)dy=0
 \end{empheq}


Que corresponde a la ecuación, escrita de manera tradicional, \linebreak$y'=-M(x,y)/N(x,y)$.



 Dada $f:\rr\times\rr\to\rr$, satisfaciendo las condiciones del \href{https://es.wikipedia.org/wiki/Teorema_de_la_funci%C3%B3n_impl%C3%ADcita}{Teorema de la Función Implícita}, la expresión
 \begin{empheq}[box=\tcbhighmath]{equation}\label{eq:ecua_impli}
  f(x,y)=c
 \end{empheq}

define una  familia paramétrica de curvas, con parámetro $c$. Derivanda la expresión podemos representar esta familia como soluciones de la EDO

\[
 \frac{\partial f}{\partial x}+\frac{\partial f}{\partial y}y'=0\Longleftrightarrow \frac{\partial f}{\partial x}dx+\frac{\partial f}{\partial y}dy=0
 \Longleftrightarrow df=0.
\]
 Esto \marginpar{Si una fuerza $F$ está dada por $(M,N)=\nabla f$ (campo coservativo) reemplazando en \eqref{eq:trabajo} y llamando $U=-f$ obtenemos que $E:=m|v(t)|^2/2+U(x(t))$ es constante. Se dice que $E$ es una magnitud conservada y se la denomina en este caso energía. } nos sugiere la idea de que, dada una ecuación diferencial cualquiera, indaguemos si se puede escribir o puede ser transformada de alguna manera en una ecuación de la forma $df=0$.  Para que la ecuación \eqref{eq:ecua_forma}, se pueda expresar como $df=0$ se debe cumplir que  $M=\partial f/\partial x$ y $N=\partial f/\partial y$.
 Vale decir el campo vectorial $(x,y)\mapsto (M(x,y),N(x,y))$ es un \href{http://es.wikipedia.org/wiki/Fuerza_conservativa}{campo
gradiente o conservativo}, con potencial $f$.

No todo campo es un campo gradiente, recordemos el siguiente teorema del análisis vectorial.

 \begin{teorema}[Caracterización de campos conservativos]{teo:campo_cons} Sea $\mathcal{O}$ un conjunto abierto y simplemente conexo de $\rr^n$. Son equivalentes
 \begin{enumerate}
  \item\label{item:cons1} El campo $F:\mathcal{O}\to\rr^n$ es un gradiente.
  \item\label{item:cons2} Si $C$ es cualquier camino cerrado entonces
  \[\ointctrclockwise_C F\cdot d x=0.\]
  \item\label{item:cons3} \[\frac{\partial F^i}{\partial x_j}=\frac{\partial F^j}{\partial x_i},\quad\text{ para }i,j=1,\ldots,n\]
 \end{enumerate}

\end{teorema}

Pensando al campo $F$ como un campo de  fuerzas sobre el espacio euclideo tridimensional,  el ítem \ref{item:cons2} expresa que el trabajo realizado por la fuerza a lo largo de un camino cerrado es cero. El ítem \ref{item:cons3} afirma que $\nabla\times F=0$, la fuerza es irrotacional.


El item \ref{item:cons3}  es simple de chequear. Una vez establecido que un campo es conservativo tendremos el problema de hallar el potencial $f$.
Ilustremos esto con el campo $(x,y)\mapsto (M(x,y),N(x,y))$. Supongamos que $\mathcal{O}$ es abierto de $\rr^2$ y
\[\frac{\partial M}{\partial y}=\frac{\partial N}{\partial x},\quad\text{ para } (x,y)\in \mathcal{O}.\]
En primer lugar debemos tener un campo escalar $f$ tal que
\[M=\frac{\partial f}{\partial x}\Rightarrow f=\int Mdx +C(y).\]
Ahora como $f_y=N$
\[N=\frac{\partial f}{\partial y}=\frac{\partial}{\partial y}\int Mdx +C'(y)\Rightarrow C'(y)=N-\frac{\partial}{\partial y}\int Mdx .\]
 Para que esta ecuación tenga solución $N-\frac{\partial}{\partial y}\int Mdx$ debe ser sólo función de $y$. Pero la condición necesaria y suficiente para ello es
\[\begin{split}0&=\frac{\partial}{\partial x}\left(N-\frac{\partial}{\partial y}\int Mdx\right)\\
&= \der{N}{x}-\frac{\partial^2}{\partial x\partial y}\int Mdx\\
&=\der{N}{x}-\frac{\partial^2}{\partial y\partial x}\int Mdx\\
&=\der{N}{x}-\der{M}{y}.
   \end{split}
\]

 Pero estamos bajo ese supuesto, entonces
 \begin{empheq}[box=\tcbhighmath]{equation}
  f= \int Mdx + \int\left( N-\frac{\partial}{\partial y}\int Mdx \right)dy.
 \end{empheq}
\begin{ejemplo}{} Resolver $e^ydx+(xe^y+2y)dy=0$.
 \end{ejemplo}


\noindent\textbf{Solución:} Aquí
\[M=e^y\quad\text{ y }\quad N=xe^y+2y.\]
Así
\[\der{M}{y}=e^y=\der{N}{x}.\]
La ecuación es exacta. El potencial $f$ debe cumplir
\[f=\int e^ydx=xe^y+C(y).\]
Luego
\[C(y)=\int\left( xe^y+2y -\frac{\partial}{\partial y} xe^y\right)dy= y^2\]
Tener en cuenta que la función potencial $f$ no es única, queda deerminada hasta una constante aditiva de integración que podemos elegir a gusto ya que
debemos encontrar sólo un potencial. Entonces podemos tomar
\[f= xe^y+y^2.\]
La solución general de la ecuación estará dada por
\[xe^y+y^2=c,\quad c\in\rr.\]
Como no sabemos despejar $\boxed{y}$ de aquí dejamos indicada de esta manera la solución.



\section{Factores integrantes}

 Las ecuaciones exactas son raras, no obstante tenemos un recurso para llevar algunas ecuaciones no exactas a una equivalente y exacta.

 Supongamos que la ecuación  \eqref{eq:ecua_forma} no es  exacta. La idea es encontrar una función $\mu(x,y)$ llamada
\href{http://es.wikipedia.org/wiki/Ecuación_diferencial_exacta\#Factor_integrante.}{factor integrante} que haga exacta la ecuación
\[\mu\left(Mdx+Ndy\right)=0.\]
 Para ello se debe cumplir que
\begin{equation}\label{carac_factor}
  \der{\mu M}{y}=\der{\mu N}{x}\Longleftrightarrow \boxed{ \mu\der{M}{y}+\der{\mu}{y}M=\der{N}{x}\mu+N\der{\mu}{x}}.
\end{equation}

\begin{proposicion}[Existencia de factores integrantes] Toda ecuación de primer
 orden \eqref{eq:ecua_forma}, con $N\neq 0$,  que tiene una solución general
que se escribe como en \eqref{eq:ecua_impli}, con $\partial f/\partial y\neq 0
$  tiene un factor integrante.
\end{proposicion}
\noindent\textit{Comentario:} La suposición $N\neq 0 \neq\der{f}{y}$,  es razonable pues $N=0$ imlpicaría que no podemos despejar $y'$ de \eqref{eq:ecua_forma}  y $\partial f/\partial y$ contradice las hipótesis del Teorema de la Función Implícita, herramienta necesaria para suponer que $y$ es función de $x$

\begin{demo} Si derivamos \eqref{eq:ecua_impli} conseguimos
\[\der{f}{x}dx+\der{f}{y}dy=0.\]
De esta ecuación y \eqref{eq:ecua_forma}  vemos que
\[-\frac{M}{N}=y'=-\frac{\partial f /\partial x}{\partial f/\partial y}\Longrightarrow \frac{ \partial f /\partial x}{M}=\frac{ \partial f/\partial y}{N}=:\mu(x,y)\]
 De la igualdad de arrriba se deduce que
\[\der{f}{x}=\mu M\quad\text{ y }\quad \der{f}{y}=\mu N.\]
Es decir $\mu$ es factor integrante.
\end{demo}

La proposición anterior nos dice que, mientras la ecuación sea resoluble, siempre  existe un factor integrante. Pero no ayuda a hallarlo dado que parte de la solución que es lo que queremos hallar. Otra alternativa es resolver la ecuación \eqref{carac_factor} que es una ecuación en derivadas parciales para $\mu$, que normalmente es más dificil que resolver  la original. Así, mientras que el método es siempre aplicable, en la práctica es útil en situaciones específicas.

Hay que señalar que sólo necesitamos una solución de \eqref{carac_factor} y no su solución general. En la práctica se suele hacer alguna suposición
sobre $\mu$ que simplifique la expresión. Es decir proponer alguna forma específica para $\mu$ que haga la ecuación \eqref{carac_factor} más sencilla de resolver. Por supuesto, a priori el factor integrante, si bien existe, no tiene ninguna forma predeterminada. Lo que hacemos es lo que en lenguaje culto se conoce como \href{http://es.wikipedia.org/wiki/Ansatz}{ansatz}, y en lenguaje coloquial, que aquí es más significativo, estamos haciendo un lance o tratando de adivinar $\mu$, sin mucho más criterio que buscarlo entre equellas funciones con una forma (simple) predeterminada. Por ejemplo, es común suponer que $\mu$ es sólo función de una de las variables. Si por ejemplo asumimos que $\mu=\mu(x)$ las ecuación
\eqref{carac_factor} se escribe

\[\mu\der{M}{y}=\mu\der{N}{x}+N\mu'(x) \Longrightarrow\boxed{\frac{\mu'}{\mu}=\frac{\partial M/\partial y-\partial N/\partial x}{N}}\]
Este \href{http://es.wikipedia.org/wiki/Ansatz}{ansatz} no siempre funcionará,
para que lo haga,  la función en el segundo miembro $(\partial M/\partial
y-\partial N\partial x)/N$
debe depender sólo de $x$. Si eso ocurre
\begin{empheq}[box=\tcbhighmath]{equation}\label{eq:factor_int}
 \mu(x)=e^{\int \frac{\partial M/\partial y-\partial N/\partial x}{N}dx}.
\end{empheq}

es un factor integrante. Recordar que sólo necesitamos hallar uno, por ese motivo omitivos constantes de integración.

De manera similar, si la función
\[\frac{\partial N/\partial x-\partial M/\partial y}{M}\]
depende sólo de $y$ tenemos que
\[\mu(y)=e^{\int \frac{\partial N/\partial x-\partial M/\partial y}{M}dy}\]
es un factor integrante que, en este caso, sólo depende de $y$.

\begin{ejemplo}{} $ydx+(x^2y-x)dy=0$.
 \end{ejemplo}


Primero chequeemos la posible exactitud.
\[\der{M}{y}=1\text{ y } \der{N}{x}=2xy-1\Longrightarrow\text{no exacta}.\]
Ahora
\[\frac{\partial M/\partial y-\partial N/\partial x}{N}=\frac{2-2xy}{x(xy-1)}=-\frac{2}{x}.\]
El factor integrante es
\[\mu(x)=e^{-2\ln |x|}=\frac{1}{x^2}.\]





\section{Aplicación: antenas y espejos parabólicos}

\begin{ejemplo}{}
\textit{Espejos, antenas parabólicos.} Hallar la forma del espejo curvo tal que el reflejo de todo haz de luz que viaja paralelo al eje $x$ con dirección
negativa repecto a este eje pasa por el $(0,0)$.
\end{ejemplo}


\begin{ejercicio}{} 
  Dejamos como ejercicio demostrar que un haz de luz que se refleja sobre un espejo lo hace de tal manera que los ángulos que se forman con los rayoscde incidencia y refracción y la tangente al espejo en el punto de incidencia son iguales ( $\beta=\alpha$ en el dibujo). Para resolver esto hay que usar el principio
de mínimo tiempo de Fermat.
\end{ejercicio}
 \marginpar{%
 \includegraphics[scale=.1]{imagenes/antena.jpg}\\
Antena parabólica
}

\begin{wrapfigure}[12]{L}{8cm}
\psset{unit=1cm}
\begin{pspicture}(-0.8,0)(7,4)
\psline{->}(-0.8,.5)(7,.5)
\psline{->}(2,0                                                                                                                                                                                                                                                                                                                                                                                                                                                                                                                                                                                                                                                                                                                                                                                                                                                                                                                                                                                                                         
                                                                                                                                                                                                                                                                                                                                                                                                                                                                                                                                                                                                                                                  )(2,4)
\rput(6.8,.25){$x$}
\rput(1.8,3.8){$y$}
\psline[linestyle=dashed](-.7,.5)(5,4)
\psbezier(1,.5)(1,1.25)(3,3.7)(7,4)
\pscircle*(2.9,2.7){.1}
\psline(2,0.5)(2.9,2.7)
\psline{->}(2,0.5)(2.3,1.21)
\psline{->}(2.9,2.7)(7,2.7)
\psarc{->}(2,0.5){1}{0}{67.75}
\psarc{-}(2.9,2.7){0.7}{213}{247}
\rput(2.5,0.8){$\theta$}
\rput(3.2,2){$\alpha$}
\rput(0.2,0.7){$\phi$}
\psline{->}(3,2)(2.9,2)(2.6,2.5)
\psarc{-}(2.9,2.7){2}{0}{33}
\rput(4,2.9){$\beta$}
\end{pspicture}
\caption{Reflejo en un espejo curvo}
\end{wrapfigure}






 \noindent\textbf{Solución.} Sea $(x,y)$ el punto de incidencia. Apelando a la geometría elemental, $\phi=\beta$ y $\theta=\alpha+\phi=2\beta$. Como $\tan\theta=\frac{y}{x}$
  y como
  \[\begin{split} \tan\theta &=\tan 2\beta\\&=\frac{2\tan\beta}{1-\tan^2\beta},
    \end{split}
\]
deducimos que
\[\frac{y}{x}=\frac{2 dy/dx}{1-(dy/dx)^2}.\]
Despejando
\begin{equation}\label{eq:antena}
 (\pm\sqrt{x^2+y^2}+x)dx+ydy=0
\end{equation}
La expresión sugiere introducir coordenadas polares
\[x=r\cos\theta, \quad y=r\sen\theta.\]
Atendiendo a \eqref{eq:formas} tenemos:
\[dx=\cos\theta dr-r\sen\theta d\theta,\quad dy=\sen\theta dr+r\cos\theta d\theta.\]
Sustituyendo en \eqref{eq:antena}

\[\mp r^{2} \sin{\left (\theta \right )} d\theta + \left(\pm r \cos{\left (\theta \right )} + r\right) dr=0.\]
Notar que
\[\frac{N_{\theta}-M_r}{M}=-\frac{1}{r}\]
es independiente de $\theta$. Tenemos que $\mu(r)=r^{-1}$ es factor integrante. Multiplicando por $\mu$ llegamos a la ecuación exacta
\[\mp r\sen\theta d\theta+(\pm\cos\theta+ 1)dr=0.\]
Siguiendo las técnicas expuestas llegamos a la función potencial
\[\Phi(\theta,r)=\pm r\cos\theta +r.\]
Convirtiendo nuevamente a coordenadas cartesianas, encontramos que la solución general es
\[\pm x + \sqrt{x^2+y^2}=c.\]
Despejando
\[y^2=\mp 2xc+c^2=2c\left(\mp x+\frac{c}{2}\right)\]
Que es la familia de todas las parábolas con eje de simetría $x$  y con foco en $(0,0)$. 
El signo $\mp$ refleja el hecho que el espejo podría mirar hacia un lado u otro. 

\section{Ecuaciones Lineales}
\begin{definicion}[Ecuación lineal]{} Se llama \href{http://es.wikipedia.org/wiki/Ecuación_diferencial_lineal}{ecuación diferencial lineal}
a una ecuación que es lineal
respecto a   la/s variables
dependientes. La siguiente es la ecuación diferencial lineal general de primer orden
\begin{empheq}[box=\tcbhighmath]{equation}\label{eq:lineal}
 y'+p(x)y=q(x)
\end{empheq}

y la  de segundo orden
\begin{empheq}[box=\tcbhighmath]{equation}\label{eq:lineal_orden2}
 y''+p(x)y'+q(x)y=r(x).
\end{empheq}
 
Aquí $p,q$ y $r$ son funciones de $x$ usualmente definidas sobre un intervalo abierto de $\rr$.
\end{definicion}
La ecuación puede ser no lineal repecto a la variable independiente.

Es costumbre introducir los operadores  diferenciales $L_1[y]=y'+py$  y $ L_2[y]=y''+py'+qy$.
Para una ecuación lineal, los operadores $L_1$ y $L_2$ son lineales. Es decir, $L_1[y_1+y_2]=L_1[y_1]+L_1[y_2]$.


 Vamos a resolver la ecuación lineal de primer orden \eqref{eq:lineal}. Esto es sencillo pues la forma diferencial asociada
 \begin{equation}\label{eq:lineal2}dy+(p(x)y-q(x))dx=0
  \end{equation}
tiene un factor integrante que depende sólo de $x$. En efecto como $M=p(x)y-q(x)$ y $N=1$.
 \[\frac{\mu'}{\mu}=\frac{\partial M/\partial y-\partial N/\partial x}{N}=p(x).\]
 Entonces $\mu(x)=e^{\int pdx}$ es factor integrante. Luego si multiplicamos por $\mu$ en \eqref{eq:lineal2},  la expresión  es exacta.
 \[e^{\int pdx}dy+p(x)e^{\int pdx}ydx=q(x)e^{\int pdx}dx.\]
Podemos identificar rápidamente, sin necesidad de hacer cálculos, el correspondiente potencial.
 \[d\left(e^{\int pdx}y\right)=d\left(\int q(x)e^{\int p} dx \right).\]
Integrando
\[e^{\int pdx}y=\int e^{\int pdx}q(x)dx+C.
 \]
Entonces
\begin{empheq}[box=\tcbhighmath]{equation}\label{SolGenLin}
  y=e^{-\int pdx}\left\{\int e^{\int pdx}q(x)dx+C\right\} 
\end{empheq}


\begin{ejemplo}{}{} Resolver $y'+y/x=3x$.
 \end{ejemplo}


\noindent\textbf{Solución.} En la práctica, para evitar recordar fórmulas, se suele repetir el procedimiento que llevo a la fórmula \eqref{SolGenLin}, ahora, dado la cercanía
de su derivación, vamos a usarla  de manera directa. La solución general es

\[\begin{split} y(x)&=e^{-\int\frac{1}{x}dx}\left\{\int e^{\int\frac{1}{x}dx}3xdx+C\right\}\\
   &=\frac{1}{|x|}\left\{\int |x| 3xdx+C\right\}\\
   &=x^2+\frac{C}{|x|}\\
   &=x^2+\frac{C}{x}\\
  \end{split}
\]




\section{Reducción de orden}

Algunas ecuaciones de segundo orden
\begin{empheq}[box=\tcbhighmath]{equation}\label{Gen2Or}
 F(x,y,y',y'')=0
\end{empheq}

se pueden reducir a una de primer orden. Por ejemplo si $F$ no depende de $y$. Es decir la ecuación es
\begin{empheq}[box=\tcbhighmath]{equation}
 F(x,y',y'')=0
\end{empheq}



Aquí introducimos la nueva variable dependiente $\boxed{p=y'}$, que resuelve
\[F(x,p,p')=0.\]
Que es una ecuación de primer orden. Supuesto que la podemos resolver y encontrar una solución general para  $p$, tendremos
\begin{empheq}[box=\tcbhighmath]{equation}
 y=\int pdx+C
\end{empheq}


Es la solución general de la ecuación de segundo orden.


Si la ecuación general de segundo orden \eqref{Gen2Or} no depende de $x$, es decir tenemos
\boxed{F(y,y',y'')=0}
entonces nuevamente usaremos $\boxed{p=y'}$ como nueva variable depeniente
pero también  $\boxed{y}$ como nueva variable independiente. Como
\[y''=p'=\frac{dp}{dx}=\frac{dp}{dy}\frac{dy}{dx}=\frac{dp}{dy}p\]
La ecuación se reduce a la siguiente ecuación de primer orden
\begin{empheq}[box=\tcbhighmath]{equation}\label{RedOrdSinInd}
 F\left(y,p,\frac{dp}{dy}p\right)=0
\end{empheq}

% 


\section{Aplicaciones}


\subsection{Velocidad de escape}\label{pag:vel_esc}


\begin{mdframed}[style=MiEstilo]\relax%
 \textbf{Velocidad de escape.} Que velocidad hay que imprimirle a un proyectil que es lanzado verticalmente desde la superficie de la Tierra si nuestra pretensión
es que el proyectil se escape al infinito. La velocidad más chica con esta cualidad se llama
\href{http://es.wikipedia.org/wiki/Velocidad_de_escape}{velocidad de escape}.
\end{mdframed}


\begin{wrapfigure}[15]{R}{5cm}

\begin{pspicture}(0,0)(4,4.3)
\psset{unit=1cm}
\pscircle[linestyle=none,fillstyle=solid,fillcolor=color8](1.5,1.5){1.5}
\psline(1.5,1.5)(2,2.91)
\psline(1.5,1.5)(2.5,4.32)
\pscircle*(2.5,4.32){.1}
\rput(1.32,1.55){\rotatebox{70}{\psline[linewidth=.01](0,-.2)(0,.2)
\psline[linewidth=.01](0,0)(0.6,0)
\psline[linewidth=.01](0.95,0)(1.5,0)
\psline[linewidth=.01](1.5,-.2)(1.5,.2)
\psline[linewidth=.01](1.5,0)(2.1,0)
\psline[linewidth=.01](2.45,0)(3,0)
\psline[linewidth=.01](3,-.2)(3,.2)
}}
\rput(1.6,2.3){\rotatebox{70}{$R$}}
\rput(2.1,3.7){\rotatebox{70}{$x$}}
\rput(.7,1.8){$M$}
\rput(2.85,4.32){$m$}
\end{pspicture}
\caption{Velocidad de Escape}\label{fig:vel_esc}
\end{wrapfigure}

 \noindent\textbf{Solución.} Para resolver este problema hay que tomar en consideración la
\href{http://es.wikipedia.org/wiki/Ley_de_gravitación_universal}{Ley de gravitación universal}
de Newton. En la parte que nos interesa, esta Ley afirma
que el módulo de la fuerza de gravedad que se ejercen entre si dos cuerpos de masa $m_1$ y $m_2$ separados una distancia $r$ es proporcional al producto de las masas
e inversamente proporcional al cuadrado de las distancia que los separa. Vale decir
\[|F|=G\frac{m_1m_2}{r^2},\]
donde $G$ es la constante de proporcionalidad.
Cuando los cuerpos no son puntos masa, sino esferas de dendsidad uniforme la distancia de separación hay que medirla entre los centros de masa de los cuerpos.



Hay que aclarar que usando el 
\href{https://docs.google.com/file/d/0B80iJ0HgObRRWll6MlJFSjFNMGc/edit}{Principio conservación energía mecánica}\linebreak podemos resolver
el problema de una manera más simple. Incluso podemos ver que la suposición de que el tiro es vertical no es necesaria, es decir la velocidad de escape es la misma aunque
el tiro sea oblicuo. Discutiremos esa solución durante la clase. Lamentablemente,  esta solución no usa ecuaciones diferenciales.
Vamos a dar una solución, quizás un poco más complicada, pero que invoca las técnicas
discutidas.


Supondremos a la Tierra una esfera de radio $R$, masa $M$ y su centro de masa en el centro de la esfera.   Al proyectil lo supondremos un punto masa
con masa $m$ y su posición en el momento $t$, denotada $x=x(t)$, la mediremos sobre un eje vertical con origen en la
superficie de la Tierra.  Todo como está indicado en la figura \ref{fig:vel_esc}.
Luego la distancia Tierra-proyectil será igual a $R+x$ donde $x$ es la posición del proyectil. Utilizando la 
\href{http://es.wikipedia.org/wiki/Leyes_de_Newton\#Segunda_ley_de_Newton_o_ley_de_fuerza}{Segunda ley de Newton},
$F=ma$, obtenemos
\[mx''(t)=-\frac{GMm}{(R+x)^2}.\]
Es una ecuación de la forma
\[F(x,x',x'')=0.\]
Con variable dependiente $x$ e independiente $t$. Pero  $F$ no depende de $t$ y por consiguiente, como vimos, se puede convertir en una ecuación de primer orden
tomando como nuevas variables: 1) independiente $x$ 2) dependiente $v=x'$. En estas variables
\[\frac{d^2x}{dt^2}=\frac{dv}{dt}=\frac{dv}{dx}\frac{dx}{dt}=v\frac{dv}{dx}.\]
La ecuación se convierte en
\[v\frac{dv}{dx}=-\frac{GM}{(R+x)^2}\Longrightarrow vdv+\frac{GM}{(R+x)^2}dx=0.\]
Que es una ecuación en variables separables y también es exacta. Usaremos la técnica discutida para
ecuaciones exactas \marginpar{Siempre las ecuaciones en variables separables son exactas pues se escriben de la forma
\[{ \scriptscriptstyle M(x)dx+N(y)dy=0,}\]
por consiguiente tienen potencial
\[{ \scriptscriptstyle f=\int M(x)dx +\int N(y)dy}\]}, los que nos indica que la solución general se expresa de la siguiente forma
\begin{equation}\label{energia}
 \frac{v^2}{2}-\frac{GM}{(R+x)}=E=\text{cte}.
\end{equation}


La igualdad anterior es precisamente  consecuencia directa del \href{https://docs.google.com/file/d/0B80iJ0HgObRRWll6MlJFSjFNMGc/edit}{Principio conservación energía mecánica},
La hemos deducido como consecuencia de que la ecuación era exacta. 
Sea $v_0$ la velocidad inicial para $t=0$. 
Como $E$ es constante y $x=0$ en $t=0$ debemos tener
\begin{equation}\label{energia_cero}
 E=\frac{v_0^2}{2}-GM/R
\end{equation}
Como $v^2\geq 0$ y por \eqref{energia} y \eqref{energia_cero}.
\[-\frac{GM}{(R+x)}\leq\frac{v^2}{2}-\frac{GM}{(R+x)}=\frac{v_0}{2}-\frac{GM}{R}\]
Queremos encontrar $v_0$ tal que $x\to\infty$. Si tomamos
límite cuando $x\to\infty$ en la expresión anterior obtenemos
\[0\leq \frac{v_0^2}{2}-GM/R\]
De aquí deducimos que el valor mínimo de velocidad de escape es 
$\sqrt(2*GM/R)$.

\subsection{Curvas de persecución}

\begin{mdframed}[style=MiEstilo]\relax%
\textbf{Curvas de persecución.} Supongamos que un conejo se mueve sobre una
línea recta con rapidez uniforme $a$ y de un punto por fuera de la recta parte un perro que lo
persigue con rapidez uniforme $b$. Encontrar la trayectoria del perrro.
\end{mdframed}







\begin{figure}[h]
\begin{center}
\animategraphics[controls, scale=.4]{15}{pursuit/pursuit-}{0}{40}
 \caption{\small Persecución en un pentágono estrellado.
\href{https://www.sciencenews.org/article/art-pursuit-3}{Art of Pursuit},
Ivars Peterson
}
\end{center}
\end{figure}



\begin{wrapfigure}[15]{R}{5cm}



\scalebox{.6} % Change this value to rescale the drawing.
{
\begin{pspicture}(0,-4.51)(8.68,4.51)
\psline[linewidth=0.04cm,arrowsize=0.05291667cm 2.0,arrowlength=1.4,arrowinset=0.4]{->}(0.8,-4.49)(0.8,4.49)
\psline[linewidth=0.04cm,arrowsize=0.05291667cm 2.0,arrowlength=1.4,arrowinset=0.4]{->}(0.0,-3.49)(8.66,-3.53)
\psbezier[linewidth=0.04](2.52,2.11)(2.52,1.33)(3.44,-3.49)(7.34,-3.51)
\psline[linewidth=0.04cm,linestyle=dashed,dash=0.16cm 0.16cm](3.4,-0.77)(0.82,3.09)
\rput(1.9,3.12){$C=(0,at)$}
\rput(0.3,4){$y$}
\rput(8.66,-3.8){$x$}
\rput(2.2,-0.77){$P=(x,y)$}
\rput(7.34,-3.8){$(c,0)$}
\rput(5.4,-1.8){$s$}
\psdots[dotsize=0.12](3.36,-0.75)
\psdots[dotsize=0.124](3.34,-0.75)
\psdots[dotsize=0.154](3.34,-0.73)
\psbezier[linewidth=0.012](7.34,-3.13)(6.34,-3.15)(4.66,-1.79)(3.92,-0.45)
\psline[linewidth=0.0139999995cm](4.16,-0.31)(3.64,-0.63)
\psline[linewidth=0.0139999995cm](7.34,-2.85)(7.34,-3.39)
\psdots[dotsize=0.012](0.84,3.09)
\psdots[dotsize=0.012](0.84,3.09)
\psdots[dotsize=0.042](0.8,3.05)
\psdots[dotsize=0.103999995](0.82,3.05)
\psdots[dotsize=0.162](0.84,3.09)
\end{pspicture} 
}
\caption{Curva persecución}\label{fig:curva_per}
\end{wrapfigure}



Supongamos que el perro parte del punto $(c,0)$, el conejo de $(0,0 )$ y se mueve 
en línea recta en la dirección positiva  del eje $y$.
Vamos a suponer que el perro sigue la trayectoria donde
la tangente a su movimiento, en un momento dado, 
intersecta a la posición del conejo correspondiente a ese momento.


Pasado un tiempo $t$, el conejo estará en el punto $(0,at)$ y el perro en un punto de su trayectoria que forma un arco de
 longitud $s=bt$ hasta el punto $(c,0)$. Ese punto, donde está el perro, lo denotaremos $(x,y)$. Como hemos supuesto que la tangente a la trayectoria del perro en $(x,y)$ pasa
 por la posición del conejo $(0,at)$ se debe cumplir que
 \begin{equation}\label{eq:persec}\frac{dy}{dx}=\frac{y-at}{x}\Longrightarrow xy'-y=-at.\end{equation}
 En esta ecuación hay tres variables, $t$ , $x$ e $y$. No hemos definido cuales 
 son independiente y cuales depenientes.   Generalmente el tiempo $t$ se considera  
 variable  indepeniente, pero en la expresión de arriba aparece la derivada de $y$ 
 respecto a $x$.
 Claramente deberíamos eliminar una de las variables. Conviene eliminar $t$, dado que 
 al no aparecer en la derivación no tendremos que hacer un cambio de variables allí, 
 donde siempre es un poco más engorroso.   Por otra parte, la intuición del problema, 
 nos dice que  a cada $t$ corresponde uno, y sólo un, $x$\footnote{De lo contrario el 
 perro no se hubiera movido en la dirección horizontal entre dos momentos, lo que es absurdo}, lo que indica que $t$ es función de $x$ y por consiguiente es de esperar poder escribir la ecuación \eqref{eq:persec} en términos de $x$ e $y$.   Tener en cuenta que no es  razonable en matemática, como en la política,  pensar que lograremos tener un beneficio (menos variables) sin pagar algún precio, pues, como dice el dicho,
 ``Cuando la limosna es grande hasta el santo desconfía'' .
 En este caso, el costo que pagaremos
 es incrementar el orden de la ecuación.
 Como hemos dado algunas técnicas de  resolver ecuaciones de orden dos, quizás estemos en condiciones de pagar este precio.

Para eliminar $t$ de la ecuación \eqref{eq:persec}, derivamos $\eqref{eq:persec}$ respecto a $x$, para obtener
\[xy''=-a\frac{dt}{dx}.\]
Como $ds/dt=b$
\[\frac{dt}{dx}=\frac{dt}{ds}\frac{ds}{dx}=-\frac{\sqrt{1+y'(x)^2}}{b}.\]
Hemos usado la relación $s=\int_x^c\sqrt{1+y'^2}dx$.
 Entonces
 \begin{empheq}[box=\tcbhighmath]{equation}xy''=\frac{a\sqrt{1+y'(x)^2}}{b}.
   \end{empheq}


Que es una ecuación que no contiene $y$. De modo que usando $p=y'$ como variable dependiente reducimos el orden de la ecuación. Nos queda
\[\frac{dp}{\sqrt{1+p^2}}=\frac{a}{b}\frac{dx}{x}.\]
Que es una ecuación en variable separables. Tomando la integral definida entre $c$ y $x$, y considerando que si $x=c$ entonces $p=0$, tenemos
\[\ln\left(p+\sqrt{1+p^2}\right)=\ln\left( \frac{x}{c}\right)^{\tfrac{a}{b}}.\]
 Si despejamos $p$ conseguimos
 \begin{empheq}[box=\tcbhighmath]{equation}p=\frac{dy}{dx}=\frac{1}{2}\left[\left(\frac{x}{c}\right)^{a/b}-\left(\frac{c}{x}\right)^{a/b}\right]. 
 \end{empheq}

Para hallar $y$ hay que recordar que $y'=p$ e $y(1)=0$.

Usemos \texttt{SymPy} para completar este cálculo y hacer los gráficos. El siguiente código evalúa la integral, halla la constante de integración para que $y(1)=0$ y grafica.


Los resultados son:

\begin{tabular}{m{4cm} m{4cm} m{4cm}}
\begin{center}
\includegraphics[scale=.3]{imagenes/perse_a_1_b_2.png}

$a<b$
\end{center}
&
\begin{center}
\includegraphics[scale=.3]{imagenes/perse_a_1_b_1.png}

$a=b$
\end{center}
&
\begin{center}
\includegraphics[scale=.3]{imagenes/perse_a_2_b_1.png}

$a>b$
\end{center}

\end{tabular}



Lo anterior constituye un ejemplo de lo que se conoce como\marginpar{
 \includegraphics[scale=.4]{imagenes/persecucion2.jpg}\\
Persecución en un pentágono estrellado. Extraído de\\
\href{https://www.sciencenews.org/article/art-pursuit-3}{Art of Pursuit},
} \href{https://en.wikipedia.org/wiki/Pursuit_curve}{curva de persecución}. Este es un tema muy interesante que tiene varias generalizaciones, por ejemplo el \href{https://en.wikipedia.org/wiki/Mice_problem}{problema de los ratones} donde se colocan en cada vértice de un polígono ratones cada uno de los cuales persigue al vecino en sentido antihorario (u horario, da lo mismo). Se consiguen patrones geométricos myu bellos




\section{Oscilador armónico}\label{resortito}




Un \href{http://es.wikipedia.org/wiki/Oscilador_armónico}{oscilador armónico} es el más simple de los sistemas físicos vibratorios. Podemos definirlo como un sistema
elástico que obedece a la \href{http://es.wikipedia.org/wiki/Ley_de_Hooke}{Ley de elasticidad de Hooke}, en honor a su descubridor
\href{http://es.wikipedia.org/wiki/Robert_Hooke}{Robert Hooke (1635-1703)}.
\marginpar{
\includegraphics[scale=.15]{imagenes/Hooke.JPG}\\
Robert Hooke
}

Suele citarse al \href{http://es.wikipedia.org/wiki/Resorte}{resorte} como un ejemplo familiar de oscilador armónico. Esto debido a que, cuando las oscilaciones de un resorte son
pequeñas,  se satisface aproximadamente la Ley de elasticidad de  Hooke. Esta ley  afirma que la fuerza que ejerce un resorte sobre una masa $m$ conectada a él por uno
de sus extremos es proporcional en magnitud al desplazamiento
del resorte desde la posición de equilibrio. Además la fuerza de elasticidad actúa en sentido opuesto al desplazamiento.

\begin{figure}[h]
 \begin{center}
  \animategraphics[controls, scale=.4]{15}{resorte/resorte-}{0}{47}
 \caption{Resorte}\label{fig:resortito}
 \end{center}
 \end{figure}

 Supongamos que tenemos un resorte, en unos de sus extremos fijado en una pared y unido a una masa $m$ por el otro extremo. Supongamos que no actúa otra fuerza
 sobre la masa mas que la del resorte. Ver la animación de la figura \ref{fig:resortito}.  Pongamos un eje de coordenadas en la dirección del movimiento, con origen en la posición de equilibrio
 del resorte. Esta posición es el punto donde el resorte no ejerce fuerza. Supongamos que la dirección positiva es la dirección donde el resorte se expande. Denotemos
 por $x(t)$ la posición de la masa en el momento $t$.    Entonces según la Segunda Ley de Newton y la Ley de Elasticidad de Hooke, tenemos que

 \begin{empheq}[box=\tcbhighmath]{equation}mx''(t)=-kx(t).\label{eq:resorte}
 \end{empheq}


La constante de proporcionalidad $k$ se llama \href{http://es.wikipedia.org/wiki/Rigidez}{constante elástica}. La ecuación \eqref{eq:resorte} se denomina la
ecuación del oscilador armónico o ecuación del resorte.

La ecuación del oscilador armónico se escribe $0=f(t,x,x',x'')$, donde \linebreak $f(t,x,y,z)=kx+mz$ es independiente de $t$. Podemos intentar usar $x$ como variable
independiente y $z=x'$ como dependiente. Como vimos  $x''(t)=dz/dt=dz/dx z$. Así la ecuación queda
\[\begin{split}
   m\frac{dz}{dx}z=-kx &\Longrightarrow mzdz=-kxdx\Longrightarrow m\frac{z^2}{2}=-k\frac{x^2}{2}+C_1\\
   &\Longrightarrow z=\pm\sqrt{-\frac{k}{m}x^2+C_1}\\
   &\Longrightarrow x'(t)=\pm\sqrt{-\frac{k}{m}x^2+C_1}.\\
  \end{split}
\]
Debe ser $C_1\geq 0$ de lo contrario el dominio de la función sería vacío. Nos queda una nueva ecuación para $x'$.
Esta ecuación es en variables separables
\[ \frac{dx}{\sqrt{-\frac{k}{m}x^2+C_1}}=dt.
\]
 Integrando
\[\begin{split}
   t+C_2
   &=\int \frac{dx}{\sqrt{-\frac{k}{m}x^2+C_1}}\\
   &= \sqrt{\frac{1}{C_1}} \int \frac{dx}{\sqrt{-\frac{k}{C_1m}x^2+1}} \\
   &= \sqrt{\frac{m}{k}} \int \frac{du}{\sqrt{1-u^2}}\quad \left(\text{haciendo } u=\sqrt{\frac{k}{C_1m}}x\right)\\
   &=\sqrt{\frac{m}{k}}\arcsen u.
  \end{split}
\]
Entonces
\[\begin{split}
    x=\frac{C_1m}{k}u &=\frac{C_1m}{k}\sen \left(\sqrt{\frac{k}{m}}(t+C_2)\right)\\
    &=\boxed{C_3\sen \sqrt{\frac{k}{m}}t+C_4\cos \sqrt{\frac{k}{m}}t}.
  \end{split}
 \]
Que es la solución general\footnote{Esta afirmación se justificará en el capítulo 3}  de la ecuación del oscilador armónico. Como vemos el movimiento es oscilatorio con frecuencia
\[\boxed{f=\sqrt{\frac{k}{m}} }.\]
En particular, no importan las condiciones iniciales, la frecuencia es siempre la misma.






\section{EDP, método características}

Saber resolver ecuaciones ordinarias de primer orden nos permite resolver algunas ecuaciones en derivadas parciales de primer orden.  Vamos a exponer este punto a través del \href{https://en.wikipedia.org/wiki/Method_of_characteristics}{método de características}.

Para tener un problema bien planteado con  ecuaciones en derivadas parciales no es suficiente conocer el valor de la función en un punto (como en una EDO de primer orden). Una condición típica extra, para lograr este propósito,  es consignar el valor de la función a lo largo de una curva, que por simplicidad asumiremos que es una recta .

\begin{ejemplo}{} Resolver
\begin{equation}\label{eq:EDP_gral_1orden}
  \left.\begin{array}{l}
  a(x,y)u_x+b(x,y)u_y=c(x,y,u)\\
  u(x,0)=f(x)\\
\end{array}\right\}
\end{equation}
Aquí $x,y$ son variables independientes y $u$ dependiente. La primera línea es la ecuación diferencial, que incluye derivadas parciales de la incognita y la segunda línea podemos denominarla condición inicial (pensando que la variable $y$ representa tiempo).
\end{ejemplo}

El método de características consiste en encotrar $u$ a lo largo de las soluciones de la EDO

\begin{empheq}[box=\tcbhighmath]{equation}\frac{dy}{dx}=\frac{b(x,y)}{a(x,y)}\label{eq:ecua_caract}
\end{empheq}
Una solución de esta ecuación es normalmente una curva en el plano  $x,y$. La idea es que las soluciones de \eqref{eq:ecua_caract} forman un flia uniparamétrica
 de curvas que llena una parte $\Omega$ del plano $x,y$.  Así terminamos conociendo el valor de $u$ sobre este conjunto $\Omega$. Para que \eqref{eq:ecua_caract} tenga sentido debemos tener $a\neq 0$. De todas formas si $a=0$ podemos invertir los roles de $x$ e $y$.

Supongamos $y(x)$ solución de \eqref{eq:ecua_caract}, entonces pongamos por abuso de notación $u(x)=u(x,y(x))$. Se tiene que

\begin{empheq}[box=\tcbhighmath]{equation}\frac{du}{dx}=u_x+u_yy'=u_x+u_y\frac{b}{a}=\frac{c(x,y(x),u(x))}{a(x,y)}\label{eq:ecua_caract2}
 \end{empheq}

Que es otra ecuación ordinaria. Podemos escribir  \eqref{eq:ecua_caract} y \eqref{eq:ecua_caract2} en una ecuación más simétrica
\begin{empheq}[box=\tcbhighmath]{equation}\frac{du}{c}=\frac{dx}{a}=\frac{dy}{b}\label{eq:caract}
 \end{empheq}


Estas ecuaciones se llaman \emph{ecuaciones características}. Las soluciones de estas ecuaciones son un familia de curvas que suele llenar un conjunto abierto de  $\mathbb{R}^3$.  La gráfica de la solución se obtiene eligiendo entre estas curvas las que pasan por los puntos de la gráfica de $u$ especificados en la condición inicial, es decir $(x,0,f(x))$. Esto, en los casos favorables, forma una superficie que es la gráfica de la solución. La proyección de estas curvas en el plano $x,y$ se denominan \emph{caracterísitcas}. Son la familia de soluciones de $y'=b/a$.


\begin{ejemplo}{} Resolver
\begin{equation}\label{eq:EDP_gral_1orden}
  \left.\begin{array}{l}
  u_x+u_y=yu\\
  u(x,0)=f(x)\\
\end{array}\right\}
\end{equation}
\end{ejemplo}
En este caso
\[\frac{dy}{dx}=\frac{b}{a}\Rightarrow y'=1\Rightarrow y(x)=x+\mu,\quad\mu=\hbox{cte}.\]
Entonces
\[\frac{du}{dx}=\frac{c}{a}\Rightarrow u'=yu=(x+\mu)u\Rightarrow \ln|u|=\frac{x^2}{2}+\mu x+C(\mu).\]
Notar que la nueva constante de integración $C(\mu)$ debe depender de la primera $\mu$. Entonces
\[u=\pm e^{\frac{x^2}{2}}e^{(y-x)x}e^{C(y-x)}.\]
Ahora
\[u(x,0)=f(x)\Rightarrow f(x)=\pm e^{\frac{x^2}{2}}e^{-x^2}e^{C(-x)}\Rightarrow C(\mu)=\frac{\mu^2}{2}+\ln|f(-\mu)|.\]
Entonces
\[u(x,y)=\pm e^{\frac{x^2}{2}}e^{(y-x)x}e^{\frac{(y-x)^2}{2}+\ln|f(x-y)|        }
=e^{\frac{y^2}{2}}f(x-y).\]


\begin{subappendices}

\section{La braquistócrona}


\begin{mdframed}[style=MiEstilo]\relax% 
 Dados dos puntos $A$ y $B$ en las proximidades de la superficie terrestre, uno mas abajo respecto al suelo que el otro,
queremos diseñar el tobogán óptimo entre los dos,
esto es el tobogán que nos lleve de $A$ hasta $B$ en el menor tiempo. La curva solución a este problema se llama curva
\href{http://es.wikipedia.org/wiki/Curva_braquistócrona}{braquistócrona} (braquistos - el más corto, cronos - tiempo).
\end{mdframed}


\begin{center}
\animategraphics[controls,scale=.5]{15}{braquis/braquis-}{0}{105}
\end{center}


 Este problema fue resuelto por primera vez por Johann Bernoulli y es unos de los problemas precursores de la rama de las matemáticas que se denomina
\href{http://es.wikipedia.org/wiki/Cálculo_variacional}{cálculo de
variaciones}. Vamos a dar la solución del problema encontrada por  \marginpar{
    \includegraphics[scale=.2]{imagenes/Bernoulli.jpg}\\
    Johann Bernoulli (1667-1748)
}Johann Bernoulli que es muy elegante y está basada en un resultado de óptica llamado
 el \href{http://es.wikipedia.org/wiki/Principio_de_Fermat}{Principio de Mínimo Tiempo} de \href{http://es.wikipedia.org/wiki/Fermat}{Fermat}\marginpar{
    \includegraphics[scale=.3]{imagenes/Fermat.jpg}\\
    Pierre de Fermat (1601-1665)
}.

\begin{mdframed}[style=MiEstilo]\relax%
\href{http://es.wikipedia.org/wiki/Principio_de_Fermat}{\textbf {Principio de Mínimo Tiempo}} \textbf{de} \href{http://es.wikipedia.org/wiki/Fermat}{\textbf{Fermat}}
 La luz sigue para ir de un punto a otro el recorrido que minimiza el tiempo.
\end{mdframed}


 La primera impresión   es que ese reccorrido debería ser la línea recta. Si embargo esto no es así debido a que la velocidad de la luz cambia
de acuerdo al \href{http://es.wikipedia.org/wiki/Velocidad_de_la_luz_en_un_medio_material}{medio que atraviesa} .
La velocidad de la luz en el vacío es 299.792,458km/h y en el diamante 124.034,943 km/h. La velocidad de la luz cambia no sólo con la sustancia sino con sus cualidades,
como la densidad.

 Si la luz se mueve dentro de un medio homogéneo, el camino que sigue es la línea recta. Esto ya no es más así cuando la luz cambia de medio de propagación. Por ejemplo
cuando pasa del aire al vidrio.

\begin{wrapfigure}[12]{R}{5cm}

\begin{pspicture}(0,0)(5,4)
\psset{unit=1cm}
\psframe[linestyle=none,fillstyle=solid,fillcolor=color8](0,.3)(5,2)
\psline(0,2)(5,2)
\psline(0.5,2)(0.5,4)
\psline(0.5,3.7)(3.5,2)(4.7,0.3)
\psline(4.7,0)(4.7,2)
\psline[linestyle=dashed](3.5,0)(3.5,4)
\rput(3.75,1.3){$\alpha_2$}
\psarc{-}(3.5,2){1}{270}{306}
\rput(3.3,2.3){$\alpha_1$}
\psarc{-}(3.5,2){0.7}{90}{150}
\rput(1.1,3.8){$(a,0)$}
\rput(4.3,0){$(c,b)$}
\rput(3.3,1.8){$x$}
\pscircle*(0.5,3.7){.1}
\pscircle*(4.7,0.3){.1}

\end{pspicture}
\caption{Refracción de la luz}
\end{wrapfigure}
 Supongamos que la luz une los puntos $A$ y $B$ del plano y en el camino atravieza de un medio a otro, siendo la velocidad
 de la luz en cada uno de ellos $v_1$ y $v_2$.  Supongamos que $A=(a,0)$ y $B=(c,b)$ y el eje $x$ es la frontera entre los medios.

Como sabemos que mientras se mueva en un medio homogéneo la luz sigue en línea recta,
el tiempo que emplea la luz para ir $A$ a $B$ es
 \[t=\frac{\sqrt{a^{2} + x^{2}}}{v_{1}} + \frac{\sqrt{{\left(c - x\right)}^{2} + b^{2}}}{v_{2}}\]
Para determinar la trayectoria es suficiente encontrar $x$, el punto donde la luz choca con la interfaz entre los medios.
El principio de Fermat afirma que el tiempo es mínimo de modo que hallaremos un punto crítico de $t$ respecto a $x$.
\[ \frac{dt}{dx}=-\frac{c - x}{\sqrt{{\left(c - x\right)}^{2} + b^{2}} v_{2}} +
\frac{x}{\sqrt{a^{2} + x^{2}} v_{1}}=\frac{\sen\alpha_1}{v_1}-\frac{\sen\alpha_2}{v_2} \]
Deducimos que en un punto crítico
\begin{empheq}[box=\tcbhighmath]{equation}\frac{\sen\alpha_1}{v_1}=\frac{\sen\alpha_2}{v_2}\label{eq:ley_snell1}
\end{empheq}

que se denomina \href{http://es.wikipedia.org/wiki/Ley_de_Snell}{Ley de Snell}. El punto crítico es mínimo pues $\left.\frac{dt}{dx}\right|_{x=0}=-\tfrac{c}{bv_2}<0$ y
$\left.\frac{dt}{dx}\right|_{x=c}=\tfrac{c}{\sqrt{c^2+a^2}v_1}>0$.

 A la razón entre la velocidad de la luz dentro de un determinado medio y la velocidad de la luz en el vacio se lo denomina
\href{http://es.wikipedia.org/wiki/Índice_de_refracción}{índice de refracción} y se lo denota con la letra $n$. La Ley de Snell se la suele escribir
\[\boxed{n_1\sen\alpha_1=n_2\sen\alpha_2 }.\]
Que pasa si la luz atraviesa un medio que va cambiando de manera continua de índice de refracción. Por ejemplo, el índice de refracción en la atmósfera
va cambiando de manera continua con la altitud respecto a la superficie terrestre, ya que la densidad del aire va cambiando con la altitud. La Ley de
Snell en este caso es
\[\boxed{\frac{\sen\alpha}{v}=\text{cte}}\]
Aquí el ángulo $\alpha$ y la velocidad $v$ cambian respecto a alguna variable/s real/es, por ejemplo la altitud.


¿Que tienen en común el recorrido de la luz y la braquistócrona? Bernoulli se dió cuenta que la situación en los dos casos es la misma, ya que en los dos casos
se trata de minimizar el tiempo del recorrido. De modo que la braquistócrona también tiene que satisfacer la Ley de Snell.
 Ahora supongamos un sistema de coordenadas con origen en el punto $A$, inicial del recorrido.  Además supongamos que el movil  parte
del reposo. Con estas suposiciones $x(t_0)=0$ y $v(t_0)=0$. Por la conservación de la energía
\[\frac{m}{2}|v(t)|^2=-mgy(t)=mg|y(t)|.\]
 Así por la ley de Snell
\[\frac{\sen\alpha}{|v(t)|}=\frac{\sen\alpha}{\sqrt{2g|y|}}=c=\text{ cte}.\]
En \eqref{cos_alpha} habíamos expresado el $\cos\alpha$ (en realidad del ángulo opuesto por el vértice, pero es igual) mediante la derivada. Luego
\[\sen\alpha=\sqrt{1-\cos^2\alpha}=\frac{1}{\sqrt{1+y'(x)^2}}.\]
Entonces tenemos
\[\sqrt{2g|y|}\sqrt{1+y'(x)^2}=c=\hbox{ cte}\]
Despejando llegamos a la ecuación diferencial
\[\boxed{\sqrt{\frac{y}{c-y}}y'=1}.\]
Es una ecuación con variables separables. La constante $c$ no tiene el mismo valor que en la ecuación anterior.

La solución se obtiene resolviendo la integral
\[x=\int dx=\int \sqrt{\frac{y}{c-y}}dy.\]
lo que no es tan sencillo. Hacemos el cambio de variables
\[\sqrt{\frac{y}{c-y}}=\tan\phi\Longrightarrow y=c\sen^2\phi\Longrightarrow dy=2c\sen\phi\cos\phi d\phi.\]
Luego
\[x=2c\int\sen^2\phi d\phi=\frac{c}{2}\left(2\phi-\sen 2\phi\right)+C_1.\]
Como tiene que pasar por $x=0$ e $y=0$ debe ser $C_1=0$.  Tenemos que
 \[\left\{\begin{array}{l l l}
	      y&=c\sen^2 \phi&=\frac{c}{2}(1-\cos2\phi)\\
	      x&=\frac{c}{2}(2\phi-\sen2\phi)\ &\\
          \end{array}\right.
\]
Conviene llamar $2\phi=\theta$ y $a=c/2$
 \[\left\{\begin{array}{l l }
	      y&= a(1-\cos\theta)\\
	      x&=a(\theta-\sen\theta)\
          \end{array}\right.
\]
Que son la ecuaciones paramétricas de una curva conocida con el nombre de \href{http://es.wikipedia.org/wiki/Cicloide}{cicloide}. Podemos usar \texttt{SymPy} para graficar esta curva
\begin{pyverbatim}
theta=symbols('theta')
from sympy.plotting import *
plot_parametric(theta-sin(theta),1-cos(theta),(theta,0,10*pi))
\end{pyverbatim}

\begin{figure}[h]
\begin{center}
\includegraphics[scale=.3]{imagenes/cicloide.png}
\end{center}\caption{Cicloide}\label{fig:cicloide}
\end{figure}


\begin{figure}[h]
\begin{center}
\animategraphics[controls,scale=.5]{15}{cicloide/cicloide-}{0}{31}
\end{center}\caption{Cicloide}\label{fig:cicloide}
\end{figure}


Si in tentamos resolver las ecuaciones con \texttt{SymPy} el resultado no es muy alentador.

\begin{pyverbatim}
x,c=symbols('x,c')
y=Function('y')(x)
MiEcua=Eq(y.diff(x),sqrt((c-y)/y))
f=dsolve(MiEcua,y,hint='separable')
\end{pyverbatim}

\textbf{Resultado:}
\[
 \begin{cases} - i \sqrt{c} \sqrt{-1 + \frac{1}{c} y{\left (x \right )}} \sqrt{y{\left (x \right )}} - i c \operatorname{acosh}{\left (\frac{1}{\sqrt{c}} \sqrt{y{\left (x \right )}} \right )} & \text{for}\: \left\lvert{\frac{1}{c} y{\left (x \right )}}\right\rvert > 1 \\ \- \frac{\sqrt{c} \sqrt{y{\left (x \right )}}}{\sqrt{1 - \frac{1}{c} y{\left (x \right )}}} + c \operatorname{asin}{\left (\frac{1}{\sqrt{c}} \sqrt{y{\left (x \right )}} \right )} + \frac{y^{\frac{3}{2}}{\left (x \right )}}{\sqrt{c} \sqrt{1 - \frac{1}{c} y{\left (x \right )}}} & \text{otherwise} \end{cases} = C_{1} + x
\]



\section{La tautócrona}







  Vamos a ver otra propiedad notable de la ciclode. Supongamos que dejamos caer el cuerpo del reposo desde un punto intermedio, digamos en $(x_0,y_0)$. Sea  $\theta_0$
 el valor del paŕámetro $\theta$ correpondiente a este punto. ¿Cuánto tardara en llegar el cuerpo al punto mínimo de la curva que ocurre cuando $\theta=\pi$? 
% \begin{center}
% \animategraphics[controls,scale=.5]{15}{tautocrona/tauto-}{0}{79}
% \end{center}


  
  

 Tenemos
\[
 \left\{ \begin{array}{l l}
 \frac{dx}{d\theta}&=a(1-\cos\theta)\\
 \frac{dy}{d\theta}&=a\sen\theta 
 \end{array}\right.
\]
% 
Como el cuerpo ahora no parte de $(0,0)$ tendremos
\[|v|=\sqrt{2g(y_0-y)}.\]


  
Por \eqref{2ley} $ds/dt=|v|$. Si llamamos $T$ al tiempo que demanda en llegar a $\theta=\pi$, y llamamos  $s_0$ y $s_1$ a los arcos correspondientes al punto inicial
y final.  Tenemos
 \[T=\int_0^Tdt=\int_{s_0}^{s_1}\frac{dt}{ds}ds=\int_{s_0}^{s_1}\frac{1}{\sqrt{2g(y_0-y)}}ds.\]
Cambiando la variable de integración a $\theta$. Como 
\[
 \frac{ds}{d\theta}=\sqrt{\left(\frac{dx}{d\theta}\right)^2+\left(\frac{dy}{d\theta}\right)^2}=\sqrt{2}a\sqrt{1-cos\theta}.
\]
 Tenemos
\[T=\sqrt{\frac{a}{g}}\int_{\theta_0}^{\pi}\frac{\sqrt{1-\cos\theta}}{\cos\theta_0-\cos\theta}d\theta=
\sqrt{\frac{a}{g}}\int_{\theta_0}^{\pi}\frac{\sen\frac{\theta}{2}}{\cos^2\frac{\theta_0}{2}-\cos^2\frac{\theta}{2}}d\theta.
\]
Ahora hacemos la sustitución
\[u=\frac{\cos\frac{\theta}{2}}{\cos\frac{\theta_0}{2}}\Longrightarrow du=-\frac{\sen\frac{\theta}{2}}{2\cos\frac{\theta_0}{2}}d\theta.\]
Vemos que
\[
 T=2\sqrt{\frac{a}{g}}\int_0^1\frac{1}{\sqrt{1-u^2}}du.
\]
Que es una expresión independiente de $\theta_0$. En consecuencia el tiempo $T$ que demanda  el cuerpo para llegar $\theta=\pi$ es siempre el mismo no importa
desde donde se deje caer.

\begin{figure}[h]
\begin{center}
\animategraphics[controls,scale=.5]{15}{tautocrona/tauto-}{0}{79}
\end{center}\caption{Tautócrona}\label{fig:tautocrona}
\end{figure}

\section{Solución al problema del espejo con Sympy}
En este apéndice resolvemos el problema usando formas diferenciales
El submódulo diffgeom de Sympy implementa formas diferenciales. Del
sub-submodulo sympy.diffgeom.rn  importamos el objeto R2. 
Para comprender el funcionamiento del código remitimos a la 
\href{http://docs.sympy.org/latest/modules/diffgeom.html}{documentación de Sympy}.


\begin{pyblock}
from sympy import *
from sympy.diffgeom.rn import R2
eq=(-sqrt(R2.x**2+R2.y**2)+R2.x)*R2.dx+R2.y*R2.dy
M=eq.rcall(R2.e_theta)
N=eq.rcall(R2.e_r)
x,y,theta=symbols('x,y,theta')
r=symbols('r',positive=True)
subst={R2.x:r*cos(theta),R2.y:r*sin(theta),R2.theta:theta,R2.r:r}
M=M.subs(subst).simplify()
N=N.subs(subst).simplify()
mu=(N.diff(theta)-M.diff(r))/M
FactInt=exp(Integral(mu,r).doit())
phi=(M/r).integrate(theta)
g=Function('g')(r)
phi=phi+g
dsolve(phi.diff(r)-N/r,g)
phi=(M/r).integrate(theta)+r
phi
\end{pyblock}


$\py{latex(phi)}$

\end{subappendices}

  \bibliographystyle{apalike-url}
  \bibliography{diferenciales_ecuaciones,diferenciales_ecuaciones_sim}
%\end{document}

%\chapter{Teoría de Lie y ecuaciones diferenciales}

% 
\section{Introducción histórica}




\begin{quote}
<<Marius Sophus Lie fue un matemático noruego (17 de diciembre de 1842-18 de febrero de 1899) que creó en gran parte la teoría de la simetría continua, y la aplicó al estudio de la geometría y las ecuaciones diferenciales.
La herramienta principal de Lie, y uno de sus logros más grandes fue el descubrimiento de que los grupos continuos de transformación (ahora llamados grupos de Lie), podían ser
 entendidos mejor "linealizándolos", y estudiando los correspondientes campos vectoriales generadores (los, así llamados, generadores infinitesimales).
Los generadores obedecen una versión linealizada de la ley del grupo llamada el corchete o conmutador, y tienen la estructura de lo que hoy, en honor suyo, llamamos un álgebra de Lie.>>
\end{quote}
\marginpar{\includegraphics[scale=.28]{imagenes/Sophus_Lie.jpg}
}

\begin{flushright}
Wikipedia
\end{flushright}


 
\begin{quote}
<< La historia del análisis de simetrías comenzó a mediados del siglo XIX cuando S. Lie
Y F. Klein se reunieron en Berlín. Ambos matemáticos contribuyeron mucho a la teoría de las
simetrías. S. Lie presentó su famosa obra para examinar las simetrías en relación con
ecuaciones algebraicas y diferenciales. En su programa de Erlangen, Klein desarrolló las contrapartes discreta y algebraica de la aplicación de las simetrías a las funciones. S. Lie
creo un gran campo de las ecuaciones diferenciales, que fue muy útil para
clasificar las ecuaciones diferenciales de una forma nueva. La teoría desarrollada por Lie es
muy laboriosa, si se hace a mano. Esta es una de las razones por las que la aplicación de
esta teoría desapareció en la práctica de resolver problemas. Muy pocas personas usaron los procedimientos de  Lie para examinar ecuaciones diferenciales. Uno de ellos fue Birkhoff,
quien en la década de 1950 aplicó la teoría a problemas hidrodinámicos. En los últimos años,
 se prestó más atención a la teoría de Lie como uno de los métodos raros para obtener
soluciones, especialmente para ecuaciones diferenciales no lineales. Hoy el procedimiento de Lie es
accesible para su aplicación, si es usado el poder computacional del álgebra computacional. 
Los cálculos algebraicos muy extendidos hoy en día se llevan a cabo por computadoras. 
En los últimos 20 años, ha habido un enorme aumento de la potencia de  las computadoras y del desarrollo de lenguajes simbólicos, permitiendo abordar problemas en una
manera más sencilla.>>
\end{quote}
  

\begin{flushright}

Symmetry Analysis of
Differential Equations
with Mathematica\textregistered\\
\cite{GerdBaumann578}

\end{flushright}




\section{Cambios de Variables}



Vamos a seguir estudiando ecuaciones no lineales de primer orden

\boxedeq{\frac{dy}{dx}=f(x,y)}{eq:gral_orden1}
o, utilizando  la escritura en  \emph{forma diferencial}
\boxedeq{M(x,y)dx+N(x,y)dy=0.}{eq:gral_orden1_FormDif}
En el apéndice \ref{section:fromas} discutimos muy sumariamente el concepto de forma diferencial.

\begin{problema}[Cambio de variables]
 Dada la ecuación \eqref{eq:gral_orden1} o \eqref{eq:gral_orden1_FormDif} en las variables $x,y$. Queremos encontrar nuevas variables 
 \begin{equation}\label{eq:cambio_prin}
  \left \{\begin{array}{cc}
	    \hat{x}&=\hat{x}(x,y)\\
	    \hat{y}&=\hat{y}(x,y)
          \end{array}
 \right. .
 \end{equation}
tales que la ecuación se transforme en una que podamos resolver.
\end{problema}

Para no correr riesgos de perder información en la ecuación transformada, es conveniente que 
$x$ e $y$ también se expresan en función de $\hat{x}$ e $\hat{y}$, vale decir que la transformación pueda invertirse:
 \begin{equation}\label{eq:cambio_inv}
  \left \{\begin{array}{cc}
	    x&=x(\hat{x},\hat{y})\\
	    y&=y(\hat{x},\hat{y})
          \end{array}
 \right. .
 \end{equation}
\subsection{Cómputos de cambios de variables}\marginpar{Es costumbre recurrir a un abuso de notación que suele producir confusión al estudiante. Nos referimos a distinguir las derivadas $\partial y/\partial x$ y  $d y/d x$. Tratándose $y$ en el caso que nos ocupa, de una función de $x$ e $\hat{y}$,    la derivada $\partial y/\partial x$ representa la derivada parcial de $y$ respecto a su primera variable. Como $\hat{y}$ es a su vez función de $x$, por  $d y/d x$ denotamos la derivada de $y$ atendiendo a que la segunda variable también depende de $x$.}
Vamos a estudiar en primer lugar como computar cambios de variables. Empezaremos por casos más sencillos hasta ir a la situación más general.

\subsubsection{Cambio de la variable dependiente manteniendo la independiente}

 Supongamos que el conjunto de variables se relacionan  por las identidades $x=\hat{x}$ e  $y=y(x,\hat{y})$. Notar que en este caso usamos la relación inversa. Entoces, derivando $y$ respecto a $x$ y usando la regla de la cadena (que sería más apropiado llamarla regla de cambio de variables para la derivada)

\[\frac{dy}{dx}=\frac{\partial y}{\partial x}+\frac{\partial y}{\partial \hat{y}}\frac{d\hat{y}}{dx}.\]

La ecuación se convierte

\[\frac{\partial y}{\partial x}+\frac{\partial y}{\partial \hat{y}}\frac{d\hat{y}}{dx}=f(x,y(x,\hat{y})).\]
Que es una expresión sólo en $\hat{y}$ y $x$. Parece más complicada, pero en un ejemplo concreto puede ser más simple.







\begin{ejemplo}{} Hacer el cambio de variable propuesto en la  ecuación indicada
\[y=\frac{e^{\hat{y}}}{x}\quad\text{en}\quad  y'=\left[\ln(xy)\right]^2xy-\frac{y}{x}.\]
 1) Expresemos $dy/dx$ sólo con $x$, $\hat{y}$ y $d\hat{y}/dx$.
\[\frac{dy}{dx}=-\frac{e^{\hat{y}}}{x^2}+\frac{e^{\hat{y}}}{x}\frac{d\hat{y}}{dx}.\]
 2) Remplacemos $y'$ e $y$ en la ecuación
\[-\frac{e^{\hat{y}}}{x^2}+\frac{e^{\hat{y}}}{x}\frac{d\hat{y}}{dx}=\left[\ln\left(x \frac{e^{\hat{y}}}{x} \right)\right]^2x\frac{e^{\hat{y}}}{x}-\frac{\frac{e^{\hat{y}}}{x} }{x}.\]
 3) Simplifiquemos
\boxedeq{\frac{d\hat{y}}{dx}=\hat{y}^2x.}{}




\end{ejemplo}

  \emph{Importante:} Observar  que en un cambio de variable, ya se cambie la variable dependiente, independiente o ambas, las derivadas (tratándose de la velocidad de cambio de unas variables repecto a otras)  también hay que cambiarlas.


Podemos resolver los cambios de variables con \texttt{SymPy}, lo cual es muy útil por dos motivos. El primero porque nos permite hacer cambios de variables en expresiones muy grandes, resolviendo operaciones que a mano son sumamente tediosas. El segundo, y no menos importante para nosotros, es que es muy rico explorar un procedimiento enmarcándolo en un contexto muy distinto. En este caso, el procedimiento es relizar un cambio de variables y estamos explorando el mismo a través de un breve código que lo implementa en un lenguaje de programación.

\begin{sympyblock}[][frame=single]
x=symbols('x') #unico simbolo primitivo
y_n=Function('y_n')(x) #variables nuevas, funciones de x
y=exp(y_n)/x #relacion entre y, y_n
eq=Eq(y.diff(x)-(ln(x*y))**2*x*y+y/x,0) #la ecuacion
eq1=simplify(eq) # simplifica expresiones
\end{sympyblock}
Obtenemos la ecuación

\[\sympy{eq1},\]
que \texttt{SymPy} no simplifica a nuestro gusto

% 
% 
% 
\subsubsection{Cambio de la variable independiente manteniendo la dependiente}

Supongamos  $\hat{x}=\hat{x}(x)$. Usamos la relación
\[\frac{dy}{dx}=\frac{dy}{d\hat{x}}\frac{d\hat{x}}{dx}.\]
Suponiendo que la relación $\hat{x}=\hat{x}(x)$ se invierte en $x=x(\hat{x})$, todo lo que resta es sustituir  $x$ por su igual en términos de $\hat{x}$

\[\frac{dy}{d\hat{x}}=f(x(\hat{x}),y) \left[\left.\frac{d\hat{x}}{dx}\right|_{x=x(\hat{x})}\right]^{-1}.\]
Que es una expresión sólo en $\hat{x}$ e $y$. Describir el procedimiento  en general puede hacer parecer que es más dificil de lo que en realidad es en un caso concreto.



\begin{ejemplo}{} Hacer el cambio de variable en la  ecuación indicados
\[x=\cos \hat{x}\quad\text{en}\quad  -\frac{dy}{dx}+\frac{x}{\sqrt{1-x^2}}y=0.\]
1) $\hat{x}=\arcsen x$
\[\frac{dy}{dx}=\frac{dy}{d\hat{x}} \frac{d\hat{x}}{dx}  =-\frac{1}{\sqrt{1-x^2}}\frac{dy}{d\hat{x}}.\]
 2) Remplacemos $x$ e $y'$ en la ecuación
\[\frac{1}{\sqrt{1-x^2}}\frac{dy}{d\hat{x}}+ \frac{x}{\sqrt{1-x^2}}y=0\]
3) Reemplazando $x$ por $\cos(\hat{x})$ y simplificando
\[\frac{dy}{d\hat{x}}+\cos(\hat{x}) y=0.\]

\end{ejemplo}


Para hacer esto con \texttt{SymPy} (de ahora en más omitiremos la sentencia de importanción del módulo, esta operación se hace sólo una vez por sesión).

\begin{sympyblock}[][frame=single]
x,x_n=symbols('x,x_n')
x_n=acos(x)
y=Function('y')(x_n)
Ecuacion=-y.diff()+1/(sqrt(1-x**2))*y
xn=symbols('xn')
Eq=Ecuacion.subs(x,cos(xn))
\end{sympyblock}

Obtenemos la ecuación
\[\sympy{latex(Eq)}\]
Nuevamente \texttt{SymPy} no simplifica a nuestro gusto, esto ocurre aún con aquellas expresiones que parece muy evidente como se simplifican. El caso es que la operación de simplificación es por un lado  subjetiva, depende de un supuesto tácito de a que expresión se quiere arribar y por otro algunas expresiones, por ejemplo $\ln\exp(z)$, se simplifican en determinados campos numéricos y en otros no. En el caso del ejemplo $\ln\exp(z)$, la expresión es simplificable si $z\in\rr$, pero no lo es si por ejemplo $z\in\mathbb{C}$. De modo que no puede esperarse que Sympy efectúe esta simplificación a menos que conozca que se trabaja en el campo numérico indicado. Hay que distinguir que supuestos tácitos está haciendo uno y hay que indicarselos a Sympy.    El lector debe tener en cuenta que en ningún momento uno le dijo a \texttt{SymPy} que tipo de ente estaba manipulando en expresiones del tipo \texttt{x\_n=acos(x)}. Puede parecer natural que se trata de números reales, no obstante esta 
información nunca fue comunicada al interprete de \texttt{SymPy}. ¿Porqué el habría de entender que \texttt{x} es real? Si al fin y al cabo \texttt{x\_n=acos(x)} tiene sentido si \texttt{x} es complejo y aún si es una matríz. Muchas veces las operaciones que se simplifican en un campo no lo pueden hacer en otro. El comando  \texttt{symbols} tiene la opción de informar a \texttt{SymPy} que tipo de ente representa \texttt{x} de la siguiente forma
\texttt{x=symbols('x',real=True)}. De esta forma se consiguen mejores resultados en las simplificaciones.


\subsubsection{Cambio de variable general $\hat{x}=\hat{x}(x,y)$, $\hat{y}=\hat{y}(x,y)$}
\begin{enumerate}
  \item Calculamos $d\hat{y}/d\hat{x}$ en las variables $x,y$
    \boxedeq{
      \frac{d\hat{y}}{d\hat{x}}=\frac{\frac{d\hat{y}}{dx}}{\frac{d\hat{x}}{dx}}=\frac{\frac{\partial\hat{y}}{\partial x}+\frac{\partial\hat{y}}{\partial y}y'}{\frac{\partial\hat{x}}{\partial x}+\frac{\partial\hat{x}}{\partial y}y'}=\frac{\frac{\partial\hat{y}}{\partial x}+\frac{\partial\hat{y}}{\partial y}f(x,y)}{\frac{\partial\hat{x}}{\partial x}+\frac{\partial\hat{x}}{\partial y}f(x,y)}.
	    }{eq:subsder}

   \item En la expresión resultante sustituímos $x,y$ por las tansformaciones 			inversas $x=x(\hat{x},\hat{y})$ y  $y=y(\hat{x},\hat{y})$
\end{enumerate}



\begin{ejemplo}{ej:cambio_forma} Transformar a polares
 \[
  \frac{dy}{dx}=\frac{y^3+x^2y-x-y}{x^3+xy^2-x+y}.
 \]
\end{ejemplo}

Dado que el cálculo es extenso lo haremos con \texttt{SymPy}, el procedimiento seguido ilustra como hacerlo a mano. Es ilustrativo hacer esto último para apreciar la utilidad de usar un sistema de álgebra computacional (SAC) como \texttt{SymPy}.

\begin{sympyblock}[][frame=single]
x=symbols('x')
y=Function('y')(x)
r=sqrt(x**2+y**2)
theta=atan(y/x)
Expr2=r.diff(x)/theta.diff(x)
\end{sympyblock}



Obtenemos la siguiente expresión 
\[\sympy{latex(Expr2)}.\]
Ahora sustituímos $y'(x)$ usando la ecuación diferencial, redefinimos $r,\theta$ fundamentalmente para limpiar el valor que tenían asignado en el código previo, que era una expresión de $x,y$, y finalmente sustituímos $x$ e $y$ por su expresión en polares.

\begin{sympyblock}[][frame=single]
Expr3=Expr2.subs(y.diff(x),(y**3+x**2*y-x-y)/(x**3+x*y**2-x+y))
r,theta=symbols('r,theta',positive=True)
Expr4=Expr3.subs([(y,r*sin(theta)),(x,r*cos(theta))])
Expr5=simplify(Expr4)
\end{sympyblock}


 Encontramos que en polares la ecuación es mucho más simple
\[\frac{dr}{d\theta}=-r^3+r.\]

 Quizás  usar la notación como  forma diferencial sea más efectivo. Como $r$ y $\theta$ son funciones de $x$ e $y$, ellas son 0-formas. Usando las reglas de la diferencial, hay que reemplazar
\[
\begin{array}{ll}
x=r\cos\theta; &dx=\cos\theta dr-\sen\theta r d\theta\\
 y= r\sen\theta; & dy=\sen\theta dr+\cos\theta r d\theta\\
\end{array}
\]
en la 1-forma:
\[(y^3+x^2y-x-y)dx-(x^3+xy^2-x+y)dy.\]

Sympy posee un módulo para operar con formas diferenciales, en un apéndice describimos como utilizarlo para resolver este ejemplo.

\section{Grupos}

\marginpar{Las raíces históricas de la teoría de grupos son la teoría de las ecuaciones algebraicas, la teoría de números y la geometría. Euler, Gauss, Lagrange, Abel y Galois fueron los creadores que ponen los cimientos de esta rama del álgebra abstracta.  Otros importantes matemáticos que contribuyen son Cayley, Emil Artin, Emmy Noether, Peter Ludwig Mejdell Sylow, A.G. Kurosch, Iwasawa entre muchos otros.
\raggedright{(Wikipedia)}
}
 \subsection{Definición y ejemplos}
\begin{definicion}[Grupo]
Un \href{https://es.wikipedia.org/wiki/Grupo_(matem%C3%A1tica)}{grupo} es un par $(G,\cdot)$ donde $G$ es un conjunto y $\cdot :G\times G\to G$ una \href{https://es.wikipedia.org/wiki/Operaci%C3%B3n_matem%C3%A1tica}{operación binaria interna} que satisface
\begin{enumerate}
\item $(g_1\cdot g_2)\cdot g_3=g_1\cdot (g_2\cdot g_3)$, para todos $g_1,g_2,g_3\in G$,
\item Existe $e\in G$ tal que $e\cdot g=g\cdot e=g$,  para todo $g\in G$.
\item Para todo $g\in G$ existe $h\in G$ tal que $g\cdot h=h\cdot g=e$. Esta función $h$ se denota  $g^{-1}$.
\end{enumerate}
\end{definicion}

Es usual omitir el punto para indicar la operación, i.e. escribimos $g\cdot h=gh$.



\begin{ejemplo}{} Sea $\Pi$ un plano euclideano y $G$ el conjunto de todas las transformaciones rígidas de $\Pi$ en si mismo. Entonces $G$ es un grupo con la operación de composición. Se llama el \emph{grupo de transformaciones rígidas}.
 \end{ejemplo}


\begin{ejemplo}{}  Sea $X=\{x_1,\ldots,x_n\}$ un conjunto de $n$ elementos y $S_n$ definido por
\[S_n=\{\sigma|\sigma:X\to X\hbox{ y }\sigma \hbox{ es biyectiva }\}\]
Entonces $S_n$ es un grupo  con la operación de composición. Se denomina \href{http://es.wikipedia.org/wiki/Grupo_simétrico}{\emph{grupo simétrico}}.
 \end{ejemplo}

\begin{ejemplo}{} Sea $\Delta$ un polígono regular de $n$ lados  en un plano euclideano $\Pi$ y $D_{2n}$ el conjunto de todas las transformaciones rígidas de $\Pi$ en si mismo que llevan $\Delta$ en si mismo. $D_{2n}$ se llama el \href{http://es.wikipedia.org/wiki/Grupo_diedral}{\emph{grupo diedral}}  de orden $2n$. Para un triángulo equilatero:
\begin{center}
\includegraphics[scale=.4]{imagenes/SimTria.jpg}
\end{center}
\end{ejemplo}

Recordemos el siguiente concepto.

\begin{definicion}[Acción de un grupo sobre un conjunto]
Diremos que un grupo $G$ actúa sobre el conjunto $X$ si existe una operación binaria externa  $\star:G\times X\to G$ que satisface:
\begin{enumerate}
\item Si $e$ es el neutro de $G$, $e \star x=x$,  para todo $x\in X$.
\item Para todos $g,h\in G$ y $x\in X$,    $g\star (h\star x)=(gh)\star x$.
\end{enumerate}
\end{definicion}

Como en el caso de grupos suele nos escribirse el símbolo de la operación binaria, i.e. se escribe $g\star x= gx$.




\section{Grupos continuos de simetrías}

\subsection{Grupos y cambios de variables}

Los cambios de variables de un conjunto de dos variables, digamos $x$ e $y$, son funciones $\Gamma$, invertibles,  de clase $C^1$, donde $\Gamma:\Omega_1\to\Omega_2$, con $\Omega_1,\Omega_2$ abiertos de $\rr^2$.  Acostumbraremos escribir $(\hat{x},\hat{y})=\Gamma(x,y)$ y diremos que $(\hat{x},\hat{y})$ son la variables nuevas y $(x,y)$ las viejas.

\begin{ejemplo}{} \textbf{Coordenadas polares.} Es más facil describir la transformación que lleva coordenadas polares en cartesianas. En este caso $(x,y)=\Gamma(r,\theta)$ y
\[
\begin{array}{ll}
\Gamma(r,\theta)&=(r\cos(\theta),r\sen(\theta)),\\
\Omega_1&=(0,\infty)\times (-\pi,\pi),\\
\Omega_2&=\rr^2-\{(x,y)|y=0,x\leq 0\}\\
\end{array}
\]
\end{ejemplo}

\subsection{Grupos de Lie uniparamétricos}

\begin{definicion}[Grupos de Lie uniparamétricos]  Supongamos dada una acción del grupo  $(\rr,+)$ en $\rr^2$. En este caso vamos a adoptar una notación funcional, i.e. en lugar de escribir $\epsilon\star (x,y)$, con $\epsilon\in\rr$ y $(x,y)\in\rr^2$ pondremos $\Gamma_{\epsilon}(x,y)$. Denotaremos por $\{\Gamma_{\epsilon}\}$ a la acción introducida.  Notar que $\{\Gamma_{\epsilon}\}$  satisface que 
\begin{enumerate}
 \item $\forall\epsilon_1,\epsilon_2\in\rr:\Gamma_{\epsilon_1}\circ \Gamma_{\epsilon_2}=\Gamma_{\epsilon_1+\epsilon_2}$.

\item $\Gamma_0=I$.

\item $\Gamma_{\epsilon}$ es invertible y $\left(\Gamma_{\epsilon}\right)^{-1}=\Gamma_{-\epsilon}$
\end{enumerate}

La acción $\Gamma_{\epsilon}$ se denomina un  \href{http://es.wikipedia.org/wiki/Grupo_uniparamétrico}{\emph{grupo de Lie uniparamétrico}} si además:

\begin{enumerate}
\item[4.] $\forall\epsilon\in\rr:\Gamma_{\epsilon}$ es un difeomorfismo sobre $\rr^2$.


\item[5.] Si $\Gamma_{\epsilon}(x,y)=\left(\hat{x}(x,y,\epsilon),\hat{y}(x,y,\epsilon)\right)$ entonces   las funciones  $\hat{x}(x,y,\epsilon)$ y $\hat{y}(x,y,\epsilon)$  se desarrollan en serie de potencias respecto a $\epsilon$. Es decir para todo $\epsilon_0\in\rr$ existen coeficientes $a_j$ y $b_j$, $j=0,1,\ldots$, y $r>0$ tales que
\begin{equation}\label{eq:des_serie}
\begin{array}{cc}
\hat{x}(x,y,\epsilon)&=a_0(x,y)+a_1(x,y)(\epsilon- \epsilon_0)+\cdots\\
\hat{y}(x,y,\epsilon)&=b_0(x,y)+b_1(x,y)(\epsilon- \epsilon_0)+\cdots\\
\end{array}
\end{equation}
para $|\epsilon-\epsilon_0|<r$.
\end{enumerate}
\end{definicion}



\begin{ejemplo}{} Demostrar que las siguientes aplicaciones inducen grupos de Lie uniparamétricos
\begin{enumerate}
\item $\Gamma_{\epsilon}(x,y)=(x+\epsilon,y)$ y $\Gamma_{\epsilon}(x,y)=(x,y+\epsilon)$.
\item $\Gamma_{\epsilon}(x,y)=(e^{\epsilon}x,y)$
\item$\Gamma_{\epsilon}(x,y)=\left(\frac{x}{1-\epsilon x},\frac{y}{1-\epsilon x} \right)$
\item$\Gamma_{\epsilon}(x,y)=\begin{pmatrix} \cos(\epsilon) & -\sen(\epsilon)
\\ \sen(\epsilon) & \cos(\epsilon)
\end{pmatrix} \begin{pmatrix} x\\ y
\end{pmatrix}
$
\end{enumerate}
\end{ejemplo}

Vamos a desarrollar sólo el ejemplo de $\Gamma_{\epsilon}(x,y)=(x+\epsilon,y)$. La propiedad 1 en la definición la chequearemos con  \texttt{SymPy}. 

\begin{sympyblock}[][frame=single]
x,y,epsilon,epsilon1,epsilon2=symbols('x,y,epsilon,epsilon1,epsilon2')
T=Matrix([x+epsilon,y])
x_copete=T.subs(epsilon,epsilon1)[0]
y_copete=T.subs(epsilon,epsilon1)[1]
PropGrupo=T.subs([(x,x_copete),(y,y_copete),(epsilon,epsilon2)])\
-T.subs(epsilon,epsilon1+epsilon2)
\end{sympyblock}

\[\sympy{latex(PropGrupo)}.\]


 La propiedad 2 en la definición es evidente y la propiedad 3 es siempre consecuencia de 1. y 2. 
Se incluyó en la lista sólo para resaltar su cumplimiento, pero no es necesario chequearla. La propiedad 4. es clara la 5. también lo es, notar que el desarrollo en serie \eqref{eq:des_serie} es válido en este caso con $a_0(x,y)=x$, $a_1(x,y)=1$, $a_j(x,y)=0$, $j\geq 2$, $b_0(x,y)=y$ y $b_j(x,y)=0$, $j\geq 1$. La justificaciones correspondientes al resto de los ejemplos queda como ejercicio.


 Las reflexiones en el plano, por ejemplo $\Gamma(x,y)=(-x,y)$,  no pueden  pertenecer a un  grupo de Lie uniparamétrico. Más generalmente:
 
 \begin{teorema}{}
   Una transformación  $T:\rr^2\to\rr^2$ tal que el determinante de la matriz Jacobiana $\det(DT)$ sea negativo en algún punto no puede pertenecer a un grupo de Lie uniparamerico. Dicho de otro modo, debe ocurrir que $\det(D\Gamma_{\epsilon})>0$ para todo $\epsilon$ y todo grupo $\{\Gamma_{\epsilon}\}$.
 \end{teorema}
\begin{proof}
 Si $\Gamma_{\epsilon}$ es un grupo y supongamos por ejemplo que existe $\epsilon_0\in\rr$ y $(x_0,y_0)\in\rr^2$ tal que:

\[J(x_0,y_0,\epsilon_0):=\det\begin{pmatrix} \frac{\partial\hat{x}}{\partial x}&  \frac{\partial\hat{x}}{\partial y}\\
 \frac{\partial\hat{y}}{\partial x} &  \frac{\partial\hat{y}}{\partial y}\\
\end{pmatrix}<0,
\]
donde las derivadas son tomadas en $\epsilon=\epsilon_0, x=x_0$ y $y=y_0$.
 Por otro lado tenemos que $J(x_0,y_0,0)=1$. Como $J(x_0,y_0,\epsilon)$ es continua respecto a $\epsilon$ debería existir $\epsilon'$ con $J(x_0,y_0,\epsilon')=0$. Esto implica que la matriz jacobiana $D\Gamma_{\epsilon}$ es singular y esto contradice que $\Gamma:\rr^2\to \rr^2$ es difeomorfismo ($D\Gamma D\Gamma^{-1}=I$).
\end{proof}


 
 
 



 En el caso de la reflexión $\Gamma(x,y)=(x,-y)$, se  genera un grupo discreto, ya que $\Gamma^2=\Gamma\circ \Gamma=I$. Luego $\Gamma$ genera el grupo finito $G=\{I,\Gamma\}$ que es isomorfo a $\mathbb{Z}_2$. En este caso diremos que    $\{I,\Gamma\}$ es un \emph{grupo discreto}.





\subsection{Grupos de simetrías de EDO}
\begin{definicion}[Grupo de simetrías de una ecuación]
 Consideremos una ecuación
\boxedeq{y'=f(x,y).}{eq:principi}
Una transformación o cambio de variables $\Gamma$ se denomina una \emph{simetría} de la ecuación si el cambio de variables dado por $(\hat{x},\hat{y})=\Gamma(x,y)$ deja invariante  la ecuación.  Diremos que un grupo de Lie uniparamétrico $\{\Gamma_{\epsilon}\}$ es un grupo uniparamétrico de simetrías de \eqref{eq:principi} si $\Gamma_{\epsilon}$ es simetría de la ecuación para cada $\epsilon$.

\end{definicion}


De acuerdo con \eqref{eq:subsder} para que $(\hat{x},\hat{y})=\Gamma(x,y)$ sea una simetría de \eqref{eq:principi} se debe cumplir que
 \boxedeq{\frac{\frac{\partial\hat{y}}{\partial x}+\frac{\partial\hat{y}}{\partial y}f(x,y)}{\frac{\partial\hat{x}}{\partial x}+\frac{\partial\hat{x}}{\partial y}f(x,y)}=f(\hat{x},\hat{y})}{eq:cond_sim}

 Esta ecuación se llama \emph{condición de simetría}. Es una ecuación en derivadas parciales, en principio más compleja que la ecuación original. Tiene varios grados de libertad, por lo que suele haber muchas simetrías.  Es común que encontremos soluciones a  traves de un  \href{http://es.wikipedia.org/wiki/Ansatz}{ansatz}. 





 \begin{ejemplo}{} Consideremos la ecuación
   \begin{equation}\label{eq:trivial}y'=0.
    \end{equation}
    \begin{wrapfigure}[10]{r}{5.5cm}
   \vspace{-.5cm}\includegraphics[scale=.3]{imagenes/sol_trivial.png}
 \end{wrapfigure}
La condición de simetría se reduce a 
\[
\frac{\frac{\partial\hat{y}}{\partial x}}{\frac{\partial\hat{x}}{\partial x}}=0
\]
Debemos tener que $\frac{\partial\hat{y}}{\partial x}=0$. Vale decir $\hat{y}$ es independiente de $x$. De allí la forma general de una simetría es 
\[\hat{x}=\hat{x}(x,y)\quad \hat{y}=\hat{y}(y).\]
\end{ejemplo}
 Hay muchas simetrías. Las traslaciones en cualquier dirección $(x,y)\mapsto (x+\alpha ,y+\beta)$ ($\alpha,\beta\in\rr$). Cambios de escala en ambos ejes  $(x,y)\mapsto (e^{\epsilon}x,y)$, $(x,y)\mapsto (x,e^{\epsilon}y)$. Reflexiones respecto ambos ejes  $(x,y)\mapsto (-x,y)$, $(x,y)\mapsto (x,-y)$.  Observar que el gráfico de las soluciones posee las mismas simetrías, pues en general \emph{las simetrías de una ecuación llevan soluciones en soluciones}.

 De todas las simetrías encontradas $\Gamma_{\epsilon}(x,y)=(e^{\epsilon}x,y)$, $\Gamma_{\epsilon}(x,y)=(x+\epsilon,y)$ y  $\Gamma_{\epsilon}(x,y)=(x+\epsilon,y)$ se llaman \emph{triviales} pues llevan una curva solución en si misma.  Cualquier cambio de la forma $\hat{x}=\hat{x}(x,y)\quad \hat{y}=y$ es trivial. \emph{Estamos interesados en hallar grupos de Lie uniparamétricos de simetrías no triviales.}




 \begin{ejemplo}{} Hallar simetrías de
\[\frac{dy}{dx}=f(x).\]
\end{ejemplo}
De acuerdo con \eqref{eq:subsder} se debe cumplir que
 \[\frac{\frac{\partial\hat{y}}{\partial x}+\frac{\partial\hat{y}}{\partial y}f(x)}{\frac{\partial\hat{x}}{\partial x}+\frac{\partial\hat{x}}{\partial y}f(x)}=f(\hat{x})\]
La forma de la ecuación sugiere el  \href{http://es.wikipedia.org/wiki/Ansatz}{ansatz}
   \[\boxed{\hat{x}=x},\quad \frac{\partial\hat{y}}{\partial x}=\frac{\partial\hat{x}}{\partial y}=0,\quad
   \frac{\partial\hat{y}}{\partial y}=\frac{\partial\hat{x}}{\partial x}. \]
 Luego 
\[\frac{\partial\hat{y}}{\partial y}=1\Rightarrow \boxed{\hat{y}=y+\epsilon} \]
con $\epsilon$ constante arbitraria. Hallamos que
\[\Gamma_{\epsilon}(x,y)=(x,y+\epsilon)\]
es un grupo de Lie uniparamétrico de simetrías. De manera similar
\[\Gamma_{\epsilon}(x,y)=(x+\epsilon,y)\]
es un grupo uniparamétrico de simetrías para 
\[\frac{dy}{dx}=f(y).\]
Geométricamente en el primer caso todas las soluciones se obtienen trasladando una cualquiera verticalmente y en el segundo caso horizontalmente.



\begin{tabular}{cc}
\includegraphics[scale=.3]{imagenes/sol_paralelas.png} &\hspace{-1.5cm}\includegraphics[scale=.3]{imagenes/sol_paralelas2.png} \\
Soluciones de $y'=x^3-x$ &\hspace{-1.5cm}Soluciones de $y'=y^3-y$
\end{tabular}





\begin{ejemplo}{}
 Demostrar que las rotaciones alrededor del origen es un grupo de Lie uniparamétrico de simetrías de
 \begin{equation}\label{eq:ejemplo_polar}
  \frac{dy}{dx}=\frac{y^3+x^2y-x-y}{x^3+xy^2-x+y}.
 \end{equation}
\end{ejemplo}
Sea $\Gamma_{\epsilon}$ la transformación que rota un ángulo $\epsilon$ alrededor del origen. Es un ejercicio demostrar que  $\{\Gamma_{\epsilon}|\epsilon\in\rr\}$ es un grupo uniparamétrico de simetrías. Se tiene la representación matricial
\[
\Gamma_{\epsilon}(x,y)= \begin{pmatrix} \hat{x}\\ \hat{y}
\end{pmatrix}=\begin{pmatrix} \cos(\epsilon) & -\sen(\epsilon)
\\ \sen(\epsilon) & \cos(\epsilon)
\end{pmatrix} \begin{pmatrix} x\\ y
\end{pmatrix}
\]

\[
\Gamma^{-1}_{\epsilon}(\hat{x},\hat{y})= \begin{pmatrix} x\\ y
\end{pmatrix}=\begin{pmatrix} \cos(\epsilon) & \sen(\epsilon)
\\ -sen(\epsilon) & \cos(\epsilon)
\end{pmatrix} \begin{pmatrix} \hat{x}\\ \hat{y}
\end{pmatrix}
\]


Para el cálculo recurrimos a \texttt{SymPy} (usamos \texttt{x\_n} en lugar de $\hat{x}$)

\begin{sympyblock}[][frame=single]
x,theta=symbols('x,theta')
y=Function('y')(x)
x_n=cos(theta)*x-sin(theta)*y
y_n=sin(theta)*x+cos(theta)*y
Expr2=y_n.diff(x)/x_n.diff(x)
Expr3=Expr2.subs(y.diff(),\
(y**3+x**2*y-x-y)/(x**3+x*y**2-x+y))
x_n,y_n=symbols('x_n,y_n')
Expr4=Expr3.subs([(y, -sin(theta)*x_n+cos(theta)*y_n),\
(x,cos(theta)*x_n+sin(theta)*y_n)])
Expr5=simplify(Expr4) 
\end{sympyblock}






\begin{wrapfigure}[15]{r}{5.5cm}
 \includegraphics[scale=.4]{imagenes/sol_rotadas.png}
\end{wrapfigure}
La ecuación resultante es \emph{la misma}
\[\sympy{latex(Expr5)}\]
A la misma conclusión arribábamos si recordabamos que en en coordenadas polares la ecuación se escribe
\[\frac{dr}{d\theta}=r-r^3,\]
y que esta ecuación tiene las simetrías $\Gamma_{\epsilon}:(r,\theta)\mapsto (r,\theta+\epsilon)$. Si rotamos un ángulo fijo el gráfico de una solución obtenemos el gráfico de otra solución.








\begin{ejemplo}{ejem:tras} Supongamos que $y'=f(x,y)$ tiene  el grupo de Lie uniparamétrico de simetrías
\boxedeq{(\hat{x},\hat{y})=\Gamma_{\epsilon}(x,y)=(x,y+\epsilon)}{eq:tras_lie}
Usando la condición de simetrías \eqref{eq:cond_sim} tenemos
\[f(x,y)=f(\hat{x},\hat{y})=f(x,y+\epsilon).\]
La igualdad vale para todo $\epsilon$, luego poniendo $\epsilon=-y$ vemos que $f(x,y)=f(x,0):=f(x)$. Vale decir que $f$ es independiente de $y$ y la ecuación
\[y'=f(x),\]
se resuelve simplemente integrando.
\end{ejemplo}



\section{Órbitas, tangentes y curvas invariantes}


\begin{definicion}[Órbitas] Dado un grupo uniparamétrico de simetrías $G=\{\Gamma_{\epsilon}|\epsilon\in\rr\}$, y $(x_0,y_0)\in\rr^2$ llamamos \emph{órbita $(x_0,y_0)$ bajo la acción de  $G$} (simplemente órbita si es claro quien es $G$) a la curva
\[\{\Gamma_{\epsilon}(x_0,y_0)|\epsilon\in\rr\} \]
\end{definicion}

\begin{wrapfigure}[15]{r}{5.5cm}
\includegraphics[scale=.3]{imagenes/sol_trivialB.png}
\end{wrapfigure}
Si $G$ es un grupo de simetrías no trivial, entonces es de esperar que la órbita de  $(x_0,y_0)$ cruce transversalmente las curvas solución. La órbita se usará como una nueva coordenada.
 La órbita atraves de $(x,y)$ es el conjunto de puntos de coordenadas
\begin{equation}\label{eq:orb} (\hat{x}(x,y,\epsilon),\hat{y}(x,y,\epsilon))=\Gamma_{\epsilon}(x,y),
\end{equation}
donde
\[(\hat{x}(x,y,0),\hat{y}(x,y,0))=(x,y).\]
Las ecuaciones \eqref{eq:orb} son ecuaciones parámetricas (parámetro $\epsilon$) de una  curva en el plano.

\begin{definicion}[Puntos invariantes]
Un punto $(x,y)$ se llama invariante si su órbita se reduce a $\{(x,y)\}$, vale decir
\[(x,y)=\Gamma_{\epsilon}(x,y),\quad\forall \epsilon>0\]
\end{definicion}






\begin{ejemplo}{} La órbita de $(x,y)$ bajo la acción del grupo  de Lie uniparamétrico
\[
\Gamma_{\epsilon}(x,y)= \begin{pmatrix} \hat{x}\\ \hat{y}
\end{pmatrix}=\begin{pmatrix} \cos(\epsilon) & -\sen(\epsilon)
\\ \sen(\epsilon) & \cos(\epsilon)
\end{pmatrix} \begin{pmatrix} x\\ y
\end{pmatrix}
\]
Son circunsferencias con centro en el origen. El punto $(0,0)$ es invariante.

\end{ejemplo}


\begin{definicion}[Campo vectorial de tangentes]
 Dado un grupo de Lie uniparamétrico $(\hat{x}(x,y,\epsilon),\hat{y}(x,y,\epsilon))=\Gamma_{\epsilon}(x,y)$  definimos el campo vectorial
\[(\xi(x,y),\eta(x, y))=\left(\left.\frac{d\hat{x}}{d\epsilon}\right|_{\epsilon=0}, \left.\frac{d\hat{y}}{d\epsilon}\right|_{\epsilon=0}   \right).\]
 $\xi$ y $\eta$ se llaman \emph{símbolos infinitesimales}.
 
 
Como $\hat{x},\hat{y}$ eran analíticas respecto a $\epsilon$ tenemos las fórmulas.

 
 \begin{equation}\label{eq:des_serie_infi}
\begin{array}{cc}
\hat{x}&=x+\epsilon\xi(x,y)+O(\epsilon^2)\\
\hat{y}&=y+\epsilon\eta(x,y)+O(\epsilon^2)\\
\end{array}
\end{equation}
 En un punto invariante \fbox{$\xi(x,y)=\eta(x,y)=0$}.
 
\end{definicion}





\begin{ejemplo}{}
Campo vectorial de infitesimales para las rotaciones. 
\end{ejemplo}
\begin{figure}[h]
 \begin{center}
  \includegraphics[scale=.3]{imagenes/CampoVectorial.png}
 \end{center}
\caption{Campo vectorial de infinitesimales $(\xi,\eta)$ }
\end{figure}

\begin{definicion}
 Más generalmente, un conjunto $C\subset\rr^2$ se dice \emph{invariante} por un grupo uniparamétrico de 
simetrías de Lie $\Gamma_{\epsilon}$ si y sólo si $\forall \epsilon: \Gamma_{\epsilon}(C)\subset C$, i.e. las órbitas de puntos en $C$ permanecen en $C$.
\end{definicion}

\noindent\textbf{Observaciones}
\begin{enumerate}
\item El conjunto unitario $\{(x,y)\}$ es invariante si y sólo si el punto $(x,y)$ 
es invariante.

\item Se $C$ es una curva plana suave, supongamos que $C$ viene dada por la ecuación implícita $g(x,y)=0$, con $\nabla g\neq 0$.  Supongamos que  $(\xi,\eta)\neq (0,0)$ sobre $C$. Entonces $C$ es invariante si y sólo si la tangente a  $C$ en cada punto $(x,y)$ es
paralela a $(\xi(x,y),\eta(x,y))$. La demostración de este hecho se propone como ejercicio.  
\item Si en particular $C$ es una curva que viene descripta como el gráfico de una función $y=h(x)$, entonces $C$ es invariante si y sólo si
\boxedeq{Q(x,y,y')=\eta(x,y)-h'\xi(x,y)\equiv 0.}{eq:inv}
En efecto, en este caso $g(x,y)=y-h(x)$ (notar que $\nabla g=(-h'(x),1)\neq 0$). Como $\nabla g$ es perpendicular a $C$ tenemos la ecuación \eqref{eq:inv}. La ecuación \eqref{eq:inv} se llama \emph{ecuación característica}.






\item Si además de lo anterior, $\Gamma_{\epsilon}$ es un grupo de Lie de simetrías de $y'=f(x,y)$ e $y(x)$ es una curva invariante y solución de la ecuación entonces:
\boxedeq{\overline{Q}(x,y)=\eta(x,y)-f(x,y)\xi(x,y)\equiv 0.}{eq:sol_inv}
A esta ecuación la llamamos  \emph{ecuación característica reducida}.

 \end{enumerate}


El recíproco también es cierto.
\begin{teorema}{teo:invariantes}
 Supongamos que $y(x)$ es solución de la ecuación característica y que
 \[\left.\frac{\partial \overline{Q}}{\partial y}\right|_{y=y(x)}\neq 0.\]
 Entonces $y(x)$ es una solución invariante de la ecuación.
\end{teorema}
\begin{proof} Será completada más adelante.
 
\end{proof}




\begin{ejemplo}{} La EDO
\begin{equation}\label{eq:eq_simple}y'=y
 \end{equation}
tiene simetrías de escala
\[(\hat{x},\hat{y})=\Gamma_{\epsilon}(x,y)=(x,e^{\epsilon}y).
 \]
Luego 
\[(\xi,\eta)=\left(\left.\frac{d\hat{x}}{d\epsilon}\right|_{\epsilon=0}, \left.\frac{d\hat{y}}{d\epsilon}\right|_{\epsilon=0}   \right)=(0,y)\]
Cualquier punto en el conjunto $\{(x,0)|x\in\rr\}$  es invariante. La ecuación característica reducida es.
\[\overline{Q}(x,y)=0\Rightarrow y=0\]
Esta formada enteramente por puntos invariantes.
\end{ejemplo}



\begin{ejemplo}{} Demostrar que la siguiente expresión es un grupo de Lie uniparamétrico de simetrías
\[(\hat{x},\hat{y})=\Gamma_{\epsilon}(x,y)=(e^{\epsilon}x,e^{(e^{\epsilon}-1)x}y),\]
para la ecuación \eqref{eq:eq_simple}.
Para este grupo tenemos
\[(\xi,\eta)=(x,xy)\]
Todo punto en $x=0$ es invariante. La ecuación característica reducida es
\[\overline{Q}(x,y)=0\Rightarrow xy-xy=0\]
De modo que estas simetrías actuan trivialmente sobre las soluciones. Llevan una solución en si misma. Chequeemos esta afirmación de manera directa. 
\end{ejemplo}

Buscamos el cambio de variables inverso
\[
\left.
\begin{array}{ll} 
  \hat{x}&= e^{\epsilon}x\\
  \hat{y}&=e^{(e^{\epsilon}-1)x}y\\
\end{array}
\right\} 
\Rightarrow  
    \left.
\begin{array}{ll} 
  x&= e^{-\epsilon}\hat{x}\\
  y&=e^{(e^{-\epsilon}-1)\hat{x}}\hat{y}\\
\end{array}
\right\} 
\]    

Las soluciones son $y=ke^x$, sustituímos en esta expresión, luego de unas operaciones, llegamos a $\hat{y}=ke^{\hat{x}}$.

\begin{ejemplo}{} La ecuación de Riccati
\[y'=xy^2-\frac{2y}{x}-\frac{1}{x^3},\quad x\neq 0\] 
 Tiene el grupo de Lie de simetrías
\[(\hat{x},\hat{y})=\Gamma_{\epsilon}(x,y)=(e^{\epsilon}x,e^{-2\epsilon}y).\]
Tenemos
\[(\xi,\eta)=(x,-2y).\]
La característica reducida
\[\overline{Q}(x,y)=\frac{1}{x^2}-x^2y^2=0.\]
Tenemos dos soluciones invariantes
\[y=\pm\frac{1}{x^2}.\] 

\end{ejemplo}

\section{Simetrías a partir de Infinitesimales }

 La mayoría de los métodos de simetría usan  $(\xi,\eta)$ en lugar de las simetrías en si mismas.
Por otra parte  $(\xi(\hat{x},\hat{y}),\eta(\hat{x},\hat{y}))$ determinan las simetrías a traves de las ecuaciones

\boxedeq{
\left\{
\begin{array}{ll}
\frac{d\hat{x}}{d\epsilon}&=\xi(\hat{x},\hat{y})\\
\frac{d\hat{y}}{d\epsilon}&=\eta(\hat{x},\hat{y})\\
\hat{x}(x,y,0)&=x\\
\hat{y}(x,y,0)&=y\\
\end{array}
\right.
}{eq:ecua_infinitesimales}
\textbf{Justificación de las ecuaciones.}
De la propiedad de grupo $ \Gamma_{\epsilon_1} \circ  \Gamma_{\epsilon}=
\Gamma_{\epsilon+\epsilon_1}$, deducimos 
\[\hat{x}(x,y,\epsilon+\epsilon_1)=\hat{x}(\hat{x}(x,y,\epsilon),
\hat{y}(x,y,\epsilon),\epsilon_1).\]
Derivando respecto a $\epsilon_1$ y evaluando en $\epsilon_1=0$ justificamos las EDO en 
\eqref{eq:ecua_infinitesimales}.




















El \emph{sistema de ecuaciones} \eqref{eq:ecua_infinitesimales}    puede ser difícil de resolver. Pero en algunos casos puede ser posible.

 \begin{ejemplo}{} Encontrar el grupo de simetrías para los infinitesimales $\xi(x,y),\eta(x,y))=(x^2,xy)$.
   \end{ejemplo}



\[ \frac{d\hat{x}}{d\epsilon}=\xi(\hat{x},\hat{y})=\hat{x}^2\text{ y }\hat{x}(x,y,0)=x
\Rightarrow \hat{x}=\frac{x}{1-\epsilon x}\]
y

\[ \frac{d\hat{y}}{d\epsilon}=\eta(\hat{x},\hat{y})=\hat{x}\hat{y}\text{ y }\hat{y}(x,y,0)=y
\Rightarrow \hat{y}=\frac{y}{1-\epsilon x}\]

\section{Condición de Simetría Linealizada}
En esta sección discutiremos una técnica para encontrar simetrías de ecuaciones. 


Desarrollando en serie de Taylor las funciones $\hat{x}$ e $\hat{y}$ en $\epsilon$ alrededror de $\epsilon=0$
\[
 \begin{split}
  \hat{x}&=x+\epsilon\xi+\mathcal{O}(\epsilon^2)\\
  \hat{y}&=y+\epsilon\eta+\mathcal{O}(\epsilon^2)
  \end{split}
\]
Reemplazando en la condición de simetría, que recordemos es
\[\frac{\frac{\partial\hat{y}}{\partial x}+\frac{\partial\hat{y}}{\partial y}
 f(x,y)}{\frac{\partial\hat{x}}{\partial x}+\frac{\partial\hat{x}}{\partial y}f(x,y)}
 =f(\hat{x},\hat{y}),
\] 
obtenemos
\[\frac{f+\epsilon\{\eta_x+f\eta_y\}+\mathcal{O}(\epsilon^2)}
{1+\epsilon\{\xi_x+f\xi_y\}+\mathcal{O}(\epsilon^2)}
=f(x+\epsilon\xi+\mathcal{O}(\epsilon^2),y+\epsilon\eta+\mathcal{O}(\epsilon^2))
\]



Desarrollando en serie de Taylor para $x$ e $y$ el segundo miembro y luego de algunas operaciones

\[f+\epsilon\{\eta_x+(\eta_y-\xi_x)f-\xi_yf^2\}+\mathcal{O}(\epsilon^2)=
 f+\epsilon\{\xi f_x+\eta f_y\}+\mathcal{O}(\epsilon^2)
\]
Cancelando $f$ dividiendo por $\epsilon$ y haciendo $\epsilon\to 0$ llegamos a la 
\emph{Condición de Simetría Linealizada}

\boxedeq{\eta_x+(\eta_y-\xi_x)f-\xi_yf^2=\xi f_x+\eta f_y}{eq:CondSimLin}

Recordando la característica reducida $\overline{Q}= \eta-f\xi$, la fórmula anterior se escribe más sintéticamente
\boxedeq{\overline{Q}_x+f\overline{Q}_y=f_y\overline{Q}}{eq:CondSimLin2}

Cada solución de \eqref{eq:CondSimLin2} conlleva una infinita cantidad de grupos de Lie de simetrías, porque si $\overline{Q}$ resueleve  \eqref{eq:CondSimLin2}  entonces para toda función $\xi$ el par $(\xi, \overline{Q}+f\xi)$ son infinitesimales para un grupo de Lie de simetrías de la ecuación. La solución trivial $\overline{Q}\equiv 0$ de  \eqref{eq:CondSimLin2}  se corresponde con simetrías triviales. En principio podríamos utilizar el método de características, que hemos visto en la unidad anterior, para resolver la ecuación lineal en derivadas parciales de primer orden \eqref{eq:CondSimLin2} que, recordando lo visto, se puede escribir
\[ \frac{dx}{1}=\frac{dy}{f}=\frac{d\overline{Q}}{f_y\overline{Q}}.\]
La primera igualdad en la ecuación anterior equivales a $dy/dx=f$, que es al fin y al cabo, la ecuación que queremos resolver. Así estas consideraciones parecen habernos llevado al origen de nuestro problema. No obstante, algunas veces es posible encontrar una solución de \eqref{eq:CondSimLin} recurriendo a un ansatz.

 \begin{ejemplo}{} Encontrar un grupo de Lie de simetrías no trivial de $y'=\frac{y}{x}+x$.
   \end{ejemplo}


 
 La condición de simetría linealizada es
 \[\eta_x+(\eta_y-\xi_x)\left(\frac{y}{x}+x\right)-\xi_y\left(\frac{y}{x}+x\right)^2
 =\xi \left(1-\frac{y}{x^2}\right)
+\frac{\eta}{x},\] 

que luce intimidante. Hagamos el ansatz $\xi=0$ y $\eta=\eta(x)$. Conseguimos
\[\eta_x-\frac{\eta}{x}=0\]
Cuya solucion general es $\boxed{\eta=cx}$. Ahora podemos encontrar simetrías de la ecuación. Recordando \eqref{eq:ecua_infinitesimales}, tenemos
\[
\left\{
\begin{array}{ll}
\frac{d\hat{x}}{d\epsilon}&=0\\
\frac{d\hat{y}}{d\epsilon}&=c\hat{x}\\
\hat{x}(x,y,0)&=x\\
\hat{y}(x,y,0)&=y\\
\end{array}
\right. 
\Rightarrow 
\left\{
\begin{array}{ll}
\hat{x}&=x\\
\hat{y}&=c\epsilon x+y\\
\end{array}
\right.
\]

Podemos constatar de manera directa que $\hat{x}=x$ e $\hat{y}=c\epsilon x+y$ constituyen un grupo de Lie uniparamétrico de simetrías de la ecuación, para el cual $\overline{Q}\neq 0$.


 \begin{ejemplo}{} Encontrar valiendose de SymPy los infinitesimales  de la ecuación \eqref{eq:ejemplo_polar}.
   \end{ejemplo}
   
\begin{sympyblock}[][frame=single]
from sympy import *
init_printing()
x,y=symbols('x,y',real=True)
f=(y**3+x**2*y-x-y)/(x**3+x*y**2-x+y)
\end{sympyblock}

Cómo en muchos otras situaciones que se presentan cuando intentamos resolver de manera exacta ecuaciones diferenciales,  debemos recurrrir a una heurística \marginpar{\textbf{Heurística:} ...significa «hallar, inventar» (etimología que comparte con eureka)... Cuando se usa como sustantivo, se refiere a la disciplina, el arte o la ciencia del descubrimiento... Cuando aparece como adjetivo, se refiere a cosas más concretas, como estrategias, reglas, silogismos y conclusiones heurísticas. Estos dos usos están íntimamente relacionados, ya que la heurística usualmente propone estrategias que guían el descubrimiento. El término fue utilizado por Albert Einstein en la publicación sobre efecto fotoeléctrico (1905). 
\begin{flushright}
 Wikipedia
\end{flushright}
 } con el objeto de resolver  el problema. Por ejemplo, SymPy incorpora una estrategia de busqueda entre distintos candidatos a infinitesimales, ver \cite{Lie_Sympy,cheb1997computer} para los detalles. Por ejemplo, la primer heurística que se enumera en \cite{Lie_Sympy} es la que usamos en el ejemplo anterior, es decir $\xi=0$ y $\eta=\eta(x)$. En el ejemplo actual, vamos a usar la siguiente heurística
 \[\xi=ax+by+h,\quad \eta=cx+dy+k.\]
\begin{sympyblock}[][frame=single]
a,b,c,d,h,k=symbols('a,b,c,d,h,k',real=True)
xi=a*x+b*y+h
eta=c*x+d*y+k
Q=eta-f*xi
\end{sympyblock}

Reemplazamos el valor de $Q$ en la condición de simetría linealizada.
\begin{sympyblock}[][frame=single]
CondSim=Q.diff(x)+f*Q.diff(y)-f.diff(y)*Q
CondSim=simplify(CondSim)
\end{sympyblock}
Llegamos a una ecuación de la forma
\[
 \frac{P_1(x,y)}{P_2(x,y)}=0,
\]
donde $P_1$ y $P_2$  son polinomios bivariados. Obviamente sólo nos interesa $P_1$, que podemos extraer de la siguiente forma 
\begin{sympyblock}[][frame=single]
P1,P2=fraction(CondSim)
P1
\end{sympyblock}
Obtenemos la siguiente expresión  monstruosa de $P_1$, que sólo reproducimos a modo anecdótico

\[
\begin{split}
P_1 =&- a x^{4} - 2 a x^{2} y^{2} - a x^{2} + 2 a x y - a y^{4} + a y^{2} - b x^{2} - 2 b x y\\
&+ b y^{2} - c x^{2} - 2 c x y+ c y^{2} - d x^{4} - 2 d x^{2} y^{2}
+ d x^{2} - 2 d x y\\
&- d y^{4} - d y^{2} + h x^{4} y - 2 h x^{3} + 2 h x^{2} y^{3} - 2 h x^{2} y - 2 h x y^{2} + h y^{5}\\
&- 2 h y^{3} + 2 h y - k x^{5} - 2 k x^{3} y^{2} + 2 k x^{3} - 2 k x^{2} y - k x y^{4} + 2 k x y^{2}\\
&- 2 k x - 2 k y^{3}
\end{split}
\]

Hay que resolver $P_1=0$. A pesar de la apariencia atemorizante de esta ecuación, es completamente inofensiva si tenemos en mente que el ``juego'' consiste en hallar los coeficientes $a,b,c,d,h,k$ que resuelven la ecuación para todo $x$ e $y$. De modo que podemos reemplazar $x$ e $y$ por distintos valores y resolver las ecuaciones resultantes. En el código debajo usamos el método \texttt{collect(x)} que agrupa los términos de una expresión que tienen la misma potencia para $x$.

\begin{sympyblock}[][frame=single]
P1.subs(y,0).collect(x)
\end{sympyblock}
Resulta
\[
 0=- k x^{5} - 2 k x + x^{4} (- a - d) + x^{3} (- 2 h + 2 k) + x^{2} (- a - b - c + d)
\]
Es un polinomio en $x$ que debería ser identicamente $0$, debe para ello tener todos los coeficientes iguales a $0$. Inferimos  inmediatamente de esto que $k=h=0$, $d=-a$ y $c=-2a-b$. Reemplacemos ahora $y$ por $1$.  

\begin{sympyblock}[][frame=single]
P11=P1.subs({h:0,k:0,d:-a,c:-2*a-b})
P11.subs(y,1).collect(x)
\end{sympyblock}
Resulta en la expresión $8ax$,  que si es identicamente $0$ fuerza  que $a=0$. Luego $d=0$ y $c=-b$. El valor de $c$ es arbitrario, se puede elegir cualquiera, elejimos $c=1$ y por consiguiente $b=-1$.   
 
 \begin{sympyblock}[][frame=single]
xi=xi.subs({d:0,a:0,b:-1,c:1,h:0,k:0})
eta=eta.subs({d:0,a:0,b:-1,c:1,h:0,k:0})
xi,eta
\end{sympyblock}
Cocnluímos $\xi=-y$ y $\eta=x$. Todavía no estamos seguros que estos $\xi$ y $\eta$ resuelvan nuestro problema. El razonamiento anterior está incompleto sino chequeamos que la solución obtenida resuelve la ecuación de simetría linealizada para todo $x$ e $y$, esto porque con la metodología  utilizada solo estamos seguros que los valores obtenidos para los coeficientes   solucionan el problema en los valores que elegimos para $y$ ($y=0$ e $y=1$). Ahora podemos chequear que en realidad se resuelve la ecuación de simetría linealizada para cualquier valor de $x$ e $y$.  
 \begin{sympyblock}[][frame=single]
CondSimLin=Q.diff(x)+f*Q.diff(y)-f.diff(y)*Q
CondSimLin.simplify()
\end{sympyblock}


Que definitivamente establece que la elección que hemos hecho resuelve el problema.

 
\textbf{Completación de la demostración  del Teorema \ref{teo:invariantes}}.
Restaba demostrar que si $y(x)$ resuelve la ecuación $\overline{Q}(x,y(x))=0$ y $\overline{Q}_y\neq 0$ en los puntos a lo largo de la curva $(x,y(x))$ entonces $y(x)$ es solución invariante de la ecuación $y'=f$. 

Por el Teorema de la función implícita,
la condición de simetría linealizada \eqref{eq:CondSimLin2} y $\overline{Q}=0$
\[y'(x)=-\frac{\overline{Q}_x}{\overline{Q}_y}=f-f_y\frac{\overline{Q}}{\overline{Q}_y}=f\]
Luego $y$ es solución de la ecuación. El hecho de que es invariante es simplemente la igualdad 
$\overline{Q}=0$.









\section{Coordenadas canónicas}

\subsection{Definición y ejemplos}
\begin{definicion}[Coordenadas canónicas]   Diremos que las coordenadas $(r,s)$ son canónicas respecto a el grupo de Lie de simetrías $\Gamma_{\epsilon}$  si en las coordenadas $(r,s)$ la acción de grupo es la traslación

\boxedeq{(\hat{r},\hat{s}):= (r(\hat{x},\hat{y}),s(\hat{x},\hat{y}))=(r(x,y),s(x,y)+\epsilon).}{eq:coorcan}
\end{definicion}
 \begin{ejemplo}{} Las coordenadas polares son canónicas respecto al grupo de Lie de rotaciones. Las rotaciones en coordenadas cartesianas y polares se escriben
\[
 \begin{pmatrix} \hat{x}\\ \hat{y}
\end{pmatrix}=\begin{pmatrix} \cos(\epsilon) & -\sen(\epsilon)
\\ \sen(\epsilon) & \cos(\epsilon)
\end{pmatrix} \begin{pmatrix} x\\ y
\end{pmatrix},\quad \begin{array}{ll} \hat{r}&=r\\ \hat{\theta}&=\theta+\epsilon\\ 
\end{array}
\]
\end{ejemplo}

Derivando las ecuaciones \eqref{eq:coorcan} respecto a $\epsilon$ obtenemos
\boxedeq{
\begin{array}{ll}
\xi(x,y)\frac{\partial r}{\partial x}+\eta(x,y)\frac{\partial r}{\partial y}&=0\\
\xi(x,y)\frac{\partial s}{\partial x}+\eta(x,y)\frac{\partial s}{\partial y}&=1
\end{array}
}{eq:coor_can_dif}
 Los cambios de coordenadas deben ser invertibles, de modo que pediremos la condición de no degeneración
\boxedeq{\frac{\partial r}{\partial x}\frac{\partial s}{\partial y}-\frac{\partial r}{\partial y}\frac{\partial s}{\partial x}\neq 0}{eq:no_deg}





\noindent\textbf{Observaciones:}
\begin{enumerate}
\item El vector tangente en cualquier punto no invariante es paralelo a la curva $r=\hbox{cte}$ que pasa por ese punto. Luego esa curva continene las órbitas de cada punto en ella. Las órbitas son invariantes, así  $r$ se llama la \emph{coordenada invariante}. Las curvas $s=\hbox{cte}$ son transversales a las órbitas.
\item Las coordenadas canónicas no estan definidas en un punto $(x,y)$ invariante pues en esos puntos $\xi(x,y)=\eta(x,y)=0$.

\item Las coordenadas canónicas estan definidas en un entorno de cualquier   punto no invariante. Esta afirmación requiere una demostración que utiliza métodos y conceptos que estan fuera de los objetivos de este curso.

\item Las coordenadas canónicas no son únicas. De hecho si $(r,s)$ son canónicas $(\tilde{r},\tilde{s})=(F(r),G(r)+s)$ lo son para cualquier $F$ y $G$ con $F'(r)\neq 0$ (para la no degeneración).

\end{enumerate}






\section{Encontrando coordenadas canónicas}

 \subsection{Integrales primeras}
\begin{definicion}[Integrales primera] Una integral primera de la EDO $y'=f(x,y)$ es una función $\phi(x,y)$ que es constante a lo largo de cualquier curva solución de la EDO. Se la denomina también \emph{magnitud conservada}.
 \end{definicion}



 \begin{teorema}[Propiedad de conservación de coordenadas canónicas]{} Si $(r,s)$ son coordenadas canónicas de una grupo de Lie de simetrías entonces $r$ es una integral primera de la ecuación
\begin{equation}\label{eq:int_pri} \frac{dy}{dx}=\frac{\eta(x,y)}{\xi(x,y)}.
\end{equation}
\end{teorema}
\textbf{Dem.} La afirmación del teorema es consecuencia de que el campo $(\xi,\eta)$ es tangente a las curvas $r=c$, con $c$ constante. Justifiquemos el teorema del siguiente modo.  Supongamos $y(x)$ solución de la EDO, es suficiente demostrar que $\frac{d}{dx}r(x,y(x))=0$. Pero, en efecto
\[
\begin{array}{lll}
 \frac{d}{dx}r(x,y(x)) &=\frac{\partial r}{\partial x}+\frac{\partial s}{\partial y}y' &\text{(regla cadena)}\\
&=\frac{\partial r}{\partial x}+\frac{\partial s}{\partial y}\frac{\eta(x,y)}{\xi(x,y)}& \text{(Ec. \eqref{eq:int_pri})}\\
&=0 & \text{(Ec. \eqref{eq:coor_can_dif})}\\
\end{array}
\]
\qed

\begin{ejemplo}{} Ya conocemos las coordenadas canónicas de las rotaciones,
 \[
 \begin{pmatrix} \hat{x}\\ \hat{y}
\end{pmatrix}=\begin{pmatrix} \cos(\epsilon) & -\sen(\epsilon)
\\ \sen(\epsilon) & \cos(\epsilon)
\end{pmatrix} \begin{pmatrix} x\\ y
\end{pmatrix}\Rightarrow  \begin{pmatrix} \xi\\ \eta
\end{pmatrix}= \begin{pmatrix} -y\\ x
\end{pmatrix}
\]
hallemosla por el método propuesto.
\end{ejemplo}
Hay que resolver
\[\frac{dy}{dx}=-\frac{x}{y}\Rightarrow ydy=-xdx\Rightarrow y^2+x^2=C\]
Luego $r=\sqrt{x^2+y^2}$ es una integral primera. 



La coordenada $r$ es constante sobre los puntos en la gráfica de una solución de \eqref{eq:int_pri}. Sobre esos puntos $(x,y(x))$, la coordenada $s$ satisface:

\begin{equation} \label{eq:coor_can_s}
\begin{array}{lll}
\frac{ds}{dx}&=\frac{\partial s}{\partial x}+\frac{\partial s}{\partial y} y'(x) & \hbox{(Regla cadena)}\\
&=\frac{\partial s}{\partial x}+\frac{\partial s}{\partial y} \frac{\eta}{\xi} &
  \eqref{eq:int_pri}\\
&=\frac{1}{\xi} &\hbox{\eqref{eq:coor_can_dif}}.
\end{array}
\end{equation}
Ahora podemos aprovechar que ya conocemos $r$ y  expresar $y$ como función de $r,x$. Luego

\begin{teorema}[Expresión para $s$]{}
\boxedeq{s=\int\frac{ds}{dx}dx=\int\frac{dx}{\xi(x,y(r,x))}.}{eq:expr_coor_s}
\end{teorema}
 

En la igualdad resultante  reemplazamos $r$ por su expresión en las variables $x,y$.


Si ocurriese que $\xi=0$ y $\eta\neq 0$. Entonces por \eqref{eq:int_pri} $r_y=0$, de modo que $r$ es sólo función de $x$. Se puede asumir $r=x$. Además $\eta s_y=1$, entonces
\boxedeq{s=\int\frac{dy}{\eta(r,y)} .}{eq:s_xi_0}


\begin{ejemplo}{} Retornando al ejemplo de las rotaciones, donde hallamos que $r=\sqrt(x^2+y^2)$, vemos que
\[s=\int\frac{dx}{-y}=-\int\frac{dx}{\sqrt{r^2-x^2}}=\arccos\left(\frac{x}{r}\right).\]
Por consiguiente  $s$ es el ángulo polar.
\end{ejemplo}


\label{pag_ejem_canon1}
\begin{ejemplo}{} Encontrar coordenadas canónicas para el grupo de Lie de simetrías
\[(\hat{x},\hat{y})=(e^{\epsilon}x,e^{k\epsilon}y)\quad k>0.\]
\end{ejemplo}
El vector tangente es
\[(\xi,\eta)=\left(\left.\frac{d\hat{x}}{d\epsilon}\right|_{\epsilon=0},\left.\frac{d\hat{y}}{d\epsilon}\right|_{\epsilon=0}\right)=(x,ky).\]
Resolvamos la ecuación  \eqref{eq:int_pri} 
\[\frac{dy}{dx}=\frac{ky}{x}\Rightarrow y=Cx^k.\]
Luego $\Phi=y/x^k$ es integral primera. Entonces podemos tomar $r=y/x^k$.  Para $s$
\label{pag_ejem_canon2}
\[s=\int\frac{dx}{\xi}=\frac{dx}{x}=\ln|x|.\]
Entonces $(r,s)=(yx^{-k},\ln|x|)$ son coordenadas canónicas. No estan definidas en $x=0$.
Podemos encontrar coordenadas canónicas definidas en $x=0$ del siguiente modo. Recordamos que para todas $F$ y $G$
\[(\tilde{r},\tilde{s})=(F(r),G(r)+s)=(F(x^{-k}y),G(x^{-k}y)+\ln|x|).\]
Son canónicas también. Si tomamos $F(r)=1/r$ y $G(r)=\frac{1}{k}\ln|r|$, evitamos la singularidad. Luego

\[(\tilde{r},\tilde{s})=(x^ky^{-1},\frac{1}{k}\ln|y|).\]
Son canónicas, están definidas en $x=0$ pero no en $y=0$.  

%%%%%%%%%%%%%

\begin{ejemplo}{} Encontrar coordenadas canónicas para
\[(\hat{x},\hat{y})=\left( \frac{x}{1-\epsilon x},\frac{y}{1-\epsilon x}\right).\]
\end{ejemplo}

\[(\xi,\eta)=\left(\left.\frac{d\hat{x}}{d\epsilon}\right|_{\epsilon=0},\left.\frac{d\hat{y}}{d\epsilon}\right|_{\epsilon=0}\right)=(x^2,xy).\]
\label{pag:can_ejem_p}
\[\frac{dy}{dx}=\frac{\eta}{\xi}=\frac{y}{x}\Rightarrow \frac{y}{x}=\hbox{cte}.\]

Podemos tomar $r=y/x$. Para $s$
\[s=\int\frac{dx}{x^2}=-\frac{1}{x}.\]
Luego $(r,s)=(\frac{y}{x},-\frac{1}{x})$ son canónicas.

En este caso los puntos sobre $x=0$ son invariantes, no podemos definir coordenadas canónicas allí.

Lo podemos desarrollar con \texttt{SymPy}. 

\begin{sympyblock}[][frame=single]
x,y,epsilon=symbols('x,y,epsilon')
T=Matrix([x/(1-epsilon*x),y/(1-epsilon*x)])
xi=T[0].diff(epsilon).subs(epsilon,0)
print(xi)
eta=T[1].diff(epsilon).subs(epsilon,0)
print(eta/xi)
y=Function('z')(x)
sol=dsolve(y.diff(x)-y/x,y)
print(sol)
Integral(1/xi,x).doit()
\end{sympyblock}





\subsection{Infinitesimales$\to$Simetrías (Revisitado)}
Las coordenadas canónicas nos dan otra manera de encontrar simetrías a partir de los infinitesimales siguiendo el procedimiento:

\begin{enumerate}
\item Determinar las coordenadas canónicas  (sólo necesitamos conocer los infinitesimales).
\item Expresamos las relaciones $\hat{r}=r$ y $\hat{s}=s+\epsilon$ en las coordenadas $x,y$.
\end{enumerate}


\begin{ejemplo}{} Hallar el grupo de simetrías asociado a los infinitesimales 
$\xi=x^2$ y $\eta=xy$.
\end{ejemplo}


\begin{enumerate}
\item En la sección \ref{pag:can_ejem_p} hallamos las coordenadas canónicas $(r,s)=\left(\frac{y}{x},-\frac{1}{x}\right)$ asociadas a los infinitesimales dados.
\item Entonces
\[\begin{split}
(\hat{r},\hat{s})&=(r,s+\epsilon)\Rightarrow \frac{\hat{y}}{\hat{x}}=\frac{y}{x}\text{ , }-\frac{1}{\hat{x}}=-\frac{1}{x}+\epsilon\\
&\Rightarrow \hat{x}=\frac{x}{1-\epsilon x} \text{ , } \hat{y}=\frac{y}{1-\epsilon x}.
\end{split}
\] 
\end{enumerate}



\section{Resolviendo EDO con grupos de Lie de simetrías}
\subsection{Método de solución}
Finalmente, toda la teoría expuesta nos permite elaborar un método que puede resolver una ecuación dada supuesto que conocemos un grupo de Lie de simetrías de ella.  Supongamos dado  un grupo de Lie de simetrías $(\hat{x},\hat{y})$ de la ecuación
\boxedeq{y'=f(x,y)}{eq:eq_princi}
 Supongamos  que las simetrías son no triviales. Según \eqref{eq:sol_inv} debemos tener
 \[\eta(x,y)\not\equiv f(x,y)\xi(x,y)\]
La razón de esta condición es que si fuese falsa entonces la ecuación \eqref{eq:int_pri} es la misma que la ecuación \eqref{eq:eq_princi} y el método es inútil.

Supongamos $(r,s)$ coordenadas canónicas. La ecuación en las coordenadas $(r,s)$, según \eqref{eq:subsder}, se escribirá
\boxedeq{\frac{ds}{dr}=\hat{f}(r,s):=\frac{s_x+f(x,y)s_y}{r_x+f(x,y)r_y}.}{eq:prin_en_canon}

Las coordenadas canónicas se definen por \eqref{eq:coorcan} de modo que el grupo de simetrías actúe por traslación $(\hat{r},\hat{s})=(r,s+\epsilon)$.
Cómo se justifica en el Ejemplo  \ref{ejem:tras}, $\hat{f}$ es independiente de $s$, por consiguiente la ecuación se reduce a
\boxedeq{\frac{ds}{dr}=\hat{f}(r)}{eq:ecu_prin_canonica}
que se resuelve integrando.


\begin{ejemplo}{} Resolver
\[y'=xy^2-\frac{2y}{x}-\frac{1}{x^3},\quad x\neq 0,\]
sabiendo que la ecuación es invariante para el grupo de simetrías
\[(\hat{x},\hat{y})=(e^{\epsilon}x,e^{-2\epsilon}y).\]
\end{ejemplo}
Por los resultados de la sección \ref{pag_ejem_canon1}:
\[(r,s)=(x^2y,\ln|x|)\]
son canónicas. Según \eqref{eq:prin_en_canon} la ecuación en $(r,s)$ es:
\[\frac{dr}{ds}=\frac{\frac{1}{x}}{2xy+x^2\left(  xy^2-\frac{2y}{x}-\frac{1}{x^3}     \right)}=\frac{1}{x^4y^2-1}=\frac{1}{r^2-1}\]
\begin{wrapfigure}[15]{r}{7cm}
\begin{center}
\includegraphics[scale=.3]{imagenes/SolGrup.png}
\caption{Soluciones de  $y'=xy^2-\frac{2y}{x}-\frac{1}{x^3}$}\label{fig:sol_met_lie}
\end{center}
\end{wrapfigure}
Como sabíamos que debía suceder el resultado del segundo miembro no depende sólo de $r$. Integrando
\[s=\frac12\ln\left( \frac{r-1}{r+1}  \right)+C.\]
Sustituyendo
\[\ln|x|=\frac12\ln\left( \frac{x^2y-1}{x^2y+1}  \right)+C.\]
Despejando
\begin{equation}\label{eq:sol_ejem}y =-\frac{x^2 + C}{x^2(x^2 - C)}\end{equation}
La ecuación característica reducida \eqref{eq:sol_inv} es para la ecuación de este ejemplo:
\[0=\overline{Q}=-2y- \left(xy^2-\frac{2y}{x}-\frac{1}{x^3}  \right)x=-x^2y^2+\frac{1}{x^2}\]
Cuyas soluciones son
\[y=\pm\frac{1}{x^2}.\]
Que además son solución de la ecuación diferencial. La curva $y=-1/x^2$ se obtiene de \eqref{eq:sol_ejem} con $C=0$. La  curva $y=-1/x^2$. En la figura \ref{fig:sol_met_lie} graficamos las distintas soluciones.Las curvas azules y naranjas se corresponden con las gráficas de \eqref{eq:sol_ejem} con $c>0$ y $c<0$ respectivamente. La verde es la de $y=1/x^2$ y la roja de $y=-1/x^2$.









\begin{ejemplo}{} Resolver
\[y'=\frac{y+1}{x}+\frac{y^2}{x^3},\quad x\neq 0,\]
sabiendo que la ecuación es invariante para el grupo de simetrías
\[(\hat{x},\hat{y})=\left(\frac{x}{1-\epsilon x},\frac{y}{1-\epsilon x}   \right).\]
\end{ejemplo}
Ya hemos computado las coordenadas canónicas en la subsección \ref{pag:can_ejem_p}:
\[(r,s)=\left(\frac{y}{x},\frac{1}{x}\right).\]
Por \eqref{eq:prin_en_canon} la ecuación se escribe
\[\frac{dr}{ds}=\frac{-\frac{1}{x^2} }{-\frac{y}{x^2}+\frac{1}{x}\left(
\frac{y+1}{x}+\frac{y^2}{x^3}\right)}=\frac{1}{1+r^2},\]
cuya solución es
\[s=\arctan(r)+C\Rightarrow y=-x\tan\left(\frac{1}{x}+C\right).\]



\subsection{Ecuaciones homogéneas}
\begin{ejemplo}{} En este ejemplo deduciremos nuevamente el método de solución de ecuaciones homogéneas apelando a las simetrías. Es decir, queremos resolver la ecuación
\boxedeq{y'=F\left(\frac{y}{x}\right).}{eq:homogenea}
\end{ejemplo}
Aquí tenemos el Grupo de Lie de simetrías de cambio de escalas
\[(\hat{x},\hat{y})=(e^{\epsilon}x,e^{\epsilon}y).\] 
Por los resultados de la subsección \ref{pag_ejem_canon1}, $(r,s)=(y/x,\ln|x|)$ son canónicas y la ecuación se escribe
\[\frac{ds}{dr}=\frac{\frac{1}{x}}{-\frac{y}{x^2}+\frac{F\left(\frac{y}{x}\right)}{x}}=\frac{1}{F(r)-r}.\]
La solución general es 
\[\ln|x|=\int^{y/x}\frac{dr}{F(r)-r}+c.\]




\subsection{Método de Lie y \texttt{SymPy}}
\text{SymPy} Incorpora distintas estrategías para resolver ecuaciones por el método de Lie. Hay mucho por indagar al respecto, pero sólo vamos a mencionar una función para calcular los infinitesimales $(\xi,\eta)$.  Aprovechamos para mostrar como luce una consola de \texttt{ipython}, otra manera de usar \texttt{Python} y \texttt{SymPy}.

\begin{sympyblock}[][frame=single]
x=symbols('x')
y=Function('y')(x)
from sympy.solvers.ode import infinitesimals
xi_eta=infinitesimals((y+1)/x+y**2/x**3-y.diff(x))
 
\end{sympyblock}

\[\sympy{latex(xi_eta)}\]



\section{Diagrama conceptual}

\begin{center}
 \includegraphics[scale=.5]{imagenes/diagrama.png}
\end{center}


\begin{subappendices}
 

\section{Formas Diferenciales, una introducción ingenua}%\label{section:fromas}





 La  expresión \eqref{eq:gral_orden1} es más asimétrica, entre las variables $x$ e $y$ una de ellas es independiente ($x$) y la otra independiente  ($y$).  La expresión \eqref{eq:gral_orden1_FormDif}  es  más simétrica, las dos variables tienen el mismo estatus.

 Las expresiones del tipo \eqref{eq:gral_orden1_FormDif} representan un ente matemático importante llamado \href{http://es.wikipedia.org/wiki/Forma_diferencial}{forma diferencial} . No disponemos del tiempo necesario para dotar con entidad matemática el concepto de forma diferencial. Tampoco  resulta de vital importancia. Pero las reglas que rigen la combinación de las formas diferenciales son muy simples y nos gustaría describir este aspecto de las formas diferenciales brevemente; como motivación para un estudio posterior más profundo y para que el lector gane en confianza en su manipulación. Daremos así una pintura de las formas diferenciales incompleta, por cuanto sólo diremos como ellas se combinan pero no contestaremos la pregunta de que son. Sólo digamos, para advertir al lector sobre los requisítos teóricos que se requieren, que las formas diferenciales se encuentran asociados al concepto de variedad diferencial, concepto este que generaliza al de curva y superficie.  Más 
concretamente, las formas diferenciales son funciones que toman valores en los duales de los espacios tangentes a las variedades diferenciales. En particular hay formas diferenciales asociadas a los espacios euclideo, que podemos identificar con  $\rr^n$.

Aqui vamos a pensar a las formas diferenciales como objetos puramente formales sobre los que actúa un operador $d$, leyes de composición internas y externas. Se construyen formas diferenciales invocando un sistema de coordenadas $x_1,\ldots,x_n$. Estas coordenadas, deben ser coordenadas de alguna variedad diferencial, pero, por simplicidad, supondremos que son coordenadas cartesianas ortogonales de un espacio euclideo. Las formas diferenciales forman un espacio vectorial con una operación de suma $+$ y producto por un escalar. Además tienen definido un producto $\wedge$, llamado \href{https://es.wikipedia.org/wiki/Producto_exterior}{producto exterior} , que posee  algunas particularidades.  Como los polinomios, las formas diferenciales tienen grado. Por eso se dice que forman un álgebra graduada. A diferencia de los polinomios, este grado no supera la dimensión $n$ del espacio ($\rr^n$ o más generalmente una variedad) a la que están asociadas. Para ser más exactas, la única forma 
de 
grado $k>n$ es la trivial, esto es la nula. Denotamos $\Lambda_k$ las formas de grado $k$, siendo $\Lambda =\bigoplus_{k=0}^{n}\Lambda_k$ el espacio vectorial de todas ellas. Luego $d:\Lambda\to\Lambda$ y $\wedge:\Lambda\times\Lambda\to \Lambda$. Ahora describimos unas simples reglas de las formas,  $d$ y $\wedge$.



\begin{enumerate}
   \item Una 0-forma diferencial es una función $g(x_1,\ldots,x_n)$.

  \item Si $\alpha\in\Lambda_k$ y $\beta\in\Lambda_p$ entonces $\alpha \wedge \beta\in \Lambda_{k+p}$.
  \item El producto $\wedge$ es asociativo, distributivo y satisface una especia de anticonmutatividad, especídficamente si $\alpha\in\Lambda_k$ y $\beta\in\Lambda_p$ entonces $\alpha\wedge \beta=(-1)^{kp}\beta\wedge\alpha$. En particular $\alpha\wedge\alpha=0$ cuando $\alpha$ es un $k$-forma con $k$ impar.
  \item El diferencial satisface
  \begin{enumerate}
    \item Si $\omega$ es una $k$-forma diferencial $d\omega$ es una $k+1$ forma diferencial.
    \item  $d^2\omega=d(d\omega)=0$, para toda $\omega\in\Lambda$.
    \item Si $\alpha\in\Lambda_k$ entonces  $d(\alpha\wedge\beta)=d\alpha\wedge\beta+(-1)^k\alpha\wedge d\beta$.
    \item En el caso de $0$-forma (función) $g(x_1,\ldots,x_n)$ el diferencial se define
    \[dg=\frac{\partial g}{\partial x_1}dx_1+\cdots+\frac{\partial g}{\partial x_n}dx_n\]

  \end{enumerate}
  \item Las expresiones $dx_i$, $i=1,\ldots,n$ forman una especie de base del espacio de las formas $\Lambda$, en el sentido que cualquier $k$ forma $\alpha$ se expresa de la siguiente manera:
  \[\alpha=\sum_{1\leq i_1<\cdots i_k\leq n}g_{i_1\ldots i_k} dx_{i_1}\wedge\cdots\wedge dx_{i_n},\]
  para ciertas funciones $g_{i_1\ldots i_k}$. Observar que la suma se extiende sobre todos los subconjuntos ordenados de $\{1,\ldots,n\}$.
\end{enumerate}


Una $k$-forma diferencial $\alpha$ se llama \href{https://es.wikipedia.org/wiki/Formas_diferenciales_cerradas_y_exactas}{exacta}  cuando es el diferencial de una $k-1$-forma y se dice \href{https://es.wikipedia.org/wiki/Formas_diferenciales_cerradas_y_exactas}{cerrada}  cuando $d\alpha=0$. Las propiedades de las formas implican que toda forma exacta es cerrda. Un famoso \href{https://es.wikipedia.org/wiki/Formas_diferenciales_cerradas_y_exactas#Lema_de_Poincar.C3.A9}{Lema de Poincare}  trata con el recíproco de esta afirmación.

Las reglas anteriores permiten computar cualquiera de las operaciones.
\begin{ejemplo}{} Si $\alpha=M(x,y)dx+N(x,y)dy$ es una 1-forma de $\rr^2$, entonces
\[ \begin{split}
    d\alpha&=dM\wedge dx-Md^2x + dN\wedge dy-Nd^2y\\
    &=\left(\frac{\partial M}{\partial x}dx+ \frac{\partial M}{\partial y}dy\right)\wedge dx+
    \left(\frac{\partial N}{\partial x}dx+ \frac{\partial N}{\partial y}dy\right)\wedge dy\\
    &= \left(\frac{\partial N}{\partial x}- \frac{\partial M}{\partial y}\right) dx\wedge dy
   \end{split}
\]
Recordemos el famoso \href{https://es.wikipedia.org/wiki/Teorema_de_Green}{Teorema de Green} , que afirmaba que si $D\subset \rr^2$ era una región cuyo borde $C=\partial D$ era una curva cerrada simple, entonces
\[\ointctrclockwise_{\partial D} Mdx +Ndy=\iint_D \left(\frac{\partial N}{\partial x}- \frac{\partial M}{\partial y}\right) dx dy.\]
Utilizando formas diferenciales este resultado se escribe de la manera, mucho más compacta y sugerente
\[\ointctrclockwise_{\partial D} \alpha = \iint_Dd\alpha,\]
donde $\alpha =M(x,y)dx+N(x,y)dy$.  Esto es una relación clave de las formas diferenciales que se generaliza en un teorema fundamental de la matemática llamado \href{https://es.wikipedia.org/wiki/Teorema_de_Stokes}{Teorema de Stokes} .
\end{ejemplo}




\begin{ejemplo}{} Computemos $d^2g$, cuando $g$ es una función (0-forma).

\[
\begin{split}
d^2g&=d\left(\frac{\partial g}{\partial x_1}dx_1+\cdots+\frac{\partial g}{\partial x_n}dx_n\right)\\
&= d\left(\frac{\partial g}{\partial x_1}\right)\wedge dx_1+\cdots+d\left(\frac{\partial g}{\partial x_n}\right)\wedge dx_n
    -\frac{\partial g}{\partial x_1}\wedge d^2x_1-\cdots-\frac{\partial g}{\partial x_n}\wedge d^2x_n\\
    &=\left(\sum_{i=1}^n\frac{\partial^2g}{\partial x_i\partial x_1} dx_i\right)\wedge dx_1+\cdots+\left(\sum_{i=1}^n\frac{\partial^2g}{\partial x_i\partial x_n} dx_i\right)\wedge dx_n\\
    &= \sum_{i\neq 1}^n\frac{\partial^2g}{\partial x_i\partial x_1} dx_i\wedge dx_1+\cdots
    \sum_{i\neq n}^n\frac{\partial^2g}{\partial x_i\partial x_n} dx_i\wedge dx_n\\
    &=\sum_{1\leq i<j\leq n}^n\left\{\frac{\partial^2g}{\partial x_i\partial x_j}- \frac{\partial^2g}{\partial x_j\partial x_i}\right \} dx_i\wedge dx_j=0.
\end{split}
\]
Que es lo que tenía que ser.


\end{ejemplo}

\section{Formas diferenciales en Sympy}

En este apéndice resolveremos el problema planteado en el ejemplo \ref{ej:cambio_forma} usando formas diferenciales y Sympy.  Usaremos el módulo de geometría diferencial, donde se pueden definir formas diferenciales
y otros entes propios de la geometría diferencial: campos escalares  y
vectoriales. No es posible entender completamente este módulo sin introducir conceptos básicos de geometría diferencial, cosa que no haremos pues nos alejaría del propósito del curso. De modo que vamos a discutir superficialmente la solución de este problema.  Es suficiente importar del módulo de geometría diferencial el submodulo R2, que ya nos
crea las formas diferenciales $dx$, $dy$  (se accede por \texttt{R2.dx}, \texttt{R2.dy}), como
así también las formas $dr$ y $d\theta$ (\texttt{R2.dr}, \texttt{R2.dtheta}). Tecnicamente hablando R2 es una variedad diferencial --en este caso el espacio euclideano bidimensional-- con toda la estructura algebraica que lleva consigo.

En el siguiente código, luego de las sentencias de importanción, cada línea efectúa lo siguiente: 4-define la forma diferencial en coordenadas cartesianas, 5 y 6-calcula el valor de M y N en coordenadas polares --$Md\theta+N dr$--. Sin embargo el resultado, todavía tiene apelaciones a las coordenadas $x$ e $y$. Por este motivo en las líneas 7 y 8 introducimos los símbolos $r$ y $theta$ y en las líneas 9 a 11 sustituímos $x$ e $y$ por $r\cos\theta$ y $r\sen\theta$ respectivamente. Hay que notar que los simbolos que introdujimos $r$ y $\theta$ son diferentes de \texttt{R2.theta} y \texttt{R2.r}, a los que sympy despliega en negritas en la consola de ipython.  Por ese motivo, también sustituímos   \texttt{R2.theta} y \texttt{R2.r} por $\theta$ y $r$ respectivamente.

 \lstinputlisting[language=Python]{scripts/sust_formas.py}

La forma obtenida es $-rdr+(-r^4+r^2)d\theta$.




\section{Teoría de grupos computacional: GAP}


\begin{quote}
 <<
GAP (Groups, Algorithms, Programming) es un sistema de álgebra discreta computacional, con especial énfasis en la teoría de grupos computacional. GAP proporciona un lenguaje de programación, una biblioteca de miles de funciones implementando algoritmos algebraicos escritos en el lenguaje GAP, así como bibliotecas de datos de objetos algebraicos de gran tamaño [...] GAP se utiliza en la investigación y la enseñanza para estudiar grupos y sus representaciones, anillos, espacios vectoriales, álgebras, estructuras combinatorias, y más. El sistema, incluida la fuente, se distribuye libremente.>>
\end{quote}
\begin{flushright}
 \href{https://www.gap-system.org/}{(Página Oficial de GAP)} 
\end{flushright}

Hasta donde sabemos Sympy no implementa objetos de teoría de grupos. Sin embargo existe software libre para estudiar grupos. El más conocido de ellos es GAP. Este sistema provee un lenguaje muy sencillo y transparente. Lamentablemente el estudio de esta herramienta está más allá de los alcances de este documento. Desarrollemos un simple ejemplo.

\begin{ejemplo}{} En el siguiente ejemplo de sesión con línea de comandos de GAP introducimos sentencias que quedan indicadas en cada línea iniciada con el símbolo de sistema (prompt) \verb~>gap~. Las líneas que no inician con el símbolo de sistema indican la salida correspondiente a la sentencia que antecede dichas líneas. En línea 1 introducimos el grupo simétrico de orden 3 y lo llamamos \texttt{G}. En línea 3 pedimos que nos enumere los elementos de \texttt{G}. En línea 5, introducimos el elemento de \texttt{G} que en notación cíclica se escribe $(1,3,2)$. En líneas 7 y 9 calculamos potencias de \texttt{r}. En 11 definimos como \texttt{H} el subgrupo generado por
\texttt{r} y en 13 enumeramos los elementos de \texttt{H}. En 15 averiguamos si \texttt{H} es normal. Siendo este el caso, podemos calcular el grupo cociente en 17 y lo llamos \texttt{K}. Por último averiguamos el orden de \texttt{K} en 19.


\end{ejemplo}
\lstinputlisting[language=GAP]{scripts/gap_basico.g}



\end{subappendices}
\nocite{*}
  \bibliographystyle{apalike}
  \bibliography{Grupos_Simetrias.bib}


%\end{document}


%\chapter{Teoremas de Existencia y Unicidad}



\subsection{Introducción}

\begin{quote}

 
 <<El determinismo es una doctrina filosófica que sostiene que todo acontecimiento físico, incluyendo el pensamiento y acciones humanas, está causalmente determinado por la irrompible cadena causa-consecuencia, y por tanto, el estado actual determina en algún sentido el futuro...En física, el determinismo sobre las leyes físicas fue dominante durante siglos, siendo algunos de sus principales defensores Pierre Simon Laplace y Albert Einstein.>>
 
<<Podemos mirar el estado presente del universo como el efecto del pasado y la causa de su futuro. Se podría condensar un intelecto que en cualquier momento dado sabría todas las fuerzas que animan la naturaleza y las posiciones de los seres que la componen. Si este intelecto fuera lo suficientemente vasto para someter los datos al análisis, podría condensarse en una simple fórmula de movimiento de los grandes cuerpos del universo y del átomo más ligero; para tal intelecto nada podría ser incierto y el futuro, así como el pasado, estaría frente sus ojos>>

\end{quote}
\begin{flushright}
Laplace, citado en \cite{ wiki:determinismo} \cite{Mazzo}
\end{flushright}


La ciencia se vale de estructuras matemáticas para describir a traves de modelos la evolución de sistemas reales. Usualmente esas estructuras son ecuaciones (diferenciales) donde el tiempo es una de sus variables. Conocer el estado actual de un sistema (las posiciones que menciona Laplace) es conocer las condiciones iniciales. Mientras que las ecuaciones en si mismas se derivan de  las fuerzas que animan la naturaleza.  Así, para que la matemática colabore con la filosofía determinista, debería ella misma plantear problemas deterministas.  En ese sentido, hablando específicamente de ecuaciones diferenciales, para que podamos decir que un PVI sea determinista, el problema debería tener solución (existencia de soluciones), caso contrario este PVI no produciría ninguna predicción del futuro. Pero también esta solución debería ser única (unicidad de soluciones), porque de no ser así, el problema produciría múltiples predicciones y por tanto el estado actual 
no determinaría el futuro. Un problema que satisfaga estas condiciones lo denominaremos \emph{bien planteado}\footnote{Es común en la literatura requerir, además de la existencia y unicidad, la estabilidad para hablar de problema bien planteado} . 

Por lo expuesto, es de trascendencia para la ciencia en general que se pueda establecer que los problemas matemáticos que modelizan problemas de estas ciencias sean bien planteados. Este es el propósito de este capítulo.


Para nuestra sorpresa, PVIs sin ningun problema aparente a la vista no son problemas bien planteados.

\begin{ejemplo}{} Considerar el siguiente PVI:
\[
  \left\{
 \begin{array}{ll}
    y'(x)&=y(x)^{2/3}\\
    y(0)&=0\\
 \end{array}
 \right.
\]

La ecuacion  s en variables separables. Deberíamos dividir por $y(x)$, pero notar antes de hacer ello que $y\equiv 0$ es solución. Supongamos $y(x)\neq 0$, una vez dividido
\[\frac{dy}{y^{2/3}}=dx\Rightarrow 3y^{1/3}=x+C\Rightarrow y(x)=\left(\frac{x}{3}+C\right)^3.\]
A pesar de la limitación original de que $y(x)\neq 0$, la expresión para $y(x)$ que hemos hallado es solución para todo $x\in\rr$. Notar que  $y(x)=0$ cuando $x=-C/3$. Sin embargo, el valor problemático $y=0$ nos trae aparejado un inesperado problema. Ocure  que la función $y_C(x):=\left(\frac{x}{3}+C\right)^3$ tiene derivada igual a cero en $x=-C/3$. Esto implica que si $C_1<C_2$ entonces la función 
\[
  y_{C_1,C_2}(x):=
  \left\{
 \begin{array}{ll}
    y_{C_1}(x)&\hbox{ si } x\leq-C_2\\
    0         &\hbox{ si } x\in [C_1,C_2]\\
    y_{C_2}(x) &\hbox{ si } x\geq -C_1\\
  \end{array}
 \right.
\]
está bien definida y es diferenciable en $\rr$. Además de ello, cualquiera de estas funciones con $C_1\leq 0 \leq C_2$ resuelven el PVI.  De esta manera hemos encontrado un PVI con infinitas soluciones.
\end{ejemplo}

\begin{tabular}{m{6cm} m{6cm}}
\begin{minipage}{6cm}
  \lstinputlisting[language=Python]{scripts/no-unicidad.py}
\end{minipage}

&
\includegraphics[scale=.4]{imagenes/no-unicidad.png}
 \end{tabular}




\section{Sistemas de Ecuaciones Diferenciales}
\subsection{Definición y ejemplos}
Un \emph{sistema de ecuaciones diferenciales de primer orden} es un conjunto  de ecuaciones que relacionan una variable independiente, digamos $x$, un conjunto de variables dependientes, digamos $y_1(x),\ldots,y_n(x)$, y sus derivadas respecto a $x$. A las ecuaciones diferenciales con una incognita y una ecuación las denominaremos \emph{ecuaciones escalares}.

\begin{definicion} Un sistema de ecuaciones diferenciales es una conjunto de ecuaciones de la forma
\begin{equation}\label{eq:sist_ecua}
\left\{
\begin{split}
 \frac{dy_1}{dx}(x)&=f_1(x,y_1(x),\ldots,y_n(x))\\
  \frac{dy_2}{dx}(x)&=f_2(x,y_1(x),\ldots,y_n(x))\\
       &\,\,\vdots                                   \\
 \frac{dy_n}{dx}(x)&=f_n(x,y_1(x),\ldots,y_n(x))\\
\end{split}\right.,
\end{equation}
donde $f_j:[a,b]\times\rr^n\to\rr$, $j=1,\ldots,n$, son funciones.

Una manera alternativa y compacta de denotar un sistema se logra introduciendo las funciones $y=(y_1,\ldots,y_n)$  y $f=(f_1,\ldots,f_n)$. Entonces $y:[a,b]\to\rr^n$ es una función con valores en $\rr^n$ y $f:[a,b]\times \rr^n\to\rr^n$ es un campo vectorial dependiente de $x$. Con estas notaciones el sistema se escribe:
\begin{equation}\label{eq:sist_ecua_comp}
 y'(x)=f(x,y(x)).
\end{equation}

\end{definicion}






\begin{ejemplo} [Ecuación del péndulo]{} Si $x(t)$ es el ángulo que forma un péndulo, de longitud $l$, con la vertical en el tiempo $t$ y $v(t)=x'(t)$, entonces $x(t)$ y$v(t)$ deben satisfacer el siguiente sistema de  ecuaciones:
\begin{equation}\label{eq:sist_pend}
\left\{
\begin{split}
 x'(t)&=v(t)\\
  v'(t)&=-\frac{g}{l}\sen(x(t)).\\
\end{split}\right.
\end{equation}


\end{ejemplo}


\begin{ejemplo}[Sistemas de ecuaciones de Lotka-Volterra]{}
 En 1925 y 1926, Alfred J. Lotka y Vito Volterra respectivamente, introdujeron
 las \href{https://es.wikipedia.org/wiki/Ecuaciones_Lotka%E2%80%93Volterra}{ecuaciones de Lotka-Volterra}. \marginpar{\includegraphics[scale=.28]{imagenes/Vito_Volterra.jpg}\\
 Vito Volterra (1860-1940)}  Se trata de un sistema de dos ecuaciones diferenciales de primer orden que se usan para describir dinámicas de sistemas biológicos en el que dos especies interactúan, una como presa y otra como depredador. Se definen como:
\begin{equation}\label{eq:lotka_volterra}
\left\{
\begin{split}
 x'(t) &= x(t) ( \alpha - \beta  y(t) )\\
  y'(t)&=-y(t)(\gamma-\delta x(t))\\
\end{split}\right.
\end{equation}
La variable $y$ representa el número de individuos de algún predador (por ejemplo, un lobo) y $x$ es el número de sus presas (por ejemplo, conejos), $t$ representa el tiempo; y  $\alpha,\beta,\gamma$ y $\delta$ son parámetros (positivos)

\end{ejemplo}


\subsection{Sistemas de ecuaciones y ecuaciones de orden superior}

\emph{Es posible convertir el sistema \eqref{eq:sist_ecua} de $n$-ecuaciones diferenciales de primer orden  en una ecuación escalar de orden $n$
\begin{equation}\label{eq:orden_n}
y^{(n)}=f(x,y,y',\ldots,y^{(n-1)}).
\end{equation}
y viceverza.}

Para justificar la aseveración anterior supongamos que $y=y(x)$ resuelven  \eqref{eq:orden_n} y escribamos:
\[y_1=y,\, y_2=y',\,y_3=y'',\ldots, y_{n}=y^{(n-1)}.\]
Entonces notar que
\begin{equation}\label{eq:sist_ecua_conv}
\left\{
\begin{array}{l l l}
 y_1'(x)&=y_2(x)\\
  y_2'(x)&=y_3(x)\\
       &\,\,\vdots\\
  y_{n-1}'(x)&=y_n(x)\\
 y_n'(x)&=f(x,y_1(x),\ldots,y_n(x))\\
\end{array}\right.,
\end{equation}

Reciprocamente, supongamos que $y_1,\ldots,y_n$ resuelven \eqref{eq:sist_ecua}. Por simplicidad vamos a suponer que las $f_j$ son independientes de $x$, el procedimiento general sigue las mismas líneas que el caso que discutimos aquí. Se toma $y=y_n$ (podríamos usar cualquier $y_j$, $j=1,\ldots,n$). Ahora derivamos sucesivamente $n$-veces respecto a $x$ la ecuación para $y_n$,  y reemplazamos cada derivada $y_j'$ por $f_j$ (vamos a omitir los argumentos de $f_j$ que son en todos los casos $(y_1,\ldots,y_n)$):

\[%\label{eq:sist_ecua_conv}
\begin{array}{c c c}
y_n'(x) =& f_n(y_1,\ldots,y_n) &\eqqcolon g_1(y_1,\ldots,y_n)\\
 y_n''(x)=&\sum\limits_{j=1}^n\frac{\partial f_n}{\partial y_j}f_j+
 & \eqqcolon g_2(y_1,\ldots,y_n)\\
  y_n'''(x)=&\sum\limits_{k,j=1}^n\frac{\partial^2 f_n}{\partial y_k \partial y_j}f_kf_j+
  \sum\limits_{k,j=1}^n\frac{\partial f_n}{\partial y_j}\frac{\partial f_j}{\partial y_k}f_k & \eqqcolon g_k(y_1,\ldots,y_n)\\
  &\,\,\vdots &\,\,\vdots \\
  y^{(n)}_n(x) =&\cdots & \eqqcolon g_n(y_1,\ldots,y_n)
  \\
\end{array}
\]
%\end{equation}
Las igualdades anteriores tienen la estructura $z=G(y)$, donde $y=(y_1,\ldots,y_n)$, $z=(y_n',y_n'',\ldots,y_n^{(n)})$ y $G=(y_1,\ldots,y_n)$. Si la función $G:\rr^n\to\rr^n$ es invertible y escribimos $H=G^{-1}$ entonces
\[y_n=H_n(z)=H_n(y_n',y_n'',\ldots,y_n^{(n)}).\]
La anterior es una ecuación escalar de orden $n$ para $y_n$. Por consiguiente hemos logrado reducir el sistema de $n$-ecuaciones a una ecuación de orden $n$. Observar que si resolvemos esta ecuación, encontrando $y_n$, podemos hallar el resto de las incognitas $y_j$, $j=1,\ldots,n-1$,  usando que $y_j=H_j(y_n',y_n'',\ldots,y_n^{(n)})$.


\begin{ejemplo}[Ecuaciones de Lotka-Volterra.]{} Reduzcamos las ecuaciones de Lotka-Volterra a una ecuación de orden 2 y luego revirtamos el camino.

La primera parte la resolvemos con Sympy:
\lstinputlisting[language=Python]{scripts/Sist-Ecua.py}
Resulta en
\begin{equation}\label{eq:lotka_escalar}\frac{d^{2}}{d t^{2}}  y{\left (t \right )- \frac{d}{d t} y{\left (t \right )} +
\left(- \alpha \gamma - \alpha\right) y{\left (t \right )} + \left(\beta \gamma + \beta\right) y^{2}{\left (t \right )} }=0
\end{equation}

El camino inverso es más sencillo. Llamamos $z=y'$. Usando las variables $y,z$ la ecuación \eqref{eq:lotka_escalar} se escribe
\[
 \left\{
 \begin{array}{ll}
    y'(t)&=v(t)\\
    v'(t)&=v+\left( \alpha \gamma + \alpha\right) y{\left (t \right )} - \left(\beta \gamma + \beta\right) y^{2}{\left (t \right )}
 \end{array}
 \right.
 \]

 No llegamos a la ecuación de partida. Hay que tener presente que una ecuación tiene diferentes representaciones en diferentes variables y que las variables que hemos elegido para el camino de vuelta $y,v$, no son las originales del problema $y,x$.


\end{ejemplo}


\section{Método de iteraciones de Picard}

En esta sección vamos a describir la estrategia que emplearemos para la demostración de la existencia de soluciones. El método que seguiremos fue ideado por \href{https://es.wikipedia.org/wiki/Charles_%C3%89mile_Picard}{Émile Picard} \marginpar{
    \includegraphics[scale=.3]{imagenes/Picard.jpg}\\
    Émile Picard (1856-1941)
}

Introducimos previamente algunas definiciones elementales.

\begin{definicion}
 Sea $\alpha$ una función  definida en un intervalo $[a,b]\subset \rr$ con valores en $\rr^n$, $n\geq 1$, tal que cada componete es una función integrable. Entonces, como es usual, escribimos
 \[\int_a^b\alpha(s)ds=\left(\int_a^b\alpha_1(s)ds,\ldots,\int_a^b\alpha_n(s)ds\right) \]
\end{definicion}

Es un ejercicio muy sencillo  demostrar que vales las propiedades elementales de las integrales, para esta extensión del concepto de integral a funciones con valores vectoriales. En particular vale el Teorema Fundamental del Cálculo: \emph{Si $\alpha$ es continuamente diferenciable entonces
\[ \int_a^b\alpha'(s)ds=\alpha(b)-\alpha(a).\]
}

Además vale la \emph{desigualdad triangular:} si $a<b$
\[\left\|\int_a^b \alpha(s)ds\right\|\leq \int_a^b \left\|\alpha(s)\right\|ds.\]
Nuestra intención es demostrar la existencia de soluciones de un PVI para el sistema de EDOs
\boxedeq{
\left\{
\begin{array}{ll}
 y'(x)&=f(x,y)\\
 y(x_0)&=y_0\\
\end{array}
\right. ,
}{eq:PVI_prin}

donde $f:\Omega \subset (a,b)\times \rr^n\to \rr^n$, $\Omega$ abierto, $y:(a,b)\to \rr^n$ (se debe satisfacer que $(x,y(x))\in \Omega$, para $x\in [a,b]$), $x_0\in (a,b)$
e $y_0\in\rr^n$ son dados con $(x_0,y_0)\in \Omega$.



Integremos la ecuación diferencial en un intervalo de extremos $x_0$ y $x$ y tomando en consideeración las condiciones iniciales obtenemos una nueva ecuación para $y(x)$,  en este caso una \href{https://es.wikipedia.org/wiki/Ecuaci%C3%B3n_integral}{ecuación integral} :

\boxedeq{
y(x)=y_0+\int_{x_0}^xf(t,y(t))dt.
}{eq:ecua_int}

Esta ecuación presenta una estructura muy particular. Podemos pensar el miembro derecho de \eqref{eq:ecua_int} como una trasformación (función) $T$ que lleva la función $y=y(x)$ en la función
\[
 T(y)(x):=y_0+\int_{x_0}^xf(t,y(t))dt.
\]
Pensando de esta manera  \eqref{eq:ecua_int} se escribe sencillamente
\[
 T(y)=y.
\]
Vale decir, la ecuación integral expresa el hecho que $T$ lleva a la función $y$ en si misma. Esto en matemática es conocido como un \href{https://es.wikipedia.org/wiki/M%C3%A9todo_del_punto_fijo}{punto fijo}. En análisis numérico los puntos fijos son utilizados para resolver ecuaciones algebraicas del tipo $h(x)=x$, donde $h:\rr\to\rr$. En aquel copntexto se ve que un procedimiento para aproximar soluciones es interar la función $h$, i.e. dado un $x_0$ cualquiera considerar la sucesión de \emph{aproximaciones sucesivas}
\[x_n=h(x_{n-1}),\quad n=1,2,\ldots.\]
El fundamento de proceder así es que si la sucesión $x_n$ converge a algún valor $x^*$ y si $h$ es continua entonces $x^*$ resuelve $h(x)=x$ pues
\begin{equation}
\label{eq:pto_fijo}h(x^*)=h\left(\lim_{n\to\infty} x_n\right)=\lim_{n\to\infty} h(x_n)=\lim_{n\to\infty} x_{n+1}=x^*.
\end{equation}

Vamos a proceder por analogía y proponer el siguiente proceso iterativo que genera las funciones $\varphi_k$, $k=0,1,\ldots$:

\boxedeq{
\left\{
\begin{array}{ll}
\varphi_0&=y_0\\
\varphi_{k+1}(x)&=y_0+\int_{x_0}^xf(t,\varphi_k(t))dt.\\
\end{array}
\right. .
}{eq:iter_picard}

Este proceso se denomina \emph{método de las aproximaciones sucesivas de Picard} y fué propuesto por E. Picard en  \cite{EmilePicard1893} y luego generalizado por \href{https://es.wikipedia.org/wiki/Ernst_Leonard_Lindel%C3%B6f}{Ernest Lindelöf}  en \cite{ErnestLindelof1894}. \marginpar{
    \includegraphics[scale=.3]{imagenes/Lindelof.jpg}\\
   Ernst L. Lindelöf (1870—1946)
}



Deberemos indagar por condiciones que nos aseguren que la sucesión $\varphi_k$ converja a una función $\varphi$ y que esta función $\varphi$ es solución del PVI. Antes de adentrarnos en los detalles de las demostraciones, constatemos que la idea funciona en algunos ejemplos sencillos.

\begin{ejemplo}{} Resolver
\[
 \left\{
 \begin{array}{ll}
 y'(x)&=y(x)\\
 y(0)&=y_0\\
 \end{array}
  \right. ,
\]
donde $y:\rr\to\rr$ e $y_0$ es un punto arbitrario de $\rr$.

Aplicando el método de picard:
\[
 \begin{array}{lll}
  \varphi_0(x)&=y_0 &\\
  \varphi_1(x) &= y_0+\int_{0}^x \varphi_0(t)dt &= y_0(1+x)\\
  \varphi_2(x) &= y_0+\int_{0}^x \varphi_1(t)dt &= y_0(1+x+\frac{x^2}{2})\\
               &\,\vdots                        &\,\vdots                 \\
  \varphi_k(x) &= y_0+\int_{0}^x \varphi_k(t)dt &= y_0(1+x+\cdots+\frac{x^k}{k!})\\
 \end{array}
\]
Se aprecia que en $\varphi_k$ aparecen las sumas parciales del desarrollo en serie de Taylor alrededor de $0$ de la función exponencial. Luego tenemos que $\varphi_k$ converge a $y_0e^x$ que es justamente la solución delm PVI propuesto.
\end{ejemplo}

Más ejemplos será tratados en la actividad práctica.

\section{\href{https://es.wikipedia.org/wiki/Teorema_del_punto_fijo_de_Banach}{Teorema de punto fijo Banach}}

\begin{definicion} Sea $(X,d)$ un espacio métrico completo. Una función $K:X\to X$ se denominará una contracción si existe un $\theta\in [0,1)$ tal que
\[d(K(x),K(y))\leq \theta d(x,y),\quad x,y\in X.\]
\end{definicion}

Dada una función $K:X\to X$ escribiremos
\[K^n(x)=\underbrace{K\left( K \left( K\cdots K(x)\cdots \right)\right)}_{n-\hbox{veces}}.\]
Cuando $n=0$ ponemos $K^0(x)=x$.

\begin{teorema}[Principio de contracción de Banach]{teo:banach}  Sea $(X,d)$ un espacio métrico completo y $K:X\to X$  una contracción. Entonces $K$ tiene un único punto fijo $x^*$. Además para todo $x$
\[d(K^n(x),x^*)\leq \frac{\theta^n}{1-\theta}d(K(x),x).\]
\end{teorema}\marginpar{
    \includegraphics[scale=.3]{imagenes/banach.jpg}\\
   Stefan Banach (1892—1945)
}
\begin{proof} Sea $x\in X$ un punto arbitrario y definimos $x_n:=K^n(x)$. Entonces si $m<n$
\[
\begin{split}
 d(K^n(x),K^m(x))&\leq d(K^n(x),K^{n-1}(x))+\cdots+ d(K^{m+1}(x),K^m(x))\\
 &\leq \theta^{n-1} d(K(x),x)+\cdots + \theta^m d(K(x),x)\\
 &\leq \frac{\theta^m}{1-\theta}d(K(x),x)
 \end{split}
\]
Como $\theta<1$ vemos que si $m,n$ son suficientemente grandes podemos hacer que $ d(K^n(x),K^m(x))$ sea arbitrariamente chico. Hemos probado que la sucesión $K^n(x)$ es de Cauchy, de allí tiene límite, al que llamaremos $x^*$. Que $x^*$ es punto fijo se justifica como en \eqref{eq:pto_fijo} (notar que una contracción es continua).

Que el punto fijo es único, se deduce de suponer que existe otro $z^*$ y aplicar lque $K$ es contracción
\[d(x^* , z^*)=d(K(x^*) , K(z^*))\leq \theta d(x^* , z^*)<d(x^* , z^*).\]
Lo que es una contradicción.
\end{proof}

\begin{corolario}{cor:banach}  Sea $(X,d)$ un espacio métrico completo y $K:X\to X$ continua y $K^m$ contracción para algún $m\in\mathbb{N}$. Entonces $K$ tiene un único punto fijo $x^*$ y para todo $x\in X$ se tiene
\[ \lim_{n\to\infty} K^n(x)=x^*.\]
 \end{corolario}
 
 \begin{proof} Sea $x^*$ el único punto fijo de $K^m$ dado por el principio de contracción de Banach aplicado a $K^m$. Sean $x\in X$ y $0\leq r<m$. Consideremos las sucesiones $\{K^{km+r}(x)\}_{k\in\mathbb{N}}$. Tenemos $m$ de estas sucesiones, una para cada valor de $r=0,\ldots,m-1$. Todas ellas son subsucesiones de $\{K^n(x)\}_{n\in\mathbb{N}}$. Además 
  \[\bigcup_{r=0}^{m-1}\{K^{km+r}(x)\}_{k\in\mathbb{N}}=\{K^n(x)\}_{n\in\mathbb{N}}.\]
  Vamos a demostrar que estas $m$ subsucesiones convergen todas a $x^*$. Esto es facil consecuencia de que $K^m$ es contracción. Pues de ello obtenemos que $\left(K^{m}\right)^n(y)\to x^*$, cuando $n\to\infty$ para todo $y\in X$. Luego
  \[\lim_{k\to\infty}K^{km+r}(x)=\lim_{k\to\infty}\left(K^{m}\right)^k(K^r(x))=x^*.\]
  Esto establece que $ \lim_{n\to\infty} K^n(x)=x^*$.
  
  Veamos que $x^*$ es punto fijo de $K$. 
  \[K(x^*)=K\left(\lim_{n\to\infty} K^{n}(x)\right)=\lim_{n\to\infty} K\left( K^{n}(x)\right)
  =\lim_{n\to\infty}  K^{n+1}(x)=x^*.\]
  
  Para la unicidad, observar que un punto fijo de $K$ lo es de $K^m$. Cómo $K^m$ tiene un único punto fijo, lo mismo podemos afirmar de $K$.
  
 \end{proof}


\begin{definicion} Sea $f:\Omega \subset \rr^n\to \rr^m$. Diremos que $f$ es \emph{lipschitziana} (o que es una función Lipschitz) si existe una constante $K\geq 0$ tal que
\[\|f(x)-f(y)\|\leq L\|x-y\|,\quad \forall x,y\in \Omega .\]

Si $f:\Omega\subset\rr\times \rr^n\to\rr^m$, diremos que $f=f(t,x)$ es lipschitziana respecto a la segunda variable si existe  $K\geq 0$ tal que
\[\|f(t,x)-f(t,y)\|\leq L\|x-y\|,\quad \forall (t,x),(t,y)\in \Omega .\]
 
\end{definicion}



\begin{teorema}[Picard- Lindelöf]{teo:picard} Sean $x_0\in\rr$, $y_0\in\rr^n$ y $f:\Omega\to\rr^n$ continua y lipschitziana respecto a la segunda variable, donde $\Omega=I_a\times B_b$,  $I_a=[x_0-a,x_0+a]$ y $B_b=\{y\in\rr^n|\|y-y_0\|\leq b\}$. Como $\Omega$ es compacto y $f$ es continua, existe $M\geq 0$ tal que $\|f\|\leq M$. Entonces existe una única solución del PVI:
\[
 \left\{\begin{array}{ll}
	  y'(x)&=f(x,y(x)),\\
	  y(x_0)&=y_0\\         
        \end{array}
\right. ,
\]
en $I_{\delta}=(x_0-\delta,x_0+\delta)$ con $\delta:=\min\{a,b/M\}$.
 \end{teorema}

 \begin{proof}
 
 Definamos
\[X=C(I_{\delta},B_b):=\{\varphi| \varphi:I_a\to B_b\hbox{ continua }\}.\]
\begin{figure}[h]
  \begin{center}
\scalebox{.6} % Change this value to rescale the drawing.
{
\begin{pspicture}(0,-4.81)(10.393595,4.81)
\definecolor{color123b}{rgb}{0.9098039215686274,0.8431372549019608,0.8431372549019608}
\definecolor{color105b}{rgb}{0.8627450980392157,0.7607843137254902,0.7607843137254902}
\psellipse[linewidth=0.04,linestyle=dashed,dash=0.16cm 0.16cm,dimen=outer,fillstyle=solid,fillcolor=color105b](3.86,0.85)(0.44,1.3)
\psellipse[linewidth=0.04,linestyle=dashed,dash=0.16cm 0.16cm,dimen=outer](5.04,0.88)(0.4,1.29)
\psellipse[linewidth=0.04,linestyle=dashed,dash=0.16cm 0.16cm,dimen=outer,fillstyle=solid,fillcolor=color123b](6.07,0.86)(0.39,1.31)
\pspolygon[linewidth=0.04](0.0,-4.79)(0.04231884,2.37)(2.92,4.79)(2.86,-2.55)(2.88,-2.49)
\psbezier[linewidth=0.04](1.42,2.19)(0.96,2.05)(0.94,1.23)(0.92,0.93)(0.9,0.63)(0.92,-0.31)(1.42,-0.39)
\psbezier[linewidth=0.04,linestyle=dashed,dash=0.16cm 0.16cm](1.48,-0.41)(1.98,-0.27)(1.98,0.51)(1.98,0.93)(1.98,1.35)(1.9,2.27)(1.48,2.15)
\psline[linewidth=0.04cm](1.44,2.17)(7.1,2.15)
\psline[linewidth=0.04cm](1.44,-0.41)(7.02,-0.43)
\psellipse[linewidth=0.04,dimen=outer](7.0,0.86)(0.54,1.29)
\psline[linewidth=0.04cm,linestyle=dashed,dash=0.16cm 0.16cm](7.0,0.83)(1.46,0.81)
\psdots[dotsize=0.24](7.06,0.83)
\psdots[dotsize=0.12](1.5,0.83)
\psline[linewidth=0.04cm,arrowsize=0.05291667cm 2.0,arrowlength=1.4,arrowinset=0.4]{->}(1.5,0.77)(1.78,1.97)
\psline[linewidth=0.04cm,arrowsize=0.05291667cm 2.0,arrowlength=1.4,arrowinset=0.4]{->}(1.3,-1.77)(9.52,-1.71)
\usefont{T1}{ptm}{m}{n}
\rput(2.4290624,3.5){$\mathbb{R}^n$}
\usefont{T1}{ptm}{m}{n}
\rput(1.3590626,1.48){$b$}
\psline[linewidth=0.04cm,linestyle=dashed,dash=0.16cm 0.16cm](5.04,0.83)(5.0,-1.77)
\psline[linewidth=0.04cm,linestyle=dashed,dash=0.16cm 0.16cm](7.1,0.79)(7.1,-1.83)
\psline[linewidth=0.04cm,linestyle=dashed,dash=0.16cm 0.16cm](3.9,0.87)(3.86,-1.79)
\usefont{T1}{ptm}{m}{n}
\rput(5.0590625,-1.96){$x_0$}
\usefont{T1}{ptm}{m}{n}
\rput(6,-1.96){$x_0+\delta$}
\usefont{T1}{ptm}{m}{n}
\rput(3.8490624,-1.96){$x_0-\delta$}
\usefont{T1}{ptm}{m}{n}
\rput(1.5590626,0.52){$y_0$}
\psbezier[linewidth=0.04,arrowsize=0.05291667cm 2.0,arrowlength=1.4,arrowinset=0.4]{->}(6.4,3.39)(5.6,3.05)(6.6,2.49)(5.52,1.77)
\usefont{T1}{ptm}{m}{n}
\rput(6.4590626,3.42){$\Omega$}
\usefont{T1}{ptm}{m}{n}
\rput(9.049064,-1.48){$\mathbb{R}$}
\psline[linewidth=0.04cm,linestyle=dashed,dash=0.16cm 0.16cm](6.06,0.79)(6.08,-1.71)
\usefont{T1}{ptm}{m}{n}
\rput(7.2690625,-1.94){$x_0+a$}
\psbezier[linewidth=0.04,arrowsize=0.05291667cm 2.0,arrowlength=1.4,arrowinset=0.4]{->}(4.36,2.51)(4.5,1.53)(5.48,2.21)(5.12,0.95)
\psdots[dotsize=0.2](5.06,0.79)
\usefont{T1}{ptm}{m}{n}
\rput(4.2745314,2.78){$(x_0,y_0)$}
\end{pspicture} 
}
 \caption{Conjunto $I_{\delta}\times B_b$}\label{fig:exi_uni}
  \end{center}
\end{figure}
\begin{ejercicio} La función
\[d(\varphi_1,\varphi_2):=\max\left\{\|\varphi_1(x)-\varphi_2(x)\| \,\big| x\in I_{\delta}\right\}\]
  define una métrica sobre el conjunto $X$ y con esta métrica $(X,d)$ es completo. 
\end{ejercicio}

Para $\varphi\in X$, definamos una nueva función $K(\varphi)$ por:
\[K(\varphi)(x)=y_0+\int_{x_0}^xf(t,\varphi(t))dt.\]
\noindent\textbf{Afirmación 1:} $K:X\to X$. 

En efecto, si $x\geq x_0$ y $x\in I_{\delta}$ (si $x\leq x_0$ queda como ejercicio)
\[\|K(\varphi)(x)-y_0\|\leq \int_{x_0}^x\|f(t,\varphi(t))\|dt\leq M|x-x_0|\leq b.\]
Esto muestra que $K(\varphi):I_{\delta}\to B_b$. 

La continuad se establece del siguiente modo. Si $x_1<x_2$
\[
\|K(\varphi)(x_1)-K(\varphi)(x_2)\|\leq \int_{x_1}^{x_2}\|f(t,\varphi(t))\|dt\leq M|x_1-x_2|.\]

Lo que implica la continuidad. Entonces $K(\varphi)\in X$. 

\noindent\textbf{Afirmación 2:} $K^m$ es  contracción para $m$ suficientemente grande. 

La afirmación es consecuencia de la siguiente desigualdad, para $\varphi_1,\varphi_2\in X$ y $x\in I_{\delta}$
\[\|K^m(\varphi_1)(x)-K^m(\varphi)(x)\|\leq \frac{L^m|x-x_0|^m}{m!}d(\varphi_1,\varphi_2).\]
La prueba procede por induccón sobre $m$. Cuando $m=0$ es inmediata. Supuesta válida la desigualdad para $m$ vemos que (otra vez asumimos $x_0\leq x$)

\[
  \begin{split}
    \|K^{m+1}(\varphi_1)(x)&-K^{m+1}(\varphi)(x)\|=   \|K\left(K^{m}(\varphi_1)\right)(x)-K\left(K^{m}(\varphi)\right)(x)\|\\
    &=\left\|   
    \int_{x_0}^{x} f(t,K^m(\varphi_1)(t))- f(t,K^m(\varphi_2)(t)) dt
        \right\|\\
     &\leq    
    \int_{x_0}^{x} \left\|f(t,K^m(\varphi_1)(t))- f(t,K^m(\varphi_2)(t))\right\| dt\\
    &\leq     L\int_{x_0}^{x} \left\|K^m(\varphi_1)(t)- K^m(\varphi_2)(t))\right\| dt\\
    &\leq  \frac{L^{m+1}}{m!}d(\varphi_1,\varphi_2) \int_{x_0}^{x}  |t-x_0|^m dt\\
    &=\frac{L^{m+1}|x-x_0|^{m+1}}{(m+1)!}d(\varphi_1,\varphi_2)
  \end{split}
\]

Notar que $L^m|x-x_0|^m/m!$ es el valor absoluto del término general de la serie de Taylor de $e^{Lx}$ alrededor de $x_0$.Como esta serie tiene radio de convergencia infinito, el término general tiende a cero. Por consiguiente existe $m$ suficientemente grande para que $L^m|x-x_0|^m/m!<1$. Para este $m$, $K^m$ será contracción. Luego, por el Corolario \ref{cor:banach},  deducimos que $K$ tiene un único punto fijo. Vale decir que existe $y\in X$ con
\[
 y(x)=y_0+\int_{x_0}^xf(t,y(t))dt.
\]
Notar que $y(x_0)=y_0$, i.e. $y$ satisface la condición inicial. Por el teorema fundamental del cálculo $y'(x)=f(x,y(x))$, i.e. $y$ satisface la ecuación y por tanto, finalmente, es solución del PVI.\end{proof}


\begin{corolario}{} Sea $\Omega$ abierto de $\rr\times \rr^n$ y $f:\Omega\to\rr^n$ continua con $\partial f_j/\partial y_k)$ tambien continuas $j,k=1,\ldots,n$. Entonces para todo $(x_0,y_0)\in\Omega$ existe un entorno $V=[x_0-\delta,x_0+\delta]\times B_b(y_0)$ tal que el PVI
 \[
 \left\{\begin{array}{ll}
	  y'(x)&=f(x,y(x)),\\
	  y(x_0)&=y_0\\         
        \end{array}
\right. ,
\]
tiene solución única en $[x_0-\delta,x_0+\delta]$. Además el gráfico de esta solución está contenido en $V$..
\end{corolario}

\begin{proof} Sea $(x_0,y_0)\in\Omega$ arbitrario. Como $\Omega$ es abierto, existen $a,b>0$ tal que $[x_0-a,x_0+a]\times B_b(y_0)\subset \Omega$. Como $\partial f_j/\partial y_k)$ son continuas y $V$ es compacto, tenemos que existe $M>0$ con $|\partial f_j/\partial y_k)|\leq M$ en $V$, $j,k=1,\ldots,n$. El Teorema del valor medio para funciones en varias variables implica que existe una constante $C>0$ tal que si $(t,x),(t,z)\in V$
\[\|f(t,x)-f(t,z)\|\leq C M\|x-z\|.\]
i.e. $f$ es Lipschitziana respecto a la segunda variable en $V$. La conclusión del Corolario sigue del Teorema de Picard- Lindelöf. 
\end{proof}


   
\nocite{JorgeSotomayor513}.


% 
% 
%   \bibliographystyle{apalike}
%   \bibliography{diferenciales_ecuaciones,diferenciales_ecuaciones_sim}

%\documentclass{article}
%\documentclass[hyperref={colorlinks=true}]{beamer}
%\documentclass[handout,hyperref={colorlinks=true}]{beamer}


%%%%%%%%%%%%%%%%%%%%%%%%%%%%%%Paquetes%%%%%%%%%%%%%%%%%%%%%%%%%%%%%%%%%%%%%%%%%%%%%%%5
%%%%%%%%%%%%%%%%%%%%%%%%%%%%%%%%%%%%%%%%%%%%%%%%%%%%%%%%%%%%%%%%%%%%%%%%%%%%%%%%%%%%%
\usepackage{empheq}
\usepackage[spanish]{babel}
%\usepackage[utf8x]{inputenc}
\usepackage{times}
%\usepackage[T1]{fontenc}
\usepackage[latin1]{inputenc}
\usepackage{amssymb,amsmath,amsthm}
\usepackage{enumerate}
\usepackage{verbatim}
\usepackage{ esint }
%\usepackage{pst-all}
%\usepackage{pstricks-add}
\usepackage{array}
%\usepackage[T1]{fontenc}
\usepackage{animate}
%\usepackage{media9}
\usepackage{xparse}
\usepackage{listings}
\usepackage{ wasysym }
%\usepackage{sagetex}
\usepackage{yfonts,mathrsfs,eufrak}
\usepackage{hyperref}
\usepackage{color}
\usepackage{url}
\usepackage{theorem}
\usepackage{boiboites}
\usepackage{wrapfig}
\usepackage{ esint }
\definecolor{mygreen}{rgb}{0,0.6,0}
\definecolor{mygray}{rgb}{0.5,0.5,0.5}
\definecolor{mymauve}{rgb}{0.58,0,0.82}

\lstset{ %
  backgroundcolor=\color{white},   % choose the background color; you must add \usepackage{color} or \usepackage{xcolor}
  basicstyle=\footnotesize,        % the size of the fonts that are used for the code
  breakatwhitespace=false,         % sets if automatic breaks should only happen at whitespace
  breaklines=true,                 % sets automatic line breaking
  captionpos=b,                    % sets the caption-position to bottom
  commentstyle=\color{mygreen},    % comment style
  deletekeywords={...},            % if you want to delete keywords from the given language
  escapeinside={\%*}{*)},          % if you want to add LaTeX within your code
  extendedchars=true,              % lets you use non-ASCII characters; for 8-bits encodings only, does not work with UTF-8
  frame=single,	                   % adds a frame around the code
  keepspaces=true,                 % keeps spaces in text, useful for keeping indentation of code (possibly needs columns=flexible)
  keywordstyle=\color{blue},       % keyword style
  language=Python,                 % the language of the code
  otherkeywords={*,...},           % if you want to add more keywords to the set
  numbers=none,                    % where to put the line-numbers; possible values are (none, left, right)
  numbersep=5pt,                   % how far the line-numbers are from the code
  numberstyle=\tiny\color{mygray}, % the style that is used for the line-numbers
  rulecolor=\color{black},         % if not set, the frame-color may be changed on line-breaks within not-black text (e.g. comments (green here))
  showspaces=false,                % show spaces everywhere adding particular underscores; it overrides 'showstringspaces'
  showstringspaces=false,          % underline spaces within strings only
  showtabs=false,                  % show tabs within strings adding particular underscores
  stepnumber=2,                    % the step between two line-numbers. If it's 1, each line will be numbered
  stringstyle=\color{mymauve},     % string literal style
  tabsize=2,	                   % sets default tabsize to 2 spaces
  title=\lstname,                   % show the filename of files included with \lstinputlisting; also try caption instead of title
  belowskip=-0.5cm
}


%%%%%%%%%%%%%%%%%%%%%%%%%%Nuevos comandos entornos%%%%%%%%%%%%%%%%%%%%%%%%%%%%%%%%
%%%%%%%%%%%%%%%%%%%%%%%%%%%%%%%%%%%%%%%%%%%%%%%%%%%%%%%%%%%%%%%%%%%%%%%%
\newenvironment{demo}{\noindent\emph{Dem.}}{$\square$ \newline\vspace{5pt}}

\newcommand{\com}{\mathbb{C}}
\newcommand{\dis}{\mathbb{D}}
\newcommand{\rr}{\mathbb{R}}
\newcommand{\oo}{\mathcal{O}}
%\renewcommand{\emph}[1]{\textcolor[rgb]{1,0,0}{#1}}
\newcommand{\der}[2]{\frac{\partial #1}{\partial #2}}
\renewcommand{\v}[1]{\overrightarrow{#1}}
\renewcommand{\epsilon}{\varepsilon}
%\newcommand{\defverbatim}{\def{#1}}
\renewcommand{\b}[1]{\boldsymbol{#1}}
\renewenvironment{frame}[1]{}{}
%\newcommand{\end{proof}}{$\square$}
\DeclareMathOperator{\atan2}{atan2}
\DeclareMathOperator{\sen}{sen}


%%%%%%%%%%%%%%%%%%%%%%%%Colores
\definecolor{myblue}{rgb}{.8, .8, 1}
\definecolor{dblackcolor}{rgb}{0.0,0.0,0.0}
\definecolor{dbluecolor}{rgb}{0.01,0.02,0.7}
\definecolor{dgreencolor}{rgb}{0.2,0.4,0.0}
\definecolor{dgraycolor}{rgb}{0.30,0.3,0.30}
\newcommand{\dblue}{\color{dbluecolor}\bf}
\newcommand{\dred}{\color{dredcolor}\bf}
\newcommand{\dblack}{\color{dblackcolor}\bf}


%%%%%%%%%%Definimos una caja con color
\newlength\mytemplen
\newsavebox\mytempbox
\makeatletter
\newcommand\mybluebox{%
\@ifnextchar[%]
{\@mybluebox}%
{\@mybluebox[0pt]}}
\def\@mybluebox[#1]{%
\@ifnextchar[%]
{\@@mybluebox[#1]}%
{\@@mybluebox[#1][0pt]}}
\def\@@mybluebox[#1][#2]#3{
\sbox\mytempbox{#3}%
\mytemplen\ht\mytempbox
\advance\mytemplen #1\relax
\ht\mytempbox\mytemplen
\mytemplen\dp\mytempbox
\advance\mytemplen #2\relax
\dp\mytempbox\mytemplen
\colorbox{myblue}{\hspace{1em}\usebox{\mytempbox}\hspace{1em}}}
\makeatother
\DeclareDocumentCommand\boxedeq{ m g }{%
{\begin{empheq}[box={\mybluebox[2pt][2pt]}]{equation}% #1%
\IfNoValueF {#2} {\label{#2}}%
#1
\end{empheq}
}%
}


%%%%%%%%%%%%%%%%%%
\newboxedtheorem[boxcolor=orange, background=blue!5, titlebackground=blue!20,
titleboxcolor = black,thcounter=section]{problema}{Problema}{thcounter1}

\newboxedtheorem[boxcolor=orange, background=blue!5, titlebackground=blue!20,
titleboxcolor = black,thcounter=section]{teorema}{Teorema}{thcounter2}

\newboxedtheorem[boxcolor=orange, background=blue!5, titlebackground=blue!20,
titleboxcolor = black,thcounter=section]{definicion}{Definici\'on}{thcounter3}

\newboxedtheorem[boxcolor=orange, background=blue!5, titlebackground=blue!20,
titleboxcolor = black,thcounter=section]{lema}{Lema}{thcounter4}

\newboxedtheorem[boxcolor=orange, background=blue!5, titlebackground=blue!20,
titleboxcolor = black,thcounter=section]{corolario}{Corolario}{thcounter5}

\newboxedtheorem[boxcolor=orange, background=blue!5, titlebackground=blue!20,
titleboxcolor = black,thcounter=section]{proposicion}{Proposici\'on}{thcounter6}

\newboxedtheorem[boxcolor=orange, background=blue!5, titlebackground=blue!20,
titleboxcolor = black,thcounter=section]{codigo}{Funci�n SymPy}{}




\newcounter{ejemplo_cont}
\setcounter{ejemplo_cont}{1}

\newenvironment{ejemplo}{\noindent\textbf{Ejemplo  \arabic{ejemplo_cont}.} }{\addtocounter{ejemplo_cont}{1}}
%%%%%%%%%%%%%%%%%%%%%%%%%%%%%%%%%%%%%%%%%%%%%%%%%%%%%%%%%%%%%%%%%%%%%%%%%%%%%%%%%%%%%%%%%%%%%%%%%%%%%%%%%%%
%%%%%%%%%%Para escibir en clase articulo o similar






\title{Ecuaciones Lineales de Segundo Orden}
\date{}

\begin{document}




\tableofcontents




\section{Introducci�n}


\begin{quote}
``Me convert� en ateo porque como estudiante de  post-grado en f�sica cu�ntica, la vida parec�a ser reducible a ecuaciones diferenciales de segundo orden. Matem�ticas, qu�mica y f�sica ten�an todo y yo no veo ninguna necesidad de ir m�s all� de eso.''
\end{quote}
\begin{flushright}
Francis Collins
\end{flushright}




\begin{definicion}[Ecuaci�n lineal general de segundo orden]
 \boxedeq{\frac{d^2y}{dx^2}+p(x)\frac{dy}{dx}+q(x)y=r(x),}{ec_2_gen}
donde $p,q,r$ son funciones definidas en un intervalo $I=(a,b)$ de $\rr$ con valores en $\rr$.Si $r\equiv 0$ se llama homog�nea
\boxedeq{\frac{d^2y}{dx^2}+p(x)\frac{dy}{dx}+q(x)y=0,}{ec_2_gen_hom}
\end{definicion}







\begin{teorema}[Teorema de existencia y unicidad de soluciones]
Supongamos $p,q,r$ continuas sobre $I$. Sean $x_0\in I$ e $y_0,y_1\in\rr$ dados. Entonces existe una �nica soluci�n del PVI
\[\left\{
\begin{array}{l l l}
\frac{d^2y}{dx^2}+&p(x)\frac{dy}{dx}+q(x)y=r(x),&x\in I\\
y(x_0)&=y_0&\\
y'(x_0)&=y^1_0&\\
\end{array}\right.
\]
\end{teorema}
\begin{proof} M�s adelante.

\end{proof}


\section{Estructura del conjunto de soluciones}
\begin{teorema}
Si $y_1$ e $y_2$ son soluciones de \eqref{ec_2_gen_hom} y $c_1,c_2\in\rr$ entonces $c_1y_1+c_2y_2$ es soluci�n. Vale decir, el conjunto de soluciones
es un espacio vectorial. En particular $y\equiv 0$ es una soluci�n, a la que llameremos \emph{trivial}.
\end{teorema}

\begin{proof}
 El operador
\[L[y]:=y''+py'+qy\]
es lineal, por consiguiente
$L[c_1y_1+c_2y_2]=c_1L[y_1]+c_2L[y_2]=0.$
\end{proof}


\begin{teorema}
 Supongamos que $y_p$ es una soluci�n particular de \eqref{ec_2_gen} y que $y_g=y_g(x,c_1,c_2)$ es una soluci�n
general de \eqref{ec_2_gen_hom}. Entonces $y=y_p+y_g$ es soluci�n general de \eqref{ec_2_gen}.
\end{teorema}



\begin{proof}
El operador
\[L[y]:=y''+py'+qy\]
es lineal, por consiguiente
$L[y_g+y_p]=L[y_g]+L[y_p]=0+r=r.$
Rec�procamente supongamos $y$ soluci�n de $L[y]=r$, entonces
$L[y-y_p]=L[y]-L[y_p]=r-r=0.$
Luego debe haber $c_1$ y $c_2$ con $y(x)-y_p(x)=y_g(x,c_1,c_2)$.

\end{proof}


Volviendo a las ecuaciones homog�neas, supongamos que tenemos dos soluciones de \eqref{ec_2_gen_hom} $y_1$ e $y_2$. Entonces la expresi�n
\begin{equation}\label{comb_lin}
c_1y_1+c_2y_2,\quad c_1,c_2\in\rr
\end{equation}
es soluci�n tambi�n. Notar que en la expresi�n aparecen dos constantes y hab�amos dicho que era de esperar que la soluci�n general de una ecuaci�n de orden 2 contuviese
precisamente dos constantes de integraci�n. De modo que podemos conjeturar que \eqref{comb_lin} es soluci�n general de \eqref{ec_2_gen_hom}.
Hay una situaci�n especial, si, por ejemplo, $y_1=ky_2$, $k\in\rr$,entonces $c_1y_1+c_2y_2=(c_1k+c_2)y_2=cy_2$. Vale decir la combinaci�n lineal \eqref{comb_lin}
termina siendo s�lo combinaci�n lineal de la funci�n $y_2$ y por ende siendo esencialmente una expresi�n uniparam�trica.

\begin{definicion}[Independencia lineal]
 Un conjunto finito de funciones $\{y_1,\ldots,y_n\}$ se dir� linealmente independiente sobre un conjunto $I$,
si la �nica soluci�n de $c_1y_1(t)+\cdots+c_ny_n(t)=0$, para $t\in I$, es $c_1=c_2=\cdots=c_n=0$.
\end{definicion}




\begin{definicion}[Definici�n wronskiano]
 Dadas $n$ fuciones $\{y_1,\ldots,y_n\}$ con dominio $I$ el wronskiano $W(x)=W(y_1,y_2,\ldots,y_n)(x)$ de estas funciones en un punto $x\in I$ se define por
\boxedeq{W(x)=\det\begin{pmatrix}
y_1(x) & y_2(x) & \cdots &y_n(x)\\
y_1'(x) & y_2'(x) & \cdots &y_n'(x)\\
\vdots & \vdots &\ddots& \vdots\\
y_1^{(n-1)}(x) & y_2^{(n-1)}(x) & \cdots &y_n^{(n-1)}(x)\\
\end{pmatrix}}{wronskiano}
\end{definicion}




\begin{lema}[Propiedades Wronskiano I]
 Sea $\{y_1,\ldots,y_n\}$ un conjunto de $n$ funciones. Si existe un $x_0\in I$ con $W(x_0)\neq 0$ entonces $\{y_1,\ldots,y_n\}$ son
linealmente independientes
\end{lema}


\begin{proof} Supongamos que $c_1y_1+\cdots+c_ny_n\equiv 0$. Derivando $n-1$ veces esta igualdad y evaluando el resultado en $x_0$ obtenemos
\[
\begin{split}
c_1y_1(x_0)+\cdots+c_ny_n(x_0)&=0\\
c_1y_1'(x_0)+\cdots+c_ny'_n(x_0)&=0\\
\vdots \quad& \quad\vdots\\
c_1y_1^{(n-1)}(x_0)+\cdots+c_ny^{(n-1)}_n(x_0)&=0\\
\end{split}
\]
Las igualdades anteriores dicen que el vector $(c_1,\ldots,c_n)^t$ pertenece al nucleo de la matriz
\[
\begin{pmatrix}
y_1(x_0) & y_2(x_0) & \cdots &y_n(x_0)\\
y_1'(x_0) & y_2'(x_0) & \cdots &y_n'(x_0)\\
\vdots & \vdots &\ddots& \vdots\\
y_1^{(n-1)}(x_0) & y_2^{(n-1)}(x_0) & \cdots &y_n^{(n-1)}(x_0)\\
\end{pmatrix}
\]
Como por hip�tesis la matr�z es no singular, debe ocurrir que $c_1=c_2=\cdots c_n=0$.
\end{proof}

\begin{teorema}[Teorema. Propiedades wronskiano II, F�rmula de Abel]
 Supongamos que $y_1$ e $y_2$ son soluci�n de
\begin{equation}\label{eq2orden}\frac{d^2y}{dx^2}+p(x)\frac{dy}{dx}+q(x)y=0,\quad x\in I=(a,b)\end{equation}
Entonces existe $c\in\rr$ que satisface
\boxedeq{W(y_1,y_2)(x)=ce^{-\int p dx}.}{formu_abel}
Esta expresi�n se denomina \href{http://en.wikipedia.org/wiki/Abel's_identity}{f�rmula de Abel}. En particular vale que
\[\exists x_0\in I: W(x_0)\neq 0 \Longleftrightarrow \forall x\in I: W(x)\neq 0 .\]
\end{teorema}

\begin{proof} Tenemos que
\[W(x)=y_1(x)y_2'(x)-y_1'(x)y_2(x).\]
Derivando y usando \eqref{eq2orden}
\[\begin{split}W'(x)&=y_1y_2''-y_1y_2''\\
&=y_1(-py_2'-qy_2)-y_2(-py_1'-qy_1)\\
&=-pW.
\end{split}
\]
Vale decir $W$ resuelve la ecuaci�n $W'=-pW$ la cual es facilmente resoluble, mostrando su resoluci�n que se satisface \eqref{formu_abel}
\end{proof}

\begin{teorema}[Propiedades wronskiano III]
 Sean $y_1$ e $y_2$ soluciones de \eqref{eq2orden}. Entonces son equivalentes
\begin{enumerate}
\item\label{item1} $y_1$ e $y_2$ son linealmente indepenientes en $I$.
\item\label{item2} $W(y_1,y_2)(x)\neq 0$ para todo $x\in I$.
\end{enumerate}
\end{teorema}


\begin{proof} Que \ref{item2} implica \ref{item1} es consecuencia de la propiedad del wronskiano I.
Veamos que \ref{item1} implica \ref{item2}. Supongamos que exista un $x_0$ con $W(x_0)=0$. Esto quiere decir que una de las columnas
de la matr�z wronskiana en $x_0$ es m�ltiplo de la otra. Supongamos que $y_2(x_0)=ky_1(x_0)$ e $y'_2(x_0)=ky'_1(x_0)$. Esto quiere decir que $y_2$ y $ky_1$
resuelven el mismo pvi. Por lo tanto $y_2(x)=ky_1(x)$ para todo $x$. Lo que nos dice lo contrario de \ref{item1}
\end{proof}


\begin{teorema}[Estructura del conjunto de soluciones, ecuaci�n  homog�nea]
 Si $y_1$ e $y_2$ son soluciones linealmente independientes de
\[\frac{d^2y}{dx^2}+p(x)\frac{dy}{dx}+q(x)y=0,\quad x\in I=(a,b)\]
entonces
\boxedeq{ y(x,c_1,c_2)=c_1y_1+c_2y_2}{comb_lin2}
es soluci�n general.
\end{teorema}

\begin{proof} Que la expresi�n \eqref{comb_lin2} es soluci�n ya lo hemos dicho. Restar�a ver que cualquier soluci�n se escribe como en \eqref{comb_lin2}.
Sea $y$ cualquier soluci�n y $x_0\in I$. La matriz wronskiana
\[\begin{pmatrix}
y_1(x_0) & y_2(x_0)\\
y'_1(x_0) & y'_2(x_0)\\
\end{pmatrix}
\]
Es no singular dado que el determinante es no nulo. Por este motivo el sistema
\[ \begin{split}
c_1 y_1(x_0) + c_2y_2(x_0) &=y(x_0)\\
c_1y'_1(x_0) + c_2y'_2(x_0) &=y'(x_0)\\
\end{split}
\]
tiene soluci�n para $c_1$ y $c_2$. De este modo vemos que la funci�n $c_1y_1+c_2y_2$ resuelve el PVI
\[\left\{
\begin{array}{l l l}
\frac{d^2z}{dx^2}+&p(x)\frac{dz}{dx}+q(x)z=0,&x\in I\\
z(x_0)&=y(x_0)&\\
z'(x_0)&=y'(x_0)&\\
\end{array}.\right.
\]
Evidentemente $y$ es soluci�n tambi�n, por el Teorema de Existencia y Unicidad vemos que $y=c_1y_1+c_2y_2$ \end{proof}

\section{Reducci�n de orden}

Como conclusi�n de los anterior, vemos que si queremos resolver \eqref{eq2orden} debemos conseguir dos soluciones linealmente independientes.
Suponiendo que ya contamos con una soluci�n no trivial vamos a describir un m�todo
que posibilita encontrar otra soluci�n $y_2$ linealmente independiente de $y_1$.
El m�todo consiste en proponer que $y_2$ se escribe
\[\boxed{y_2(x)=v(x)y_1(x)}.\]
Sustituyendo este ansatz en la ecuaci�n
\[
\begin{split}
0&=y_2''+py_2'+qy_2\\
&=y_1v''+2v'y_1'+vy_1''+pv'y_1+pvy_1'+qvy_1\\
&=y_1 v''+(2y_1'+py_1)v'+v(y_1''+py_1'+qy_1)\\
&=y_1 v''+(2y_1'+py_1)v'
\end{split}
\]
La f�rmula anterior es nuevamente una ecuaci�n de segundo orden para $v$,
pero en este caso afortunadamente contamos con herramientas para resolverla puesto que se trata de una ecuaci�n donde la
variable dependiente $v$ no aparece expl�citamente, sino que aparecen sus derivadas $v'$ y $v''$. Hay que intentar la sustituci�n $w=v'$.
Luego
\[
y_1w''+(2y_1'+py_1)w=0
\]
Recordar que $y_1$ la asumimos conocida y que $p$ es obviamente conocida, as� $2y_1'+py_1$ es una funci�on conocida. La ecuaci�n es una ecuaci�n lineal homog�nea de primer orden.
Usando la f�rmula para resolver este tipo de ecuaci�n, obtenemos
\[w(x)=Ce^{-\int \frac{y_1'}{y_1}+p dx}=Ce^{-2\ln|y_1|}e^{-\int p dx}=C\frac{1}{y_1^2}e^{-\int p dx} \]
Es suficiente encontrar s�lo una funci�n $v$, de all� podemos tomar $C=1$.
\boxedeq{w(x)= \frac{1}{y_1^2}e^{-\int p dx}\Longrightarrow v(x)= \int \frac{1}{y_1^2}e^{-\int p dx}dx }{reduc_orden}


Otra manera de testear la independencia lineal de dos funciones $y_1$ e $y_2$ es notar que si fueran linealmente dependientes e $y_1\neq 0$
en un conjunto $J\subset I$ entonces $y_2/y_1$
ser�a constante.
Luego uno chequear�a independencia si comprobase que $y_2/y_1$ no es constante en alg�n subdominio $J\subset I$.
En el caso anterior $y_2/y_1=v$, luego
deber�amos tener $v$ no constante sobre alg�n subconjunto $J$. Pero $v$ constante implicar�a $y_1^{-2}e^{-\int pdx}=0$ y esto claramente no ocurre. De modo que por el
m�todo anterior encontramos dos soluciones independientes.

\section{Ecuaciones homog�neas con coeficientes constantes}

Consideramos la ecuaci�n
\boxedeq{y''+py'+qy=0,\quad p,q\in\rr}\label{2orden_coef_ctes}
Propongamos una soluci�n de la forma
\[\boxed{y(x)=e^{\lambda x},\quad \lambda\in\mathbb{C}}\]
Reemplazando en la ecuaci�n
\[(\lambda^2+\lambda p+q)e^{\lambda x}=0.\]
Se debe satisfacer la llamada \emph{ecuaci�n caracter�stica}
\boxedeq{\lambda^2+p\lambda+q=0}{ecua_carac}
Tenemos tres casos acorde al valor de $\Delta:=p^2-4c$

\begin{enumerate}

\item \noindent \textbf{ $\b{\Delta=p^2-4c>0}$, raices reales distintas $\b{\lambda_1$, $\lambda_2}$}. Este es el caso m�s sencillo de todos, obtenemos las soluciones
\[y_1(x)=e^{\lambda_1 x}\quad\text{y}\quad y_2(x)=e^{\lambda_2 x}.\]
Para chequear la independencia
\[\frac{y_2}{y_1}=e^{(\lambda_2-\lambda_1)x}\neq\text{cte}.\]
Luego
\boxedeq{y(x,c_2,c_2)=c_1e^{\lambda_1 x}+c_2e^{\lambda_2 x}.}{sol_gen1}
es soluci�n general


\item \noindent  \textbf{$\b{\Delta=p^2-4c<0}$, raices complejas conjugadas $\lambda_1=\mu+i\nu$, $\lambda_2=\mu-i\nu$, $\mu,\nu\in\rr$}.
Proponemos una soluci�n de la forma
\[
y(x)=e^{\mu x}v(x)
\]
Hagamos los c�lculos con \texttt{SymPy}





\begin{lstlisting}
>>> x,p,q=symbols('x,p,q')
>>> v=Function('v')(x)
>>> y=exp(-p/2*x)*v
>>> ecua=y.diff(x,2)+p*y.diff(x)+q*y
>>> simplify(ecua/exp(-p/2*x))

\end{lstlisting}

\[-\frac{p^{2}}{4} v{\left (x \right )} + q v{\left (x \right )} + \frac{d^{2}}{d x^{2}}  v{\left (x \right )}=0
 \]

Como $\nu^2:=-\frac{1}{4}(p^2-4q)>0$, $v$ resuelve  la ecuaci�n del oscilador arm�nico con frecuencia $\nu$. Recordar que la soluci�n general
para $v$ es
\[v(x)=C_1\cos \nu x +C_2\sen \nu x,\]
y de all�
\boxedeq{y(x)=e^{\mu x}\left\{C_1\cos \nu x +C_2\sen \nu x\right\}}{sol_gen_2caso}

Seguidamente presentamos las gr�ficas de las soluciones para distiontos valores de $\mu$.

\begin{center}
 \begin{tabular}{c c c}
\includegraphics[scale=.1]{imagenes/mu_neg.png} & \includegraphics[scale=.1]{imagenes/mu_pos.png} &
\includegraphics[scale=.1]{imagenes/mu_cero.png}\\
$\mu<0$ & $\mu>0$ & $\mu=0$\\
\end{tabular}

\end{center}








\item \noindent\textbf{ $\b{\Delta=p^2-4c=0}$, raices iguales }. Conocemos una soluci�n $\boxed{y_1=e^{-\frac{p}{2}x}}$. Podemos hallar otra
por el m�todo de reducci�n de orden. Esto consiste en proponer otra soluci�n de la forma $y_2(x)=y_1(x)v(x)$ Dejemos que lo haga \texttt{SymPy}


\begin{lstlisting}
>>> x,p=symbols('x,p')
>>> y=Function('y')(x)
>>> v=Function('v')(x)
>>> y=v*exp(-p/2*x)
>>> ecua=y.diff(x,2)+p*y.diff(x)+p**2/4*y
>>> ecuav=simplify(ecua/exp(-p/2*x))
>>> ecuav
\end{lstlisting}
Se obtiene
\[\frac{d^{2}}{d x^{2}}  v{\left (x \right )}=0.\]
La soluci�n general para $v$ es $v=c_1+c_2x$. As� el m�todo mencionado proporciona la soluci�n extra
\[\boxed{y_2(x)=xe^{-\frac{p}{2}x}}.\]
\end{enumerate}


\section{Ecuaci�n no homog�nea}
\subsection{M�todo coeficientes indeterminados}
Intentamos resolver
\boxedeq{\frac{d^2y}{dx^2}+p\frac{dy}{dx}+qy=r(x),}{ec_2_nohom}
donde $p,q,r \in \rr$, $r\in C(I)$ $r\neq 0$. El m�todo consiste en buscar soluciones en la misma clase de funciones a la que pertenece $r(x)$. Funciona de manera met�dica s�lo para algunos tipos de funciones $r(x)$.
Concretamente para $r(x)$ combinaci�n lineal de funciones polin�micas, exponenciales $e^{\alpha x}$ o trigonom�tricas $\cos \alpha x$ y $\sen \alpha x$.
Lo vamos a ilustrar con ejemplos para cada caso.



\begin{enumerate}
\item \textbf{Caso $\b{r(x)=e^{a x}}$ y $\b{a^2+pa+q\neq 0}$.}
En esta situaci�n se propone como soluci�n una funci�n de la forma $\boxed{y(x)=Ae^{ax}}$. Usamos \texttt{SymPy} para el c�lculo
\begin{lstlisting}
>>> x,p,q,a,A=symbols('x,p,q,a,A')
>>> y=A*exp(a*x)
>>> ecua=y.diff(x,2)+p*y.diff(x)+q*y-exp(a*x)
>>> ecua=simplify(ecua/exp(a*x))
>>> ecua
A*a**2 + A*a*p + A*q - 1
>>> solve(ecua,A)
[1/(a**2 + a*p + q)]
\end{lstlisting}


Si $a^2+pa+q\neq 0$, encontramos la soluci�n particular $\boxed{y(x)=\frac{1}{(a^2+pa+q)}e^{ax}}$.




\item \textbf{Caso $\b{r(x)=e^{a x}}$ y $\b{a^2+pa+q= 0}$.}
En esta situaci�n diremos que la ecuaci�n est� en \emph{resonancia}. M�s generalmente, diremos que se presenta resonancia cuando $r(x)$ es soluci�n
del problema homog�neo.
Propongamos como soluci�n $y(x)=Axe^{ax}$. Hagamos los c�lculos con \texttt{SymPy}.


\begin{lstlisting}
>>> x,p,q,a,A=symbols('x,p,q,a,A')
>>> y=A*x*exp(a*x)
>>> ecua=y.diff(x,2)+p*y.diff(x)+q*y-exp(a*x)
>>> ecua=simplify(ecua/exp(a*x))
>>> ecua
A*a*(a*x + 2) + A*p*(a*x + 1) + A*q*x - 1
>>> ecua.subs(q,-a**2 - a*p).simplify()
2*A*a + A*p - 1

\end{lstlisting}

Luego, si $2a+p\neq 0$, $\boxed{y(x)=\frac{1}{2a+p}xe^{ax}}$ resuelve el problema.





\item \textbf{Caso $\b{r(x)=e^{a x}}$, $\b{a^2+pa+q= 0}$ y $ \b{2a+p=0}$.}
Si $2a+p=0$, como tambi�n $a^2+pa+q=0$, tenemos que $a$ es una ra�z doble de la ecuaci�n $\lambda^2+p\lambda+q=0$.
En este caso, proponemos como soluci�n $y(x)=Ax^2e^{ax}$.

\begin{lstlisting}
>>> x,p,q,a,A=symbols('x,p,q,a,A')
>>> y=A*x**2*exp(a*x)
>>> ecua=y.diff(x,2)+p*y.diff(x)+q*y-exp(a*x)
>>> ecua=simplify(ecua/exp(a*x))
>>> ecua
A*p*x*(a*x + 2) + A*q*x**2 + A*(a**2*x**2 + 4*a*x + 2) - 1
>>> ecua.subs([(q,-a**2 - a*p) , (p,-2*a)]).simplify()
2*A - 1
\end{lstlisting}

Hay que tomar $\boxed{y(x)=\frac{1}{2}x^2e^{ax}}$







\item \textbf{Caso $\b{r(x)=\sen bx}$.}
Proponemos
\[y(x)=A\cos x+ B\sen x,\]
como candidato a soluci�n.
\begin{lstlisting}
>>> x,p,q,a,b,A,B=symbols('x,p,q,a,b,A,B')
>>> y=A*cos(b*x)+B*sin(b*x)
>>> ecua=y.diff(x,2)+p*y.diff(x)+q*y-sin(b*x)
>>> ecua.simplify()
-b**2*(A*cos(b*x) + B*sin(b*x)) - b*p*(A*sin(b*x) -
B*cos(b*x)) + q*(A*cos(b*x) + B*sin(b*x)) - sin(b*x)
\end{lstlisting}


La expresi�n en el miembro de la izquierda es una combinaci�n lineal de las funciones $\cos bx$ y $\sen bx$. Como estas funciones son linealmente independientes
debemos tener que los coeficientes en la combinaci�n lineal deben ser cero

\begin{lstlisting}
>>> ecua.expand().coeff(sin(b*x))
-A*b*p - B*b**2 + B*q - 1
>>> ecua.expand().coeff(cos(b*x))
-A*b**2 + A*q + B*b*p
\end{lstlisting}

Obtenemos un sistema de ecuaciones
\begin{equation}\label{sist_lin_1}
\left\{\begin{array}{l l}
-Abp - (b^2 - q)B & = 1\\
Bbp - (b^2 - q)A &=0
\end{array}
\right.
\end{equation}
%
Para que el sistema tenga soluci�n la matriz de coeficientes debe ser no singular
\[
0\neq\det \begin{pmatrix}
-bp & -(b^2-q)\\
-(b^2-q) & bp
\end{pmatrix} = -(b^2p^2+(b^2-q)^2)
\]
Podemos suponer $b\neq 0$, de lo contrario la ecuaci�n hubiese sido homog�nea. entonces la condici�n de arriba ocurre si y s�lo si
$p\neq 0$ o $b^2\neq q$. En esa situaci�n encontraremos una soluci�n de la forma
\[
\boxed{y(x)=A\cos bx + B\sen bx},
\]
donde $A$ y $B$ resuelven \eqref{sist_lin_1}.

\label{eq:forz_res}
Cuando $p=0$ y $b^2= q$ el sistema \eqref{sist_lin_1} puede no tener soluci�n. Notar que en este caso la ecuaci�n queda
\[
y''+b^2y=\sen bx
\]
Es una ecuaci�n de un oscilador arm�nico no homog�nea. Hab�amos visto que justamente $r(x)=\sen bx$ es una soluci�n del problema homog�no. Nuevamente
estamos en una situaci�n de resonancia. Como en casos anteriores hay que proponer como soluci�n
\[y(x)=x\left(A\cos x+ B\sen x\right),\]




\item \textbf{Caso $\b{r(x)=\sen bx}$ con resonancia}

\begin{lstlisting}
>>> x,b,A,B=symbols('x,b,A,B')
>>> y=x*(A*cos(b*x)+B*sin(b*x))
>>> ecua=y.diff(x,2)+b**2*y-sin(b*x)
>>> eq1=ecua.expand().coeff(sin(b*x))
>>> eq2=ecua.expand().coeff(cos(b*x))
>>> H=solve([eq1,eq2],[A,B])
>>> H
{B: 0, A: -1/(2*b)}
>>> y.subs(H)
-x*cos(b*x)/(2*b)
\end{lstlisting}



Encontramos la soluci�n general
\[\boxed{y(x)=-\frac{x}{2b}\cos bx}.\]
El caso donde $r(x)=\cos bx$ se trata de manera completamente similar.



\item \textbf{Caso $\b{r(x)}$ polinomio}
 Hay que proponer como soluci�n un polinomio, en primera instancia, del mismo grado.

 Supongamos

 \boxedeq{\frac{d^2y}{dx^2}+p\frac{dy}{dx}+q(x)y=c_0+c_1x+\cdots+c_nx^n}{eq:no_hom_pol}

Se propone $y=a_0+a_1x+\cdots+a_nx^n$. Luego

\[\begin{array}{l}
2a_2+3\cdot2x+\cdots+n(n-1)x^{n-2}+\\
pa_1+p2a_2x+\cdots+pna_nx^{n-1}+\\
qa_0+qa_1x+\cdots+qa_nx^n=c_0+c_1x+\cdots+c_nx^n
\end{array}
.\]

Como las funciones $1,x,\ldots,x^n$ son linealmente independientes, los coeficientes en ambos lados de la igualdad deben ser iguales.

\[
\begin{split}
2a_2+pa_1+qa_0&=c_0\\
3\cdot 2+2pa_2+qa_1&=c_1\\
 &\hspace{2mm} \vdots\\
n(n-1)+p(n-1)a_{n-1}+qa_{n-2} &=c_{n-2}\\
pna_{n}+qa_{n-1} &=c_{n-1}\\
qa_n &=c_n\\
\end{split}
\]
Es �til escribir estas igualdades matricialmente.
\[\begin{pmatrix} q &\cdots  &\cdots&\cdots&\cdots\\
&q & \cdots & \cdots&\cdots \\
\vdots & &\ddots&& \vdots\\
\vdots&&&q&pn\\
\vdots&&&&q\\
\end{pmatrix}
\begin{pmatrix}
a_0\\
a_1\\
a_2\\
\vdots\\
a_{n-1}\\
a_n
\end{pmatrix}
=
\begin{pmatrix}
c_0\\
c_1\\
c_2\\
\vdots\\
c_{n-1}\\
c_n
\end{pmatrix}
\]
Es un sistema tri�ngular superior que se resuelve por sustituci�n ascendente. Esto siempre que $q\neq 0$. En caso contrario la matr�z es singular y es posible que el sistema no tenga soluci�n.

El caso $q=0$ es una forma de resonancia. Puede ser tratado como las anteriores resonancias, pero notando que la ecuaci�n se reduce a $y''+py'=r$ conviene
tomar $v=y'$ como
nueva variable dependiente y reducir la ecuaci�n a una de primer orden.

\end{enumerate}

Por �ltimo se�alemos que si deseamos resolver un problema de la forma
\[L[y]\equiv y''+py'+qy=r_1(x)+\cdots +r_n(x),\]
donde las funciones $r_i$ son de alguna de las formas descriptas en los casos previos,
entonces la linealidad de $L$ implica que, si $y_i$
resuelve $L[y_i]=r_i$, $y=y_1+\cdots +y_n$ resuelve la ecuaci�n deseada.


\subsection{M�todo de variaci�n de los par�metros}

Queremos resolver la ecuaci�n
\begin{equation}\label{eq:2orden_gen}
y''(x)+p(x)y'(x)+q(x)y(x)=r(x).
\end{equation}
Supongamos que contamos con un par de soluciones $y_1$, $y_2$ linealmente independientes de la ecuaci�n homog�nea asociada
\begin{equation}\label{eq:hom_asoc}
y''(x)+p(x)y'(x)+q(x)y(x)=0.
\end{equation}
El m�todo de \href{http://en.wikipedia.org/wiki/Variation_of_parameters}{variacion de los par�metros} consiste en proponer una soluci�n de la forma
\boxedeq{y(x)=c_1(x)y_1(x)+c_2(x)y_2(x).}{eq:var_param0}
Hay dos funciones incognitas $c_1$ y $c_2$, pero s�lo una ecuaci�n. Tendremos
por esto libertad de introducir otra condici�n que consideremos conveniente.
Tenemos
\[
y'=c_1'y_1+c_1y_1'+c_2'y_2+c_2y_2'.
\]
Pidamos que
\begin{equation}\label{eq:var_param1}
c_1'y_1+c_2'y_2=0.
\end{equation}
Supuesta esta igualad
\[ y'= c_1y_1'+c_2y_2'.\]
Derivando
\[ y''= c_1'y_1'+c_2'y_2'+c_1y_1''+c_2y_2''.\]
Entonces
\[
\begin{split}
r(x)&=y''+py'+qy\\
&=c_1'y_1'+c_2'y_2'+c_1y_1''+c_2y_2''+p(c_1y_1'+c_2y_2')+q(c_1y_1+c_2y_2)\\
&=c_1(y_1''+py_1'+qy_1)+c_2(y_2''+py_2'+qy_2)+c_1'y_1'+c_2y_2'\\
&=c_1'y_1'+c_2'y_2'
\end{split}
\]
Esta ecuaci�n junto a \eqref{eq:var_param1} nos dan el sistema
\boxedeq{
\left\{\begin{array}{c c}
c_1'y_1+c_2'y_2&=0\\
c_1'y_1'+c_2'y_2'&=r
\end{array}
\right.
}{eq:var_param_sis}
Las incognitas son $c_1'$ y $c_2'$. El determinante de la matriz de coeficientes es precisamente el Wronskiano $W$ de las soluciones $y_1$ e $y_2$, por la suposici�n
de independencia $W\neq 0$ y por lo tanto el sistema tiene soluci�n �nica.
Se tiene
\[c_1'=-\frac{\det\begin{pmatrix}
0 & y_2\\
r & y_2'
\end{pmatrix}
}{W}=-\frac{ry_2}{W}
\]
y
\[c_2'=-\frac{\det\begin{pmatrix}
y_1 & 0\\
y_1' & r
\end{pmatrix}
}{W}=\frac{ry_1}{W}
\]
En consecuencia
\boxedeq{c_1=-\int\frac{ry_2}{W}dx}{eq:c1}
y
\boxedeq{c_2=\int\frac{ry_1}{W}}{eq:c2}
Usando estas f�rmulas y \eqref{eq:var_param0} obtenemos una soluci�n particular del sistema. La soluci�n general es la suma de la particular m�s
una soluci�n general del homog�neo. Esta �ltima soluci�n general se escribe como una combinaci�n lineal gen�rica entre $y_1$ e $y_2$.

\begin{ejemplo} Resolver el siguiente pvi $y''+y=\csc x$,
$y\left(\tfrac{\pi}{2}\right)=0$ y $y'\left(\tfrac{\pi}{2}\right)=1$.
La ecuaci�n homog�nea asociada tiene el par $y_1(x)=\cos(x)$ e $y_1(x)=\sen(x)$
   de soluciones linealmente independientes.   El wronskiano es $W\equiv 1$ y
entonces una soluci�n paarticular es
\[\begin{split}
  y(x)&=-\int \frac{ry_2}{W}dx y_1 +\int \frac{ry_1}{W}dx y_2\\
       &=-\int dx \cos(x)+\int \frac{1}{\tan(x)}dx \sen(x)\\
       &=-x\cos(x) +\ln(\sen(x))\sen(x).
 \end{split}
\]

 La soluci�n general es
 \[
  y(x)=-x\cos(x) +\ln(\sen(x))\sen(x)+c_1\cos(x)+c_2\sen(x).
  \]
Las condiciones $y\left(\tfrac{\pi}{2}\right)=0$ y
$y'\left(\tfrac{\pi}{2}\right)=1$ nos conducen a $c_2=0$ y $c_1=\frac{\pi}{2}$.



\end{ejemplo}
\section{Conclusiones}

\begin{enumerate}
\item Si podemos encontrar dos soluciones linealmente independientes de una ecuaci�n lineal homog�nea de segundo orden, tenemos la soluci�n general a traves de
combinaciones lineales.
\item Si tenemos una soluci�n no trivial de una ecuaci�n lineal homog�nea de segundo orden podemos hallar otra por el m�todo de reducci�n de orden.
\item Podemos resolver completamente una ecuaci�n lineal homog�nea de segundo orden con coeficientes constantes.
\item Podemos resolver algunos problemas no homog�neos por el m�todo de coeficientes indeterminados.
\item Si conocemos las soluciones del problema homog�neo podemos resolver, en teor�a, el no homog�neno para cualquier $r(x)$ por el m�todo de variaci�n
de los par�metros
\end{enumerate}

\section{Aplicaciones}
\subsection{Vibraciones mec�nicas }
\begin{problema} Estudiar el movimiento de un resorte (c�mo el de la unidad
anterior) pero suponer que adem�s de actuar sobre la masa la fuerza el�stica del
resorte,
tenemos una fuerza de fricci�n debida a la resistencia del medio. Por la acci�n de esta fuerza, se dice que es un sistema resorte-masa amortiguado.
Adem�s suponemos que hay otra fuerza $F$ externa y que s�lo depende de $t$. Por ejemplo si el resorte se colocase verticalmente y se dejase suspendida
la masa, $F$ ser�a la fuerza de gravedad. Si la masa estuviese hecha de metal, $F$ podr�a ser una fuerza provista por un im�n. Por la acci�n de esta fuerza el sistema se
dice forzado. Por consiguiente el sistema completo, con la acci�n de las tres fuerzas, se denomina un sistema resorte-masa, amortiguado y forzado.
\end{problema}

La fuerza el�stica del resorte se modeliza con la Ley de Hooke.
Para la amortiguaci�n, supongamos que el m�dulo de la fuerza es proporcional a
la velocidad de la masa. La constante de proporcionalidad $c$ se llama
coeficiente de
\href{http://es.wikipedia.org/wiki/Viscosidad}{viscosidad}. La direcci�n y sentido de la fuerza amortiguadora es siempre contraria
al movimiento. Por el principio de conservaci�n de la energ�a, vemos que la fuerza de amortiguaci�n siempre realiza un trabajo $W$ negativo, por consiguiente
hace perder energ�a cin�tica. De la fuerza externa $F$ no sabemos nada en principio. Por todo lo expuesto, si ponemos un sistema
de coordenadas con origen en la posici�n de equilibrio del sistema masa-resorte y si $x(t)$ es la posici�n de la masa en el momento $t$, la ecuaci�n que gobierna
el sistema masa-resorte con amortiguaci�n y forzamiento es
\boxedeq{mx''(t)\underbrace{=}_\text{2� Ley Newton}\underbrace{-kx(t)}_
\text{Hooke}\underbrace{-cx'(t)}_\text{Amortiguaci�n}+\underbrace{F(t)}_\text{Fuerza externa}}{eq:res_amor_for}


\subsubsection{Vibraciones amortiguadas no forzadas ($\b{c>0}$, $\b{F=0}$) }

Escribamos la ecuaci�n \eqref{eq:res_amor_for} de la siguiente froma
\begin{equation}\label{eq:res_amor}
\boxed{x''(t)+2\mu x'(t)+\omega^2x=0}\quad
\mu:=\frac{c}{2m},\omega:=\sqrt{\frac{k}{m}}.
\end{equation}
Las ra�ces de la ecuaci�n caracter�stica   son
\[
\boxed{\lambda_{1,2}=-\mu\pm\sqrt{\Delta},\quad \Delta:=\mu^2-\omega^2}
\]
\textbf{Caso $\Delta>0$.} Aqu� la viscocidad es ``grande'' relativa ala rigidez
$k$. Se dice que el sistema est� sobreamortiguado.
En este caso tenemos dos soluciones
linealmente independientes del problema homog�neo y la soluci�n general de
este es de la forma
\[x(t)=c_1e^{\lambda_1t}+c_2e^{\lambda_2t}\]
Notar que $\lambda_1,\lambda_2<0$. Supongamos que el sistema masa-resorte parte
del resposo $x'(0)=0$ y de una posici�n indeterminada $x_0$. Resolvamos este pvi

\begin{lstlisting}
>>> from sympy import *
>>> init_printing()
>>> lambda1,lambda2,t,x0,c1,c2=symbols('lambda1,lambda2,t,x0,c1,c2')
>>> x=c1*exp(lambda1*t)+c2*exp(lambda2*t)
>>> C=solve([x.subs(t,0)-x0,x.diff(t).subs(t,0)], [c1,c2])
>>> C
\end{lstlisting}

\[\begin{Bmatrix}c_{1} : - \frac{\lambda_{2} x_{0}}{\lambda_{1} -
\lambda_{2}}, & c_{2} : \frac{\lambda_{1} x_{0}}{\lambda_{1} -
\lambda_{2}}\end{Bmatrix}\]

\begin{lstlisting}
>>> x=x.subs(C[0])
\end{lstlisting}
\boxedeq{
x(t)= x_0\left\{ \frac{\lambda_{1}
e^{\lambda_{2} t}}{\lambda_{1} - \lambda_{2}}-\frac{\lambda_{2} e^{\lambda_{1} t}}{\lambda_{1} - \lambda_{2}}\right\}
}{eq:sol_sub_amor}


\begin{lstlisting}
>>> x=x.subs({lambda1:-1,lambda2:-2,x0:1})
>>> plot(x,(t,0,10))
\end{lstlisting}


\begin{figure}[h]
\begin{center}
\includegraphics[scale=.2]{imagenes/sobreamortiguado.png}
\caption{ Vibraciones amortiguadas no forzadas ($c>0$, $F=0$) }\label{fig:VibrAmorNoFor}
\end{center}
\end{figure}



\begin{figure}[h]
\begin{center}
\animategraphics[controls, scale=.4]{15}{res_sobre/res_sobre-}{0}{60}
\vspace{.5cm}
\caption{Masa-resorte sobreamortiguado}\label{ani:VibrAmorNoFor}
\end{center}
\end{figure}

Como se observa en la gr�fica \ref{fig:VibrAmorNoFor} y la animaci�n \ref{ani:VibrAmorNoFor}
la masa ejecuta una oscilaci�n, lo cual le demanda un tiempo infinito. Podr�a
haber pasado por la posici�n de equilibrio s�lo en el pasado,
puesto que $x(t)=0$ cuando
\[\boxed{t=\frac{1}{\lambda_1-\lambda_2}\ln\frac{\lambda_1}{\lambda_2}<0}\]


\noindent\textbf{Caso $\Delta=0$.} En esta situaci�n se dice que hay amortiguaci�n cr�tica. Las ra�ces son iguales $\lambda_1=\lambda_2=-\mu$. Sabemos que
\boxedeq{x_1(t)=c_1e^{-\mu t}+c_2te^{-\mu t}=e^{-\mu t}\{c_1+c_2t\}}{eq:sol_gen_crit}
Nuevamente la soluci�n puede pasar a lo sumo una vez por la posici�n de equilibrio, siempre y cuando
$C_2\neq 0$. El compportamiento cualitativo de la soluci�n es muy parecido al caso anterior.


\noindent\textbf{Caso $\Delta<0$, caso subamortiguado.} $\lambda_{1,2}=-\mu\pm\nu i$ con $\nu=\sqrt{|\Delta|}=\sqrt{|\omega^2-\mu^2|}$.
La soluci�n general viene dada por
\boxedeq{x(t)=e^{-\mu t}\left\{ c_1\cos \nu t+c_2\sen \nu t \right\}}{eq:sol_gen_sub}
Est� funci�n tiene por gr�fica una onda sinusoidal modulada por una funci�n exponencial
decreciente.

\begin{lstlisting}
>>> c1,c2,mu,nu,x0,t=symbols('c1,c2,mu,nu,x0,t')
>>> x=exp(-mu*t)*(c1*cos(nu*t)+c2*sin(nu*t))
>>> C=solve([x.subs(t,0)-x0,x.diff(t).subs(t,0)],[c1,c2])
>>> x=x.subs(C).subs({mu:.1,nu:4,x0:1})
>>> plot(x,(t,0,100))
\end{lstlisting}
\begin{center}
\includegraphics[scale=.15]{imagenes/subamortiguado.png}
\end{center}

{ Vibraciones amortiguadas no forzadas ($c>0$, $F=0$) }
\begin{figure}[h]
\begin{center}
\animategraphics[controls, scale=.4]{15}{res_sub/res_sub-}{0}{60}
%\includegraphics[scale=.4]{res_sub/res_sub-0.jpg}
\vspace{.5cm}
\caption{Masa-resorte subamortiguado}
\end{center}
\end{figure}

Se suele escribir la ecuaci�n \eqref{eq:sol_gen_sub} de otra forma. Expresemos el vector $(c_1,c_2)$ en
coordenadas polares.
\[
c_1=\rho\cos\alpha,\quad c_2=\rho\sen\alpha.
\]
Entonces usando las
\href{https://es.wikipedia.org/wiki/Identidades_trigonom%C3%A9tricas}{f�rmulas de adici�n}
de las funciones trigonom�tricas
\[
x(t)=e^{-\mu t}\left\{ c_1\cos \nu t+c_2\sen \nu t \right\}=\boxed{\rho e^{-\mu t}\cos(\nu t-\alpha)}.
\]
Llamaremos este r�gimen \emph{movimiento cuasi-oscilatorio}. Se ejecutan vibraciones, que se disipan con el tiempo,
de \href{http://es.wikipedia.org/wiki/Frecuencia}{frecuencia}
\[
f=\frac{1}{\text{per�odo}}=\frac{\nu}{2\pi},\quad \nu=\sqrt{\omega^2-\mu^2}=\sqrt{\left(\frac{k}{m}\right)^2-
\left(\frac{c}{2m}\right)^2}.
\]
En lugar de la frecuencia se suele considerar la
\href{http://luz.izt.uam.mx/mediawiki/index.php/Frecuencia_angular}{frecuencia angular} que se define como
$2\pi f$. La ventaja de esta definici�n es que la frecuencia �ngular de la funci�n de arriba es
$\nu$.


\noindent\textbf{Ejercicio:} Demostrar que en cualquiera de las situaciones descriptas, $x(t)\to 0$ y $x'(t)\to 0$, cuando $t\to\infty$.
Es decir, el movimiento se va deteniendo.


\subsubsection{ Vibraciones no amortiguadas y forzadas ($\b{c=0}$, $\b{F\neq 0}$) }
Vamos a considerar una fuerza externa oscilatoria de frecuencia angular $\omega_0$ y amplitud $F_0$. Tenemos que resolver
\boxedeq{x''(t)+\omega^2 x(t)=F_0\cos(\omega_0 t).}{eq:ecua_2orden_nohom}
Usaremos el m�todo de coeficientes indeterminados.  Recordar que si $\omega=\omega_0$ estamos en
resonancia. Tendremos que considerar ese caso por separado. Supongamos pues $\omega\neq\omega_0$.
Tenemos que  reemplazar en la ecuaci�n
\[x(t)=A\cos(\omega_0 t)+B\sen(\omega_0 t),\]
y hallar $A$ y $B$ que logren que $x(t)$ sea soluci�n. Usamos SymPy

\begin{lstlisting}
>>> from sympy import *
>>> init_printing()
>>> t,omega,omega0,F0,A,B=symbols('t,omega,omega0,F0,A,B')
>>> x=A*cos(omega0*t)+B*sin(omega0*t)
>>> eq=x.diff(t,2)+omega**2*x-F0*cos(omega0*t)
>>> eqL1=eq.factor().coeff(sin(omega0*t))
>>> eqL2=eq.factor().coeff(cos(omega0*t))
>>> Incog=[A,B]
>>> Ecuas=[eqL1,eqL2]
>>> Matrix([[ecu.coeff(inco) for inco in Incog] for ecu in Ecuas])

\end{lstlisting}
La matr�z de coeficientes del sistema de ecuaciones para $A$ y $B$ es
\[\left[\begin{matrix}0 & \omega^{2} - \omega_{0}^{2}\\\omega^{2} - \omega_{0}^{2} & 0\end{matrix}\right]\]
claramente es una matr�z no singular, cuando $\omega\neq \omega_0$, y por consiguiente  en no resonancia
tiene una soluci�n
\begin{lstlisting}
>>> SolAB=solve([eqL1,eqL2],[A,B])
>>> x=x.subs(SolAB)
>>> x
\end{lstlisting}

\[x(t)=\frac{F_{0} \cos{\left (\omega_{0} t \right )}}{\omega^{2} - \omega_{0}^{2}}.
\]
La soluci�n general del problema es la soluci�n particular que acabamos de obtener m�s una soluci�n
general del homog�neo que sabemos es una combinaci�n lineal generica entre $\cos \omega t$ y $\sin \omega t$.

\boxedeq{x(t)=\frac{F_{0} \cos\left(\omega_{0} t\right)}{\omega^{2} - \omega_{0}^{2}}+
c_1\cos(\omega t)+c_2\sin(\omega t).}{eq:sol_gen_noamort_forz}
Como ya hemos visto, considerando las coordenadas polares $\rho$ y $\alpha$ de $c_1,c_2$
podemos reescribir la soluci�n
\[x(t)=\frac{F_{0} \cos\left(\omega_{0} t\right)}{\omega^{2} - \omega_{0}^{2}}+
\rho\cos(\omega t-\alpha)
\]
Vemos que el movimiento es la superposici�n de dos movimientos oscilatorios de frecuencias $\omega$, que se
denomina la \emph{frecuencia natural} del resorte, y $\omega_0$ que se denomina \emph{frecuencia impresa}.




Resolvamos el pvi
\[
\left\{\begin{array}{l}
x''(t)+\omega^2x(t)=F_0\cos(\omega_0 t),\\
x'(0)=x(0)=0\\
\end{array}
\right.
\]

\begin{lstlisting}
>>> from sympy import *
>>> init_printing()
>>> var('t,c1,c2,omega,omega0,F0')
>>> x=F0/(omega**2-omega0**2)*cos(omega0*t)+c1*cos(omega*t)+c2*sin(omega*t)
>>> C=solve([x.subs(t,0),x.diff(t).subs(t,0)],[c1,c2])
>>> x=x.subs(C)
>>> x
\end{lstlisting}

\[x(t)=-\frac{F_{0} \cos\left(\omega t\right)}{\omega^{2} - \omega_{0}^{2}} + \frac{F_{0} \cos\left(\omega_{0} t\right)}{\omega^{2} - \omega_{0}^{2}}
\]

Ahora usemos la identidad $\cos(a-b)-\cos(a+b)=2\sen a\sen b$, con $a=\frac12 (\omega+\omega_0)$ y
$b=\frac12 (\omega-\omega_0)$. Deducimos
\boxedeq{%
x(t)=\frac{2F_{0} }{\omega^{2} - \omega_{0}^{2}}\sen\left( \frac{\omega-\omega_0}{2}t \right)\sen
\left(\frac{\omega+\omega_0}{2}t\right).
}{eq:sol_gen_pulsos}
Esta expresi�n la podemos ver como una onda de frecuencia grande $\omega+\omega_0$ modulada por una de frecuencia
chica $\omega-\omega_0$.

\begin{lstlisting}
x=x.subs({F0:1,omega:1,omega0:.9})
plot(x,(t,0,200))
\end{lstlisting}

\begin{figure}[h]
\begin{center}
\includegraphics[scale=.15]{imagenes/batido.jpg}
\end{center}
\caption{Fen�meno de batido}\label{fig:batido}
\end{figure}


Calculemos el l�mite $\lim_{\omega_0\to\omega}x(t)$,

\begin{lstlisting}
>>> limit(x0,omega0,omega)
\end{lstlisting}

\[x(t)=\frac{F_{0} t \sin\left(\omega t\right)}{2 \, \omega}
\]
El caso $\omega=\omega_0$ es el caso con resonancia, que debemos resolver, como fue indicado, proponiendo como soluci�n $y(x)=x\left(A\cos x+ B\sen x\right)$. El siguiente c�digo \texttt{SymPy} muestra que la soluci�n es la misma funci�n
que la obtenida por el proceso de l�mite de los casos sin resonancia.

\begin{lstlisting}
>>> var('t,omega,F0,A,B')
>>> x=t*(A*cos(omega*t)+B*sin(omega*t))
>>> eq=x.diff(t,2)+omega**2*x-F0*cos(omega*t)
>>> eqL1=eq.factor().coeff(sin(omega*t))
>>> eqL2=eq.factor().coeff(cos(omega*t))
>>> SolAB=solve([eqL1,eqL2],[A,B])
>>> x=x.subs(SolAB)
\end{lstlisting}



Se producen ``vibraciones'' no acotadas.
\begin{lstlisting}
>>> plot(x.subs({omega:1,F0:1}),(t,0,100))
\end{lstlisting}

\begin{figure}[h]
\begin{center}
\includegraphics[scale=0.3]{imagenes/osc_arm_forz_res.png}
\end{center}
\caption{Resonancia}
\end{figure}
En la wiki \href{http://wiki.sagemath.org/interact/misc\#Hearing_a_trigonometric_identity}{Hearing a trigonometric identity}
se puede escuchar ondas sonoras con los fen�menos de resonancia y batido.

\subsubsection{Vibraciones amortiguadas y forzadas ($\b{c>0}$, $\b{F\neq 0}$) }
Vamos a considerar una fuerza externa oscilatoria de frecuencia $\omega_0$ y amplitud $F_0$. Tenemos que resolver
\boxedeq{x''(t)+2\mu x'(t)+\omega^2 x(t)=F_0\cos(\omega_0 t).}{eq:ecua_2orden_amort_nohom}
Proponemos por soluci�n $x(t)=A\cos(\omega_0 t)+B\sen(\omega_0 t)$.

\begin{lstlisting}
>>> var('t,mu,omega,omega0,F0,A,B')
>>> x=A*cos(omega0*t)+B*sin(omega0*t)
>>> eq=x.diff(t,2)+2*mu*x.diff(t)+omega**2*x-F0*cos(omega0*t)
>>> eqL1=eq.factor().coeff(sin(omega0*t))
>>> eqL2=eq.factor().coeff(cos(omega0*t))
>>> Incog=[A,B]
>>> Ecuas=[eqL1,eqL2]
>>> M=Matrix([[ecu.coeff(inco) for inco in Incog] for ecu in Ecuas])
>>> M.det()
\end{lstlisting}

La matr�z $M$ del sistema lineal que satisfacen las incognitas $A$ y $B$ tiene determninante
\[\det(M)=- 4 \mu^{2} \omega_{0}^{2} - \omega^{4} + 2 \omega^{2} \omega_{0}^{2} - \omega_{0}^{4}.
\]
No queda claro si, eventualmente, puede ser singular. Calculando las posibles soluciones para $\omega$ de $\det(M)=0$
\begin{lstlisting}
>>> solve(M.det(),omega)
\end{lstlisting}
\[\left [ - \sqrt{- 2 i \mu \omega_{0} + \omega_{0}^{2}}, \quad \sqrt{- 2 i \mu \omega_{0} + \omega_{0}^{2}}, \quad - \sqrt{2 i \mu \omega_{0} + \omega_{0}^{2}}, \quad \sqrt{2 i \mu \omega_{0} + \omega_{0}^{2}}\right ]
\]
son todas complejas no reales, por consiguiente la matr�z es simpre no singular.
\begin{lstlisting}
>>> SolAB=solve([eqL1,eqL2],[A,B])
>>> x.subs(SolAB)
\end{lstlisting}

\boxedeq{x(t)= \frac{2 F_{0} \mu \omega_{0} \sin{\left (\omega_{0} t \right )}}{4 \mu^{2} \omega_{0}^{2} + \left(\omega^{2} - \omega_{0}^{2}\right)^{2}} + \frac{F_{0} \left(\omega^{2} - \omega_{0}^{2}\right) \cos{\left (\omega_{0} t \right )}}{4 \mu^{2} \omega_{0}^{2} + \left(\omega^{2} - \omega_{0}^{2}\right)^{2}}
}{eq:SolGennoHom}
Es un movimiento oscilatorio de frecuencia angular $\omega_0$. Podemos escribir $x(t)=\rho\cos(\omega_0 t-\alpha)$,
donde $(\rho,\alpha)$ son las coordenadas polares de $(A,B)$. En particular la amplitud de la oscilaci�n viene dada por  $\rho=\sqrt{A^2+B^2}$.
Recurrimos nuevamente a SymPy para calcular $\rho$.

\begin{lstlisting}
>>> rho=sqrt(A**2+B**2).subs(SolAB).simplify()
>>> rho
\end{lstlisting}
\[
\rho(\omega_0)=
\frac{F_{0}}{\sqrt{(\omega^2-\omega_0^2)^2+4\mu^2\omega_0^2}}
\]
%
% \[\alpha=\atan2\left( {{\omega^{2} -
% \omega_{0}^{2}} } , {2 \, \mu \omega_{0} } \right)
% \]
%En una consola de sage entrar \texttt{atan2?} para averiguar que funci�n es $\atan2$.
Grafiquemos la funci�n $\rho(\omega_0)$ para $\omega=5$ y $\mu=0.1$.
\begin{lstlisting}
>>> plot(rho.subs({F0:1, mu:.1,omega:5}),(omega0,0,10))
\end{lstlisting}
\begin{figure}[h]
\begin{center}
\includegraphics[scale=.3]{imagenes/rho_graf.png}
\end{center}
\caption{Gr�fico amplitud vs. frecuencia, resorte amortiguado y forzado}
\end{figure}
La funci�n tiene un notorio m�ximo cerca de $\omega_0=5$. Seguramente es debido a la aparici�n de resonancias. Hallemos el punto de m�ximo exacto, primero encontremos puntos cr�ticos.
\begin{lstlisting}
>>> sol=solve(rho.diff(omega0),omega0)
>>>sol
\end{lstlisting}
\[\left [ 0, \quad - \sqrt{- 2 \mu^{2} + \omega^{2}}, \quad \sqrt{- 2 \mu^{2} + \omega^{2}}, \quad - \sqrt{\tilde{\infty} \sqrt{F_{0}^{2}} - 2 \mu^{2} + \omega^{2}}, \quad \sqrt{\tilde{\infty} \sqrt{F_{0}^{2}} - 2 \mu^{2} + \omega^{2}}\right ]
\]
El �nico l�cito, si $2\mu^2<\omega$,  es $\hat{\omega}_0=\sqrt{- 2 \mu^{2} + \omega^{2}}$.  Podemos constatar el caracter,  si es m�ximo/m�nimo o ninguna de estas opciones. Calculando la derivada segunda
\begin{lstlisting}
>>> rho.diff(omega0,2).subs(omega0,sol[2])
\end{lstlisting}
\[
\frac{2 \sqrt{\frac{F_{0}^{2}}{4 \mu^{4} + 4 \mu^{2} \left(- 2 \mu^{2} + \omega^{2}\right)}} \left(4 \mu^{2} - 2 \omega^{2}\right)}{4 \mu^{4} + 4 \mu^{2} \left(- 2 \mu^{2} + \omega^{2}\right)}
\]
vemos que es claramente negativa. Luego tenemos un m�ximo local en $\hat{\omega}_0$. Comparando el valor de $\rho(\hat{\omega}_0)$ con los de $\rho(0)$ y $\rho(+\infty)$ determinamos si es m�ximo global. Un c�lculo, que podemos hacer con SymPy, nos muestra que
\[\lim_{\omega_0\to\infty}\rho(\omega_0)=0<\frac{F_0}{\omega^2}=\rho(0)< \frac12\frac{F_0}{\mu\sqrt{\omega^2-\mu^2}}=\rho(\hat{\omega}_0).\]
En consecuencia $\hat{\omega}_0$ es m�ximo absoluto.



En el ejemplo que graficamos
el m�ximo ocurre en
\begin{lstlisting}
>>> sol[2].subs({mu:.1,omega:5})
4.99799959983992
\end{lstlisting}

Vale decir, un oscilador arm�nico en reposo es m�s sensible a excitaciones en ciertas frecuencias, aproximadamente igual a la frecuencia
natural del resorte cuando el coeficiente de viscocidad $c=2m\mu$ es chico. Esto es utilizado para dise�ar dispositivos que captan ondas s�smicas.


Hasta aqu� hemos encontrado una soluci�n particular del sistema no homog�neo. Para encontrar una soluci�n general deber�amos adicionar a la particular que disponemos
una soluci�n general $x_g(t)$ de la ecuaci�n homog�nea. La forma de esta soluci�n general es de alguno de los tipos \ref{eq:sol_gen_sub},
\ref{eq:sol_gen_crit} o \ref{eq:sol_sub_amor}. Sin embargo no nos importa ahora la f�rmula expl�cita de estas soluciones, sino que nos interesa resaltar que
tr�tese del tipo que se trate, se satisface que $\lim_{t\to\infty}x_g(t)=0$. Por este motivo, vamos a decir que esta parte de la soluci�n es
\emph{transitoria}. En cambio la soluci�n que prevalece en el tiempo dada por \eqref{eq:SolGennoHom} la denominaremos soluci�n \emph{estacionaria}.

\subsection{Un poco de mec�nica celeste}

Vamos a considerar ahora el problema del movimiento de un planeta, digamos la Tierra, de masa $m_{\earth}$ alrededor del sol, de masa $m_{\sun}$. Como
$m_{\sun}\gg m_{\earth}$ vamos a ignorar la fuerza que act�a sobre el Sol debido a la atracci�n gravitatoria de la Tierra. Esta suposici�n, aunque falsa, la hacemos
por simplicidad. No obstante, con s�lo un poco de trabajo, el caso m�s general se reduce al tratado aqu�. Ver el \href{https://drive.google.com/file/d/0B80iJ0HgObRRbUtUQ2hFQ0FlTG8/view}{trabajo final} de la Lic. Matem�tica de Leopoldo Buri,
para una deducci�n m�s cuidadosa. Vamos a suponer adem�s que el movimiento del planeta se retringe a un plano. Esta afirmaci�n es cierta y aunque su demostraci�n
es sencilla no la desarrollaremos aqu�.
Supongamos un sistema de coordenadas cartesianas sobre el plano en que se realiza el movimiento orbital del planeta. Asumimos el Sol en el origen de coordenadas y en
reposo. Como no act�a fuerza sobre �l, permanecer� en esa situaci�n. Vamos a suponer que la posici�n de la Tierra es $\v{r}$.

Los dos ingredientes b�sicos para derivar la leyes de movimiento del planeta son la
\href{http://es.wikipedia.org/wiki/Leyes_de_Newton\#Segunda_ley_de_Newton_o_ley_de_fuerza}{Segunda Ley de Newton} y la
\href{http://es.wikipedia.org/wiki/Ley_de_gravitaci�n_universal}{Ley Gravitaci�n Universal}. Ya hemos considerado ambas con anterioridad.
Seg�n la Ley de Gravitaci�n Universal, la magnitud de la fuerza de gravedad es proporcional a $\frac{m_{\earth}m_{\sun}}{d^2}$, donde $d$ es la distancia tierra-sol.
A la constante de proporcionalidad la llamaremos, como es costumbre, $G$. La direcci�n de la fuerza gravitatoria es la de la recta que une los dos astros y
el sentido es tal que la fuerza es atractiva entre los cuerpos. Vale decir, la direcci�n y sentido de la
fuerza de gravedad vienen dados por el versor $-\v{r}/r$, donde $r=|\v{r}|$. Luego se debe satisfacer que
\[Gm_{\earth}\frac{d^2\v{r}}{dt^2}=-\frac{Gm_{\earth}m_{\sun}}{r^2}\frac{\v{r}}{r}=-Gm_{\earth}m_{\sun}\frac{\v{r}}{r^3}. \]


Es decir
\boxedeq{\frac{d^2\v{r}}{dt^2}=-\mu\frac{\v{r}}{r^3}\quad\text{donde } \mu:=Gm_{\sun}}{eq:grav}
Esta ecuaci�n se conoce como la \href{http://es.wikipedia.org/wiki/Problema_de_los_dos_cuerpos}{ecuaci�n de los dos cuerpos}.
Dado que esta ecuaci�n entra�a, a su vez, tres ecuaciones escalares, una por cada
componente de $\v{r}$, se nos presenta aqu� un \emph{Sistema de Ecuaciones Diferenciales}. No sabemos resolver sistemas de ecuaciones. No obstante vamos
a ver como podemos reducir la ecuaci�n anterior, mediante ingeniosos cambios de
variables, a ecuaciones diferenciales que sabemos resolver.


Vamos a usar coordenadas polares $(r,\theta)$ y los versores $\v{u}_r:=(\cos\theta,\sen\theta)$ y $\v{u}_{\theta}:=(-\sen\theta, \cos\theta)$. Notar que
$\v{u}_r \perp \v{u}_{\theta}$ y por consiguiente $\mathcal{B}:=\{\v{u}_r , \v{u}_{\theta}\}$ forma una base del espacio euclideano 2-dimensional. Usaremos este hecho
para representar distintos vectores como combinaci�n lineal de vectores de la base. Los c�lculos, como es ya habitual, se los dejaremos a SymPy,


Primero declaramos las variables y asignamos los vectores $\v{u}_r$, $\v{u}_{\theta}$ y el vector
$\v{r}$ al que llamamos \texttt{pos}.

\begin{lstlisting}
>>> from sympy import *
>>> init_printing()
>>> var('t,mu')
>>> x,y,r,theta=symbols('x,y,r,theta',cls=Function)
>>> u_r=Matrix([cos(theta(t)),sin(theta(t))])
>>> u_theta=Matrix([-sin(theta(t)),cos(theta(t))])
>>> pos=r(t)*u_r
\end{lstlisting}



Como vamos a necesitar representar vectores en la base $\mathcal{B}=\{\v{u}_r , \v{u}_{\theta}\}$, constru�mos una matriz
con los vectores de la base en las columnas.
\begin{lstlisting}
>>> M=u_r.row_join(u_theta)
\end{lstlisting}
Concretamente queremos representar el vector aceleraci�n $\v{a}:=\tfrac{d^2\v{r}}{dt^2}$ en la base $\mathcal{B}$, para ello debemos resolver $MX=\v{a}$, donde $X$ y
$\v{a}$ los asumimos vectores columna. Con SymPy lo hacemos en un periquete

\begin{codigo}[\texttt{linsolve}]
Resuelve un sistema $Ax=b$

\textbf{Sintaxis} (\href{http://docs.sympy.org/dev/modules/solvers/solveset.html#sympy.solvers.solveset.linsolve}{documentaci�n \texttt{SymPy}})

\texttt{linsolve((A,b), s�mbolos)}

\texttt{(A,b):} tuple con la matr�z y el t�rmino independientre del sistema.

\texttt{s�mbolos:} Muchas veces la soluci�n no es �nica, en ese caso trata de representar la soluci�n param�tricamente con los s�mbolos introducidos por \texttt{s�mbolos}.

\texttt{Retorna: } Un conjunto finito (tipo de datos de SymPy).

\end{codigo}



\begin{lstlisting}
>>> alpha=symbols('alpha')
>>> a=linsolve((M,pos.diff(t,2)),alpha )
>>> a=list(a) #convertimos de finite set a list
>>> a=a[0] #es una lista de tuples, sacamos el tuple
>>> a
\end{lstlisting}

\[
 \left\{\left ( - r{\left (t \right )} \frac{d}{d t} \theta{\left (t \right )}^{2} + \frac{d^{2}}{d t^{2}}  r{\left (t \right )}, \quad r{\left (t \right )} \frac{d^{2}}{d t^{2}}  \theta{\left (t \right )} + 2 \frac{d}{d t} r{\left (t \right )} \frac{d}{d t} \theta{\left (t \right )}\right )\right\}
\]
Obtenemos asi las dos componetes de $\v{a}$.
\[
\v{a}=\left(\begin{array}{c}
\ddot{r}-r\dot{\theta}^2 \\
r\ddot{\theta}+2\dot{r}\dot{\theta}
\end{array}
\right)=
\left(\begin{array}{c}
\ddot{r}-r\dot{\theta}^2 \\
\frac{1}{r}\frac{d}{dt} \left( r^2\dot{\theta} \right)
\end{array}
\right)
\]


El vector aceleraci�n debe ser igual a la fuerza por unidad de masa $-\mu \v{r}/r^3$. Notemos que esta fuerza es
\href{http://es.wikipedia.org/wiki/Campo_central}{central}, es decir tiene componente nula
respecto al vector $\v{u}_{\theta}$. Por consiguiente se debe satisfacer que
\[\frac{1}{r}\frac{d}{dt} \left( r^2\dot{\theta} \right)=0\Longleftrightarrow \exists h\in\mathbb{R}: \boxed{ r^2\dot{\theta}=h}. \]




\begin{wrapfigure}[12]{l}{5cm}
\includegraphics[scale=.3]{imagenes/2leykepler.jpeg}\\
\end{wrapfigure}
Hemos derivado la \href{http://es.wikipedia.org/wiki/Leyes_de_Kepler}{Segunda Ley de Kepler}: El radio vector barre �reas iguales en tiempos iguales.



En la direcci�n radial $\v{u}_r$ la componente de la fuerza es $-\mu/r^2$. Es decir se satisface la ecuaci�n
\[
\ddot{r}-r\dot{\theta}^2=-\frac{\mu}{r^2}
\]
Notar que esta ecuaci�n entra�a dos incognitas $r$ y $\theta$, pero $\dot{\theta}$ puede ser remplazado por $h/r^2$ por la segunda Ley de Kepler.
Declaremos la variable $h$ que juega un rol importante y reemplacemos $\dot{\theta}$ en la ecuaci�n

\begin{lstlisting}
>>> var('h')
>>> ed=(a[0]).subs((theta(t)).diff(t),h/r(t)**2)
>>> ed+=mu/r(t)**2
\end{lstlisting}
Resulta
\[- \frac{h^{2}}{r^{3}{\left (t \right )}} + \frac{\mu}{r^{2}{\left (t \right )}} + \frac{d^{2}}{d t^{2}}  r{\left (t \right )}=0.
\]
Conseguimos una ecuaci�n no lineal de segundo orden para $r$. De los m�todos que hemos visto, ninguno se aplica a esta ecuaci�n.
El truco m�gico consiste en considerar la nueva variable dependiente $z=1/r$ y la nueva variable independiente
$\theta$.

\begin{lstlisting}
>>> z=Function('z')(theta(t))
>>> r=1/z
>>> ed2=r.diff(t,2)+mu/r**2-h**2/r**3
>>> ed2
\end{lstlisting}
Se obtiene
\[
\begin{split}
 0&=- h^{2} z^{3}{\left (\theta{\left (t \right )} \right )} + \mu z^{2}{\left (\theta{\left (t \right )} \right )} + \frac{1}{z^{2}{\left (\theta{\left (t \right )} \right )}} \left(- \frac{d}{d t} \theta{\left (t \right )}^{2} \frac{d^{2}}{d \theta{\left (t \right )}^{2}}  z{\left (\theta{\left (t \right )} \right )}\right.\\
 &\quad\left.- \frac{d^{2}}{d t^{2}}  \theta{\left (t \right )} \left. \frac{d}{d \xi_{1}} z{\left (\xi_{1} \right )} \right|_{\substack{ \xi_{1}=\theta{\left (t \right )} }}+ \frac{2 \frac{d}{d t} \theta{\left (t \right )}^{2}}{z{\left (\theta{\left (t \right )} \right )}} \left. \frac{d}{d \xi_{1}} z{\left (\xi_{1} \right )} \right|_{\substack{ \xi_{1}=\theta{\left (t \right )} }}^{2}\right)
\end{split}
\]
En la complicada ecuaci�n resultante nuevamente aparece $\theta'$ y adem�s ahora aparece $\theta''$. Tenemos que reemplazar
$\theta'$ por $hz^2$ y $\theta'$ por $\tfrac{d}{dt}hz^2$.
\begin{lstlisting}
>>> theta2diff=(h*z**2).diff(t).subs(theta(t).diff(t),h*z**2)
>>> ed3=ed2.subs({theta(t).diff(t):h*z**2,theta(t).diff(t,2):theta2diff})
>>> ed4=(ed3/z**2/h**2).expand()
>>> ed4
\end{lstlisting}
Resulta
\[- z{\left (\theta{\left (t \right )} \right )} - \frac{d^{2}}{d \theta{\left (t \right )}^{2}}  z{\left (\theta{\left (t \right )} \right )} + \frac{\mu}{h^{2}}=0.
\]
La ecuaci�n del oscilador arm�nico. Sabemos resolver esta ecuaci�n y S tambi�n!!
\begin{lstlisting}
>>> var('theta')
>>> z=Function('z')(theta)
>>> ed5=ed4.subs(theta(t),theta)
>>> dsolve(ed5,z).simplify()
\end{lstlisting}
Obtenemos
\[
z{\left (\theta \right )} = C_{1} \sin{\left (\theta \right )} + C_{2} \cos{\left (\theta \right )} + \frac{\mu}{h^{2}}.
\]
Ahora si escribimos $C_1=\rho\cos\omega$ y $C_2=-\rho\sen\omega$ y recordamos que $z=1/r$, deducimos
\[r=\frac{1}{\frac{\mu}{h^2}+\rho\sen(\theta-\omega)}\]
Llamando $p=\tfrac{h^2}{\mu}$ y $e=\tfrac{\rho h^2}{\mu}$
\boxedeq{r=\frac{p}{1+e\sen(\theta-\omega)}}{eq:orb_elipse}

\noindent\textbf{Ejercicio:} La ecuaci�n \eqref{eq:orb_elipse} es la ecuaci�n de una c�nica con foco en el origen y excentricidad $e$.
Recordemos que la variable $\theta$ es el �ngulo polar. Hagamos algunos gr�ficos para distintas excentricidades.
\begin{lstlisting}
from sympy.plotting import plot_parametric
""" Gr�fico de excentricidades menores a uno  """
cant=20
e_s=[.04*j for j in range(cant)]
e=e_s[0]
p=plot_parametric(1/(1+e*cos(theta))*cos(theta),\
                  1/(1+e*cos(theta))*sin(theta),\
                  (theta,0,6.28),show=False)


for e in e_s[1:]:
    p1=plot_parametric(1/(1+e*cos(theta))*cos(theta),\
                       1/(1+e*cos(theta))*sin(theta),\
                       (theta,0,6.28),show=False)
    p.append(p1[0])

""" Gr�fico de excentricidades mayores a uno  """
e_s=[.04*j+1 for j in range(1,cant)]
for e in e_s:
    p1=plot_parametric(1/(1+e*cos(theta))*cos(theta),\
                       1/(1+e*cos(theta))*sin(theta),\
                       (theta,-pi*2/3,pi*2/3),show=False,line_color=(0,1,0))
    p.append(p1[0])

p.show()
\end{lstlisting}

\begin{center}
\includegraphics[scale=.5]{imagenes/conicas.png}
\end{center}


Hemos logrado encontrar $r$ como funci�n de $\theta$. No obstante
no hemos logrado resolver a�n el problema de los dos cuerpos \eqref{eq:grav}, para ello deber�amos encontrar $\v{r}(t)$, es decir poner a $
\v{r}$ como funci�n de $t$. Esto nos servir�a para decir que punto de la �rbita ocupa el planeta en un dado momento. Este problema no lo desarrollaremos aqu� dado
que su soluci�n se aparta del tema de las ecuaciones diferenciales.


\section{Ecuaciones lineales de orden superior}


\begin{definicion}[Ecuaci�n lineal general de orden $n$]
Es una ecuaci�n de la forma
\boxedeq{y^{(n)}(x)+p_{n-1}(x)y^{(n-1)}(x)+\cdots+p_0(x)y(x)=r(x)}{eq:ecua_lin_gen_orden_n}
donde $p_i,r$, $i=0,\ldots,n-1$ son funciones definidas en un intervalo $I$
\end{definicion}


Los resultados y t�cnicas que hemos desarollado para ecuaciones de orden $2$ se aplican con cambios previsibles a ecuaciones de mayor orden. Exponemos de manera sumaria
estos resultados.

\subsection{Existencia y unicidad de soluciones}

\begin{teorema}[Teorema de existencia y unicidad de soluciones]
 Supongamos $p_i,r$, $i=0,\ldots,n-1$ continuas sobre $I$. Sean $x_0\in I$ e $y_0,y_1,\ldots y_{n-1}\in\rr$ dados. Entonces existe una �nica soluci�n del PVI
 \[\left\{
 \begin{array}{l l l}
   y^{(n)}(x)+&p_{n-1}(x)y^{(n-1)}(x)+\cdots+p_0(x)y(x)=r(x),x\in I\\
   y(x_0)&=y_0&\\
   y'(x_0)&=y_1&\\
   &\vdots\\
  y^{(n-1)}(x_0)&=y_{n-1}&\\
  \end{array}\right.
\]

\end{teorema}


\subsection{Estructura del conjunto de soluciones}

\begin{teorema}[Estructura conjunto de soluciones ecuaciones homog�neas]
Supongamos $y_1,y_2,\ldots,y_n$ soluciones linealmente independientes de
\[y^{(n)}(x)+p_{n-1}(x)y^{(n-1)}(x)+\cdots+p_0(x)y(x)=0.\]
Entonces
\[y=c_1y_1+\cdots+c_ny_n,\quad c_i\in\rr, i=1,\ldots,n,\]
es soluci�n general. Vale decir, el conjunto de soluciones es un espacio vectorial $n$-dimensional.
\end{teorema}



\begin{teorema}[Estructura conjunto de soluciones ecuaciones no homog�neas]
Una soluci�n general de la ecuaci�n no homog�nea
\[y^{(n)}(x)+p_{n-1}(x)y^{(n-1)}(x)+\cdots+p_0(x)y(x)=r(x),\]
es la suma de una soluci�n particular de esta ecuaci�n m�s una soluci�n general de la ecuaci�n homog�nea asociada.
\end{teorema}






\begin{teorema}[F�rmula Abel]
Si $y_1,y_2,\ldots,y_n$ son soluciONES de la ecuaci�n  homog�nea
\boxedeq{y^{(n)}(x)+p_{n-1}(x)y^{(n-1)}(x)+\cdots+p_0(x)y(x)=0,}{eq:lin_hom_orden_n}
Entonces el Wronskiano satisface
\boxedeq{W(y_1,\ldots,y_n)(x)=W(y_1,\ldots,y_n)(x_0)e^{-\int_{x_0}^xp_{n-1}(t)dt}.}{eq:formu_abel}
En particular  $\{y_1,y_2,\ldots,y_n\}$ son linalmente independientes si y solo si $W(x)\neq 0$ para todo $x\in I$.
\end{teorema}

\textbf{Dem.} Las demostraciones de los resultados anteriores es tan similar a su an�logo de orden 2 que no vale la pena invertir tiempo en ellas. La demostraci�n de la f�rmula de Abel, si nos parece lo suficientemente interesante para dejarla como \textbf{ejercicio}.



\subsection{Ecuaciones homog�neas con coeficientes constantes}
Las ecuaciones lineales de orden $n$, homogeneas con coeficientes constantes  se resuelven por m�todos an�logos a los considerados para ecuaciones de segundo orden. Se propone $y(x)=e^{rx}$ como soluci�n. Reemplazando esta funci�n
en \eqref{eq:lin_hom_orden_n} vemos que $y$ ser�a soluci�n si y solo si $r$ es soluci�n de la ecuaci�n caracter�stica
\boxedeq{r^n+p_{n-1}r^{n-1}+\cdots+p_1r+p_0=0.}{eq:ecua_carac}
Ahora se presentan diversos casos.



\subsubsection{Raices reales distintas.} Si la ecuaci�n caracter�stica \eqref{eq:ecua_carac} tiene $n$ ra�ces $r_1,\ldots,r_n$ reales y distintas entonces
$y_1(x)=e^{r_1x},\ldots,y_n(x)=e^{r_nx}$ son soluciones linealmente independientes y por ende
\[\boxed{y(x)=c_1e^{r_1x}+\cdots+c_ne^{r_nx}}\]
es soluci�n general.

Dejamos como ejercicio la demostraci�n de la independencia lineal. En lugar de usar el Wronskiano, se puede utilizar  la siguiente idea basada en m�todos operacionales. Denotemos por $D$  el operador diferenciaci�n, esto es $D$ es sencillamente la funci�n  definida sobre el conjunto de funciones diferenciables sobre un intervalo abierto y que act�a derivando, esto es $Dy=y'(x)$. Dado un polinomio $p(X)=p_nX^n+p_{n-1}X^{n-1}+\cdots+p_1X+p_0$, $p_i\in\rr$, $i=0,\ldots,n$, denotamos por  $p(D)$ el operador definido por
\[p(D)y=p_ny^{(n)}+p_{n-1}y^{(n-1)}+\cdots+p_1y'+p_0y.\]
Diremos que $p(D)$ es un \emph{operador diferencial polinomial}.

\noindent\textbf{Ejercicio 1} Dado que dos polinomios se pueden sumar y multiplicar, resultando estas operaciones en un nuevo polinomio, es posible hacer lo propio con operadores diferenciales polinomiales. Demostrar que el producto de dos de tales operadores  $p(D)$ y $q(D)$ es conmutativo. Esta propiedad es lo mismo que afirmar que el producto de polinomios es conmutativo. Analizar que ocurrir�a si permiti�semos que los coeficientes $p_i$ fuesen funciones de $x$, $p_i=p_i(x)$.


\noindent\textbf{Ejercicio 2} Supongamos ahora que $r_1,\ldots,r_n$ son n�meros reales distintos y que
\[y=c_1e^{r_1x}+c_2e^{r_2x}+\cdots+c_ne^{r_nx}.\]
Considerar el operador diferencial
\[p(D)=(D-r_1)\cdots(D-r_{i-1})(D-r_{i+1})\cdots(D-r_n).\]
Demostrar que
\[p(D)y=(r_i-r_1)\cdots(r_i-r_{i-1})(r_i-r_{i+1})\cdots(r_i-r_n)e^{r_ix}.\]
Deducir de esto la independencia lineal de $\{e^{r_1x},\ldots,e^{r_nx}\}$.



\subsubsection{Raices reales repetidas.}
Supongamos que la ecuaci�n caracter�stica \eqref{eq:ecua_carac} tiene  ra�ces reales repetidas. Sean $r_1,\ldots,r_k$ las raices distintas y $m_j$ la multiplicidad de la ra�z $r_j$. Por cada ra�z $r_j$ considerar las $m_j$  funciones

\[\mathcal{B}_j:=\{y_j^0(x)=e^{r_jx}, y_j^1(x)=xe^{r_jx},\ldots y_j^{m_j-1}(x)=x^{m_j-1}e^{r_jx}\}.\]

 \noindent\textbf{Ejercicio 3} Demostrar que $\mathcal{B}:=\mathcal{B}_1\cup\cdots\cup \mathcal{B}_k$ forma
 un conjunto de  $n$  soluciones linealmente independientes.

\subsubsection{Raices complejas.}
Supongamos que la ecuaci�n caracter�stica \eqref{eq:ecua_carac} tiene  ra�ces complejas. Como la ecuaci�n caracter�stica tiene coeficientes reales, las raices aparecen de a pares conjugados $\mu\pm\nu i$. Si las raices son simples por cada uno de estos pares hay que considerar las soluciones
\[e^{\mu x}\cos x\quad\text{y}\quad e^{\mu x}\sen x.\]
Si son m�ltiples con multiplicidad $k$ hay que considerar las $2k$ soluciones
\[x^0e^{\mu x}\cos x,\ldots, x^{k-1}e^{\mu x}\cos x\quad\text{y}\quad x^0e^{\mu x}\sen x,\ldots x^{k-1}e^{\mu x}\sen x .\]
Dejamos los detalles que es necesario completar como trabajo pr�ctico.




\subsection{Aplicaci�n: osciladores arm�nicos acoplados}
Supongamos dos masas $m_1$ y $m_2$ sujetas a dos puntos fijos por sendos resortes y, a su vez, unidas entre si por un tercer resorte.
\begin{center}
\animategraphics[controls, scale=.4]{15}{osciladores_acoplados/osci-}{0}{751}
 \end{center}

Vamos a medir la posici�n de las masas desde origenes distintos situados en los respectivos puntos de equilibrio. Por consiguiente las posiciones $x$ e $y$ de las masas representan tambi�n su desplazamiento desde el equilibrio.  Utilizando la segunda ley de Newton, la Ley de Elasticidad de Hooke  y tomando en consideraci�n que el resorte central est� desplazado desde su estado de equilibrio en una cantidad igual a $y-x$, vemos que se debe satisfacer que
% \[\left\{ \begin{array}{l l}
%              m_1x''(t)=-k_1x +k_3(y-x)\\
%              m_2y''(t)=-k_3(y-x)-k_2y
%             \end{array} \right.
% \]
\begin{subequations}
\label{eq:tot}
\begin{empheq}[left=\empheqlbrace]{align}
             m_1x''(t)=-k_1x +k_3(y-x)\label{eq:sis1}\\
             m_2y''(t)=-k_3(y-x)-k_2y\label{eq:sis2}
\end{empheq}
\end{subequations}

\begin{center}
  \includegraphics[scale=.4]{imagenes/osci_aco.png}
\end{center}


Se nos present� un sistema de ecuaciones de segundo orden. Vamos a poder convertirlo en una ecuaci�n pagando el precio de incrementar el orden. El procedimiento es derivar \eqref{eq:sis1} (obviamente es lo mismo empezar por \eqref{eq:sis2}) dos veces respecto a $t$. En el resultado sustitu�mos $y''$ por su igual seg�n \eqref{eq:sis2}. El resultado es una ecuaci�n que todav�a tiene la variable $y$, pero podemos usar \eqref{eq:sis1} para sustituir $y$ por una expresi�n que s�lo tiene $x$ y sus derivadas. Todo esto lo haremos con SymPy.



\begin{lstlisting}
>>> from sympy import *
>>> init_printing()
>>> var('x,y,t,m1,m2,k1,k2,k3')
>>> x,y=symbols('x,y',cls=Function)
>>> eq1=m1*x(t).diff(t,2)+k1*x(t)-k3*(y(t)-x(t))
>>> eq2=m2*y(t).diff(t,2)+k2*y(t)+k3*(y(t)-x(t))
>>> sust1,=solve(eq2,y(t).diff(t,2))
>>> sust2,=solve(eq1,y(t))
>>> eq3=eq1.diff(t,2).subs({y(t).diff(t,2):sust1, y(t):sust2})
>>> eq3.expand().collect(x(t))
\end{lstlisting}
 Obtenemos la ecuaci�n de cuarto orden
\[
\begin{split}
 m_{1} \frac{d^{4}}{d t^{4}} & x{\left (t \right )} + \left(\frac{k_{1} k_{2}}{m_{2}} + \frac{k_{1} k_{3}}{m_{2}} + \frac{k_{2} k_{3}}{m_{2}}\right) x{\left (t \right )}\\ &+ \left(k_{1} + \frac{k_{2} m_{1}}{m_{2}} + \frac{k_{3} m_{1}}{m_{2}} + k_{3}\right) \frac{d^{2}}{d t^{2}}  x{\left (t \right )}=0.
\end{split}
\]
 Vamos a suponer todos los par�metros igual a 1. Encontremos y resolvamos la ecuaci�n caracter�stica
 \begin{lstlisting}
>>> eq4=eq3.subs({m1:1,m2:1,k1:1,k2:1,k3:1})
>>> eq4
\end{lstlisting}
\[3 x{\left (t \right )} + 4 \frac{d^{2}}{d t^{2}}  x{\left (t \right )} + \frac{d^{4}}{d t^{4}}  x{\left (t \right )}=0.\]





\begin{lstlisting}
>>> r=var('r')
>>> z=exp(r*t)
>>> eq5=eq4.subs({x(t):z,x(t).diff(t,4):z.diff(t,4),x(t).diff(t,2):z.diff(t,2)})/z
>>> eq5.expand()
\end{lstlisting}
Las soluciones de la ecuaci�n caracter�stica son ${sol}$.
\begin{lstlisting}
>>> sol=solve(eq5,r)
\end{lstlisting}

Seg�n lo que hemos dicho antes, la soluci�n general ser�
\boxedeq{ x(t)=A\cos(t)+B\sen(t)+C\cos(\sqrt(3)t) +D\sen(\sqrt(3)t)}{eq:sol_gen_arm_aco}

La soluci�n es una superposici�n de ondas con frecuencias \href{http://es.wikipedia.org/wiki/Conmensurabilidad_(matem�tica)}{inconmensurables}. Decimos que dos magnitudes no nulas son inconmensurables cuando su cociente es irracional.

�Tendra el sistema de osciladores acoplados soluciones peri�dicas? Notar que una superposici�n de funciones peri�dicas no necesariamente es peri�dica.   La pregunta nos lleva a una pregunta m�s general. Si $f,g:\rr\to\rr$ son funciones peri�dicas de per�odo $T_1$ y $T_2$, ser� $f+g$ per�odica. La respuesta es el teorema de abajo.

\begin{teorema}  Si $f,g:\rr\to\rr$ son funciones no constantes, peri�dicas y continuas de per�odo $T_1$ y $T_2$, la funci�n $f+g$ ser� per�odica si y s�lo si $T_1$ y $T_2$ son conmensurables.
\end{teorema}

\begin{proof} La demostraci�n descansa sobre varios hechos, que poco tienen que ver con las ecuaciones diferenciales. Pero, el teorema nos parece tan interesante, que vamos a dar algunos detalles y otros los dejaremos como ejercicio.
 \end{proof}


\noindent \textbf{Ejercicio} Sea $F:\rr\to\rr$ una funci�n per�odica y sea $\mathfrak{P}$ el conjunto de todos los per�odos. Entonces
\begin{enumerate}
 \item $\mathfrak{P}$ es un subgrupo aditivo  de $\rr$. Si $F$ es continua $\mathfrak{P}$ es cerrado.
 \item Si $\mathfrak{P}$ es cualquier subgrupo aditivo propio y cerrado de $\rr$  entonces  $\mathfrak{P}$ es un grupo ciclico, es decir existe $a>0$ con $\mathfrak{P}=a\mathbb{Z}$.  Como corolario, si $\mathfrak{P}$ es cualquier subgrupo  aditivo cerrado propio  y $T_1,T_2\in \mathfrak{P}$ entonces $T_1$ y $T_2$ son conmensurables. \emph{Ayuda:} Considerar
\[a:=\inf\{x\in \mathfrak{P}:x>0\}\]
Entonces si $a>0$,  $\mathfrak{P}=a\mathbb{Z}$ y si $a=0$,  $\mathfrak{P}=\rr$.

�Qu� ocurrir� si no suponemos $\mathfrak{P}$ cerrado?

 \item Demostrar el Teorema. \emph{Ayuda:} Supongamos  que $f+g$ tiene per�odo $T>0$. Entonces:
\[F(x):=f(x+T)-f(x)=g(x)-g(x+T).\]
La funci�n $F$ tendr� per�odos $T_1$ y $T_2$.
 \end{enumerate}

Retornando al oscilador arm�nico y a su soluci�n general \eqref{eq:sol_gen_arm_aco}, el ejercicio anterior nos dice  que la soluci�n no ser� peri�dica, a menos que $A=B=0$ o $C=D=0$. Estos casos especiales de soluciones se denominan \href{http://es.wikipedia.org/wiki/Modo_normal}{modos normales}.  Usemos SymPy para encontrar y graficar algunos modos normales. Gr�ficaremos las soluciones sobre el espacio de configuraciones. Esto es decir que graficaremos las curvas $t\mapsto (x(t),y(t))$ en $\rr^2$.

\noindent\textbf{Primer modo normal}
\begin{lstlisting}
>>> var('A,B,C,D')
>>> x=A*cos(t)+B*sin(t)+C*cos(sqrt(3)*t)+D*sin(sqrt(3)*t)
>>> x1=x.subs({A:1,B:0,C:0,D:0})
>>> y1=x1.diff(t,2)+2*x1
>>> from sympy.plotting import *
>>> plot_parametric(x1,y1,(t,0,10*pi))
\end{lstlisting}

\begin{wrapfigure}[4]{l}{5cm}
  \begin{center}
   \vspace{-.5cm}\includegraphics[scale=.2]{imagenes/modo_normal1.png}
 \end{center}
\end{wrapfigure}
 Los desplazamientos de las dos masas estan sobre la recta $y=x$, vale decir las masas se mueven perfectamente en fase.


 \begin{center}
   \animategraphics[controls, scale=.4]{15}{osciladores_acoplados/primero/modo1-}{0}{9}
 \end{center}





\noindent\textbf{Segundo modo normal}

\begin{lstlisting}
>>> x1=x.subs({A:0,B:0,C:1,D:0})
>>> y1=x1.diff(t,2)+2*x1
>>> plot_parametric(x1,y1,(t,0,10*pi))
\end{lstlisting}

\begin{wrapfigure}[6]{l}{5cm}
  \begin{center}
   \vspace{-.5cm}\includegraphics[scale=.2]{imagenes/modo_normal2.png}
 \end{center}
\end{wrapfigure}
Los desplazamientos de las dos masas estan sobre la recta $y=-x$, vale decir las masas se mueven perfectamente fuera de fase. Cuando una alcanza el desplazamiento negativo menor la otra alcanza el mayor positivo.


 \begin{center}
    \animategraphics[controls, scale=.4]{15}{osciladores_acoplados/segundo/modo2-}{0}{9}
 \end{center}



\noindent\textbf{Fuera de un modo normal}
\begin{lstlisting}
>>> x1=x.subs({A:1,B:0,C:1,D:0})
>>> y1=x1.diff(t,2)+2*x1
>>> plot_parametric(x1,y1,(t,0,10*pi))
\end{lstlisting}
\begin{wrapfigure}[6]{r}{5cm}
  \begin{center}
   \vspace{-.5cm}\includegraphics[scale=.2]{imagenes/lissajous.png}
   \caption{Curva de Lissajous}
 \end{center}
\end{wrapfigure}

Se obtienen gr�ficas  bonitas llamadas \href{http://es.wikipedia.org/wiki/Curva_de_Lissajous}{Curvas de Lissajous}, que fuera de los modos normales llenan densamente un cuadrado del plano.










\section{M�todos Operacionales}


El denominado \href{http://en.wikipedia.org/wiki/Operational_calculus}{M�todo Operacional} es una t�cnica iniciada por el ingeniero el�ctrico  \href{http://es.wikipedia.org/wiki/Oliver_Heaviside}{Oliver Heavside} (1850-1925).

\begin{wrapfigure}[13]{r}{5cm}
  \begin{center}
\includegraphics[scale=.8]{imagenes/Heaviside.jpg}
 \end{center}
 \caption{Oliver Heavside}
\end{wrapfigure}
Hemos mencionado que una ecuaci�n lineal de orden $n$ con coeficientes constantes y  no homog�nea se puede pensar como
\boxedeq{p(D)y=r(x),}{eq:lin_oper}
donde $p(D)=D^n+p_{n-1}D^{n-1}+\cdots+p_1D+ p_0$ es un operador diferencial polinomial.




La idea central del m�todo es proceder desde una manera puramente formal, para afirmar que si $y$ resuelve
\eqref{eq:lin_oper} entonces
\boxedeq{y=\frac{1}{p(D)}r(x),}{eq:oper_inv}
Esto no parece m�s que un juego de s�mbolos del que no se puede desprender nada interesante. Vamos a ver que no es ese el caso.

El s�mbolo $1/p(D)$ deber�a ser interpretado como el operador inverso de $p(D)$. Por ejemplo, supongamos que $p(D)=D$. entonces $p(D)y=y'$. En este caso $1/p(D)$ puede ser definido como

\boxedeq{\frac{1}{p(D)}r=\int r(x)dx}{eq:oper_inv_int}

Si tuviesemos $p(D)=D-q$, $q\in\rr$, entonces $p(D)y=y'-qy$. En este caso, teniendo en mente que $y(x)=(1/p(D))r(x)$ deber�a resolver la ecuaci�n lineal de primer orden $y'-qy=r(x)$, y por la f�rmula expl�cita que obtuvimos para esta soluci�n, es natural definir
\boxedeq{\frac{1}{p(D)}r=e^{qx}\int e^{-qx}r(x)dx}{eq:oper_inv_lin}

\subsection{Ra�ces simples}
Supongamos ahora que $p$ es un polinomio que se factoriza en monomios de primer orden
\[p(D)=(D-p_0)\cdots(D-p_k).\]
Siguiendo la l�nea de razonamiento aneterior
definimos
\boxedeq{\frac{1}{p(D)}r=\frac{1}{D-p_0}\left(\frac{1}{D-p_1}\left(\cdots\frac{1}{D-p_k}\left(r  \right)\cdots    \right)  \right)}{eq:oper_inv_fact}


\noindent\textbf{Problema.} Resolver  $y''-3y'+2y=xe^x$. Las cuentas las haremos con SymPy. Primero veamos si el polinomio se factoriza



\begin{lstlisting}
>>> var('D')
>>> p=D**2-3*D+2
>>> p.factor()
\end{lstlisting}

Vemos que $p(D)=(D-2)(D-1)$. Ahora programemos la f�rmula \eqref{eq:oper_inv_lin} y usemosla para resolver la ecuaci�n.
\begin{lstlisting}
>>> x=var('x')
>>> LinInv=lambda r,a: exp(a*x)* integrate(exp(-a*x)*r,x)
>>> LinInv(LinInv(x*exp(x),1),2)
\end{lstlisting}
Deducimos $y=\frac{e^{x}}{2} \left(- x^{2} - 2 x - 2\right)
$ es soluci�n.



\subsubsection{Fracciones simples}
 Otra idea es descomponer $1/p(D)$ en fracciones simples. Suponiendo $p(D)=(D-p_0)\cdots(D-p_k)$, con $p_j$ ra�ces simples
\[\frac{1}{(D-p_0)\cdots(D-p_k)}=\left\{\frac{A_0}{(D-p_0)}+\cdots+\frac{A_k}{(D-p_k)}\right\}.\]
Como cada t�mino del miembro de la derecha lo tenemos definido, s�lo tenemos que sumar cada uno de ellos. Reprocesemos con esta idea el  ejemplo de antes. La funci�n \texttt{apart} de sympy hace descomposiciones en fracciones simples. En el caso del operador $p(D)=(D-2)(D-1)$ obtenemos descomposici�n en fracciones simples
\begin{lstlisting}
>> apart(1/p)
\end{lstlisting}

\[\frac{1}{p(D)}=-\frac{1}{D-1}+\frac{1}{D-2}.\]
 Entonces la siguiente expresi�n deber�a darnos una soluci�n
\begin{lstlisting}
>>> LinInv(x*exp(x),2)-LinInv(x*exp(x),1)
\end{lstlisting}
Obviamente obtenemos la misma soluci�n que obtuvimos antes.

\subsection{Series}
En algunas ocasiones es conveniente desarrollar en serie $1/p(D)$:
\boxedeq{\frac{1}{p(D)}r(x)=\left(a_0+a_1D+a_2D^2+\cdots \right)r(x).}{eq:des_serie}
Por ejemplo cuando $r$ es un polinomio, puesto que salvo una cantidad finita, todas las derivadas de orden $k$ de $r$ son cero.

\noindent\textbf{Ejemplo.} Resolver $y'''+2y''+y=x^4+2x+5$.
Como $r$ es un polinomio de grado 4, desarrollemos en serie $1/p(D)$ hasta ese orden

\begin{lstlisting}
>>> L=1/(1+2*D**2+D**3)
>>> L.series(D,0,5)
\end{lstlisting}
Obtenemos
\[\frac{1}{p(D)}=\frac{1}{1+2D^2+D^3}=
1 - 2 D^{2} - D^{3} + 4 D^{4} + \mathcal{O}\left(D^{5}\right).\]
Ahora el siguiente c�digo  define el operador diferencial asociado a la expresi�n anterior
\begin{lstlisting}
>>> coeficientes=[Q.diff(D,j).subs(D,0)/factorial(j) for j in range(5)]
        def Q_op(f):
	return sum([coeficientes[j]*f.diff(x,j) for j in range(5)])
\end{lstlisting}

Evaluamos el segundo miembro de  \eqref{eq:des_serie} en $r(x)=x^4+2x+5$.
\begin{lstlisting}
>>> Q_op(x**4+2*x+5)
\end{lstlisting}
Llegamos a la soluci�n
\[y=x^{4} - 24 x^{2} - 22 x + 101
\]

\end{document}
%\include{Uni5}
%



\chapter[Series de Potencias y de Frobenius]{Métodos de desarrollo en serie de potencias y en serie de Frobenius}
%\date{}



 \section{Series de potencias}

En esta sección recordamos algunos conceptos y teoremas sobre series de potencias. Dado que este tema es motivo de un estudio más profundo en otras materias de la licenciatura en matemática omitimos algunas demostraciones.
\subsection{Definición} %Primer Frame para información personal.
\begin{definicion}{} Una serie de potencias es una serie de la forma:
\[
	\sum\limits_{n=0}^{\infty}a_n(z-z_0)^n
\]
donde $a_n$, $n=0,1,\ldots$, $z_0$ y $z$ son elementos de $\com$.
\end{definicion}


Estamos interesados en determinar los valores de $z$ para los cuales   una serie converge.

\begin{ejemplo}{} La serie geométrica
\[
	\sum\limits_{n=0}^{\infty}z^n,
\]
es una serie de potencias. Aquí  $a_n=1$, $n=0,1,\ldots$ y $z_0=0$. Esta serie converge para $|z|<1$ a
\[ \frac{1}{1-z}\]
y no converge para cualquier otro valor de $z\in\com$.
\end{ejemplo}

\begin{ejemplo}{} Supongamos $f:I\to\rr$, donde $I$ es un intervalo abierto $I=(a,b)$ y que $f$ tiene derivadas de todo orden en  $z_0\in I$. Entonces es posible construir la serie de Taylor de $f$ en $z_0$ que es una serie de potencias. Recordemos que esta serie es
\[S(f,z_0,z)=\sum\limits_{n=0}^{\infty}\frac{f^{(n)}(z_0)}{n!}(z-z_0)^n.\]
\end{ejemplo}




\subsection{Límites superior e inferior}


\begin{definicion}{} Dada una sucesión de números reales $x_n$,
 consideramos una nueva sucesión:

\[
	A_n=\sup\{x_n, x_{n+1},\ldots \}
\]

La nueva sucesión de reales $A_n$ es  no creciente ($A_n\geq A_{n+1}$), luego tiene un límite (puede ser $\pm\infty$). A este límite lo llamamos \emph{el límite superior de $x_n$}. Lo denotamos por $\limsup_{n\to\infty} x_n$. Es decir:


\[
\limsup_{n\to\infty} x_n=\lim_{n\to\infty} A_n=\lim\limits_{n\to \infty} \sup\{x_n, x_{n+1},\ldots \}.
\]


Tomando ínfimo en lugar de supremo conseguimos \emph{el límite inferior} ($\liminf$).
\end{definicion}
\begin{ejemplo}{}  Si $x_n=(-1)^n$, entonces
\[
	\{x_n,x_{n+1},\ldots\}=\{\pm1,\mp1,\pm1,\ldots\}.
\]
El supremo de este conjunto  es para todo $n$ igual a 1 y el ínfimo igual a -1. Luego $\liminf x_n=-1$ y $\limsup x_n=1$.
\end{ejemplo}



\begin{ejemplo}{}  Si $x_n=1/n$, si $n$ es par y $x_n=1$ si $n$ es impar, entonces el conjunto

\[
	\{x_n,x_{n+1},\ldots\}
\]
tiene por supremo 1 y el ínfimo igual a $0$ . Luego $\liminf x_n=0$ y $\limsup x_n=1$.
\end{ejemplo}

\begin{teorema}[Propiedades]{} Sea $x_n$  e $y_n$ dos sucesiones de números reales, entonces:

\begin{enumerate}
	\item El $\limsup x_n$ y el $\liminf x_n$  existen si se permite que $\pm\infty$ sean sus posibles valores.
	\item $\liminf x_n \leq \limsup x_n$.
	\item $\liminf x_n=\limsup x_n$ si y solo si el $\lim x_n$ existe. En este caso todos los límites coinciden.
	\item $\liminf (x_n+y_n)\geq \liminf x_n + \liminf y_n$.
	\item $\limsup (x_n+y_n)\leq \limsup x_n + \limsup y_n$

\end{enumerate}
\end{teorema}



\subsection{Radio de convergencia}


\begin{definicion}{} Dada la serie de potencias
\[
	\sum\limits_{n=0}^{\infty}a_n(z-z_0)^n,
\]
definimos el \emph{radio de convergencia} $R$ de la siguiente forma:



\[
\frac{1}{R}=\limsup\limits_{n\to \infty} |a_n|^{1/n}.
\]

\end{definicion}

\begin{ejemplo}{} La serie
\[
	\sum\limits_{n=0}^{\infty}z^n,
\]
tiene radio de convergencia:

\[\frac{1}{R}=\limsup\limits_{n\to \infty}1^{1/n}=\lim\limits_{n\to\infty}1^{1/n}=1\]
Luego $R=1$.
\end{ejemplo}

\begin{ejemplo}{} La serie
\[
	\sum\limits_{n=0}^{\infty}\bigg(\frac{1}{M}\bigg)^nz^n,
\]
tiene radio de convergencia:

\[\frac{1}{R}=\limsup\limits_{n\to \infty}\bigg(\bigg(\frac{1}{M}\bigg)^n\bigg)^{1/n}=\lim\limits_{n\to\infty}\bigg(\bigg(\frac{1}{M}\bigg)^n\bigg)^{1/n}=\frac1M\]
Luego $R=M$.

\end{ejemplo}

\begin{ejemplo}{} Fijemos $M>0$ y $n$ un natural tal que $[n/2] >M$ (aquí $[x]$ es la parte entera de $x$). Entonces,
como $n-[n/2]\geq [n/2]>M$

\[
\begin{split}
n!=n(n-1)\cdots 1 &> n(n-1)\cdots (n-[n/2]) \\
&>\underbrace{M\cdots M }_{[n/2]-\hbox{veces}}\\
&\geq M^{[n/2]}\\
&> M^{n/3}\\
\end{split}
\]
Luego
\[\frac{1}{R}:=\limsup_{n\to\infty} (1/n!)^{1/n}\leq \lim_{n\to\infty} \bigg(\frac{1}{M^{n/3}}\bigg)^{1/n}=\frac{1}{\sqrt[3]{M}}\]
Como $M$ es arbitrario, haciendo $M\to\infty$ vemos que el radio de convergencia de la serie  $\sum\limits_{n=0}^{\infty}\frac{1}{n!}z^n$
es $R=\infty$.
\end{ejemplo}

\begin{teorema}  Supongamos que la serie:
\[
	\sum\limits_{n=0}^{\infty}a_n(z-z_0)^n,
\]
tiene radio de convergencia $R>0$. Entonces:
\begin{enumerate}
	\item Si $|z-z_0|<R$, la serie converge absolutamente en $z$.
	\item Si $|z-z_0|>R$, la serie diverge.
	\item Si $|z-z_0|=R$, no se afirma nada.

\end{enumerate}
\end{teorema}

 \begin{demo} Se puede suponer sin perdida de generalidad $z_0=0$. Supongamos $0<R<\infty$.  Sea $L=1/R$ y tomemos $\epsilon>0$ pequeño. Como
\[\lim_{n\to\infty}\sup\{|a_n|^{1/n},|a_{n+1}|^{1/n+1},\ldots\}=L\]
para $n_0$ suficientemente grande
\[\sup\{|a_n|^{1/n},|a_{n+1}|^{1/n+1},\ldots\}<L+\epsilon.\]
para $n\geq n_0$.   Así
\[|a_n|^{1/n}<L+\epsilon\quad\hbox{para } n\geq n_0.\]
Elijamos $0<r<1/(L+\epsilon)<1/L=R$. Si $|z|<r$ entonces
\[|a_n||z|^n<(L+\epsilon)^nr^n\quad\hbox{para } n\geq n_0.\]
Pero $r(L+\epsilon)<1$. La desigualdad de arriba y el teorema de comparación para series (notar que el miembro de la derecha forma una serie geométrica) implican que la serie converge absolutamente para este $z$.  Esto implica la convergencia para cualquier $|z|<R$, ya que si $|z|<R=1/L$ existe $\epsilon>0$ lo suficientemente chico para que $|z|<1/(L+\epsilon)$. Por el resultado ya demostrado la serie converge absolutamente para este $z$.\end{demo}
\begin{ejercicio} Demostrar los casos $R=0$, $R=\infty$ y el segundo inciso.
 \end{ejercicio}
\begin{teorema}  La función
\[
f(z)=	\sum\limits_{n=0}^{\infty}a_n(z-z_0)^n,
\]
es diferenciable dentro en $\{z:|z-z_0|<R\}$. Además
\[
f'(z)=g(z):=	\sum\limits_{n=1}^{\infty}na_n(z-z_0)^{n-1},
\]
teniendo esta serie el mismo radio de convergencia que el de $f$.
\end{teorema}

\begin{demo} Nuevamente supondremos $z_0=0$. La afirmación sobre el radio de convergencia es consecuencia de que $\lim_{n\to\infty}n^{1/n}=1$. Como el radio $R'$ de convergencia de $g$ es:

\[
\begin{split}
\frac{1}{R'}&=\limsup_{n\to\infty}|a_{n+1}(n+1)|^{1/(n+1)}\\
&\underbrace{=}_{\hbox{\emph{ Ejercicio}}}\limsup_{n\to\infty}|a_{n+1}|^{1/(n+1)}\lim_{n\to\infty}|(n+1)|^{1/(n+1)}\\
&=\limsup_{n\to\infty}|a_{n+1}|^{1/(n+1)}=\frac{1}{R}
\end{split}\]

Ahora veamos que  $f'=g$.   Sea $0<r<R$, $|z_0|<r$ y $N\in\mathbb{N}$. Pongamos:
\[f(z)=S_N(z)+E_N(z),\]
\[S_N(z)=\sum_{n=0}^{N}a_nz^n\quad\hbox{y}\quad E_N(z)=\sum_{n=N+1}^{\infty}a_nz^n\]
Tomemos $|h|<r-|z_0|$, así $|z_0+h|<r$.  Tenemos
\[\begin{split}
\frac{f(z_0+h)-f(z_0)}{h}-g(z_0)&= \frac{S_N(z_0+h)-S_N(z_0)}{h}-S_N'(z_0)\\
				&+S_N'(z_0)-g(z_0)\\
				&+\frac{E_N(z_0+h)-E_N(z_0)}{h}
\end{split}\]

Ahora si $\epsilon>0$

\[\begin{split}
				&\bigg|\frac{E_N(z_0+h)-E_N(z_0)}{h}\bigg|\leq\sum_{n=N+1}^{\infty}|a_n|\bigg|\frac{(z_0+h)^n-z_0^n}{h}\bigg|\\
				&=\sum_{n=N+1}^{\infty}|a_n|(|z_0|^{n-1}+|z_0|^{n-2}h+\cdots+h^{n-1})\\
				&\leq2\sum_{n=N+1}^{\infty}|a_n|nr^{n-1}<\epsilon\\
\end{split}\]
Para $N$ suficientemente grande. Además como $S_N'(z)\to g(z)$ cuando $N\to\infty$ podemos elegir, a su vez, $N$ suficientemente grande para que

\[
	|S_N'(z_0)-g(z_0)|<\epsilon
\]

 Fijemos un $N$ que satisfaga las condiciones anteriores. Ahora podemos encontrar $\delta>0$ para que $|h|<\delta$ cumpla que
\[
\bigg|\frac{S_N(z_0+h)-S_N(z_0)}{h}-S_N'(z_0)\bigg|<\epsilon.
\]

Esto muestra que $f'(z_0)=g(z_0)$ y por consiguiente $f$ es derivavble.
\end{demo}





\begin{corolario}  Una serie de potencias es infinitamente diferenciable. Las sucesivas derivadas se obtienen derivando término a término la serie. El radio de convergencia se conserva.
\end{corolario}

\begin{ejemplo}{}  Hemos visto que la serie:
\[
	f(z)=\sum_{n=0}^{\infty}\frac{1}{n!}z^n
\]
tiene radio de convergencia infinito y por ende converge en $\com$. Ahora vemos que
\[f'(z)=\sum_{n=1}^{\infty}\frac{1}{(n-1)!}z^{n-1} =\sum_{n=0}^{\infty}\frac{1}{n!}z^n
\]
Lo que nos dice que $f$ resuelve la simple ecuación diferencial $f'(z)=f(z)$. La misma ecuación es resuelta por $g(z)=e^z$. Además $f(0)=g(0)=1$. Por el Teorema de existencia y unicidad $f(z)=g(z)$ para todo $z$. Hemos probado la importante fórmula.
\boxedeq{e^z=\sum_{n=0}^{\infty}\frac{1}{n!}z^n}{eq:exp}
\end{ejemplo}
\subsection{Funciones analíticas}
\begin{definicion}{} Una función $f:\Omega\subset\com\to\com$ se dirá analítica si para cada $z_0\in\Omega$, existe  $R>0$ y $a_n\in\com$, tal que:

\[ f(z)=\sum_{n=0}^{\infty}a_n(z-z_0)^n,\quad\hbox{para } |z-z_0|<R \]
\end{definicion}


\begin{ejercicio} Si $f$ es analítica tenemos la siguiente fórmula
\[
	a_n=\frac{f^{(n)}(z_0)}{n!}
\]
para los coeficientes $a_n$.
\end{ejercicio}

 \begin{teorema}[Operaciones entre series de potencias]{teor_oper_series} Supongamos que $f(z)=\sum_{n=0}^{\infty}a_n(z-z_0)^n$ y  $g(z)=\sum_{n=0}^{\infty}b_n(z-z_0)^n$ son series de potencias con radio de convergencia mayor o igual a $R>0$. Entonces $f+g$ y $fg$ son funciones analíticas que tienen por desarrollo en serie
\[
    \begin{split}
      (f+g)(z)&=\sum_{n=0}^{\infty}(a_n+b_n)(z-z_0)^n\\
      (fg)(z)&=\sum_{n=0}^{\infty}\left(\sum_{k=0}^na_kb_{n-k}\right)(z-z_0)^n,\\
    \end{split}
\]
y los radios de convergencia de las series anteriores es, al menos, $R$. Si $g(z)\neq 0$ para $|z-z_0|<R$ entonces $f(z)/g(z)$ es analítica y se desarrolla por  una  serie de radio de convergencia al menos $R$. Es posible expresar los coeficientes del cociente en términos de los coeficientes del dividendo y divisor, pero no nos detendremos en ello.
\end{teorema}


\section{Solución de EDO mediante series de potencias. Método coeficientes indeterminados}



\subsection{Método coeficientes indeterminados}

Dada una EDO

\begin{equation} \hbox{(1)}\quad  F(x,y,y',\ldots,y^{(n)})=0\end{equation}
queremos desarrollar en series de potencias  la solución general a esta ecuación. El método que estudiaremos se denomina \emph{metodo de los coeficientes indeterminados}. Consiste en proponer el desarrollo en serie de la solución

\[y(x)=a_0+a_1(x-x_0)+a_2(x-x_0)^2+\cdots  \]
remplazar $y(x)$ por este desarrollo en  en la ecuación (1) y tratar de resolver la ecuación resultante para los coeficientes (indeterminados) $a_n$. El método suele funcionar en algunas ecuaciones. Desarrollemos un ejemplo.



\begin{ejemplo}{} Hallar el desarrollo en serie de la solución del siguiente pvi
\[\left\{\begin{array}{l l} y'&=y\\ y(0)&=1\end{array}\right.\]
La solución, es bien sabido, es $y(x)=e^x$,  pero pretendemos reencontrarla por el método expuesto. Escribimos
\[\begin{split}
   y&=a_0+a_1x+a_2x^2+\cdots+a_nx^n+\cdots\\
   y'&=a_1+2a_2x+3a_3x^2+\cdots+(n+1)a_{n+1}x^n+\cdots
  \end{split}
\]
La igualdad $y'=y$ implica que
\[\begin{split}
   a_1&=a_0\\
   a_2&=\frac{a_1}{2}\\
   a_3&=\frac{a_2}{3}\\
      &\,\,\,\,\vdots \\
   a_{n+1}&=\frac{a_{n}}{n+1}
 \end{split}
\]
Si iteramos la fórmula $a_{n+1}=a_{n}/(n+1)$, obtenemos
\[a_n=\frac{1}{n}a_{n-1}=\frac{1}{n(n-1)}a_{n-2}=\cdots=\frac{1}{n(n-1)\cdots 1}a_{0}=\frac{a_0}{n!}.\]
Pero $a_0=y(0)=1$. Luego
\boxedeq{a_n=\frac{1}{n!}}{eq:exp}






Vemos que la solución es
\boxedeq{y(x)=1+x+\frac{x^2}{2}+\frac{x^3}{6}+\frac{x^4}{24}+\frac{x^5}{120}+\frac{x^6}{720} +\cdots         }{eq:sol_sage}
\end{ejemplo}






\subsection{Relaciones de recurrencia}

La expresión $a_{n+1}=\frac{a_{n}}{n+1}$ es un ejemplo de \href{http://es.wikipedia.org/wiki/Relación_de_recurrencia}{relación de recurrencia}.
\begin{definicion}{} Una  relación de recurrencia para una sucesión $b_n$ de números reales es una sucesión de
funciones $f_n:\rr^n\to\rr$ que relaciona $b_{n+1}$ con los términos anteriores de la sucesión por medio de
la expresión
\begin{equation}\label{eq:recu} b_{n+1}=f_n(b_1,\ldots,b_n).
\end{equation}
Resolver una relación de recurrencia es encontrar una fórmula explícita de $b_n$ como función de $n$.
 \end{definicion}


Hay técnicas y métodos para resolver relaciones de recurrencia que guardan analogías con técnicas y métodos de resolver ecuaciones diferenciales.  No vamos a desarrollar este importante tema en este curso, sugerimos la \href{http://es.wikipedia.org/wiki/Relación_de_recurrencia}{correspondiente wiki} en la wikipedia. Sólo agregamos que  SymPy resuelve relaciones recurrentes a través del comando \texttt{rsolve}.

\begin{ejemplo}{} Resolvamos con SymPy la sucesión de Fibonacci $a_{n+2}=a_{n+1}+a_n$.
\begin{sympyblock}[][numbers=left,frame=single,framesep=5mm]
from sympy import *
n=symbols('n',integer=True)
y = Function('y')
f=Equality(y(n),y(n-1)+y(n-2))
sol=rsolve(f,y(n))
\end{sympyblock}

 El resultado es
 \boxedeq{a_n=\sympy{sol} }{}
Las constantes arbitrarias $C_0$ y $C_1$ aparecen porque una relación de recurrencia no tiene una única solución. Se dice que una relación de recurrencia tiene orden $k$ o es de $k$-términos si el coeficiente $a_n$ se expresa en función de los $k$  anteriores. En general la solución general de una relación de recurrencia de $k$-términos tiene $k$ constantes arbitrarias. Por consiguiente, si queremos una única solución debemos tener $k$ relaciones extras. Usualmente esto se consigue dando los valores de los $k$-primeros términos $a_0,\ldots,a_k$. Por ejemplo, en  la sucesión de Fibonacci si pedimos $a_0=a_1=1$ entonces los primeros números de Fibonacci son

\[ [1,1,2,3,5,8,13,21,34,55] \]

\end{ejemplo}

\subsection{Serie binomial}

 Se puede utilizar el método de coeficientes indeterminados para encontrar desarrollos en serie de una función  $f$. La técnica consiste en encontrar un pvi que satisfaga $f$ y le aplicamos el método de coeficientes indeterminados a ese pvi.

\begin{ejemplo}{}  Encontrar el desarrollo en serie de la función
\[y(x)=(1+x)^p\quad p\in\rr\]
La función $y(x)$ resuelve el pvi  $(1+x)y'(x)=py$, $y(0)=1$. apliquemos el método de coeficientes indeterminados a este pvi.
Como
\[\begin{split}
   y&=a_0+a_1x+a_2x^2+\cdots+a_nx^n+\cdots\\
   y'&=a_1+2a_2x+3a_3x^2+\cdots+(n+1)a_{n+1}x^n+\cdots
  \end{split}
\]
Tenemos
\[\begin{split}
   py=&pa_0+pa_1x+pa_2x^2+\cdots+pa_nx^n+\cdots\\
  (1+x)y'=&a_1+2a_2x+3a_3x^2+\cdots+(n+1)a_{n+1}x^n+\cdots\\
          &+a_1x+2a_2x^2+3a_3x^3+\cdots+na_{n}x^n+\cdots\\
-------&----------------------\\
0=(1+x)y'-py =& (a_1-pa_0)+(a_1+2a_2-pa_1)x+\cdots +((n+1)a_{n+1}+na_n-pa_n)x^n+\cdots
  \end{split}
\]
Tenemos la relación
\[ a_{n+1}=\frac{(p-n)}{n+1}a_n.
\]
Que es una relación de recurrencia de un sólo término. Estas relaciones se resuelven iterando la relación de manera sucesiva de modo de relacionar $a_n$ con $a_0$
\[a_n=\frac{(p-n+1)}{n}a_{n-1}=\frac{(p-n+1)(p-n+2)}{n(n-1)}a_{n-2}=\cdots=\frac{(p-n+1)(p-n+2)\cdots p}{n!}a_0.\]
Como $a_0=y(0)=1$ vemos que
\boxedeq{a_n=\frac{(p-n+1)(p-n+2)\cdots p}{n!}.}{eq:coef_bin}
Si $p\in\mathbb{N}$ entonces $a_n=0$ para $n>p$. Esto es claro, por otro lado, ya que en este caso $(1+x)^p$ es un polinomio. Por la fórmula del binomio de Newton los coeficientes para $p\in \mathbb{N}$  no son más que los coeficientes binomiales
\[a_n=\binom{p}{n}\]
 Cuando $p\in\mathbb{R}$ aún vamos a seguir denominado a $a_n$, dado por la fórmula \eqref{eq:coef_bin},   coeficiente binomial. La serie resultante se llama la serie binomial. Cuando $p\in\mathbb{R}-\mathbb{N}$ es una serie infinita y no  un polinomio. Notar que para $p$ no entero positivo
\[\lim\limits_{n\to\infty}\frac{|a_{n+1}|}{|a_n|}=\lim\limits_{n\to\infty}\frac{|p-n|}{|n+1|}=1\]
Luego la serie tiene radio de convergencia 1.  Hemos demostrado asi que vale la siguiente fórmula, que es una generalización de la fórmula binomial de Newton
\boxedeq{(1+x)^p=1+px+\frac{p(p-1)}{2!}x^2  +\cdots=1+\binom{p}{1}x+\binom{p}{2}x^2+\cdots}{}
Esta importante serie se denomina \href{http://en.wikipedia.org/wiki/Binomial_series}{serie binomial}.


\end{ejemplo}

\subsection{Oscilador armónico}

\begin{ejemplo}{} Consideremos la ecuación
\[y''+\omega^2y=0.\]
Esta es una ecuación de segundo orden. Veamos si el método de coeficientes indeterminados nos lleva a la solución. Se tiene
\[\begin{split}
    \omega^2y&=\omega^2a_0+\omega^2a_1x+\omega^2a_2x^2+\cdots+\omega^2a_nx^n+\cdots\\
  y''&=2a_2+2\cdot 3a_3x+\cdots+(n+1)(n+2)a_{n+2}x^n+\cdots\\
--&---------------------------\\
0=y''+\omega^2y =& (\omega^2a_0+2a_2)+(\omega^2a_1+2\cdot 3a_3)x+\cdots\\&
+(\omega^2a_n+(n+1)(n+2)a_{n+2})x^n+\cdots
  \end{split}
\]
Encontramos la relación de recurrencia de dos términos
\boxedeq{a_{n+2}=-\frac{\omega^2a_n}{(n+1)(n+2)}.}{}
Notar que  en este caso $a_1$ y, obviamente, $a_0$ no se relacionan con ningún coeficiente anterior. Por este motivo es de esperar que podamos elegir de manera arbitraria $a_0$ y $a_1$. Esto está de acuerdo con el hecho que remarcamos antes de que en una relación de recurrencia de dos términos aparecen dos constantes arbitrarias y también está de acuerdo con que la solución general de una ecuación de segundo orden tiene dos constantes arbitrarias. En este caso resolvemos la relación de recurrencia relacionando $a_n$ con $a_0$ cuando $n$ es impar y con $a_1$ cuando es impar. Concretamente si $n=2k$, $k\in\mathbb{N}$,
\[a_{2k}=-\frac{\omega^2}{2k(2k-2)}a_{2k-2}=\cdots=(-1)^k\frac{\omega^{2k}}{(2k)!}a_0.\]
En cambio si $n=2k+1$ es impar
\[a_{2k+1}=-\frac{\omega^2}{(2k+1)2k}a_{2k-1}=\cdots=(-1)^k\frac{\omega^{2k}}{(2k+1)!}a_1.\]
Agrupando los términos pares y los impares de la serie de potencias resultante queda
\boxedeq{
   \begin{split}
     y(x) &=a_0\sum_{k=0}^{\infty}(-1)^k\frac{1}{(2k)!}\left(\omega x\right)^{2k}+\frac{a_1}{\omega}\sum_{k=0}^{\infty}(-1)^k\frac{1}{(2k+1)!}\left(\omega x\right)^{2k+1}\\
&=a_0\cos\omega x+\frac{a_1}{\omega}\sen\omega x
   \end{split}
 }{}
La última igualdad es conocida de las asignaturas de análisis. Es facil verificar, lo dejamos de ejercicio, que las series involucradas tienen radio de convergencia infinito.



\end{ejemplo}

\subsection{Ecuación de Legendre. Primera aproximación}

\begin{ejemplo}{} Consideremos la \href{http://es.wikipedia.org/wiki/Polinomios_de_Legendre}{ecuación de Legendre}.
\boxedeq{(1-x^2)y''-2xy'+p(p+1)y=0,}{eq:ecua_lege}
donde $p>0$. Esta ecuación aparece en muchas aplicaciones, por ejemplo en muchos problemas que involucran funciones definidas en esferas, como es el caso de los modos normales de vibración de una esfera y en problemas de potenciales esféricos

\[\begin{split}
   p(p+1)y= p(p+1)&a_0+ p(p+1)a_1x+ p(p+1)a_2x^2+\cdots+ p(p+1)a_nx^n+\cdots\\
  -2xy'=&-2a_1x-4a_2x^2-6a_3x^3+\cdots-2na_{n}x^n+\cdots\\
(1-x^2)y''=& 2a_2+2\cdot 3a_3x+\cdots +(n+1)(n+2)a_{n+2}x^n+\cdots\\
          &-2a_2x^2-2\cdot 3a_3x^3-\cdots -(n-1)na_{n}x^n-\cdots\\
     -------&----------------------\\
0=(1-x^2)y''-2xy+p(p+1)y =& (p(p+1)a_0+2a_2)+(p(p+1)a_1-2a_12\cdot 3a_3)x+\cdots\\
     &+ \left( (p(p+1)-n(n+1)\right)a_n+n(n+1)a_{n+2})x^n+\cdots
  \end{split}
\]
Obtenemos la relación de recurrencia
\boxedeq{a_{n+2}=-\frac{(p-n)(p+n+1)}{(n+1)(n+2)} a_n}{eq:leg_rel_recu}
 Ahora vamos a dividir la serie en los términos pares e impares
\[\sum\limits_{n=0}^{\infty}a_nx^n= \sum\limits_{k=0}^{\infty}a_{2k}x^{2k}+\sum\limits_{k=0}^{\infty}a_{2k+1}x^{2k+1}\]
A cada una de estas series le podemos aplicar el criterio de la razón usando la fórmula de recuerrencia de arriba. Por ejemplo para los términos pares
\[\lim\limits_{k\to\infty}\frac{|a_{2k+2}x^{2k+2}|}{|a_{2k}x^{2k}|}=\lim\limits_{k\to\infty}\frac{|p_0-2k||p_0+2k+1|}{(2k+1)(2k+2)}|x|^2=|x|^2\]
De modo que la serie tiene radio de convergencia 1. La misma situación ocurre con la serie de términos impares. Esto muestra que la serie en su conjunto tambien tiene radio de convergencia igual a 1. Era previsible que el radio de convergencia no fuese mayor a 1, pues la forma explícita de la ecuación de la ecuación de Legendre es
\[y''-\frac{2x}{(1-x^2)}y'+\frac{p(p+1)}{(1-x^2)}y=0.\]
Se observa que $1$ y $-1$ son puntos singulares de la ecuación.

Podemos relacionar cualquier coeficiente de índice par $a_{2k}$ con el $a_0$ y cualquiera con índice impar $a_{2k+1}$ con el $a_1$. 

\[a_{2n}=\frac{(p+1)(p_0+3)\cdots (p+2n-1) \times p(p-2)\cdots (p-2n+2)}{(2n)!}a_0\]
y
\[a_{2n+1}=\frac{(p+2)(p+4)\cdots (p+2n) \times (p-1)(p-3)\cdots (p-2n+1)}{(2n+1)!}a_1\]
Podemos elegir $a_0$ y $a_1$ de manera arbitraria (esto está de acuerdo con que en una ecuación de orden 2 aparece 2 constantes de integración) y por las relaciones anteriores deducir el valor de los restantes $a_n$.

Un caso especial se plantea cuando  $p\in\mathbb{N}$. En esa situación vemos que infinitos de los $a_n$ resultan iguales a cero. De hecho una de las series, la de términos impares o la de términos pares acorde a que $p$ sea par o impar respectivamente,  se trunca. Supongamos que esto ocurre con la serie de términos impares, es decir $p$ es entero positivo impar. Si ahora tomamos $a_0=0$ toda la serie de términos pares se hará cero. Para $a_1$ elijo algún valor no nulo y esto me garantiza que los términos impares son no nulos hasta el término $a_{n}$, pero a partir de allí también se hacen cero. Es usual elegir $a_1$ para que $y(1)=1$. Nos queda definida así una función polinómica que se denomina \href{http://es.wikipedia.org/wiki/Polinomios_de_Legendre}{polinomio de Legendre} y se denota por $P_p$. Cuando $p$ es par hacemos una construcción análoga, quedando un polinomio $P_p$, con todas potencias pares, tal que $P_p(1)=1$.

Los primeros polinomios de Legendre son:

\begin{sympycode}
def Legendre(n):
    orden=n+2
    a=symbols('a0:%s' %orden)
    x=symbols('x')
    y=sum([a[i]*x**i for i in range(orden)])
    Ecua=(1-x**2)*y.diff(x,2)-2*x*y.diff(x)+n*(n+1)*y
    Ecuaciones=[Ecua.diff(x,i).subs(x,0)/factorial(i) for i in range(orden-2)]
    s=symbols('s')
    if n%2==0:
        Ecuaciones+=[a[0]-s,a[1]]
    else:
        Ecuaciones+=[a[0],a[1]-s]
    Sol_a_n=solve(Ecuaciones,a)
    y=y.subs(Sol_a_n)
    sol=solve(y.subs(x,1)-1,s)
    return y.subs(s,sol[0])

print('\\begin{align*}')
for n in range(1,10):
    print(r'P_'+str(n)+'(x)&='+latex(Legendre(n)) + r'\\')
print('\\end{align*}')
\end{sympycode}

Sus gráficas sobre el intervalo $[-1,]$ se muestran en la figura \ref{fig:legendre}.

\begin{figure}[h]
\begin{center}
\includegraphics[scale=.5]{imagenes/legendre.png}
\caption{Polinomios de Legendre hasta el orden 8}\label{fig:legendre}
\end{center}
\end{figure}



\end{ejemplo}

\section{Teorema fundamental sobre puntos ordinarios}


\begin{definicion}{} Dada la ecuación diferencial
\[y''(x)+p(x)y'(x)+q(x)y(x)=0\]
donde $p,q$ son funciones   definidas en algún intervalo abierto $I$, diremos que $x_0\in I$ es un \emph{punto ordinario} de la ecuación si $p$ y $q$ son analíticas en $x_0$. Un punto no ordinario se llama \emph{singular}.
\end{definicion}

\begin{ejemplo}{} En la ecuación del oscilador armónico
\[y''+\omega^2 y=0 \]
todo punto es ordinario.
\end{ejemplo}

\begin{ejemplo}{} En la ecuación de Legendre
\[(1-x^2)y''-2xy'+p(p+1)y=0 \]
$1$ y $-1$ son puntos singulares, otros valores de $x$ son puntos ordinarios.
\end{ejemplo}

Antes de ir al Teorema más importante de esta sección vamos a enunciar un  lema que nos resultará útil.

\begin{lema}[Comparación]{lema:comp_recu}  Supongamos que tenemos una relación de recurrencia
 \begin{equation}\label{eq:recu2} a_{n}=f_n(a_0,\ldots,a_{n-1})\quad n\geq 2\end{equation}
donde las funciones $f_n$ son crecientes respecto a sus todas las variables. Si la sucesión  $\{a_n\}$ resuelve \eqref{eq:recu2}, la sucesión  $\{b_n\}$ resuelve la desigualdad
 \[b_{n}\leq f_n(b_0,\ldots,b_{n-1})\quad n\geq 2 \]
y además vale que  $b_0\leq a_0$ y $b_1\leq a_1$, entonces $b_n\leq a_n$ para todo $n=2,3,\ldots$. En particular la afirmación se satisface cuando $f_n$ es lineal con coeficientes positivos, es decir $f_n(a_0,\ldots,a_{n-1})=\sum_{k=0}^{n-1}\alpha^n_ka_k$, con $\alpha^n_k\geq 0$ para $k=0,\ldots,n-1$.
\end{lema}
\begin{demo}
Es muy sencilla y la dejamos de ejercicio (evidentemenete hay que utilizar el principio de inducción).
\end{demo}


\begin{teorema}[Teorema Fundamental Sobre Puntos Ordinarios.]{eq:teor_ptos_ord} Sea $x_0$ un punto ordinario de la ecuación
\[y''(x)+p(x)y'(x)+q(x)y(x)=0\]
y sean $a_0,a_1\in\mathbb{R}$. Existe una solución de la ecuación que es analítica en un entorno de $x_0$ y que satisface $y(x_0)=a_0$ e $y'(x_0)=a_1$. El radio de convergencia del desarrollo en serie de $y$ es al menos tan grande como el mínimo de los radios de convergencia de los desarrollos en serie de $p$ y $q$.
\end{teorema}

\begin{demo}
Supongamos, sin perder generalidad, que $x_0=0$. Consideremos que los desarrollos en serie de potencias de $p$ y $q$.
\begin{equation}\label{eq:series_p_q}p(x)=\sum_{n=0}^{\infty}p_nx^n\quad\text{y}\quad q(x)=\sum_{n=0}^{\infty}q_nx^n.
\end{equation}
Supongamos que ambas series convergen en $|x|<R$, para cierto $R>0$. Vamos a aplicar el método de coeficientes indeterminados. Tenemos
\[
    \begin{split}
      y&=\sum_{n=0}^{\infty}a_nx^n=a_0+a_1x+\cdots+a_nx^n+\cdots\\
      y'&=\sum_{n=0}^{\infty}(n+1)a_{n+1}x^n=a_1+2a_2x+\cdots+(n+1)a_{n+1}x^n+\cdots\\
      y''&=\sum_{n=0}^{\infty}(n+1)(n+2)a_{n+2}x^n= 2a_2+2\cdot 3a_3x+\cdots+(n+1)(n+2)a_{n+2}x^n+\cdots.
    \end{split}
\]
Por el  Teorema \ref{teor_oper_series}
\[
   \begin{split}
     q(x)y&=\left(\sum_{n=0}^{\infty}a_nx^n\right)\left(\sum_{n=0}^{\infty}q_nx^n\right)=\sum_{n=0}^{\infty}\left(\sum_{k=0}^na_kq_{n-k}\right)x^n,\\
     p(x)y'&=\left(\sum_{n=0}^{\infty}(n+1)a_{n+1}x^n\right)\left(\sum_{n=0}^{\infty}p_nx^n\right)=\sum_{n=0}^{\infty}\left(\sum_{k=0}^n(k+1)a_{k+1}p_{n-k}\right)x^n.
   \end{split}
\]
Sustituyendo estos, y los anteriores, desarrollos en la ecuación, obtenemos
\[\sum_{n=0}^{\infty}\left\{ (n+1)(n+2)a_{n+2}+ \sum_{k=0}^na_kq_{n-k}+ \sum_{k=0}^n(k+1)a_{k+1}p_{n-k} \right\}x^n.\]
De esta forma deducimos la ecuación de recurrencia que se satisface en una ecuación lineal general de segundo orden en un punto ordinario.
\boxedeq{a_{n+2}=-\frac{ \sum_{k=0}^n\left\{ a_kq_{n-k}+ (k+1)a_{k+1}p_{n-k}\right\}}{n(n+1)}}{eq:rel_recu_gral}
Dados los coeficientes $a_0$ y $a_1$ la relación de recurrencia determina los $a_n$, $n\geq 2$. Queda ver que la serie así definida tiene radio de convergencia al menos $R$.

Tomemos $r$ tal que $0<r<R$. Como las series \eqref{eq:series_p_q} tienen radio de convergencia al menos  $R$,  convergen absolutamente para $x=r$. El criterio de convergencia de series conocido como criterio del resto implica que $p_nr^n,q_nr^n$ son suceciones que tienden a $0$. En particular estan acotadas, y por ello existe $M>0$ tal que
\[ |p_n|r^n,|q_n|r^n\leq M.\]
Tomando módulo en \eqref{eq:rel_recu_gral} y usando las desigualdades de arriba concluímos
\[
  \begin{split}
    |a_{n+2}|&\leq\frac{M}{(n+1)(n+2)r^n} \sum_{k=0}^n\left(|a_k|+ (k+1)|a_{k+1}|\right)r^k\\
&\leq\frac{M}{(n+1)(n+2)r^n} \sum_{k=0}^n\left(|a_k|+ (k+1)|a_{k+1}|\right)r^k+\frac{M|a_{n+1}|r}{(n+1)(n+2)}.
  \end{split}
  \]
En la última desigualdad se agregó un término en apariencia por capricho, pero este término nos servirá para complementar una expresión en el futuro. Definamos la sucesión $b_n$ como la solución de la siguiente relación de recurrencia
\begin{equation}\label{eq:bn_recu}b_{n+2}=\frac{M}{(n+1)(n+2)r^n} \sum_{k=0}^n\left(b_k+ (k+1)b_{k+1}\right)r^k+\frac{Mb_{n+1}r}{(n+1)(n+2)},
\end{equation}
con las condiciones iniciales $b_0=|a_0|$ y $b_1=|a_1|$. Por el Lema \ref{lema:comp_recu} tenemos que $|a_n|\leq b_n$ para todo $n$. Aplicando \eqref{eq:bn_recu} a $n-1$ en lugar de $n$
\begin{equation}\label{eq:bn_recu_2}b_{n+1}=\frac{M}{n(n+1)r^{n-1}} \sum_{k=0}^{n-1}\left(b_k+ (k+1)b_{k+1}\right)r^k+\frac{Mb_{n}r}{n(n+1)},
\end{equation}
Multiplicando  \eqref{eq:bn_recu} por $r$ y usando \ref{eq:bn_recu_2}
\[
\begin{split}
rb_{n+2}=&\frac{M}{(n+1)(n+2)r^{n-1}} \sum_{k=0}^{n-1}\left(b_k+ (k+1)b_{k+1}\right)r^k\\
&+\frac{Mr\left(b_n+ (n+1)b_{n+1}\right)}{(n+1)(n+2)}+\frac{Mb_{n+1}r^2}{(n+1)(n+2)}\\
=&\frac{n(n+1)b_{n+1}-Mb_nr}{(n+1)(n+2)}
+\frac{Mr\left(b_n+ (n+1)b_{n+1}\right)}{(n+1)(n+2)}\\
&+\frac{Mb_{n+1}r^2}{(n+1)(n+2)}\\
=&\frac{(n(n+1)+Mr(n+1)+Mr^2)}{(n+1)(n+2)}b_{n+1},
\end{split}
\]
Entonces
\[\lim_{n\to\infty}\frac{|b_{n+2}x^{n+2}|}{|b_{n+1}x^{n+1}|}=\lim_{n\to\infty}\frac{(n(n+1)+Mr(n+1)+Mr^2)}{(n+1)(n+2)}\frac{|x|}{r}=\frac{|x|}{r}.
\]
Luego la serie $\sum_{n=0}^{\infty}b_nx^n$ converge para $|x|<r$. Como $|a_n|\leq b_n$ la misma afirmación es cierta para $\sum_{n=0}^{\infty}a_nx^n$. Como $r<R$ fue elegido arbitrariamente, tenemos que $\sum_{n=0}^{\infty}a_nx^n$ converge en $|x|<R$.
\end{demo}


\section{Puntos singulares, método de Frobenius}

\subsection{Series de Frobenius}
\begin{definicion}[Sigularidades, Polos.]{} Sea $f$ definida en un intervalo abierto $I$ con valores en $\rr$. Diremos que $f$ posee un \emph{polo de orden $k$} en $x_0\in\rr$, si la función $(x-x_0)^kf(x)$ es analítica en un entorno de $x_0$. Vale decir que $(x-x_0)^kf(x)$ se desarrolla en serie de potencias.
\[(x-x_0)^kf(x)=\sum_{n=0}^{\infty}a_n(x-x_0)^n.\]
En consecuencia
\[f(x)=\sum_{n=0}^{\infty}a_n(x-x_0)^{n-k}=\frac{a_0}{(x-x_0)^k}+\cdots+\frac{a_{k-1}}{(x-x_0)}+a_k+a_{k+1}(x-x_0)+\cdots.\]
Este tipo de desarrollo en serie es un caso particular de serie de Laurent.

Cuando el orden de un polo es $1$ se lo denomina \emph{polo simple}.
\end{definicion}



\begin{definicion}{} Un punto singular $x_0$ de la ecuación
\[y''(x)+p(x)y'(x)+q(x)y(x)=0\]
se llama singular regular si $p(x)$ tiene un polo a lo sumo simple en $x_0$ y $q(x)$ tiene un polo a lo sumo de orden $2$ en $x_0$. Es decir
\[(x-x_0)p(x)\quad\hbox{y}\quad (x-x_0)^2q(x)\]
son analíticas en $x_0$.
\end{definicion}
Algunas de las ecuaciones más importantes de la Física-Matemática tienen puntos singulares regulares.

\begin{ejemplo}{} 1 y -1 son puntos singulares regulares de la ecuación de Legendre de orden $p$
\[y''-\frac{2x}{1-x^2}y'+\frac{p(p+1)}{1-x^2}y=0\]
\end{ejemplo}


\begin{ejemplo}{} 0 es un punto singular regular de la ecuación de Bessel de orden $p$
\[y''+\frac{1}{x}y'+\left(1-\frac{p^2}{x^2}\right)y=0\]
\end{ejemplo}

El método de coeficientes indeterminados puede fallar en los puntos donde $p$ y $q$ tienen polos. En su lugar vamos a proponer otro tipo de desarrollo en serie. Lo vamos a motivar con un ejemplo.

\begin{ejemplo}{} Consideremos la ecuación de Euler, para $p,q\in\rr$
\[y''+\frac{p}{x}y'+\frac{q}{x^2}y=0\]
o equivalentemente
\[x^2y''+pxy'+qy=0\]
Aquí es facil verificar que las funciones
\[P(x):=\frac{p}{x}\quad\hbox{ y }\quad Q(x):= \frac{q}{x^2}\]
satisfacen que
\[\frac{Q'+2PQ}{Q^{\frac{3}{2}}}\quad\text{es constante.}\]
 Cuando se daba esta condición, el ejercicio 6 de la página 102 del libro de Simmons nos enseña que podemos reducir la ecuación a una ecuación con coeficientes constantes por medio del cambio de la variable independiente
\[z=\int\sqrt{Q}dx\]
En este caso, obviando las constantes, el cambio de variables que debemos hacer es
\[z=\ln(x)\]
Aquí  asumimos $x>0$. Seguramente, a esta altura del curso,  el alumno ya hizo los cálculos que muestran que la ecuación de Euler se transforma,  por medio del cambio de variables propuesto, en la ecuación a coeficientes constantes
\[y''+(p-1)y'+qy=0.\]
Cuya ecuación característica es
\[\lambda^2+(p-1)\lambda+q=0\]
Resolviendo esta ecuación 
\[s_1= -\frac{p-1}{2} - \frac{\sqrt{p^2 - 2p - 4q + 1}}{2}   \quad\text{y}\quad s_2=-\frac{p-1}{2} +\frac{\sqrt{p^2 - 2p - 4q + 1}}{2} .\]
 Si $s_1\neq s_2$ dos soluciones linealmente independientes son:
\[y_1(z)=e^{s_1z}\quad\hbox{y}\quad y_2(z)=e^{s_2z}\]
 Si $s_1=s_2$
\[y_1(z)=e^{s_1z} \quad\hbox{y}\quad y_2(z)=ze^{s_1z}\]
son soluciones linealmente independientes. Asumamos que las raices $s_1$ y $s_2$ son reales, entonces como $z=\ln(x)$, las soluciones en términos de la variable $x$ son
\[y_1(x)=x^{s_1}\quad\hbox{y}\quad y_2(x)=x^{s_2}\quad\text{para }  s_1\neq s_2\]
y
\[y_1(x)=x^{s_1} \quad\hbox{y}\quad y_2(x)=\ln(x)x^{s_1}\quad\text{para }  s_1= s_2\]



Estas funciones, a menos que $s_1$ y $s_2$ sean enteros positivos, ya no son analíticas en cero, pues una función analítica es derivable infinitas veces y claramente hay derivadas (o las mismas funciones si las raices son negativas) de $y_1$ e $y_2$ que son discontinuas.
\end{ejemplo}


De modo que, como era de suponer, no podremos encontrar en general en un punto singular una solución analítica.  Vamos a intentar flexibilizar nuestro método para incluir otro tipo de desarrollo en serie, que está inspirado en los resultados obtenidos para la ecuación de Euler.

\begin{definicion}{} A una expresión de la forma
 \[y(x)=(x-x_0)^m(a_0+a_1(x-x_0)+a_2(x-x_0)^2+\cdots),\]
donde $m\in\rr$ y $a_0\neq 0$, lo llamaremos \emph{Serie de Frobenius}.
\end{definicion}
Las series de Frobenius no son series de potencias ni de Laurent ya que en ellas aparecen potencias no enteras.

El método de Frobenius consiste en proponer como solución de una ecuación diferencial una serie de Frobenius. Este método tiene éxito, por ejemplo, en los puntos sigulares regulares de ecuaciones diferenciales lineales de segundo orden.

\begin{ejemplo}{} En este ejemplo ilustramos el método de Frobenius. Consideremos la ecuación
\begin{equation}\label{eq:ejem_sim}y''+\left(\frac{1}{2x}+1\right)y'-\left(\frac{1}{2x^2}\right)y=0.
\end{equation}
Notar que $x=0$ es regular singular. Desarrollaremos el método de Frobenius, primero ``a mano'' y por último con SymPy.

Proponemos como solución

 \[y(x)=x^m\sum_{n=0}^{\infty}a_{n}x^n=\sum_{n=0}^{\infty}a_{n}x^{n+m}\]
y calculamos
\[
    \begin{split}
     -\frac{1}{2x^2}y&=-\sum_{n=0}^{\infty}\frac{a_n}{2}x^{m+n-2}=-\frac{a_0}{2}x^{m-2}
-\frac{a_1}{2}x^{m-1}-\cdots-\frac{a_{n+2}}{2}x^{m+n}-\cdots\\
      y'&=\sum_{n=0}^{\infty}(m+n)a_{n}x^{m+n-1}\\
      &=ma_0x^{m-1}+(m+1)a_1x^m+\cdots+(m+n+1)a_{n+1}x^{m+n}+\cdots\\
      \frac{1}{2x}y'&=\sum_{n=0}^{\infty}\frac{(m+n)a_{n}}{2}x^{m+n-2}\\
      &=\frac{ma_0}{2}x^{m-2}+\frac{(m+1)a_1}{2}x^{m-1}+\cdots+\frac{(m+n+2)a_{n+2}}{2}x^{m+n}+\cdots\\
      y''&=\sum_{n=0}^{\infty}(m+n)(m+n-1)a_{n}x^{m+n-2}\\
&= m(m-1)a_0x^{m-2}+(m+1)ma_1x^{m-1}+\cdots\\
&+(m+n+2)(m+n+1)a_{n+2}x^{m+n}+\cdots.
    \end{split}
\]
Sumando las cuatro igualdades miembro a miembro y sustiyendo en \eqref{eq:ejem_sim}


\begin{multline*}
    0= \left(-\frac{1}{2} +\frac{m}{2}+m(m-1)\right)a_0x^{m-2}\\
    + \left(-\frac{a_1}{2}+ma_0+\frac{(m+1)a_1}{2}+ (m+1)ma_1 \right)x^{m-1}+\cdots \\
  +\left( -\frac{a_{n+2}}{2}+(m+n+1)a_{n+1}+ \right.\\
  \left.  \frac{(m+n+2)a_{n+2}}{2}+(m+n+2)(m+n+1)a_{n+2} \right)x^{m+n}+\cdots\\
 =\frac12\left(2m+1\right)(m-1) a_0x^{m-2}+ \\\frac{1}{2} \, {\left( 2 \, a_{0} + (1+2(m+1))\, a_{1}\right)}m x^{m-1}+\cdots\\
 +\frac12\left(   2a_{n+1}+ (2(m+n+2)+1)a_{n+2}  \right)(m+n+1)x^{m+n}+\cdots.
\end{multline*}

Igualando a cero los coeficientes de cada exponente obtenemos las ecuaciones

\begin{equation}\label{eq:ecua_ejem_frob}
    \begin{split}
      \frac12\left(2m+1\right)(m-1) a_0&=0\\
       \frac{1}{2}  \left(   2 a_{0} + (1+2(m+1)) a_{1}\right)m&=0\\
                                      &\vdots\\
     \left(   2a_{n+1}+ (2(m+n+2)+1)a_{n+2}  \right)(m+n+1) &=0\\
    \end{split}
\end{equation}
Despejando de la última ecuación nos queda la relación de recurrencia de un término
\boxedeq{a_{n+2}=-\frac{2a_{n+1}}{2(m+n+2)+1}.}{eq:recu_ecua_ejem_frob}
Para llegar a esta igualdad debimos suponer $2(m+n+2)+1\neq 0$.
La primera de las ecuaciones en \eqref{eq:ecua_ejem_frob} es importante puesto que determina el valor de $m$. Se llama \emph{ecuación indicial}\index{ecuación indicial}. Vemos que si tomamos $m=1$ o $m=-1/2$ se resuelve la primera ecuacion. Supongamos $m=1$. Entonces \eqref{eq:recu_ecua_ejem_frob} se transforma en
\[
a_{n+2}=-\frac{2a_{n+1}}{2n+7},\quad n=-1,0,\ldots.
\]
Con ayuda de la relación de recurrencia podemos determinar el radio de convergencia de la serie sin necesidad de conocer el valor de los $a_n$

\[\lim_{n\to\infty}\frac{|a_{n+1}x^{n+1}|}{|a_{n}x^{n}|}=
\lim_{n\to\infty}\frac{2|x|}{2n+7}=0.\]
Por consiguiente el radio de convergencia es infinito.  Iterando la relación de recurrencia llegamos
\[a_{n}=-\frac{2}{2n+3}a_{n-1}=\frac{2}{(2n+3)(2n+1)}a_{n-2}=\cdots=
\frac{(-1)^n2^{n}}{(2n+3)(2n+1)\cdots 5}a_0.\]
El valor de $a_0$ determina al resto de los $a_n$ y se puede elegir arbitrariamente. Supongamos que $a_0=1$. Hemos hallado la siguiente solución
\boxedeq{y_1(x)=x\sum_{n=0}^{\infty}(-1)^n\frac{2^{n}}{(2n+3)(2n+1)\cdots 5}x^n.}{}
Cuando $m=-\frac12$, la relación de recurrencia es
\[a_{n+1}=-\frac{a_{n}}{n+1}.\]
Por ende
\[a_n=-\frac{a_{n-1}}{n}=\frac{1}{n(n-1)}a_{n-2}=\cdots=\frac{(-1)^n}{n!}a_{0}=\frac{(-1)^n}{n!}.\]
Conseguimos la solución
\boxedeq{y_2(x)=x^{-\frac12}\sum_{n=0}^{\infty}\frac{(-1)^n}{n!}x^n=\frac{e^{-x}}{\sqrt{x}}}{}
Estas dos soluciones son linealmente independientes. Para justificar esta afirmación basta ver que $y_1(0)=0$ y $\lim_{x\to 0+}y_2(x)=+\infty$. Estas igualdades hacen imposible la relación $c_1y_1+c_2y_2=0$ a menos que $c_1=c_2=0$.


\end{ejemplo}

\subsection{Ecuación de Bessel, funciones de Bessel de primera especie}\label{sec:bessel_1}

\subsubsection{Relaciones de recurrencia y solución por el método de Frobenius}

\begin{definicion}{} Recordemos a la ecuación de Bessel\index{Ecuación!Bessel} de orden $p$ ($p>0$)
 \begin{equation}\label{eq:bessel3}x^{2} y^{\prime \prime}+x y^{\prime}+\left(x^{2}-p^{2}\right) y=0\end{equation}
\end{definicion}

En $x=0$ la ecuación de Bessel tiene un punto  singular regular. Vamos a aplicarle el método de Frobenius. Suponemos que 
$$
y=x^{m} \sum_{n=0}^\infty a_{n} x^{n}=\sum_{n=0}^\infty a_{n} x^{n+m}
$$
es solución, donde $a_{0} \neq 0$ y la serie de potencias converge para  $x$. Calculado
$$
\begin{aligned} 
y^{\prime}&=\sum_{n=0}^\infty(n+m) a_{n} x^{n+m-1},\\
y^{\prime \prime}&=\sum_{n=0}^\infty(n+m-1)(n+m) a_{n} x^{n+m-2},\\
x^{2} y^{\prime \prime} &=\sum_{n=0}^\infty(n+m-1)(n+m) a_{n} x^{n+m} \\
x y^{\prime} &=\sum_{n=0}^\infty(n+m) a_{n} x^{n+m}\\
x^{2} y &=\sum_{n=2}^\infty a_{n-2} x^{n+m} \\
-p^{2} y &=\sum_{n=0}^\infty-p^{2} a_{n} x^{n+m}\\
\end{aligned}
$$
Sumando esas series e igualando a cero el coeficiente de $x^{n+m}$ se obtiene
\begin{equation}\label{eq:recu_bessel_gen}
\begin{aligned} 
    (m^2-p^2)a_0&=0\\
    ((m+1)^2-p^2)a_1&=0 \\
    ((n+m)^2-p^2) a_{n}+a_{n-2}&=0\quad n\geq 2.
\end{aligned}
\end{equation}
La primera es la ecuación indicial para la ecuación de Bessel y como $a_0\neq 0$ entonces se infiere que 

$$ m=p  \vee m=-p .$$

Consideremos el caso que $m=p$. Se tiene que $(2p+1)a_1=0$. Podría ocurrir que $p=-1/2$ o también que $a_1=0$. El caso $p=\pm1/2$ lo vamos a trabajar en un ejercicio. Supongamos pues  $a_1=0$. La ecuación en recurrencia se transforma, luego de sustituir $m$ por $p$

$$
a_{n}=-\frac{a_{n-2}}{n(2 p+n)}
$$
Ahora, como $a_{1}=0$,  aplicando repetidas veces la recurrencia se deduce que $a_{n}=0$ para todo $n$ impar. Reemplazamos $n\leftrightarrow 2n$ en la relación de recurrencia 

\boxedeq{a_{2n}=-\frac{1}{4n(p+n)}a_{2n-2}}{eq:recu_bessel}
Iterando esta relación
\[
\begin{split}
  a_{2n}&=-\frac{1}{4n(p+n)}a_{2n-2}\\
       &=\frac{1}{4n(p+n-1)}\cdot\frac{1}{4(n-1)(p+n-1)}a_{2n-4}=\cdots\\
       & =(-1)^n\frac{1}{4^nn!(p+n)(p+n-1)\cdots (p+1)}a_{0}.
\end{split}
\]
Obtenemos la solución
\boxedeq{y(x)=x^p\sum_{n=0}^{\infty}\frac{(-1)^na_0}{4^nn!(p+n)(p+n-1)\cdots (p+1)}x^{2n}}{eq:sol_bessel_1}

En la próxima sección veremos que cierta elección especial de $a_0$ no lleva a lo que denominaremos funciones de Bessel.
 

\subsubsection{Función Gamma y la función de Bessel de primera especie}

Vamos a elegir un  valor de $a_0$ apropiado de modo que la expresión de la solución \eqref{eq:sol_bessel_1} sea más simple. Si $p$ fuese un entero positivo, eligiendo $a_0$ como $1/2^pp!$ la expresión quedaría
\[y(x)=\sum_{n=0}^{\infty}\frac{(-1)^n}{n!(p+n)!}\left(\frac{x}{2}\right)^{2n+p}\]
Con el objeto de considerar  $p$  no entero, vamos a generalizar la función facorial a números reales. 

\begin{definicion}{def:gamma} Para $p>0$ definimos la \emph{función Gamma} por
\begin{equation}\label{eq:gamma}\Gamma(p):=\int_0^{\infty}t^{p-1}e^{-t}dt
\end{equation}
\end{definicion}

En la clase demostraremos que la integral impropia en la definición converge. Además demostraremos las siguientes relaciones
\boxedeq{\begin{split}\Gamma(1)&=1\\
 \Gamma(p+1)&=p\Gamma(p)\end{split}}{eq:recu_gamma}
Estas igualdades  implican, cuando $p=n$ es entero positivo, que $\Gamma(n+1)=n\Gamma(n)=\cdots=n!\Gamma(1)=n!$. Luego podemos ver a la función gamma como una extensión de la función factorial a los reales no negativos. Notar que
\[\lim_{p\to 0+}\Gamma(p)=\lim_{p\to 0+}\frac{\Gamma(p+1)}{p}=+\infty.\]
La función Gamma tiende a  infinito cuando nos acercamos a cero por derecha.

La Definición \ref{eq:gamma} es válida para $p>0$. Utilizando la relación \eqref{eq:recu_gamma} podemos extender la función a $p<0$. Concretamente,  supongamos  que $-1<p<0$, en  este caso definimos
\boxedeq{\Gamma(p):=\frac{\Gamma(p+1)}{p}.}{eq:gama_iter}
Observar que el segundo miembro esta bien definido pues $p+1>0$. Como consecuencia de esta definición tendremos
\[\lim_{p\to 0-}\Gamma(p)=\lim_{p\to 0-}\frac{\Gamma(p+1)}{p}=-\infty.\]
y
\[\lim_{p\to -1+}\Gamma(p)=\lim_{p\to -1+}\frac{\Gamma(p+1)}{p}=-\infty.\]
Ahora podemos extender $\Gamma$ a $p\in (-2,1)$. Pues  podemos usar la fórmula \eqref{eq:gama_iter} y el hecho de que ya tenemos definida la función Gamma en $(-1,0)$. Continuando de esta forma, definimos $\Gamma$ para cualquier valor de $p<0$ y $p\notin \mathbb{Z}$. Si $n$ es un entero negativo ocurre que
\[\lim_{p\to n+}\Gamma(p)=(-1)^n\infty\quad\text{y}\quad \lim_{p\to n-}\Gamma(p)=(-1)^{n-1}\infty.\]




\begin{figure}[h]
\begin{center}
\includegraphics[scale=.4]{imagenes/gamma.png}
\caption{La función gamma $\Gamma$}
\end{center}
\end{figure}
Ahora estamos en condiciones de definir la \emph{función de Bessel de primera especie} \index{Función!Bessel de primera especie} que resulta de tomar $a_0$ como $1/2^p\Gamma(p+1)$.

\begin{definicion}{def:bessel_primera} Definimos la función de Bessel de primera especie como
\begin{equation}\label{eq:func_bessel_1e}
 J_p(x)=\sum_{n=0}^{\infty}\frac{(-1)^n}{n!\Gamma(p+n+1)}\left(\frac{x}{2}\right)^{2n+p}
\end{equation}
 
\end{definicion}


En la figura \ref{fig:bessel} se muestran diversos gráficos de funciones de Bessel obteneidos con sympy (ver apéndice).

\begin{figure}[h]
\begin{center}
\includegraphics[scale=.5]{imagenes/bessel.png}
\end{center}

\caption{Funciones de Bessel $J_p$, $p=1,2,3$}\label{fig:bessel}

\end{figure}





Ahora consideremos la solución $m=-p$ de la ecuación indicial. Notar que la fórmula \eqref{eq:func_bessel_1e} continúa teniendo sentido a menos que $-p+n+1=0$, cosa que puede ocurrir cuando $p\in\mathbb{N}$.  

\begin{definicion}{def:bessel_segunda} Si $p\notin\mathbb{N}$   definimos la función de Bessel de primera especie $J_{-p}$  como
\begin{equation}\label{eq:func_bessel_2e}J_{-p}(x)=\sum_{n=0}^{\infty}\frac{(-1)^n}{n!\Gamma(-p+n+1)}\left(\frac{x}{2}\right)^{2n-p}
\end{equation}
\end{definicion}

Cuando $p\notin \nn\cup\{0\}$, la función $J_{-p}$ es no acotada cuando $x\to 0$ y esto es razón suficiente para asegurar que es linealmente independientes con $J_p$. En la figura \ref{fig:bessel2esp} graficamos algunas de estas funciones de Bessel. La soluci{on general de la ecuaci{on de Bessel para $0<p\notin\mathbb{N}$ es enconces

\[y(x)=c_1J_p(x)+c_2 J_{-p}(x).\]
Es una función acotada si y solo si $c_2=0$. 



\begin{figure}[h]
\begin{center}
\includegraphics[scale=.5]{imagenes/bessel2.png}
\end{center}
\caption{Funciones de Bessel $J_{-1/3}$, $J_{-2/3}$ y $J_{-5/3}$.}\label{fig:bessel2esp}
\end{figure}

El caso $p=0$  continua siendo problemático dado que obviamente $J_p=J_{-p}$ y no generamos una solución linealmente independiente de la que ya teníamos. 




Más adelante vamos a seguir estudiando con más profundidad la ecuación de Bessel.


\begin{observacion} Es importante enfatizar que todas las series anteriores tienen un caracter formal. Cuando decimos esto entendemos que son resultado de la aplicación del método de coeficientes indeterminados, pero todavía no está establecido que estas series representen una solución, ni tampoco siquiera que las series converjan en un disco de radio positivo. Por lo tanto podría darse el caso que estas series no definan una función más allá de $x=0$. Esta problemática la abordamos en la sección siguiente.  
 
\end{observacion}

\begin{ejercicio}{} Proponiendo una solución de la forma $y(x)=)v(x)\sqrt{x}$ econtrar de otra manera una solución de la ecuación de Bessel de orden $p=1/2$. Usar la relación $\Gamma(1/2)=\sqrt{\pi}$ para concluir

\[J_{\frac{1}{2}}=\sqrt{\frac{2}{\pi x}}\sen(x),\quad J_{-\frac{1}{2}}=\sqrt{\frac{2}{\pi x}}\cos(x).\]

\end{ejercicio}


\begin{ejercicio}{} Recordando que $\Gamma$ tiene polos en $-n=0,1,2,\ldots$  escribamos 
\[
 \frac{1}{\Gamma(-n)}=0,\quad \text{cuando } n\in\mathbb{N}.
\]
Con esta suposición la fórmula \eqref{def:bessel_segunda} tendría sentido aún con $p\in\mathbb{N}$. Demostrar que $J_{-n}=(-1)^nJ_n$. En conclusión de esta manera se encuentra una solución linealmente dependiente con $J_n$. 
\end{ejercicio}





\subsection{Teorema fundamental sobre puntos singulares regulares}\label{eq:sec_teor_fund_frob}
Habiendo visto algunos ejemplos, ahora pasaremos a discutir la situación general del método de Frobenius.

Supongamos $x=0$ un punto regular singular de la ecuación
\begin{equation}\label{eq:dif_2_orden} y''+p(x)y'+q(x)y=0.
\end{equation}
Supongamos que $xp(x)$ y $x^2q(x)$ poseen los  siguientes desarrollos en serie
\[xp(x)=\sum_{n=0}^{\infty}p_nx^n\quad\text{y}\quad x^2q(x)=\sum_{n=0}^{\infty}q_nx^n\]
Proponemos como solución
\[y=x^{m}\sum_{n=0}^{\infty}a_nx^n=\sum_{n=0}^{\infty}a_nx^{m+n}.\]
Empecemos por hallar las relaciones de recurrencia para los coeficientes $a_n$, $n=0,1,\ldots$.   Tenemos
\begin{subequations}
    \begin{align}
      y'&=\sum_{n=0}^{\infty}(m+n)a_{n}x^{m+n-1}\\
      y''&=\sum_{n=0}^{\infty}(m+n)(m+n-1)a_{n}x^{m+n-2}\notag\\
&=x^{m-2}\sum_{n=0}^{\infty}(m+n)(m+n-1)a_{n}x^{n}\label{eq:der_seg} .
    \end{align}
  \end{subequations}
Luego

\begin{equation}\label{eq:der_pri}
  \begin{split}
    p(x)y'(x)&=\frac{1}{x}\left(\sum_{n=0}^{\infty}p_nx^n\right)\left(\sum_{n=0}^{\infty}(m+n)a_{n}x^{m+n-1}\right)\\
&=x^{m-2}\left(\sum_{n=0}^{\infty}p_nx^n\right)\left(\sum_{n=0}^{\infty}(m+n)a_{n}x^{n}\right)\\
&= x^{m-2}\sum_{n=0}^{\infty}\left(\sum_{k=0}^np_{n-k}(m+k)a_k\right)x^n\\
&= x^{m-2}\sum_{n=0}^{\infty}\left(\sum_{k=0}^{n-1}p_{n-k}(m+k)a_k+p_0(m+n)a_n\right)x^n
  \end{split}
\end{equation}
Ahora desarrollemos $q(x)y(x)$,
\begin{equation}\label{eq:der_cero}
  \begin{split}
    q(x)y(x)&=\frac{1}{x^2}\left(\sum_{n=0}^{\infty}q_nx^n\right)\left(\sum_{n=0}^{\infty}a_{n}x^{m+n}\right)\\
&=x^{m-2}\left(\sum_{n=0}^{\infty}q_nx^n\right)\left(\sum_{n=0}^{\infty}a_{n}x^{n}\right)\\
&= x^{m-2}\sum_{n=0}^{\infty}\left(\sum_{k=0}^nq_{n-k}a_k\right)x^n\\
&= x^{m-2}\sum_{n=0}^{\infty}\left(\sum_{k=0}^{n-1}q_{n-k}a_k+q_0a_n\right)x^n
  \end{split}
\end{equation}
A partir de \eqref{eq:dif_2_orden}, \eqref{eq:der_cero},\eqref{eq:der_pri} y \eqref{eq:der_seg} obtenemos
\begin{equation}
\begin{split}
  0=&x^{m-2}\sum_{n=0}^{\infty}\bigg\{\left[(m+n)(m+n-1)+p_0(m+n)  +q_0  \right] a_{n}\\&+\sum_{k=0}^{n-1}a_k\left[p_{n-k}(m+k) +
q_{n-k}\right]\bigg\}x^n.
\end{split}
\end{equation}
Entonces se debe satisfacer
\boxedeq{\left[(m+n)(m+n-1)+p_0(m+n)  +q_0  \right] a_{n}+\sum_{k=0}^{n-1}a_k\left[p_{n-k}(m+k) +
q_{n-k}\right]=0,}{eq:recu_gral_frob}
para $n=0,1,\ldots$. Definamos
\boxedeq{f(m)=m(m-1)+p_0m+q_0.}{eq:func_f}
Entonces las ecuaciones \eqref{eq:recu_gral_frob} se escriben
\boxedeq{f(m+n)a_n=-\sum_{k=0}^{n-1}a_k\left[p_{n-k}(m+k) +
q_{n-k}\right],\quad n=0,1,\ldots}{eq:recu_gral_frob_2}
La primera de estas ecuaciones es
\begin{definicion}{} Definimos la  ecuación indicial por
\begin{equation}\label{eq:eq_indicial}
  f(m)=m(m-1)+p_0m+q_0=0.
\end{equation}

\end{definicion}

Si $m$ resuelve la ecuación indicial entonces $m$ resuelve la ecuación \eqref{eq:recu_gral_frob_2} para $n=0$ y el valor de $a_0$ se puede elegir arbitrariamente.

Supongamos que la ecuación indicial tiene las soluciones $m_2\leq m_1$. Podría ocurrir que las soluciones fueran complejos conjugados, pero no trataremos este caso aquí. Ahora discutamos cuando es posible resolver las relaciones de recurrencia \eqref{eq:recu_gral_frob_2}. El único problema que podría ocurrir es que  $f(m+n)=0$ para algún valor de $n=1,2,\ldots$, dado que en ese caso \eqref{eq:recu_gral_frob_2} se reduce a
\begin{equation}\label{eq:ec_enter} 0=-\sum_{k=0}^{n-1}a_k\left[p_{n-k}(m+k) +
q_{n-k}\right].
\end{equation}
Esta ecuación no nos dice nada sobre $a_n$ y no podemos esperar que se satisfaga. Sin embargo, cuando $m=m_1$ esta desafortunada situación no se da, de lo contrario  $m_1+n$ sería una raíz de la ecuación indicial distinta que $m_2$ y $m_1$. En consecuencia, \emph{si $m=m_1$ y $a_0\neq 0$ es elegido arbitrariamente las ecuaciones \eqref{eq:recu_gral_frob_2} determinan los coeficientes $a_n$, $n=1,2,\ldots$.}

Cuando $m=m_2$ si puede ocurrir que $f(m_2+n)=0$, esto es así cuando $m_2+n=m_1$ para algún $n=1,2,\ldots$, vale decir cuando $m_1-m_2\in\mathbb{Z}$.  Luego estamos en condiciones de afirmar que \emph{si $m=m_2$, $m_1-m_2\notin \mathbb{Z}$ y $a_0\neq 0$ es elegido arbitrariamente las ecuaciones \eqref{eq:recu_gral_frob_2} determinan los coeficientes $a_n$, $n=1,2,\ldots$.} En esta situación, se demuestra con facilidad que las dos soluciones obtenidas por el método de Frobenius son linealmente independientes. Por ejemplo, si estas soluciones son
\[y_1=x^{m_1}\sum_{n=0}^{\infty}a_nx^n\quad\text{y}\quad y_2=x^{m_2}\sum_{n=0}^{\infty}b_nx^n,\]
con $a_0\neq 0\neq b_0$ y si suponemos que ellas son linealmente dependientes entonces su cociente $y_1/y_2$ sería una constante $c$  no nula. Pero tomando límite en la expresión $y_1/y_2$ cuando $x\to 0$ vemos que
\[c=\lim_{x\to 0} \frac{y_1}{y_2}=\lim_{x\to 0} x^{m_1-m_2}\frac{\sum_{n=0}^{\infty}a_nx^n}{ \sum_{n=0}^{\infty}b_nx^n  }=0.\frac{a_0}{b_0}=0.\]

Si $m_1-m_2:=n_0\in\mathbb{Z}$, entonces la situación es diferente. Si, en particular,  \emph{tuviesemos $n_0=0$ ( $m_1=m_2$) entonces claramente podremos encontrar esencialemente sólo una solución en serie de Frobenius.}



Hemos discutido con cierto detalle las posibles soluciones de las ecuaciones de recurrencia. No resta considerar la cuestión de la convergencia de las series. Esto lo tratamos en el siguiente teorema.

\begin{teorema}{} Supongamos $x=0$ un punto regular singular de la ecuación
\begin{equation}\label{eq:dif_2_orden} y''+p(x)y'+q(x)y=0.
\end{equation}
Supongamos que $xp(x)$ y $x^2q(x)$ poseen los  siguientes desarrollos en serie
\[xp(x)=\sum_{n=0}^{\infty}p_nx^n\quad\text{y}\quad x^2q(x)=\sum_{n=0}^{\infty}q_nx^n\]
y que estas series convergen para $|x|<R$ ($R>0$). Supongamos que la ecuación indicial tiene la raíces reales $m_1$, $m_2$ con  $m_2\leq m_1$.  Entonces la ecuación \eqref{eq:dif_2_orden}  tiene una solución en serie de Frobenius dada por
\[y_1=x^{m_1}\sum_{n=0}^{\infty}a_nx^n\quad a_0\neq 0.\]
La serie $\sum_{n=0}^{\infty}a_nx^n$ converge en $|x|<R$. Si $m_1-m_2$ no es un entero no negativo entonces tenemos una segunda solución en serie de Frobenius con $m_2$ en lugar de $m_1$ y satisfaciendo las mismas condiciones que la primer serie.
\end{teorema}
\begin{demo} Dado que $m_1$ y $m_2$ son raíces de la ecuación indicial vale la factorización
\[f(m)=(m-m_1)(m-m_2)=m^2-(m_1+m_2)m+m_1m_2.\]
Entonces
\[f(m_1+n)=n(n+m_1-m_2)\quad\text{y}\quad f(m_2+n)=n(n+m_2-m_1).\]
Luego
\begin{equation}\label{eq:desig_aux}|f(m_1+n)|\geq n(n-|m_1-m_2|)\quad\text{y}\quad f(m_2+n)\geq n(n-|m_1-m_2|).
\end{equation}
Tomemos $r$ tal que $0<r<R$. Por las mismas razones expresadas  en el Teorema \eqref{eq:teor_ptos_ord},  existe $M>0$ tal que
\[ |p_n|r^n,|q_n|r^n\leq M.\]
Usando \eqref{eq:recu_gral_frob_2}, con $m=m_1$, \eqref{eq:desig_aux} y las desigualdades de arriba concluímos
\[n(n-|m_1-m_2|)|a_n|\leq M\sum_{k=0}^{n-1}\frac{|a_k|}{r^{n-k}}(|m_1|+k+1).\]
Ahora definimos la sucesión $b_n$ por $b_n=|a_n|$, para $0\leq n\leq|m_1-m_2|$ y para $ n>|m_1-m_2|$
\begin{equation}\label{eq:bn_iter}n(n-|m_1-m_2|)b_n= M\sum_{k=0}^{n-1}\frac{b_k}{r^{n-k}}(|m_1|+k+1).\end{equation}
Por una variación simple  del Lema \ref{lema:comp_recu} tenemos que $|a_n|\leq b_n$, para todo $n=0,1,\ldots$. Ahora veremos que la serie $\sum_{n=0}^{\infty}b_nx^n$ converge para  $|x|<r$. Por \eqref{eq:bn_iter} tenemos que
\[
   \begin{split}
     r(n+1)(n+1-|m_1-m_2|)b_{n+1}&=M\sum_{k=0}^{n}\frac{|b_k|}{r^{n-k}}(|m_1|+k+1)\\
&=M\sum_{k=0}^{n-1}\frac{b_k}{r^{n-k}}(|m_1|+k+1)+Mb_n(|m_1|+n+1)\\
&=n(n-|m_1-m_2|)b_n+Mb_n(|m_1|+n+1).
  \end{split}
\]

Luego
\[\frac{b_{n+1}}{b_n}=\frac{ n(n-|m_1-m_2|)+M(|m_1|+n+1) }{ r(n+1)(n+1-|m_1-m_2|)  }\]
Y así
\[ \begin{split}
\lim_{n\to\infty}\frac{b_{n+1}|x|^{n+1}}{b_n|x|^n}&=|x|\lim_{n\to\infty}\frac{ n(n-|m_1-m_2|)+M(|m_1|+n+1) }{ r(n+1)(n+1-|m_1-m_2|)  }=\frac{|x|}{r}
\end{split}
\]
Luego la serie  $\sum_{n=0}^{\infty}b_nx^n$   converge para $|x|<r$. Como $|a_n|\leq b_n$ tendremos que  $\sum_{n=0}^{\infty}a_nx^n$ converge absolutamente para $|x|<r$. Finalmente como $0<r<R$ fue elegido arbitrariamente tendremos que la convergencia se da para $|x|<R$.

Restaría considerar el caso $m=m_2$ cuando $m_1-m_2$ no es entero. Pero la demostración del mismo sigue el mismo camino que para $m=m_1$.

\end{demo}


\subsection{Caso raíces que difieren en un entero}

Cuando  $n_0:=m_1-m_2\in \mathbb{N}\cup\{0\}$ la ecuación de recurrencia  \eqref{eq:recu_gral_frob_2} se convierte en \label{pag:apar_sol} con $m_2$ y $n_0$ en lugar de $m$ y $n$ respectivamente, la cual puede ser imposible de satisfacer. 
 
 Una aparente solución sería tomar  $0=a_0=\cdots=a_{n_0-1}$ que resolverían la ecuación conflictiva. Podríamos elegir $a_{n_0}\neq 0$ arbitrariamente y continuar la iteración
\begin{equation}\label{eq:iter_m1}
  \begin{split}
    f(m_1+1)a_{n_0+1}&=-\sum_{k=0}^{n_0}a_k\left[p_{n_0+1-k}(m_2+k) + q_{n_0+1-k}\right]=-a_{n_0}[p_1m_1+q_1]\\
   f(m_1+2) a_{n_0+2}&=-\sum_{k=0}^{n_0+1}a_k\left[p_{n_0+2-k}(m_2+k) + q_{n_0+2-k}\right]\\
                   &=-a_{n_0}[p_2m_1+q_2]-a_{n_0+1}[p_1(m_1+1)+q_0]\\
                   &\,\,\,\vdots
  \end{split}
\end{equation}
Como se ve, la iteración termina siendo la misma que la correpondiente iteración para la raíz $m_1$. Además tendríamos que el desarrollo en serie
\[y(x)=x^{m_2}\sum_{n=0}^{\infty}a_nx^n=x^{m_2}\sum_{n=n_0}^{\infty}a_nx^n
 =x^{m_2+n_0}\sum_{n=0}^{\infty}a_{n+n_0}x^n
=x^{m_1}\sum_{n=0}^{\infty}b_nx^n
,\]
donde $b_n:=a_{n+n_0}$ resuelve el esquema de iteración \eqref{eq:iter_m1} que, como dijimos, es el esquema de iteración correspondiente a $m=m_1$. Vale decir que elegir  $0=a_0=\cdots=a_{n_0-1}$ nos lleva a una solución linealmente dependiente con la solución que ya encontramos para $m=m_1$. 

Podríamos tener la suerte que eligiendo $a_0\neq 0$ se satisfaga la ecuación \eqref{eq:ec_enter} con $m=m_2$ y $n=n_0$. Si esto ocurre podemos elegir $a_{n_0}$ arbitrariamente, por ejemplo $a_{n_0}=0$, y continuar la iteración. Obtenemos una solución que es linealmente independiente   de la correspondiente a $m_1$. En síntesis, \emph{si $n_0:=m_1-m_2\in\mathbb{Z}$, si $a_0\neq 0$ es eligido arbitrariamente, si  $a_1,\ldots,a_{n_0-1}$ son hallados mediante \eqref{eq:recu_gral_frob} y si se satisface  \eqref{eq:ec_enter} tenemos una segunda solución en serie de Frobenius para $m=m_2$}. Esta situación ocurría con la ecuación de Bessel cuando $p=(2k-1)/2$, $k\in\mathbb{N}$. 

Si falla esta última condición ya no hay otra solución en serie de Frobenius. Se puede encontrar otro  desarrollo, emparentado pero que ya no constituye una serie de Frobenius. Una manera de llegar a este es aplicando el método de reducción de orden a partir de la solución conocida $y_1=x^{m_1}\sum_{n=0}^{\infty}a_nx^n$. Este método nos decía que la otra solución venía dada a partir de las relaciones
\[y_2(x)=v(x)y_1(x),\quad\text{donde  } v'(x)=\frac{1}{y_1^2}e^{-\int p(x)dx}.\]
Teniendo en cuenta que
\[p(x)=\sum_{n=0}^{\infty}p_nx^{n-1}=\frac{p_0}{x}+p_1+p_2x+\cdots.\]
Vemos que
\[
   \begin{split}
     v'(x)&=\frac{1}{x^{2m_1}\left(\sum_{n=0}^{\infty}a_nx^n\right)^2}e^{-\int \left(\frac{p_0}{x}+p_1+p_2x+\cdots \right)dx}\\
  &= \frac{1}{x^{2m_1}\left(\sum_{n=0}^{\infty}a_nx^n\right)^2}e^{-p_0 \ln x -p_1x-\cdots }\\
  &= \frac{1}{x^{2m_1+p_0}\left(\sum_{n=0}^{\infty}a_nx^n\right)^2}e^{ -p_1x-\cdots }
   \end{split}
\]

Ahora la función
\[
  g(x)= \frac{ e^{ -p_1x-\cdots } }{\left(\sum_{n=0}^{\infty}a_nx^n\right)^2},
\]
es analítica en $0$ puesto que el denominador no se anula en cero. Por consiguiente tenemos un desarrollo en serie
\[ g(x)=\sum_{n=0}^{\infty}b_nx^n, \quad b_0\neq 0.\]
En la práctica encontrar este desarrollo en serie de manera explícita puede ser muy difícil. Llamemos $k:=2m_1+p_0$. Por la conocida fórmula para la suma de las raíces de una ecuación de segundo grado, se tiene que $m_1+m_2=1-p_0$.  Luego  $2m_1+p_0=2m_1+1-m_1-m_2=m_1-m_2+1\in\mathbb{N}$.

Tenemos que
\[v'(x)=\frac{b_0}{x^k}+ \frac{b_1}{x^{k-1}}+\cdots+\frac{b_{k-1}}{x}+b_{k}+\cdots\]
Entonces
\[v(x)=\frac{b_0}{(-k+1)x^{k-1}}+ \frac{b_1}{(-k+2)x^{k-2}}+\cdots+b_{k-1}\ln x+b_{k}x+\cdots\]
Reemplazando esta identidad en la expresión para $y_2$,
\[
   \begin{split}
     y_2&=vy_1\\
&=y_1\left(\frac{b_0}{(-k+1)x^{k-1}}+ \frac{b_1}{(-k+2)x^{k-2}}+\cdots+b_{k-1}\ln x+b_{k}x+\cdots\right)\\
       &=b_{k-1}\ln x y_1+ x^{m_1}\sum_{n=0}^{\infty}a_nx^n\left(\frac{b_0}{(-k+1)x^{k-1}}+ \frac{b_1}{(-k+2)x^{k-2}}\cdots\right)\\
       &=b_{k-1}\ln x y_1+ x^{m_1-k+1}\sum_{n=0}^{\infty}a_nx^n\left(\frac{b_0}{(-k+1)}+ \frac{b_1}{(-k+2)}x+\cdots\right)\\
   \end{split}
\]
Ahora $m_1-k+1=m_1-2m_1-p_0+1=-p_0-m_1+1$. Es sabido que la suma de las raices $m_1+m_2$ es igual a $-p_0+1$, es decir $m_1=-p_0+1-m_2$. Entonces $m_1-k+1=m_2$. Obtenemos una solución de la forma
\boxedeq{y_2(x)=b_{k-1}y_1\ln x+x^{m_2}\sum_{n=0}^{\infty}c_nx^n.}{eq:des_enter}
La segunda solución $y_2$ es la suma de una serie de Frobenius, con $m=m_2$, y un múltiplo de la función $y_1\ln x$. Esta última función no se puede desarrollar en serie de potencias alrededor de $0$.

En un caso particular se puede hallar los coeficientes $b_{k-1}$ y $c_n$ por el método de coeficientes indeterminados. Hay fórmulas para estos coeficientes, que en algunos casos pueden ser útiles. Vamos a obtener algunas de estás. 

Tratemos primero la situación cuando tenemos una única raiz $m_1=m_2$. Notar que en este caso $k=1$ y $b_{k-1}=b_0=g(0)=1/a_0^2\neq 0$. En este caso el término logarítmico siempre está presente. Sólo restan determinar los $c_n$, $n\geq 1$.    

Para un valor arbitrario de $m$ tal que $m_1-m\notin\mathbb{N}$, escribamos 
\begin{equation}\label{eq:2da_sol_bessel}
 y(x;m)=x^m\left(1+\sum_{n=1}^\infty a_n(m)x^n \right),
\end{equation}
donde $a_n$ son la solución de la ecuación en recurrencia \eqref{eq:recu_gral_frob_2} con $a_0=1$. Se tiene que

\begin{ejercicio}{} Demostrar que 
 \[
 y''+p(x)y'+q(x)y=x^{m-2}(m-m_1)^2.
\]
Notar que la fórmula anterior vuelve a verificar que $y(x;m_1)$ es solución.
 \end{ejercicio}

Derivando respecto a $m$

 \[
 \left(\frac{\partial y}{\partial m}
 \right)''+p\left(\frac{\partial y}{\partial m}
 \right)'+q\left(\frac{\partial y}{\partial m}
 \right)=x^{m-2}(m-m_1)^2\ln x + 2x^{m-2}(m-m_1).
\]
El miembro derecho es cero cuando $m=m_1$. De esta manera se puede obtener una segunda solución

\begin{multline}\label{eq:bessel_2_sol_p=0}
 y_2(x)=\frac{\partial y(x;m)}{\partial m}\bigg|_{m=m_1}\\
 =x^m\ln x\left(1+\sum_{n=1}^\infty a_n(m)x^n \right)+ x^m\left(\sum_{n=1}^\infty\frac{\partial a_n(m)}{\partial m}\bigg|_{m=m_1} x^n \right)\\
 =\ln x y_1(x)+ x^m\left(\sum_{n=1}^\infty\frac{\partial a_n(m)}{\partial m}\bigg|_{m=m_1} x^n \right).
\end{multline}
Llegamos a la conclusión que los coeficientes en \eqref{eq:des_enter} vienen dados cuando $m_1=m_2$ por
 \boxedeq{b_{k-1}=b_0=1,c_0=1, \quad c_n=\frac{\partial a_n(m)}{\partial m}\bigg|_{m=m_1},\quad n=1,2,\ldots .}{eq:for_coef1}
 
La dificultad de esta fórmula es que requiere resolvere la relación de recurrencia para $m$ arbitrario y esto puede resultar más dificil que cuando $m$ era solución de la ecuación indicial. 


En el caso de que las raíces de la ecuación indicial satisfagan que $0<k-1=m_1-m_2\in\mathbb{N}$, calculos similares, aunque bastante m{as arduos muestran que los coeficientes en \eqref{eq:des_enter} se obtienen mediante las fórmulas


\begin{empheq}[box=\tcbhighmath]{align}\label{eq:for_coef1}
b_{k-1} & =\lim_{m\to m_2}(m-m_2)a_{k-1}(m),\\c_0& =1,\quad c_n=\frac{\partial (m-m_2)a_n(m)}{\partial m}\bigg|_{m=m_2}, n=1,\ldots.
 \end{empheq} 

\subsection{Funciones de Bessel  de segunda especie, funciones de Neumann} \index{Función!Bessel de segunda especie}\index{Función!Neumann}

Vamos a ejemplificar los resultados de la subsección anterior para lograr una expresión para una solución de la ecuación de Bessel de orden $0$ linealmente independiente con $J_0$. 

Reemplazando $p$ por $0$ en las relaciones \eqref{eq:recu_bessel_gen} y despejando, llegamos cuando $m\neq 0$ a
 
 \[
 \begin{array}{ll}
  (m+1)^2a_1=0\\
  a_n=-\frac{1}{(n+m)^2}a_{n-2}
\end{array}
\]

Se infiere que $a_1\equiv a_3\equiv\cdots \equiv 0$ y consecuentemente las derivadas de estas funciones respecto a $m$ también son nulas. Para los valores pares

\begin{equation}\label{eq:calc_Y0}
 a_{2 n}(m)=\frac{(-1)^{n} a_{0}}{(2 n+m)^{2}(2 n-2+m)^{2} \cdots(4+m)^{2}(2+m)^{2}}, \quad n=1,2,3, \ldots
\end{equation}



Debemos determinar $a_n'(0)$, este cálculo puede efectuarse considerando la función
$$
f(x)=\left(x-\alpha_{1}\right)^{\beta_{1}}\left(x-\alpha_{2}\right)^{\beta_{2}}\left(x-\alpha_{3}\right)^{\beta_{3}} \cdots\left(x-\alpha_{n}\right)^{\beta_{n}},
$$
entonces
$$
\begin{aligned}
f^{\prime}(x)=& \beta_{1}\left(x-\alpha_{1}\right)^{\beta_{1}-1}\left[\left(x-\alpha_{2}\right)^{\beta_{2}} \cdots\left(x-\alpha_{n}\right)^{\beta_{n}}\right] \\
&+\beta_{2}\left(x-\alpha_{2}\right)^{\beta_{2}-1}\left[\left(x-\alpha_{1}\right)^{\beta_{1}}\left(x-\alpha_{3}\right)^{\beta_{3}} \cdots\left(x-\alpha_{n}\right)^{\beta_{n}}\right]+\cdots .
\end{aligned}
$$
De donde, para $x$ no igual a $\alpha_{1}, \alpha_{2}, \ldots, \alpha_{n}$
$$
\frac{f^{\prime}(x)}{f(x)}=\frac{\beta_{1}}{x-\alpha_{1}}+\frac{\beta_{2}}{x-\alpha_{2}}+\cdots+\frac{\beta_{n}}{x-\alpha_{n}} .
$$
Aplicado a \eqref{eq:calc_Y0}
$$
\frac{a_{2 n}^{\prime}(m)}{a_{2 n}(m)}=-2\left(\frac{1}{2 n+m}+\frac{1}{2 n-2+m}+\cdots+\frac{1}{2+m}\right),
$$
Entonces
$$
a_{2 n}^{\prime}(0)=-2\left[\frac{1}{2 n}+\frac{1}{2(n-1)}+\frac{1}{2(n-2)}+\cdots+\frac{1}{2}\right] a_{2 n}(0)
$$

De la fórmula \eqref{eq:func_bessel_1e} inferimos que $a_{2n}(0)=(-1)^n/2^{2n}(n!)^2$, luego
$$
a_{2 n}^{\prime}(0)=-H_{n} \frac{(-1)^{n} a_{0}}{2^{2 n}(n !)^{2}}, \quad n=1,2,3, \ldots
$$
donde
$$
H_{n}=\frac{1}{n}+\frac{1}{n-1}+\cdots+\frac{1}{2}+1
$$

Usando las identidades \eqref{eq:bessel_2_sol_p=0}, \eqref{eq:for_coef1}, tomando  $a_{0}=1$ y  $y_{1}(x)=J_0(x)$ se obtiene
$$
y_{2}(x)=J_{0}(x) \ln x+\sum_{n=1}^{\infty} \frac{(-1)^{n+1} H_{n}}{2^{2 n}(n !)^{2}} x^{2 n}, \quad x>0 .
$$
En vez de $y_{2}$, como  segunda solución suele usarse la siguiente combinación lineal entre $J_{0}$ e $y_{2}$. La misma se conoce como función de Bessel de segunda clase de orden cero o función de Neumann de orden $0$.  Se denota por $Y_{0}$ y se define de la siguiente forma
$$
Y_{0}(x)=\frac{2}{\pi}\left[y_{2}(x)+(\gamma-\ln 2) J_{0}(x)\right],
$$
donde  $\gamma$ es la \emph{constante de Euler-Máscheroni}\index{Constante!Euler-Máscheroni} que se define por la ecuación
$$
\gamma=\lim _{n \rightarrow \infty}\left(H_{n}-\ln n\right) \cong 0.5772 \text {. }
$$
De manera explícita
$$
Y_{0}(x)=\frac{2}{\pi}\left[\left(\gamma+\ln \frac{x}{2}\right) J_{0}(x)+\sum_{m=1}^{\infty} \frac{(-1)^{m+1} H_{m}}{2^{2 m}(m !)^{2}} x^{2 m}\right], \quad x>0
$$

Las funciones de Neumann se definen para cualquier orden $p$. Se suele seguir otro camino que exponemos de manera breve


\begin{definicion}{} Para $0<p\notin\mathbb{N}$ definimos la \emph{función de Bessel  de segunda especie}  o \emph{función de Neumann } $Y_p$ por
\begin{equation}\label{eq:bessel_2_especie}
Y_p=\frac{\cos p\pi J_p-J_{-p}}{\sen p\pi}.
\end{equation}
\end{definicion}

La función $Y_p$ es una solución no acotada cerca de $0$ dado que $J_p$ es acotada y $J_{-p}$ no. La razón de esta llamativa definición es el siguiente resultado.

\begin{lema}{lem:func_bessel_2_e} Para $n$ entero no negativo,  el  límite $\lim_{p\to n}Y_p$ existe. Por consiguiente, podemos definir
\begin{equation}\label{eq:bessel_2_especie_b}
Y_n(x):=\lim_{p\to n}Y_p(x).
\end{equation}
Esta función es solución de la ecuación de Bessel de orden $n$ y se denomina, también, función de Bessel de segunda especie. También resulta ser una función no acotada cerca de $0$ y por consiguiente, linealmente idependiente de $J_n$.
\end{lema}

\begin{proof} La omitiremos. \end{proof}
\iffalse 

% \begin{lstlisting}
% a=symbols('a0:6')
% x=symbols('x')
% y=sum([a[i]*x**i for i in range(6)])
% Ecua=y.diff(x)-y
% Ecuaciones=[Ecua.diff(x,i).subs(x,0)/factorial(i) for i in range(6)]
% Ecuaciones=Ecuaciones[:-1]+[a[0]-1]
% a_sol=solve(Ecuaciones,a)
% y.subs(a_sol)
% \end{lstlisting}

% 
% \begin{lstlisting}
% a=symbols('a0:6')
% x,p=symbols('x,p')
% y=sum([a[i]*x**i for i in range(6)])
% Ecua=(1+x)*y.diff(x)-p*y
% Ecuaciones=[Ecua.diff(x,i).subs(x,0)/factorial(i) for i in range(6)]
% Ecuaciones=Ecuaciones[:-1]+[a[0]-1]
% a_sol=solve(Ecuaciones,a)
% y.subs(a_sol)
% \end{lstlisting}
% 
% 
% 
% \boxedeq{y(x)=\frac{1}{6} \, {\left(p - 1\right)} {\left(p - 2\right)} p x^{3} + \frac{1}{2} \, {\left(p - 1\right)} p x^{2} + p x + 1}{}

% 
% Podemos hacer los cálculos anteriores con SAGE
% 
% \begin{lstlisting}
% a=symbols('a0:10')
% orden=10
% x,omega=symbols('x,omega')
% y=sum([a[i]*x**i for i in range(orden)])
% Ecua=y.diff(x,2)+omega**2*y
% Ecuaciones=[Ecua.diff(x,i).subs(x,0)/factorial(i) for i in range(orden)]
% Ecuaciones=Ecuaciones[:-2]
% a_sol=solve(Ecuaciones,a[2:])
% y.subs(a_sol)
% \end{lstlisting}
 

% 
% Esto lo resolveremos con SymPy
% 
% \begin{lstlisting}
% a=symbols('a0:10')
% orden=10
% x,p=symbols('x,p')
% y=sum([a[i]*x**i for i in range(orden)])
% Ecua=(1-x**2)*y.diff(x,2)-2*x*y.diff(x)+p*(p+1)*y
% Ecuaciones=[Ecua.diff(x,i).subs(x,0)/factorial(i) for i in range(orden)]
% Ecuaciones=Ecuaciones[:-2]
% a_sol=solve(Ecuaciones,a[2:])
% {ind:a_sol[ind].factor() for ind in a[2:]}
% 
% \end{lstlisting}
% 
% \[
% \begin{split}
% a_{2} &=- \frac{a_{0} p}{2} \left(p + 1\right),\\
% a_{3} &=- \frac{a_{1}}{6} \left(p - 1\right) \left(p + 2\right),\\
% a_{4} &=\frac{a_{0} p}{24} \left(p - 2\right) \left(p + 1\right) \left(p + 3\right), \\
% a_{5} &=\frac{a_{1}}{120} \left(p - 3\right) \left(p - 1\right) \left(p + 2\right) \left(p + 4\right),\\
% a_{6} &=- \frac{a_{0} p}{720} \left(p - 4\right) \left(p - 2\right) \left(p + 1\right) \left(p + 3\right) \left(p + 5\right), \\
% a_{7} &=- \frac{a_{1}}{5040} \left(p - 5\right) \left(p - 3\right) \left(p - 1\right) \left(p + 2\right) \left(p + 4\right) \left(p + 6\right),\\
% a_{8} &=\frac{a_{0} p}{40320} \left(p - 6\right) \left(p - 4\right) \left(p - 2\right) \left(p + 1\right) \left(p + 3\right) \left(p + 5\right) \left(p + 7\right), \\
% a_{9} &=\frac{a_{1}}{362880} \left(p - 7\right) \left(p - 5\right) \left(p - 3\right) \left(p - 1\right) \left(p + 2\right) \left(p + 4\right) \left(p + 6\right) \left(p + 8\right).
% \end{split}
% \]
% 
% 
% En general

% 
% Todos estos cálculos los programamos en la siguiente función de SymPy. Esta función tiene un argumento $n$ que debe ser un número natural y devuelve el correspondiente polinomio de Legendre.
% 
% 
% \begin{lstlisting}
% def Legendre(n):
%     orden=n+2
%     a=symbols('a0:%s' %orden)
%     x=symbols('x')
%     y=sum([a[i]*x**i for i in range(orden)])
%     Ecua=(1-x**2)*y.diff(x,2)-2*x*y.diff(x)+n*(n+1)*y
%     Ecuaciones=[Ecua.diff(x,i).subs(x,0)/factorial(i) for i in range(orden-2)]
%     s=symbols('s')
%     if n%2==0:
%         Ecuaciones+=[a[0]-s,a[1]]
%     else:
%         Ecuaciones+=[a[0],a[1]-s]
%     Sol_a_n=solve(Ecuaciones,a)
%     y=y.subs(Sol_a_n)
%     sol=solve(y.subs(x,1)-1,s)
%     return y.subs(s,sol[0])
% \end{lstlisting}
% 
% Esto es una función de Python. Se pueden copiar las lineas de arriba en una consola de Python, pero lo más usual es preparar un archivo aparte que luego se carga en una sesión de SymPy.  Las sentencias para cargar el archivo son las dadas aquí debajo, donde hemos supuesto que el archivo se llama \texttt{Legendre.py} y está en el directorio \texttt{/home/fdmazzone/Git/Ecuaciones\_Diferenciales/scripts}

% 
% Resolvamos el ejemplo con SymPy.
% 
% \begin{lstlisting}
% orden=5
% a=symbols('a0:%s' %orden)
% x,m=symbols('x,m')
% y=x**m*sum([a[i]*x**i for i in range(orden)])
% Ecua=y.diff(x,2)+(1/(2*x)+1)*y.diff(x,1)-(1/(2*x**2))*y
% #Dividimos por m-2 asi todos los exponentes son enteros positivos
% Ecua=Ecua/x**(m-2)
% Ecua=Ecua.expand()
% Ecuaciones=[Ecua.diff(x,i).subs(x,0)/factorial(i) for i in range(orden)]
% for ec in Ecuaciones:
% ...     ec
% ...
% a0*m**2 - a0*m/2 - a0/2
% a0*m + a1*m**2 + 3*a1*m/2
% a1*m + a1 + a2*m**2 + 7*a2*m/2 + 5*a2/2
% a2*m + 2*a2 + a3*m**2 + 11*a3*m/2 + 7*a3
% a3*m + 3*a3 + a4*m**2 + 15*a4*m/2 + 27*a4/2
% 
% \end{lstlisting}
% 
% Con esta parte del script hemos generado la lista de las 5 primeras ecuaciones.
% Resolvamos la ecuación indicial
% 
% \begin{lstlisting}
% Sol_Ecua_Ind=solve(Ecuaciones[0],m)
% Sol_Ecua_Ind
% [-1/2, 1]
% \end{lstlisting}
% 
% Sustituyamos los valores de $m$ en la lista de ecuaciones, agreguemos la ecuación $a_0=1$, resolvamos para los $a_i$ y sustituyamos la solución en $y$. Obtenemos la primer solución (truncada)
% \begin{lstlisting}
% Ecuaciones1=[ec.subs(m,Sol_Ecua_Ind[1]) for ec in Ecuaciones]
% Ecuaciones1
% [0, a0 + 5*a1/2, 2*a1 + 7*a2, 3*a2 + 27*a3/2, 4*a3 + 22*a4]
% Ecuaciones1[0]=a[0]-1
% Ecuaciones1
% [a0 - 1, a0 + 5*a1/2, 2*a1 + 7*a2, 3*a2 + 27*a3/2, 4*a3 + 22*a4]
% sol=solve(Ecuaciones1,a)
% y1=y.subs(sol).subs(m,Sol_Ecua_Ind[1])
% y1
% x*(16*x**4/3465 - 8*x**3/315 + 4*x**2/35 - 2*x/5 + 1)
% 
% 
% \end{lstlisting}
% La segunda solución se obtinene
% \begin{lstlisting}
% Ecuaciones2=[ec.subs(m,Sol_Ecua_Ind[0]) for ec in Ecuaciones]
% Ecuaciones2
% [0, -a0/2 - a1/2, a1/2 + a2, 3*a2/2 + 9*a3/2, 5*a3/2 + 10*a4]
% Ecuaciones2[0]=a[0]-1
% sol=solve(Ecuaciones2,a)
% y2=y.subs(sol).subs(m,Sol_Ecua_Ind[0])
% (x**4/24 - x**3/6 + x**2/2 - x + 1)/sqrt(x)
% 
% \end{lstlisting}


\begin{sympycode}
orden=5
a,x,m,p=symbols(['a0:%s' %orden, 'x','m','p'],positive=True )
y=x**m*sum([a[i]*x**i for i in range(orden)])
EDif1=y.diff(x,2)+1/x*y.diff(x,1)+(1-p**2/x**2)*y
EDif2=(EDif1/x**(m-2)).simplify()
ECoef=[EDif2.diff(x,i).subs(x,0)/factorial(i) for i in range(orden)]
SolEInd=solve(ECoef[0],m)





\end{sympycode}

Sutituyendo 
$$y=\sympy{y.expand()}+\cdots$$ 
en \eqref{eq:bessel3},  agrupando potencias iguales de $x$ y dividiendo por $x^{m-2}$ obtememos
\[
 \sympy{EDif2.collect(x).factor()}
\]



Las raíces de la ecuación indicial son
\boxedeq{m=p\quad\text{y}\quad m=-p}{eq:sol_ec_ind}
Vamos a trabajar con la raíz $m=p$.

\begin{lstlisting}
ECoefA=[ec.subs(m,Sol_Ecua_Ind[1]) for ec in ECoef]
for ec in ECoefA:
...     pprint(ec)

\end{lstlisting}

\[
\begin{split}
 0 & = 0\\
0&=2 a_{1} p + a_{1}\\
0&=a_{0} + 4 a_{2} p + 4 a_{2}\\
0&=a_{1} + 6 a_{3} p + 9 a_{3}\\
0&=a_{2} + 8 a_{4} p + 16 a_{4}\\
0&=a_{3} + 10 a_{5} p + 25 a_{5}\\
0&=a_{4} + 12 a_{6} p + 36 a_{6}\\
0&=a_{5} + 14 a_{7} p + 49 a_{7}\\
\end{split}
\]



Se puede observar que estas ecuaciones relacionan  $a_n$ con $a_{n-2}$, i.e. que son relaciones de dos términos.  Podemos hacer explícita la relación

\begin{lstlisting}
for i in range(1,orden):
    iter=solve(ECoefA[i],a[i])[0].factor()
    print str(a[i])+'='+str(iter)
a1=0
a2=-a0/(4*(p + 1))
a3=-a1/(3*(2*p + 3))
a4=-a2/(8*(p + 2))
a5=-a3/(5*(2*p + 5))
a6=-a4/(12*(p + 3))
a7=-a5/(7*(2*p + 7))
\end{lstlisting}
 
\begin{lstlisting}
p,x=symbols('p,x')
J= sum([(-1)**n/factorial(n)/gamma(p+n+1)*(x/2)**(2*n+p) for n in range(30)])
p1=plot(J.subs(p,1),(x,0,20),line_color=(1,0,0))
p2=plot(J.subs(p,2),(x,0,20),line_color=(0,1,0))[0]
p1.append(p2)
p2=plot(J.subs(p,3),(x,0,20),line_color=(0,0,1))[0]
p1.append(p2)
p1.show()
\end{lstlisting}

\fi

%\end{document}

%\include{Uni7}

\end{document}
