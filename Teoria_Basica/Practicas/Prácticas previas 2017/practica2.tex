\documentclass{article}
\usepackage{amssymb,amsmath}

\usepackage[spanish, activeacute]{babel}
\usepackage{theorem}
\usepackage{times}
\usepackage{array}
\usepackage{graphicx}
\usepackage{hyperref}
\usepackage{multirow}
\usepackage[cp1252]{inputenc}
\usepackage{hhline}
\usepackage{multicol}
\usepackage{bm}
\usepackage{natbib}
\usepackage{url}
\usepackage{tabularx}
%%%%%%%%%Estilo de la pagina%%%%%%%%%%%%%%%%%%%%%%%%%%%%%%%%%%%
%%%%%%%%%%%%%%%%%%%%%%%%%%%%%%%%%%%%%%%%%%%%%%%%%%%%%%%%%%%%%%%%%%
\newcounter{ejer}

{\theorembodyfont{\rmfamily}
\newtheorem{ejercicio}[ejer]{Ejercicio}}

\newcommand{\rr}{\mathbb{R}}
\newcommand{\qq}{\mathbb{Q}}
\newcommand{\nn}{\mathbb{N}}
\newcommand{\C}[1]{\overline{#1}}
\newcommand{\zz}{\mathbb{Z}}
\newcommand\eme {\mbox{\tiny\sc M}}
\newcommand\bfm {\mbox{\tiny\sc BFM}}
\newcommand\btuk {\mbox{\tiny\sc BT}}
\newcommand\rice {\mbox{\tiny\sc R}}
\newcommand\pc {\mbox{\tiny\sc PC}}
\newcommand\gasser {\mbox{\tiny\sc G}}
\newcommand\hall {\mbox{\tiny\sc H}}
\newcommand\h {\widehat{h}_n}
\newcommand\ta {\widehat{\tau}_n}
\newcommand\ho {h_{ {\mbox{\tiny opt}},n}}
\newcommand{\cs}{  \stackrel{c.s.}\longrightarrow  }
\newcommand{\pro}{  \stackrel{p}\longrightarrow  }
\newcommand{\di}{  \stackrel{d}\longrightarrow  }
\newcommand{\luno}{  \stackrel{L_1}\longrightarrow  }
\newcommand{\ldos}{  \stackrel{L_2}\longrightarrow  }
\newcommand{\comp}{  \stackrel{c}\longrightarrow  }
\newcommand{\seno}{\hbox{sen}}





\begin{document}


\hyphenation{excen-tri-ci-dad}
 %%%%%%%%%Estilo de la pagina%%%%%%%%%%%%%%%%%%%%%%%%%%%%%%%%%%%
%%%%%%%%%%%%%%%%%%%%%%%%%%%%%%%%%%%%%%%%%%%%%%%%%%%%%%%%%%%%%%%%%%


\noindent\begin{tabular}{m{.1\textwidth} m{.9\textwidth}}\hline\hline
\includegraphics[scale=.4]{unrc.jpg} &
%\begin{large}
\begin{bfseries}  \begin{scshape}
Facultad de Cs. Exactas, F�sico-Qu�micas y Naturales\par
        Depto de Matem\'atica.\par
        Primer Cuatrimestre de 2016\par
        Ecuaciones Diferenciales (1913) \par
        Pr�ctica 2: Ecuaciones de Primer Orden.

\end{scshape}
\end{bfseries}
%\end{large}
\\
\hline\hline
\end{tabular}
\renewcommand{\theenumi}{\alph{enumi}}





Los ejercicios marcados con * deben ser resueltos manualmente y con \texttt{SymPy}.

\begin{ejercicio} De \cite[Pag. 49-51]{simmons_esp} ejercicios 1(a)*,1(c)*,1(h)*, 2*, 3, 4, 5(a)  y 8.
\end{ejercicio}



\begin{ejercicio} De \cite[Pag. 54-55, ]{simmons_esp}  ejercicios 1*,3*,5*,7*,22.
\end{ejercicio}


\begin{ejercicio} De \cite[Pag. 61-62]{simmons_esp} ejercicios 1, 2(a)*, 2(c)*, 2(h)* y 3. 
\end{ejercicio}
 
 
\begin{ejercicio} De \cite[Pag. 63-65]{simmons_esp} ejercicios 1, 2(a)*, 2(c)*, 2(h)* y 3.
\end{ejercicio}


\begin{ejercicio} De \cite[Pag. 68]{simmons_esp} ejercicios 4 .
\end{ejercicio}

\begin{ejercicio} De \cite[Pag. 74 ]{simmons_esp} ejercicio 6. 
\end{ejercicio}

\begin{ejercicio} Resolver los siguientes problemas de valores iniciales asociados a  ecuaciones diferenciales en derivadas parciales lineales de primer orden. 
\begin{center}
\begin{tabular}{cc}
$\left.\begin{array}{ll} xu_x+u_y&=x^2\\u(x,0)&=x
    \end{array}\right\},\quad
$
&
$\left.\begin{array}{ll} yu_x-xu_y&=0\\u(0,y)&=y^2
    \end{array}\right\}
$
\end{tabular}
\end{center}

\end{ejercicio}
\bibliographystyle{plain}
\bibliography{biblio}
\end{document}
