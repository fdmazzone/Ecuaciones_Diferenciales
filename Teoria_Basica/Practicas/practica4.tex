\documentclass{article}
\usepackage{amssymb,amsmath}

\usepackage[spanish, activeacute]{babel}
\usepackage{theorem}
\usepackage{times}
\usepackage{array}
\usepackage{graphicx}
\usepackage{hyperref}
\usepackage{multirow}
\usepackage[cp1252]{inputenc}
\usepackage{hhline}
\usepackage{multicol}

%%%%%%%%%Estilo de la pagina%%%%%%%%%%%%%%%%%%%%%%%%%%%%%%%%%%%
%%%%%%%%%%%%%%%%%%%%%%%%%%%%%%%%%%%%%%%%%%%%%%%%%%%%%%%%%%%%%%%%%%
\newcounter{ejer}

{\theorembodyfont{\rmfamily}
\newtheorem{ejercicio}[ejer]{Ejercicio}}

\newcommand{\rr}{\mathbb{R}}
\newcommand{\qq}{\mathbb{Q}}
\newcommand{\nn}{\mathbb{N}}
\newcommand{\C}[1]{\overline{#1}}
\newcommand{\zz}{\mathbb{Z}}






\begin{document}


\hyphenation{excen-tri-ci-dad}
 %%%%%%%%%Estilo de la pagina%%%%%%%%%%%%%%%%%%%%%%%%%%%%%%%%%%%
%%%%%%%%%%%%%%%%%%%%%%%%%%%%%%%%%%%%%%%%%%%%%%%%%%%%%%%%%%%%%%%%%%
\setlength{\unitlength}{1cm}
%
\setlength{\extrarowheight}{5mm}
%

\noindent\begin{tabular}{m{.1\textwidth} m{.9\textwidth}}\hline\hline
\includegraphics[scale=.4]{unrc.jpg} &
%\begin{large}
\begin{bfseries}  \begin{scshape}
Facultad de Cs. Exactas, F�sico-Qu�micas y Naturales\par
        Depto de Matem\'atica.\par
        Primer Cuatrimestre de 2016\par
        Ecuaciones Diferenciales (1913)\par
        Pr�ctica 4: Ecuaciones Diferenciales  Lineales 

\end{scshape}
\end{bfseries}
%\end{large}
\\ 
\hline\hline
\end{tabular}
\renewcommand{\theenumi}{\alph{enumi}}


\setlength{\extrarowheight}{-5mm}
\hyphenation{ho-meo-mor-fo}

Realizar los siguientes ejercicios de \cite[Cap�tulo 3]{simmons_esp}. Los ejercicios marcados con (*) deben ser hechos a ``mano'' y con \texttt{SymPy}.

\begin{itemize}

\item  Secci�n 14, Pag 89-90, Ejercicios: 6a,6b,6c,9 y 10.
\item Secci�n 15, Pag 94-95, Ejercicios: 2(*), 4(*),8 y 9.
\item Secci�n 16, Pag 97-98, Ejercicios: 5(*), 6(*), 8 y 12.
\item Secci�n 17, Pag. 101-102, Ejercicios: 1(a), 1(j), 1(p),2(d), 2(f), 3,4,5(completar incisos a y b) y 6.
\item Secci�n 18, Pag. 107, Ejercicio 2(*).
\item Secci�n 19, Pag. 110, Ejercicios: 1,3(a) y 5.
\item Secci�n 20, Pag. 119, Ejercicio 3 (donde dice esf�rica poner cil�ndrica) , 4 y 5.
\item Secci�n 21, Pag. 128, Ejercicios 3(*), 4 y 5.
\item Secci�n 22, Pag. 134, Ejercicios 17 y 18.



\end{itemize}

\bibliographystyle{plain}
\bibliography{biblio}



\end{document}
