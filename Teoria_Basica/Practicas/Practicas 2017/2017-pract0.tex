\documentclass{article}
\usepackage{amssymb,amsmath}

\usepackage[spanish, activeacute]{babel}
\usepackage{theorem}
\usepackage{times}
\usepackage{array}
\usepackage{graphicx}
\usepackage{hyperref}
\usepackage{multirow}
\usepackage[utf8x]{inputenc}
%\usepackage[cp1252]{inputenc}
\usepackage{hhline}
\usepackage{multicol}
\usepackage{bm}
\usepackage{natbib}
\usepackage{url}
\usepackage{tabularx}
%%%%%%%%%Estilo de la pagina%%%%%%%%%%%%%%%%%%%%%%%%%%%%%%%%%%%
%%%%%%%%%%%%%%%%%%%%%%%%%%%%%%%%%%%%%%%%%%%%%%%%%%%%%%%%%%%%%%%%%%
\newcounter{ejer}

{\theorembodyfont{\rmfamily}
\newtheorem{ejercicio}[ejer]{Ejercicio}}
%
%\newcommand{\rr}{\mathbb{R}}
%\newcommand{\qq}{\mathbb{Q}}
%\newcommand{\nn}{\mathbb{N}}
%\newcommand{\C}[1]{\overline{#1}}
%\newcommand{\zz}{\mathbb{Z}}
%\newcommand\eme {\mbox{\tiny\sc M}}
%\newcommand\bfm {\mbox{\tiny\sc BFM}}
%\newcommand\btuk {\mbox{\tiny\sc BT}}
%\newcommand\rice {\mbox{\tiny\sc R}}
%\newcommand\pc {\mbox{\tiny\sc PC}}
%\newcommand\gasser {\mbox{\tiny\sc G}}
%\newcommand\hall {\mbox{\tiny\sc H}}
%\newcommand\h {\widehat{h}_n}
%\newcommand\ta {\widehat{\tau}_n}
%\newcommand\ho {h_{ {\mbox{\tiny opt}},n}}
%\newcommand{\cs}{  \stackrel{c.s.}\longrightarrow  }
%\newcommand{\pro}{  \stackrel{p}\longrightarrow  }
%\newcommand{\di}{  \stackrel{d}\longrightarrow  }
%\newcommand{\luno}{  \stackrel{L_1}\longrightarrow  }
%\newcommand{\ldos}{  \stackrel{L_2}\longrightarrow  }
%\newcommand{\comp}{  \stackrel{c}\longrightarrow  }
%\newcommand{\seno}{\hbox{sen}}





\begin{document}


\hyphenation{excen-tri-ci-dad}
 %%%%%%%%%Estilo de la pagina%%%%%%%%%%%%%%%%%%%%%%%%%%%%%%%%%%%
%%%%%%%%%%%%%%%%%%%%%%%%%%%%%%%%%%%%%%%%%%%%%%%%%%%%%%%%%%%%%%%%%%
\setlength{\unitlength}{1cm}
%
\setlength{\extrarowheight}{5mm}
%

\noindent\begin{tabular}{m{.1\textwidth} m{.9\textwidth}}\hline\hline
\includegraphics[scale=.4]{unrc.jpg} &
%\begin{large}
\begin{bfseries}  \begin{scshape}
Facultad de Cs. Exactas, F\'isico-Qu\'imicas y Naturales\par
        Depto de Matem\'atica.\par
        Primer Cuatrimestre de 2018\par
        Ecuaciones Diferenciales (1913) \par
        Pr\'actica 0: Python.

\end{scshape}
\end{bfseries}
%\end{large}
\\
\hline\hline
\end{tabular}
\renewcommand{\theenumi}{\alph{enumi}}





\begin{ejercicio} Hacer un programa en Python  para testear la valid\'ez de la \href{http://es.wikipedia.org/wiki/Conjetura_de_Collatz}{Conjetura de Collatz}.

\end{ejercicio}

\begin{ejercicio}
Hacer un programa en Python que verifique si un n\'umero natural dado es primo es no.
\end{ejercicio}

\begin{ejercicio}
Realizar un programa en Python que verifique, para un n\'umero natural dado, la \href{https://es.wikipedia.org/wiki/Conjetura_de_Goldbach}{Congetura de Goldbach}.

 
\end{ejercicio}


%Se prevé trabajar en el problema de hallar soluciones periódicas de sistemas dados por
%campos vectoriales definidos en espacios de Sobolev-Orlicz; concretamente, sistemas del
%tipo $\frac{d}{dt}(\frac{u'\phi(u')}{|u|'})=\nabla F(t,u(t))$, donde $\phi$ es la derivada de una N-función $\Phi$ El operador diferencial del miembro izquierdox
%de la ecuación se suele denominar $\Phi$-Laplaciano. En trabajos anteriores se obtuvieron
%resultados de existencia de soluciones periódicas. Se proseguirá en esta línea, tratando de
%extender nuestros resultados. Más específicamente, planificamos: 1) abordar el caso $F$ no
%diferenciable, usando la noción de subdiferencial de Clarke. 2) Extender resultados de
%existencia, obtenidos en el marco del p-Laplaciano por técnicas de punto silla y por técnicas
%de dualidad al contexto del $\Phi$-Laplaciano. Se considerará además extensiones a funciones
%$\Phi$ anisotrópicas y a problemas en derivadas parciales.
%En otra línea de trabajo, se aplicaran técnicas variacionales y de optimización de forma a
%problemas de Stokes dependientes del tiempo en un dominio acotado $\Omega$ en el espacio
%tridimensional, conteniendo un cuerpo inaccesible D que se mueve en un fluido viscoso. El
%objetivo es determinar D o ciertos parámetros que lo describen (por ejemplo, su centro de
%masa y su matriz de rotación) en base a datos obtenidos en alguna porción de la frontera
%de $\Omega$.
%Se trabajará en el problema inverso de las configuraciones centrales y colineales del
%problema de los n-cuerpos, en especial en una conjetura de larga data.
%
%
%\bigskip
%\bigskip
%\newpage
\end{document}