\documentclass{article}
\usepackage{amssymb,amsmath}

\usepackage[spanish, activeacute]{babel}
\usepackage{theorem}
\usepackage{times}
\usepackage{array}
\usepackage{graphicx}
\usepackage{hyperref}
\usepackage{multirow}
\usepackage[cp1252]{inputenc}
\usepackage{hhline}
\usepackage{multicol}
\usepackage{bm}
\usepackage{tabularx}
\usepackage{natbib}
\usepackage{url}

%%%%%%%%%Estilo de la pagina%%%%%%%%%%%%%%%%%%%%%%%%%%%%%%%%%%%
%%%%%%%%%%%%%%%%%%%%%%%%%%%%%%%%%%%%%%%%%%%%%%%%%%%%%%%%%%%%%%%%%%
\newcounter{ejer}

{\theorembodyfont{\rmfamily}
\newtheorem{ejercicio}[ejer]{Ejercicio}}

\newcommand{\rr}{\mathbb{R}}
\newcommand{\qq}{\mathbb{Q}}
\newcommand{\nn}{\mathbb{N}}
\newcommand{\C}[1]{\overline{#1}}
\newcommand{\zz}{\mathbb{Z}}
\newcommand\eme {\mbox{\tiny\sc M}}
\newcommand\bfm {\mbox{\tiny\sc BFM}}
\newcommand\btuk {\mbox{\tiny\sc BT}}
\newcommand\rice {\mbox{\tiny\sc R}}
\newcommand\pc {\mbox{\tiny\sc PC}}
\newcommand\gasser {\mbox{\tiny\sc G}}
\newcommand\hall {\mbox{\tiny\sc H}}
\newcommand\h {\widehat{h}_n}
\newcommand\ta {\widehat{\tau}_n}
\newcommand\ho {h_{ {\mbox{\tiny opt}},n}}
\newcommand{\cs}{  \stackrel{c.s.}\longrightarrow  }
\newcommand{\pro}{  \stackrel{p}\longrightarrow  }
\newcommand{\di}{  \stackrel{d}\longrightarrow  }
\newcommand{\luno}{  \stackrel{L_1}\longrightarrow  }
\newcommand{\ldos}{  \stackrel{L_2}\longrightarrow  }
\newcommand{\comp}{  \stackrel{c}\longrightarrow  }
\newcommand{\seno}{\hbox{sen}}





\begin{document}


\hyphenation{excen-tri-ci-dad}
 %%%%%%%%%Estilo de la pagina%%%%%%%%%%%%%%%%%%%%%%%%%%%%%%%%%%%
%%%%%%%%%%%%%%%%%%%%%%%%%%%%%%%%%%%%%%%%%%%%%%%%%%%%%%%%%%%%%%%%%%
\setlength{\unitlength}{1cm}
%
\setlength{\extrarowheight}{5mm}
%

\noindent\begin{tabular}{m{.1\textwidth} m{.9\textwidth}}\hline\hline
\includegraphics[scale=.4]{imagenes/unrc.jpg} &
%\begin{large}
\begin{bfseries}  \begin{scshape}
Facultad de Cs. Exactas, F�sico-Qu�micas y Naturales\par
        Depto de Matem\'atica.\par
        Primer Cuatrimestre de 2014\par
        Ecuaciones Diferenciales (1913) \par
        Pr�ctica SAGE: 1 

\end{scshape}
\end{bfseries}
%\end{large}
\\
\hline\hline
\end{tabular}
\renewcommand{\theenumi}{\alph{enumi}}



\setlength{\unitlength}{1cm}
%
\setlength{\extrarowheight}{5mm}

\begin{ejercicio} Usar SAGE para graficar las familias de curvas y las familias de curvas ortogonales respectivas correspondiente a los ejercicios 
\cite[Pag. 16-17, Ejercicios 1,2,5]{simmons_esp}.
\end{ejercicio}

\begin{ejercicio} Usar SAGE y el comando \texttt{desolve} para resolver \cite[Ejercicio 4, pag. 34]{simmons_esp} y  
\cite[Pag. 16-17, Ejercicios 1,2,5]{simmons_esp}.
\end{ejercicio}


\begin{ejercicio} Resolver y graficar con  SAGE  
\cite[Pag. 64, Ejercicio 5]{simmons_esp}.
\end{ejercicio}

\begin{ejercicio} Resolver con  SAGE  
\cite[Pag. 64, Ejercicio 7]{simmons_esp}.
\end{ejercicio}

\begin{ejercicio} Resolver con  SAGE  
\cite[Pag. 97-98, Ejercicios 5 y 6]{simmons_esp}.
\end{ejercicio}

\begin{ejercicio}
 Programar en SAGE una funci�n que tome por argumentos los n�meros reales $p$ y $q$ y devuelva la soluci�n general de $y''+py'+qy=0$.
\end{ejercicio}

\begin{ejercicio}
 Programar en SAGE una funci�n que tome por argumentos los n�meros reales $p$ y $q$, la funci�n $r(x)$  y devuelva la soluci�n general de $y''+py'+qy=r(x)$.
\end{ejercicio}


\bibliographystyle{plain}
 \bibliography{biblio.bib}



\end{document}
