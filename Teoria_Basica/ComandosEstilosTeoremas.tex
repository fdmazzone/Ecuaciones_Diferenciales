
% %%%%%%%%%%%%%%%%%%%%%%%%%%Nuevos comandos entornos%%%%%%%%%%%%%%%%%%%%%%%%%%%%%%%%
% %%%%%%%%%%%%%%%%%%%%%%%%%%%%%%%%%%%%%%%%%%%%%%%%%%%%%%%%%%%%%%%%%%%%%%%%
\newenvironment{demo}{\noindent\emph{Dem.}}{\hfill\qed \newline\vspace{5pt}}

% \newenvironment{observa}{\noindent\textbf{Observación:}}{}
\newcommand{\com}{\mathbb{C}}
\newcommand{\rr}{\mathbb{R}}
\newcommand{\nn}{\mathbb{N}}
% \renewcommand{\epsilon}{\varepsilon}
\renewcommand{\lim}{\mathop{\rm lím}}
\renewcommand{\inf}{\mathop{\rm ínf}}
\renewcommand{\liminf}{\mathop{\rm líminf}}
\renewcommand{\limsup}{\mathop{\rm límsup}}
\renewcommand{\min}{\mathop{\rm mín}}
\renewcommand{\max}{\mathop{\rm máx}}
\renewcommand{\b}[1]{\boldsymbol{#1}}
% \renewenvironment{frame}[1]{}{}

%%%%%%%%%%%%%%%% Funcion característica %%%%%%%%%%%5555555

\DeclareRobustCommand{\rchi}{{\mathpalette\irchi\relax}}
\newcommand{\irchi}[2]{\raisebox{\depth}{$#1\chi$}} % inner command, used by \rchi
\newcommand{\der}[2]{\frac{\partial #1}{\partial #2}} 
 
 \definecolor{mycolor}{RGB}{204,179,174}
% 
\tcbset{highlight math style={enhanced,
  colframe=red!60!black,colback=mycolor,arc=4pt,boxrule=1pt,
  drop fuzzy shadow}}

% 
% 
% 


%\renewcommand{\lim}{displaystyle\lim}
\DeclareMathOperator{\atan2}{atan2}
\DeclareMathOperator{\sen}{sen}

\pgfdeclareverticalshading{exersicebackground}{100bp}
  {color(0bp)=(black!40);color(50bp)=(black!0)}

\mdfdefinestyle{MiEstilo}{innertopmargin=10pt,linecolor=white!100,%
linewidth=2pt,topline=true,tikzsetting={shading=exersicebackground}}  



%%%%%%%%%%%%%%%%  Recuadro ecuacion %%%%%%%%%%%%%%%%%%%%%%%%


\newcommand{\boxedeq}[2]{%
\begin{empheq}[box=\tcbhighmath]{equation}\label{#2} #1 \end{empheq}}

%\newcommand{boxedeq}[1]{\textbf{#1}}












%%%%%%%%%%%%  Teorema %%%%%%%%%%%%%%%%%%%%%%%%%%55

\newcounter{teorema}[chapter] \setcounter{teorema}{0}
\renewcommand{\theteorema}{\arabic{chapter}.\arabic{section}.\arabic{teorema}}
\newenvironment{teorema}[2][]{%
\refstepcounter{teorema}%
\mdfsetup{style=MiEstilo%
}
\ifstrempty{#1}
{
\begin{mdframed}[]\relax%
\strut \textbf{Teorema~\theteorema}\label{#2}
}
{
\begin{mdframed}[]\relax%
\strut \textbf{Teorema~\theteorema~(#1)}\label{#2}
}
}{\end{mdframed}}
%%%%%%%%%%%%%%%%%%%%%%%%%%%%%
%Lemma


\newcounter{lema}[chapter] \setcounter{lema}{0}
\renewcommand{\thelema}{\arabic{chapter}.\arabic{section}.\arabic{lema}}
\newenvironment{lema}[2][]{%
\refstepcounter{lema}%
\mdfsetup{style=MiEstilo%
}
\ifstrempty{#1}
{
\begin{mdframed}[]\relax%
\strut \textbf{Lema~\thelema}\label{#2}
}
{
\begin{mdframed}[]\relax%
\strut \textbf{Lema~\thelema~(#1)}\label{#2}
}
}{\end{mdframed}}
%%%%%%%%%%%%%%%%%%%%%%%%%%%%%


%% Definicion
\newcounter{definicion}[chapter] \setcounter{definicion}{0}
\renewcommand{\thedefinicion}{\arabic{chapter}.\arabic{section}.\arabic{definicion}}
\newenvironment{definicion}[2][]{%
\refstepcounter{definicion}%
\mdfsetup{style=MiEstilo%
}
\ifstrempty{#1}
{
\begin{mdframed}[]\relax%
\strut \textbf{Definición~\thedefinicion}\label{#2}
}
{
\begin{mdframed}[]\relax%
\strut \textbf{Definición~\thedefinicion~(#1)}\label{#2}
}}{\end{mdframed}}
%%%%%%%%%%%%%%%%%%%%%%%%%%%%%
%%%%%%%%%%%%%%%%%%%%%%%%%%%%%%

%%%%%%%%%%%%%%%%% Demostracion

\newenvironment{prf}{\noindent\emph{Dem.}}{$\square$ \newline\vspace{5pt}}


%Corolario
\newcounter{corolario}[chapter] \setcounter{corolario}{0}
\renewcommand{\thecorolario}{\arabic{chapter}.\arabic{section}.\arabic{corolario}}
\newenvironment{corolario}[2][]{%
\refstepcounter{corolario}%
\mdfsetup{style=MiEstilo%
}
\ifstrempty{#1}
{
\begin{mdframed}[]\relax%
\strut \textbf{Corolario~\thecorolario}\label{#2}
}
{
\begin{mdframed}[]\relax%
\strut \textbf{Corolario~\thecorolario~(#1)}\label{#2}
}}{\end{mdframed}}
%%%%%%%%%%%%%%%%%%%%%%%%%%%%%




%%%%%%%%%%%%%%%%%%%%%%%%%%%%%%
%% Ejercicio
\newcounter{ejercicio}[chapter] \setcounter{ejercicio}{0}
\renewcommand{\theejercicio}{\arabic{chapter}.\arabic{section}.\arabic{ejercicio}}
\newenvironment{ejercicio}[2][]{%
\refstepcounter{ejercicio}%
\mdfsetup{style=MiEstilo%
}
\ifstrempty{#1}
{
\begin{mdframed}[]\relax%
\strut \textbf{Ejercicio~\theejercicio}\label{#2}
}
{
\begin{mdframed}[]\relax%
\strut \textbf{Ejercicio~\theejercicio~(#1)}\label{#2}
}}{\end{mdframed}}
%%%%%%%%%%%%%%%%%%%%%%%%%%%%%

%%%%%%%%%%%%%%%%%%%%%%%%%%%%%%
%% Codigo
\newenvironment{codigo}[1][]{%
\mdfsetup{style=MiEstilo%
}
\ifstrempty{#1}
{
\begin{mdframed}[]\relax%
\strut \textbf{Codigo}
}
{
\begin{mdframed}[]\relax%
\strut \textbf{Codigo (#1)}
}}{\end{mdframed}}
%%%%%%%%%%%%%%%%%%%%%%%%%%%%%



%%%%%%%%%%%%%%%%%%%%%%%%%%%%%
%%%%%%%%%%%%%%%%%%%%%%%%%%%%%%
%%%%%%%%%%%%%%%    Proposicion    %%%%%%%%%%%%%%%%%%%%5

\newcounter{proposicion}[chapter] \setcounter{proposicion}{0}
\renewcommand{\theproposicion}{\arabic{chapter}.\arabic{section}.\arabic{proposicion}}
\newenvironment{proposicion}[2][]{%
\refstepcounter{proposicion}%
\mdfsetup{style=MiEstilo%
}
\ifstrempty{#1}
{
\begin{mdframed}[]\relax%
\strut \textbf{Proposición~\theproposicion}\label{#2}
}
{
\begin{mdframed}[]\relax%
\strut \textbf{Proposición~\theproposicion (~#1)}\label{#2}
}}{\end{mdframed}}
%%%%%%%%%%%%%%%%%%%%%%%%%%%%%

%%%%%%%%%%Ejemplo%%%%%%%%%%%%%%%%%%%%%%%%%%%%%%5555
\newcounter{ejemplo}[chapter] \setcounter{ejemplo}{0}
\renewcommand{\theejemplo}{\arabic{chapter}.\arabic{section}.\arabic{ejemplo}}
\newenvironment{ejemplo}[2][]{%
\refstepcounter{ejemplo}%
\relax\noindent\textbf{Ejemplo~\theejemplo}\label{#2}
}{}


%%%%%%%%%%Observacion%%%%%%%%%%%%%%%%%%%%%%%%%%%%%%5555
\newcounter{observacion}[chapter] \setcounter{observacion}{0}
\renewcommand{\theobservacion}{\arabic{chapter}.\arabic{section}.\arabic{observacion}}
\newenvironment{observacion}{%
\refstepcounter{observacion}%
\relax\noindent\textbf{Observacion~\theobservacion}
}{}

%%%%%%%%%% Problema %%%%%%%%%%%%%%%%%%%%%%%%%%%%%%5555
\newenvironment{problema}[1][]{%
%
\mdfsetup{style=MiEstilo%
}
\ifstrempty{#1}
{
\begin{mdframed}[]\relax%
\strut \textbf{Problema }
}
{
\begin{mdframed}[]\relax%
\strut \textbf{Problema (~#1)}
}}{\end{mdframed}}
%%%%%%%%%%%%%%%%%%%%%%%%%%%%%
%%%%%%%%%%%%%%%Secciones%%%%%%%%%%%%%%%%%%%%%%%%%%%%%%%5


\makeatletter
\newcommand\makeSecHead[4][\fbox]{%
  \@namedef{#2}{\@ifnextchar*{\@nameuse{#2@i}}{\@nameuse{#2@ii}}}
%
    \expandafter\def\csname#2@i\endcsname*##1{\par\vspace{#4}\noindent
       #1{\parbox{\dimexpr\textwidth-2\fboxsep-2\fboxrule}{%
         \normalfont\normalsize#3\makebox[40pt][l]{}~##1}}\par\vspace{#4}}%
%
    \expandafter\def\csname#2@ii\endcsname{\@ifnextchar[{\@nameuse{#2@iii}}{\@nameuse{#2@iv}}}%
%
    \expandafter\def\csname#2@iii\endcsname[##1]##2{\par\vspace{#4}\noindent
      #1{\parbox{\dimexpr\textwidth-2\fboxsep-2\fboxrule}{%
        \refstepcounter{#2}\normalfont\normalsize#3\makebox[40pt][l]{\@nameuse{the#2}}~##2}}%
        \addcontentsline{toc}{#2}{\@nameuse{the#2}~##1}\par\vspace{#4}}%
%
   \expandafter\def\csname#2@iv\endcsname##1{\par\vspace{#4}\noindent
     #1{\parbox{\dimexpr\textwidth-2\fboxsep-2\fboxrule}{%
       \refstepcounter{#2}\normalfont\normalsize#3\makebox[40pt][l]{\@nameuse{the#2}}~##1}}%
       \addcontentsline{toc}{#2}{\@nameuse{the#2}~##1}\par\vspace{#4}}%
}
\makeatother    

\makeSecHead[\colorbox{gray!30}]{chapter}{\Huge\bfseries}{20pt}
\makeSecHead[\colorbox{gray!30}]{section}{\LARGE\bfseries}{15pt}
\makeSecHead[\colorbox{gray!30}]{subsection}{\Large\bfseries}{12pt}
\makeSecHead[\colorbox{gray!30}]{subsubsection}{\large\bfseries}{10pt}

%\renewcommand{\emph}[1]{\fontshape{it}\selectfont #1}


%%%%%%%%%%%%%%%%%%%%%Colores

\definecolor{color8}{HTML}{8E87C1}
\definecolor{color2}{rgb}{0.44,0.62,0.42}
\definecolor{color3}{rgb}{0.28, 0.51, .68}
\definecolor{color4}{rgb}{0.29,0.3,0.57}



%%%%%%%%%%%%%%%%%%%%%% Configuracion listing

\lstset{ %
  backgroundcolor=\color{white},   % choose the background color; you must add \usepackage{color} or \usepackage{xcolor}
  basicstyle=\footnotesize,        % the size of the fonts that are used for the code
  breakatwhitespace=false,         % sets if automatic breaks should only happen at whitespace
  breaklines=true,                 % sets automatic line breaking
  captionpos=b,                    % sets the caption-position to bottom
  commentstyle=\color{color2},    % comment style
  deletekeywords={...},            % if you want to delete keywords from the given language
  escapeinside={\%*}{*)},          % if you want to add LaTeX within your code
  extendedchars=true,              % lets you use non-ASCII characters; for 8-bits encodings only, does not work with UTF-8
  frame=single,	                   % adds a frame around the code
  keepspaces=true,                 % keeps spaces in text, useful for keeping indentation of code (possibly needs columns=flexible)
  keywordstyle=\color{blue},       % keyword style
  language=Python,                 % the language of the code
  otherkeywords={symbols,dsolve,solve,Eq, simplify, subs, plot,Function},           % if you want to add more keywords to the set
  numbers=left,                    % where to put the line-numbers; possible values are (none, left, right)
  numbersep=5pt,                   % how far the line-numbers are from the code
  numberstyle=\tiny\color{color3}, % the style that is used for the line-numbers
  rulecolor=\color{black},         % if not set, the frame-color may be changed on line-breaks within not-black text (e.g. comments (green here))
  showspaces=false,                % show spaces everywhere adding particular underscores; it overrides 'showstringspaces'
  showstringspaces=false,          % underline spaces within strings only
  showtabs=false,                  % show tabs within strings adding particular underscores
  stepnumber=2,                    % the step between two line-numbers. If it's 1, each line will be numbered
  stringstyle=\color{color4},     % string literal style
  tabsize=2,	                   % sets default tabsize to 2 spaces
  title=\lstname                   % show the filename of files included with \lstinputlisting; also try caption instead of title
}

\lstdefinelanguage{GAP}{%
  morekeywords={%
    Assert,Info,IsBound,QUIT,%
    TryNextMethod,Unbind,and,break,%
    continue,do,elif,%
    else,end,false,fi,for,%
    function,if,in,local,%
    mod,not,od,or,%
    quit,rec,repeat,return,%
    then,true,until,while,SymmetricGroup,Elements,Subgroup,IsNormal,FactorGroup,>gap%
  },%
  sensitive,%
  morecomment=[l]\#,%
  morestring=[b]",%
  morestring=[b]',%
}[keywords,comments,strings]

