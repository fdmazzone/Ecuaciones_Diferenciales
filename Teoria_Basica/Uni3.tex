\chapter{Teoremas de Existencia y Unicidad}
\addtocounter{ejemplo_cont}{1}


\subsection{Introducción}

%\begin{quote}

 \enquote{
 El determinismo es una doctrina filosófica que sostiene que todo acontecimiento físico, incluyendo el pensamiento y acciones humanas, está causalmente determinado por la irrompible cadena causa-consecuencia, y por tanto, el estado actual \enquote{determina} en algún sentido el futuro...En física, el determinismo sobre las leyes físicas fue dominante durante siglos, siendo algunos de sus principales defensores Pierre Simon Laplace y Albert Einstein}. \cite{ wiki:determinismo}.
 
\enquote{Podemos mirar el estado presente del universo como el efecto del pasado y la causa de su futuro. Se podría condensar un intelecto que en cualquier momento dado sabría todas las fuerzas que animan la naturaleza y las posiciones de los seres que la componen. Si este intelecto fuera lo suficientemente vasto para someter los datos al análisis, podría condensarse en una simple fórmula de movimiento de los grandes cuerpos del universo y del átomo más ligero; para tal intelecto nada podría ser incierto y el futuro, así como el pasado, estaría frente sus ojos}. Laplace, citado en \cite{ wiki:determinismo}.

La ciencia se vale de estructuras matemáticas para describir a traves de modelos la evolución de sistemas reales. Usualmente esas estructuras son ecuaciones (diferenciales) donde el tiempo es una de sus variables. Conocer el \enquote{estado actual} de un sistema (las \enquote{posiciones} que menciona Laplace) es conocer las condiciones iniciales. Mientras que las ecuaciones en si mismas se derivan de  \enquote{las fuerzas que animan la naturaleza}.  Así, para que la matemática colabore con la filosofía determinista, debería ella misma plantear problemas deterministas.  En ese sentido, hablando específicamente de ecuaciones diferenciales, para que podamos decir que un PVI sea determinista, el problema debería tener solución (existencia de soluciones), caso contrario este PVI no produciría ninguna predicción del futuro. Pero también esta solución debería ser única (unicidad de soluciones), porque de no ser así, el problema produciría múltiples predicciones y por tanto el estado actual no determinaría el futuro. Un problema que satisfaga estas condiciones lo denominaremos \emph{bien planteado}\footnote{Es común en la literatura requerir, además de la existencia y unicidad, la estabilidad para hablar de problema bien planteado} . 

Por lo expuesto, es de trascendencia para la ciencia en general que se pueda establecer que los problemas matemáticos que modelizan problemas de estas ciencias sean bien planteados. Este es el propósito de este capítulo.


Para nuestra sorpresa, PVIs sin ningun problema aparente a la vista no son problemas bien planteados.

\begin{ejemplo} Considerar el siguiente PVI:
\[
  \left\{
 \begin{array}{ll}
    y'(x)&=y(x)^{2/3}\\
    y(0)&=0\\
 \end{array}
 \right.
\]

La ecuacion  s en varialbles separables. Deberíamos dividir por $y(x)$, pero notar antes de hacer ello que $y\equiv 0$ es solución. Supongamos $y(x)\neq 0$, una vez dividido
\[\frac{dy}{y^{2/3}}=dx\Rightarrow 3y^{1/3}=x+C\Rightarrow y(x)=\left(\frac{x}{3}+C\right)^3.\]
A pesar de la limitación original de que $y(x)\neq 0$, la expresión para $y(x)$ que hemos hallado es solución para todo $x\in\rr$. Notar que  $y(x)=0$ cuando $x=-C/3$. Sin embargo, el valor problemático $y=0$ nos trae aparejado un inesperado problema. Ocure  que la función $y_C(x):=\left(\frac{x}{3}+C\right)^3$ tiene derivada igual a cero en $x=-C/3$. Esto implica que si $C_1<C_2$ entonces la función 
\[
  y_{C_1,C_2}(x):=
  \left\{
 \begin{array}{ll}
    y_{C_1}(x)&\hbox{ si } x\leq-C_2\\
    0         &\hbox{ si } x\in [C_1,C_2]\\
    y_{C_2}(x) &\hbox{ si } x\geq -C_1\\
  \end{array}
 \right.
\]
está bien definida y es diferenciable en $\rr$. Además de ello, cualquiera de estas funciones con $C_1\leq 0 \leq C_2$ resuelven el PVI.  De esta manera hemos encontrado un PVI con infinitas soluciones.
\end{ejemplo}


\end{psclip}
\end{pspicture}
\end{center}
\caption{Ecuación $x'=\frac12(x^2-1)$}\label{fig2}
\end{figure}




\section{Sistemas de Ecuaciones Diferenciales}
\subsection{Definición y ejemplos}
Un \emph{sistema de ecuaciones diferenciales de primer orden} es un conjunto  de ecuaciones que relacionan una variable independiente, digamos $x$, un conjunto de variables dependientes, digamos $y_1(x),\ldots,y_n(x)$, y sus derivadas respecto a $x$. A las ecuaciones diferenciales con una incognita y una ecuación las denominaremos \emph{ecuaciones escalares}.

\begin{definicion} Un sistema de ecuaciones diferenciales es una conjunto de ecuaciones de la forma
\begin{equation}\label{eq:sist_ecua}
\left\{
\begin{split}
 \frac{dy_1}{dx}(x)&=f_1(x,y_1(x),\ldots,y_n(x))\\
  \frac{dy_2}{dx}(x)&=f_2(x,y_1(x),\ldots,y_n(x))\\
       &\,\,\vdots                                   \\
 \frac{dy_n}{dx}(x)&=f_n(x,y_1(x),\ldots,y_n(x))\\
\end{split}\right.,
\end{equation}
donde $f_j:[a,b]\times\rr^n\to\rr$, $j=1,\ldots,n$, son funciones.

Una manera alternativa y compacta de denotar un sistema se logra introduciendo las funciones $y=(y_1,\ldots,y_n)$  y $f=(f_1,\ldots,f_n)$. Entonces $y:[a,b]\to\rr^n$ es una función con valores en $\rr^n$ y $f:[a,b]\times \rr^n\to\rr^n$ es un campo vectorial dependiente de $x$. Con estas notaciones el sistema se escribe:
\begin{equation}\label{eq:sist_ecua_comp}
 y'(x)=f(x,y(x)).
\end{equation}

\end{definicion}






\begin{ejemplo} \textbf{Ecuación del péndulo.} Si $x(t)$ es el ángulo que forma un péndulo, de longitud $l$, con la vertical en el tiempo $t$ y $v(t)=x'(t)$, entonces $x(t)$ y$v(t)$ deben satisfacer el siguiente sistema de  ecuaciones:
\begin{equation}\label{eq:sist_pend}
\left\{
\begin{split}
 x'(t)&=v(t)\\
  v'(t)&=-\frac{g}{l}\sen(x(t)).\\
\end{split}\right.
\end{equation}


\end{ejemplo}


\begin{ejemplo} \textbf{Sistemas de ecuaciones de Lotka-Volterra}
 En 1925 y 1926, Alfred J. Lotka y Vito Volterra respectivamente, introdujeron
 las \href{https://es.wikipedia.org/wiki/Ecuaciones_Lotka%E2%80%93Volterra}{ecuaciones de Lotka-Volterra}\link. \marginpar{\includegraphics[scale=.28]{imagenes/Vito_Volterra.jpg}\\
 Vito Volterra (1860-1940)}  Se trata de un sistema de dos ecuaciones diferenciales de primer orden que se usan para describir dinámicas de sistemas biológicos en el que dos especies interactúan, una como presa y otra como depredador. Se definen como:
\begin{equation}\label{eq:lotka_volterra}
\left\{
\begin{split}
 x'(t) &= x(t) ( \alpha − \beta  y(t) )\\
  y'(t)&=-y(t)(\gamma-\delta x(t))\\
\end{split}\right.
\end{equation}
La variable $y$ representa el número de individuos de algún predador (por ejemplo, un lobo) y $x$ es el número de sus presas (por ejemplo, conejos), $t$ representa el tiempo; y  $\alpha,\beta,\gamma$ y $\delta$ son parámetros (positivos)

\end{ejemplo}


\subsection{Sistemas de ecuaciones y ecuaciones de orden superior}

\emph{Es posible convertir el sistema \eqref{eq:sist_ecua} de $n$-ecuaciones diferenciales de primer orden  en una ecuación escalar de orden $n$
\begin{equation}\label{eq:orden_n}
y^{(n)}=f(x,y,y',\ldots,y^{(n-1)}).
\end{equation}
y viceverza.}

Para justificar la aseveración anterior supongamos que $y=y(x)$ resuelven  \eqref{eq:orden_n} y escribamos:
\[y_1=y,\, y_2=y',\,y_3=y'',\ldots, y_{n}=y^{(n-1)}.\]
Entonces notar que
\begin{equation}\label{eq:sist_ecua_conv}
\left\{
\begin{array}{l l l}
 y_1'(x)&=y_2(x)\\
  y_2'(x)&=y_3(x)\\
       &\,\,\vdots\\
  y_{n-1}'(x)&=y_n(x)\\
 y_n'(x)&=f(x,y_1(x),\ldots,y_n(x))\\
\end{array}\right.,
\end{equation}

Reciprocamente, supongamos que $y_1,\ldots,y_n$ resuelven \eqref{eq:sist_ecua}. Por simplicidad vamos a suponer que las $f_j$ son independientes de $x$, el procedimiento general sigue las mismas líneas que el caso que discutimos aquí. Se toma $y=y_n$ (podríamos usar cualquier $y_j$, $j=1,\ldots,n$). Ahora derivamos sucesivamente $n$-veces respecto a $x$ la ecuación para $y_n$,  y reemplazamos cada derivada $y_j'$ por $f_j$ (vamos a omitir los argumentos de $f_j$ que son en todos los casos $(y_1,\ldots,y_n)$):

\[%\label{eq:sist_ecua_conv}
\begin{array}{c c c}
y_n'(x) =& f_n(y_1,\ldots,y_n) &\eqqcolon g_1(y_1,\ldots,y_n)\\
 y_n''(x)=&\sum\limits_{j=1}^n\frac{\partial f_n}{\partial y_j}f_j+
 & \eqqcolon g_2(y_1,\ldots,y_n)\\
  y_n'''(x)=&\sum\limits_{k,j=1}^n\frac{\partial^2 f_n}{\partial y_k \partial y_j}f_kf_j+
  \sum\limits_{k,j=1}^n\frac{\partial f_n}{\partial y_j}\frac{\partial f_j}{\partial y_k}f_k & \eqqcolon g_k(y_1,\ldots,y_n)\\
  &\,\,\vdots &\,\,\vdots \\
  y^{(n)}_n(x) =&\cdots & \eqqcolon g_n(y_1,\ldots,y_n)
  \\
\end{array}
\]
%\end{equation}
Las igualdades anteriores tienen la estructura $z=G(y)$, donde $y=(y_1,\ldots,y_n)$, $z=(y_n',y_n'',\ldots,y_n^{(n)})$ y $G=(y_1,\ldots,y_n)$. Si la función $G:\rr^n\to\rr^n$ es invertible y escribimos $H=G^{-1}$ entonces
\[y_n=H_n(z)=H_n(y_n',y_n'',\ldots,y_n^{(n)}).\]
La anterior es una ecuación escalar de orden $n$ para $y_n$. Por consiguiente hemos logrado reducir el sistema de $n$-ecuaciones a una ecuación de orden $n$. Observar que si resolvemos esta ecuación, encontrando $y_n$, podemos hallar el resto de las incognitas $y_j$, $j=1,\ldots,n-1$,  usando que $y_j=H_j(y_n',y_n'',\ldots,y_n^{(n)})$.


\begin{ejemplo} \textbf{Ecuaciones de Lotka-Volterra.} Reduzcamos las ecuaciones de Lotka-Volterra a una ecuación de orden 2 y luego revirtamos el camino.

La primera parte la resolvemos con Sympy:
\lstinputlisting[language=Python]{scripts/Sist-Ecua.py}
Resulta en
\begin{equation}\label{eq:lotka_escalar}\frac{d^{2}}{d t^{2}}  y{\left (t \right )- \frac{d}{d t} y{\left (t \right )} +
\left(- \alpha \gamma - \alpha\right) y{\left (t \right )} + \left(\beta \gamma + \beta\right) y^{2}{\left (t \right )} }=0
\end{equation}

El camino inverso es más sencillo. Llamamos $z=y'$. Usando las variables $y,z$ la ecuación \eqref{eq:lotka_escalar} se escribe
\[
 \left\{
 \begin{array}{ll}
    y'(t)&=v(t)\\
    v'(t)&=v+\left( \alpha \gamma + \alpha\right) y{\left (t \right )} - \left(\beta \gamma + \beta\right) y^{2}{\left (t \right )}
 \end{array}
 \right.
 \]

 No llegamos a la ecuación de partida. Hay que tener presente que una ecuación tiene diferentes representaciones en diferentes variables y que las variables que hemos elegido para el camino de vuelta $y,v$, no son las originales del problema $y,x$.


\end{ejemplo}


\section{Funciones Lipschitzianas}








   







\begin{flushright}
 \cite{ wiki:determinismo}.
\end{flushright}


%\end{quote}



  \bibliographystyle{apalike-url}
  \bibliography{diferenciales_ecuaciones,diferenciales_ecuaciones_sim}