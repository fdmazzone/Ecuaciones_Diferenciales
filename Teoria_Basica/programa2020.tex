\documentclass[12pt]{article}
\usepackage[top=1.5cm, bottom=1.25cm, left=2cm, right=2cm]{geometry}

\usepackage[spanish,activeacute]{babel}
%\usepackage[utf8]{inputenc}
%\usepackage{ucs}
%\usepackage[latin1]{inputenc}
%\usepackage{times}
%\usepackage[T1]{fontenc}
\usepackage{amssymb,amsmath}
\usepackage{enumerate}
\usepackage{verbatim}
%\usepackage{pst-all}
%\usepackage{pstricks-add}
\usepackage{array}
%\usepackage[T1]{fontenc}
%\usepackage{animate,movie15}
\usepackage{hyperref}
\usepackage{graphicx}
%\usepackage{tabularx}
\usepackage{url}
\usepackage{fontspec}
\setsansfont{Roboto Condensed}
\renewcommand{\familydefault}{\sfdefault}
\usepackage{float}
\floatplacement{table}{H}
\begin{document}



\pagestyle{plain}


\setlength{\unitlength}{1cm}
%
%\setlength{\extrarowheight}{5mm}
%

\noindent\begin{tabular}{m{.1\textwidth} m{.9\textwidth}}
\includegraphics[scale=.25]{escudounrc.jpg} &
%\begin{large}
{
Universidad Nacional de Rio Cuarto\par
Facultad de Ciencias Exactas, Físico-Químicas y Naturales\par
Departamento de Matemática
}
%\end{large}
\\
\end{tabular}

\setlength{\parindent}{0pt} % Default is 15pt.

\begin{center}
 \textbf{FORMULARIO PARA LA PRESENTACIÓN DE LOS PROGRAMAS DE ASIGNATURAS}
\end{center}





 \textbf{CARRERA:}  Lic en Matemática.

 \textbf{PLAN DE ESTUDIOS:} 2008 versión 1

 \textbf{ASIGNATURA:}  Ecuaciones Diferenciales\quad\textbf{CÓDIGO:} 1913

\textbf{[MODALIDAD DE CURSADO:} Presencial
 
 \textbf{DOCENTES RESPONSABLES:} Fernando Mazzone

\textbf{EQUIPO DOCENTE:} Dr. Gastón Beltritti (Ay. 1º exclusivo), Dr. Fernando Mazzone (PI exclusivo).

\textbf{AÑO ACADÉMICO:} 2020

\textbf{REGIMEN DE LA ASIGNATURA:} Cuatrimestral

\textbf{UBICACIÓN EN EL PLAN DE ESTUDIO:} primer cuatrimestre, cuarto año.

\textbf{RÉGIMEN DE CORRELATIVIDADES:}

\begin{table}[h]
\begin{tabular}{|l|l|}\hline
 Aprobada & Regular\\\hline
 &
 Álgebra Lineal Aplicada (2261)\\\hline
 &
Topología (1917)\\ \hline
\end{tabular}
 \end{table}






\textbf{CARGA HORARIA TOTAL:}  
\begin{table}[h]
\begin{tabular}{|l|l|l|l|l|l|l|l|}\hline
 Teóricas & hs & Prácticas &  hs & Teóricas-Prácticas: & hs & Laboratorio: & hs \\ \hline
& 60 &  & 60 & & & & \\ \hline
\end{tabular} 
\end{table}

\textbf{CARGA HORARIA SEMANAL:}  
\begin{table}[h]
\begin{tabular}{|l|l|l|l|l|l|l|l|}\hline
 Teóricas & hs & Prácticas &  hs & Teóricas-Prácticas: & hs & Laboratorio: & hs \\ \hline
& 4 &  & 4 & & & & \\ \hline
\end{tabular} 
\end{table}


\textbf{CARÁCTER DE LA ASIGNATURA:} Obligatoria
\renewcommand{\theenumi}{\Alph{enumi}}
\renewcommand{\theenumii}{\theenumi .\arabic{enumii}}

\newpage

\begin{enumerate}
 
 \item\textbf{ FUNDAMENTACIÓN }

	La gran parte del curso versa sobre temas que ya son estándar en las ecuaciones diferenciales y se consideran básicos en el desarrollo de esta área. No creemos necesario abundar en fundamentos sobre la incorporación de ellos. Si merece fundamentarse aquellos que no son del todo habituales.

	Con frecuencia las leyes de la física o modelos matemáticos de sistemas biológicos, sociales, económicos, etc, se expresan por medio de ecuaciones y particularmente con ecuaciones diferenciales. Con igual frecuencia en nuestras aulas la enseñanza de esta, como de otras  ramas de la matemática, omite la consideración de las relaciones entre los conceptos matemáticos y otras ciencias. A lo sumo se suele presentar alguna aplicación de la teoría como medio de justificar la relevancia de ella. Se hace extensivo al quehacer pedagógico  el postulado formalista que la matemática se valida por si misma y es independiente de cualquier  realidad ajena a ella.

	Nuestro punto de vista es que las relaciones de la teoría matemática con su entorno constituye un ingrediente insoslayable en la enseñanza de la teoría.  Fundamentamos este punto de vista, en que es frecuente que el sistema físico  que es modelizado por cierta teoría, ilumine el entendiemiento de la teoría misma. Por ejemplo, el principio de máximo, que afirma que una solución de la ecuación $\triangle u=0$ en un abierto y acotado $\Omega$ alcanza su máximo en la frontera de $\Omega$, no es evidente de la ecuación en si misma, pero si lo es en gran medida de una de las situaciones físicas que modeliza:  la temperatura en estado estacionario de un cuerpo sobre el cual el calor fluye por difusión. No puede desaprovecharse el recurso de pensar una solución de la ecuación aludida en estos dos sentidos.

	También ocurre el camino inverso, esto es que el desarrollo de la teoría matemática ilumine el entendimiento del sistema físico. Al fin y al cabo, ese es el propósito primario de la modelización matemática. Por ejemplo, para quien escribe no resulta físicamente evidente que un sistema de resortes acoplados tenga esencialmente sólo dos modos normales de vibración, cosa que se demuestra en el actual curso. Por estas consideraciones, entre otras, también estamos convencidos que la demostración matemática rigurosa también constituye un ingrediente insoslayable para el entendimiento de la teoría.

	Se incorpora activamente el uso de sistemas algebraicos computacionales SAC.  Una causa es contar con asistencia para el desarrollo de cálculos que son engorrosos. Pero la causa fundamental de la introducción de SAC es que ponen al alumno en la situación de hacer un programa que implemente procedimientos de la teoría. Esto suele ser una tarea no trivial para el recién iniciado y obliga a desarrollar aptitudes de programación, pero más importante, obliga a repensar la teoría matemática para adaptarla al nuevo contexto.


 
 \item \textbf{OBJETIVOS PROPUESTOS}
      %\begin{description}

          \begin{itemize}
	    \item  Presentar la teoría de las ecuaciones diferenciales desde un perspectiva rigurosa.
	    
	    

	    \item  Poner en evidencia la retroalimentación entre teoría matemática y modelos físicos. En este sentido se desarrollan aplicaciones a la caída de cuerpos a lo largo de guías (y su relación con la óptica), problema de la braquistócrona, vibraciones de sistemas mecánicos, membranas,  potencial sobre una esfera, movimiento planetario, etc.

	    
	    
	    \item  Incorporar el uso de sistemas algebraicos computacionales en la práctica del  alumno. Se utilizaran recursos de código abierto que derivan del lenguaje Python, en particular SymPy y SAGE.

        \item  Desarrollar la capacidad de resolver problemas analíticos a traves de SAC.
        
        
       \end{itemize}


 % \end{description}





 \item\textbf{CONTENIDOS BÁSICOS DEL PROGRAMA A DESARROLLAR}
    Ecuaciones de primer orden, métodos de solución.  Teorema de existencia y unicidad. Ecuaciones lineales de orden superior. Osciladores armónicos.   Método de desarrollo en serie. Método de Frobenius.     Sistemas de Ecuaciones Lineales.

 \item\textbf{PROGRAMAS Y/O PROYECTOS PEDAGÓGICOS INNOVADORES E INCLUSIVOS:}  No se preveen.
  
 
 \item\textbf{  ACTIVIDADES A DESARROLLAR}

		  \textbf{ CLASES TEÓRICAS:} Presencial, 4 horas semanales. La metodología que se desarrollará es la exposición por parte del docente de los fundamentos teóricos de los contenidos impartidos. Se incentivará la participación de los alumnos durante la clase, requiriendo que ellos aporten, por ejemplo, demostraciones de determinados hechos o, en general, soluciones a determinadas situaciones problemáticas que plantea el desarrollo teórico de la materia.

		  \textbf{ CLASES PRÁCTICAS:} Presencial 4 horas semanales. Se espera que los alumnos trabajen sobre los ejercicios de la práctica en forma independiente fuera de los horarios de la asignatura. Posteriormente estos ejercicios se discutirán durante la clase,  el profesor tratará de limitar su participación de modo tal de favorecer que los alumnos autogestionen su aprendizaje.

\item\textbf{   HORARIOS DE CLASES:} martes y jueves de 16 a 20.

\textbf{   HORARIOS DE CLASES DE CONSULTA:} Se preveen dos horas semanales de clases de consulta, 1 hora teórica y 1 h práctica. Dado que se preveen no más de 3 alumnos en la cursada, se considera más conveniente acordar con ellos el horario para estas clases.  

 



\item\textbf{   MODALIDAD DE EVALUACIÓN:}

\textbf{Evaluaciones Parciales:}  Se le presentará al alumno una serie de problemas que deberá resolver.

\textbf{Evaluación Final:} Será oral, el alumno deberá desarrollar los ejes conceptuales y fundamentos teóricos de la materia.

\item\textbf{  CONDICIONES DE REGULARIDAD:}
Aprobar los exámenes parciales o sus respectivos recuperatorios.

\item\textbf{  CONDICIONES DE PROMOCIÓN:} no se prevé

\newpage

\begin{center}
 \textbf{PROGRAMA ANALÍTICO}
\end{center}

\end{enumerate}

\begin{enumerate}

\item\textbf{ CONTENIDOS:}

 
\begin{description}
 \item[Unidad 0. Python, SymPy, SAGE.] Panorama de instalación, distribuciones y recursos online. Tipos de datos en Python, programación elemental. \cite{Mazzo}

\item[Unidad 1. Ecuaciones de Primer Orden.]  Noción de ecuación diferencial y clasificación. Ecuaciones diferenciales ordinarias (EDO).   Problemas de valores iniciales. Familia de curvas y la famila ortogonal. Método de separación de variables. Aplicaciones: dinámica de mezclas, cuerpos en caída a lo largo de guías, curvas braquistócrona y tautócrona. Ecuaciones homogéneas. Ecuaciones exactas. Factores integrantes. Ecuaciones lineales de primer orden. Métodos de reducción de orden. Curvas de persecución. Velocidad de escape. Problema del resorte. Cambios de variables. Usando SymPy para cambiar variables.   \cite{Mazzo,simmons_esp, WilliamE.Boyce496}

\item[Unidad 2. Teorema de Existencia y Unicidad.] Presentación filosófica: determinismo científico. Funciones Lipschitzianas. Teorema de punto fijo de Banach. Método iterativo de Picard. Teorema de existencia y unicidad para sistemas de EDO de primer orden.\cite{Mazzo,simmons_esp, WilliamE.Boyce496, JorgeSotomayor513}

\item[Unidad 3. Ecuaciones Lineales de Segundo Orden.] Ecuaciones lineales. Reducción de orden. Ecuaciones homogéneas a coeficientes constantes. El problema no homogéneo. Independencia lineal. Bases de soluciones. Polinomio característico. Ecuaciones no homogéneas. Coeficientes indeterminados y variación de los parámetros. Vibraciones mecánicas. Solución del problema Kepleriano de los dos cuerpos. Osciladores armónicos acoplados. \cite{Mazzo,simmons_esp, WilliamE.Boyce496}

\item[Unidad 4. Métodos cualitativos.] Teoremas de separación y de comparación de Sturn. Aplicaciones, ceros de las funciones de Bessel.\cite{Mazzo,simmons_esp,JorgeSotomayor513}

\item[Unidad 5. Desarrollo en serie de potencias.] Repaso de series de potencias. Método de coeficientes indeterminados.  Resolución de problemas de desarrollo en serie con  SymPy.   Ecuaciones lineales de segundo orden: puntos regulares. Puntos singulares regulares. Series de Frobenius. Teoremas fundamentales.\cite{Mazzo,simmons_esp, WilliamE.Boyce496}


\item[Unidad 6.  Sistemas lineales.]  Base de soluciones. Matriz fundamental. Sistemas lineales a coeficientes constantes. Solución del problema homogéneo con formas de Jordan. Problema no homogéneo. Sistemas no-lineales. \cite{ WilliamE.Boyce496,JorgeSotomayor513}

\end{description}

\item\textbf{CRONOGRAMA DE CLASES Y PARCIALES:}

\begin{table}[H]
\begin{tabular}{|m{0.6cm} |m{5cm}|m{5cm}|m{4.8cm}|}\hline
Sem.  & Teóricos & Prácticos & Parciales/ Recuperatorios\\\hline \hline
1   & Python-Sympy\newline Ecuaciones Primer Orden  &  Python-Sympy &   \\ \hline
  2 &  Ecuaciones Primer Orden  & Ecuaciones Primer Orden  &   \\ \hline
  3 &  Ecuaciones Primer Orden    &  Ecuaciones Primer Orden &   \\ \hline
   4 &  Ecuaciones Primer Orden    &  Ecuaciones Primer Orden &   \\ \hline
  5 & Teorema de Existencia Unicidad  &  Teorema de Existencia Unicidad   &   \\ \hline
  6 &   Ecuaciones Lineales   &  Ecuaciones Lineales   &   \\ \hline
  7 &  Ecuaciones Lineales   &  Ecuaciones Lineales &   \\ \hline

  8& Ecuaciones Lineales   &  Ecuaciones Lineales &  Jueves 7/5-Primer Parcial  \\ \hline

  9 & Métodos cualitativos & Métodos cualitativos   &   \\ \hline
  10 & Serie de Potencias  & Serie de Potencias  &   \\ \hline
  11 & Serie de Potencias   & Serie de Potencias  &   \\ \hline
   12 & Serie de Potencias   & Serie de Potencias  &   \\ \hline
  13 & Sistemas lineales  & Sistemas lineales  &   \\ \hline
  14 & Sistemas lineales & Sistemas lineales  &  Jueves 11/6-Segundo Parcial \\ \hline
  15  & Consulta & Consulta &  Martes 16/6 Recuperatorio Primer Parcial\newline
  Jueves 18/6 Recuperatorio Segundo Parcial \\ \hline

\end{tabular}
\end{table}

\end{enumerate}
















%
%
  \bibliographystyle{plain}%{apalike-url}
  \bibliography{diferenciales_ecuaciones,diferenciales_ecuaciones_sim}

\end{document}
