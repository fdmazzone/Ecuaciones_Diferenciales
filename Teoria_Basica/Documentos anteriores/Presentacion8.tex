% \documentclass{article}
% %\documentclass[hyperref={colorlinks=true}]{beamer}
% %\documentclass[handout,hyperref={colorlinks=true}]{beamer}
% 
% 
% %%%%%%%%%%%%%%%%%%%%%%%%%%%%%%Paquetes%%%%%%%%%%%%%%%%%%%%%%%%%%%%%%%%%%%%%%%%%%%%%%%
% %%%%%%%%%%%%%%%%%%%%%%%%%%%%%%%%%%%%%%%%%%%%%%%%%%%%%%%%%%%%%%%%%%%%%%%%%%%%%%%%%%%%%
% 
% %\usepackage{pgfpages}
% %\pgfpagesuselayout{4 on 1}[a4paper,landscape,border shrink=5mm]
% \usepackage{empheq}
% \usepackage[spanish]{babel}
% \usepackage[utf8x]{inputenc}
% \usepackage{times}
% \usepackage[T1]{fontenc}
% \usepackage{amssymb,amsmath}
% \usepackage{enumerate}
% \usepackage{verbatim}
% \usepackage{ esint }
% \usepackage{pdfsync}
% %\usepackage{pst-all}
% %\usepackage{pstricks-add}
% \usepackage{array}
% %\usepackage[T1]{fontenc}
% \usepackage{animate}
% %\usepackage{media9}
% %\usepackage{movie15}
% \usepackage{xparse}
% \usepackage{listings}
% \usepackage{ wasysym }
% \usepackage{sagetex}
% \usepackage{hyperref}
% \usepackage{tabularx}
% \usepackage{xcolor}
% \usepackage{adjustbox}
% 
% 
% 
% %%%%%%%%%%%%%%%%%%%%%%%%%%Nuevos comandos entornos%%%%%%%%%%%%%%%%%%%%%%%%%%%%%%%%
% %%%%%%%%%%%%%%%%%%%%%%%%%%%%%%%%%%%%%%%%%%%%%%%%%%%%%%%%%%%%%%%%%%%%%%%%
% %%%%%%%%%%%%%%%%%%%%%%%%Colores
% 
% \definecolor{myblue}{rgb}{.8, .8, 1}
% \definecolor{dblackcolor}{rgb}{0.0,0.0,0.0}
% \definecolor{dbluecolor}{rgb}{0.01,0.02,0.7}
% \definecolor{dgreencolor}{rgb}{0.2,0.4,0.0}
% \definecolor{dgraycolor}{rgb}{0.30,0.3,0.30}
% \definecolor{color_teor}{HTML}{FFFDA2}
% \definecolor{color_ejer}{HTML}{F6BFBF}
% \definecolor{color_cor}{HTML}{49E6D8}
% \definecolor{color_lem}{HTML}{C6F6AB}
% \definecolor{color_defi}{HTML}{EDA07C}
% \newcommand{\dblue}{\color{dbluecolor}\bf}
% \newcommand{\dred}{\color{dredcolor}\bf}
% \newcommand{\dblack}{\color{dblackcolor}\bf}
% 
% %%%%%%%%%%%%%%%%%%%Teoremas, definiciones , etc
% 
% \newenvironment{demo}{\noindent\emph{Dem.}}{{\hspace*{\fill}$\square$} \newline\vspace{5pt}}
% \newenvironment{colbox}[2]{%
%     \begin{adjustbox}{minipage={\linewidth},margin=1ex,bgcolor=#1,env=center}
%         #2}{%
%     \end{adjustbox}%
% }
% 
% \newcounter{defi_cont}
% \newenvironment{definicion}[1]{\begin{colbox}{color_defi}{\refstepcounter{defi_cont}\textbf{Definición \arabic{defi_cont}.} #1}}{\end{colbox}}
% 
% \newcounter{teor_cont}
% \newenvironment{teorema}[1]{\refstepcounter{teor_cont}\begin{colbox}{color_teor}{\textbf{Teorema \arabic{teor_cont}.} #1}}{\end{colbox}}
% 
% \newcounter{cor_cont}
% \newenvironment{corolario}[1]{\begin{colbox}{color_cor}{\refstepcounter{cor_cont}\textbf{Corolario \arabic{cor_cont}.} #1}}{\end{colbox}}
% 
% \newcounter{lem_cont}
% \newenvironment{lema}[1]{\begin{colbox}{color_lem}{\refstepcounter{lem_cont}\textbf{Lema \arabic{lem_cont}.} #1}}{\end{colbox}}
% 
% \newcounter{ejem_cont}
% \newenvironment{ejemplo}[1]{\refstepcounter{ejem_cont}\vspace{1ex}\noindent\textbf{Ejemplo \arabic{ejem_cont}.} #1}{}
% 
% \newcounter{ejer_cont}
% \newenvironment{ejercicio}[1]{\begin{colbox}{color_ejer}{\refstepcounter{ejer_cont}\textbf{Ejercicio \arabic{ejer_cont}.} #1}}{\end{colbox}}
% 
% 
% %%%%%%%%%%%%%%%%%%%%%%%%%%%%%%%%%%%%Simbolos
% 
% \newcommand{\com}{\mathbb{R}}
% \newcommand{\dis}{\mathbb{D}}
% \newcommand{\rr}{\mathbb{R}}
% \newcommand{\oo}{\mathcal{O}}
% \renewcommand{\emph}[1]{\textcolor[rgb]{0,0,1}{#1}}
% \newcommand{\der}[2]{\frac{\partial #1}{\partial #2}}
% \renewcommand{\v}[1]{\overrightarrow{#1}}
% \renewcommand{\epsilon}{\varepsilon}
% \DeclareMathOperator{\mcd}{mcd}
% \DeclareMathOperator{\atan2}{atan2}
% \DeclareMathOperator{\sen}{sen}
% %%%%%%%%%%%%%%%%%%%%%%%%%%%%%%Formulas en cajas 
% \newlength\mytemplen
% \newsavebox\mytempbox
% \makeatletter
% \newcommand\mybluebox{%
%     \@ifnextchar[%]
%        {\@mybluebox}%
%        {\@mybluebox[0pt]}}
% 
% \def\@mybluebox[#1]{%
%     \@ifnextchar[%]
%        {\@@mybluebox[#1]}%
%        {\@@mybluebox[#1][0pt]}}
% 
% \def\@@mybluebox[#1][#2]#3{
%     \sbox\mytempbox{#3}%
%     \mytemplen\ht\mytempbox
%     \advance\mytemplen #1\relax
%     \ht\mytempbox\mytemplen
%     \mytemplen\dp\mytempbox
%     \advance\mytemplen #2\relax
%     \dp\mytempbox\mytemplen
%     \colorbox{myblue}{\hspace{1em}\usebox{\mytempbox}\hspace{1em}}}
% 
% \makeatother
% \DeclareDocumentCommand\boxedeq{ m g }{%
%     {\begin{empheq}[box={\mybluebox[2pt][2pt]}]{equation}% #1%
%         \IfNoValueF {#2} {\label{#2}}%
%        #1
%        \end{empheq}
%     }%
% }
% 
% 
% 
% 
% %%%%%%%%%%%Listing
% \lstdefinelanguage{SymPy}[]{Python}
% {morekeywords={False,sage,True,symbols,sympy,diff,solve,subs,factorial,expand,atan2,sqrt,simplify,positive,cos,sin },sensitive=true}
% \lstset{
% frame=none,
% showtabs=False,
% showspaces=False,
% showstringspaces=False,
% commentstyle={\ttfamily\color{dgreencolor}},
% keywordstyle={\ttfamily\color{dbluecolor}\bfseries},
% stringstyle={\ttfamily\color{dgraycolor}\bfseries},
% language=SymPy,
% basicstyle={\fontsize{8pt}{8pt}\ttfamily},
% aboveskip=.3em,
% belowskip=0.1em,
% numbers=none,
% numberstyle=\footnotesize
% }
% 
% 
% 
% %%%%%%%%%%%%%%%%%%%%%%%%%%%%%%%%%%%%%%%%%%%%%%%%%%%%%%%%%%%%%%%%%%%%%%%%%%%%%%%%%%%%%%%%%%%%%%%%%%%%%%%%%%%%%%%%%%%%%%%%%%%%%%%%%%%%%%%%%%%%%%%%%%%%%%%%%%%%%%%%%%%%%%%%%%%%%%%%%%%%%%%%%%%%%%%
% 
% 


%\begin{document}
\chapter[Separacion de Variables]{Ecuación de la membrana, separación de variables}

El movimiento de una membrana elástica sujeta a un aro circular viene gobernada por la ecuación de \emph{ondas bidimensional}

\[ u_{tt}=\Delta u:=u_{xx}+u_{yy}\]

Esta ecuación es un ejemplo de \emph{ecuación en derivadas parciales}. Aquí la función incognita $u$ depende de tres variables independientes $x,y,t$. No hemos tratado hasta aquí este tipo de ecuaciones, en esta breve unidad vamos a ver como los resultados que desarrollamos antes  nos sirven para encontrar algunas soluciones de esta ecuación.  

En la deducción de esta ecuación, que no haremos, se supone que no actúa otra fuerza más que la tensión de la membrana, que el material de esta membrana es uniforme, que la dirección de desplazamientos de un punto sobre la membrana es perpendiculares al plano que contiene al aro de sujeción. 

El significado de las variables es el siguiente: $t$ es el tiempo, las variables $x,y,u$ son las coordenadas de un punto sobre la membrana en un sistema de coordenadas cartesiano ortogonal. Las coordenadas $x,y$     se toman en el plano que contiene al aro de sujeción y por ende, como el desplazamiento es solo en la dirección vertical, son independientes de $t$. El aro de sujeción  se supone de radio $1$ y centro en $(0,0)$.  La función  $u$ depende de  $x,y$ y  $t$. La ecuación  se satisface en la bola de radio $1$ y centro $(0,0)$ de $\mathbb{R}^2$.  Se impone además una \emph{condición de contorno} que da cuenta del hecho que la membrana esta fija al aro, esta condición es  $u=0$ en $\partial B$. Por último necesitamos  una \emph{condición inicial} que nos dice cual es el estado de la membrana en $t=0$, vamos a suponer que en ese momento la membrana está en reposo, esto es  $u_t(x,y,0)=0$.

 Buscaremos soluciones de la ecuación de ondas en variables separadas, esto es supondremos que $u$ se escribe  $u(x,y,t)=v(x,y)T(t)$. Estas soluciones tienen el significado físico de ser \emph{tonos normales}, esto es regímenes de vibración donde todos los puntos de la membrana vibran a la misma frecuencia. Asumamos además  que $u$ parte del reposo, esto es $u_t(x,y,0)=0$. 

Reemplazando $u(x,y,t)=v(x,y)T(t)$ en la ecuación obtenemos

\begin{equation}\label{eq:sep_var}v(x,y)T''(t)=T(t)\left(v_{xx}(x,y)+v_{yy}(x,y) \right).
\end{equation}
Vale decir que
\[\frac{T''(t)}{T(t) }=\frac{v_{xx}(x,y)+v_{yy}(x,y)}{v(x,y)}.\]
Las variables $t,x,y$ son independientes entre si.  En el miembro de la izquierda sólo aparece $t$ y en el de la derecha sólo $x,y$.  Por consiguiente  podríamos cambiar $t$ en el miembro de la izquierda dejando $x,y$ fijos. La conclusión es entonces que el miembro de la izquierda no cambia por cambiar $t$, vale decir $T''/T$ es una función constante y por lo tanto $\Delta v/v$ es constante. Debe exixtir $\lambda>0$ tal que
 
 \[\frac{T''(t)}{T(t) }=\frac{v_{xx}(x,y)+v_{yy}(x,y)}{v(x,y)}=-\lambda.\]
Tenemos así las dos ecuaciones
\begin{colbox}{myblue}{
\begin{align}
  T''(t)+\lambda T(t)&=0,\label{eq:eq_t}\\
  v_{xx}(x,y)+v_{yy}(x,y)+\lambda v(x,y)&=0\label{eq:eq_xy}
\end{align}
}
\end{colbox}

Además se deben satisfacer la condiciones de contorno  $u(x,y,t)=0$ para  $(x,y)\in \partial B$. Esta condición implica que \footnote{Otra posible solución de la condición de contorno sería tomar  $T=0$,pero esto  nos llevaría a la solución trivial}
\boxedeq{ v=0 \hbox{ en } \partial B. }{eq:cond_contor}

No olvidemos  la condición inicial $u_t(x,y,0)=0$. Atendiendo a la separación de variables debe ocurrir que $T'(0)v(x,y)=0$ y esto implica que  $T'(0)=0$ o $v(x,y)\equiv 0$. La segunda opción nos llevaría a  la solución trivial de la ecuación, lo que no es de nuestro interés. De modo que aceptaremos

\boxedeq{T'(0)=0.}{eq:cond_ini}

La ecuación \eqref{eq:eq_t} es una ecuación de segundo orden, lineal con coeficientes constantes, que hemos considerado en multiples oportunidades. Si $\lambda\leq 0$ obtendríamos soluciones que tienden a infinito cuando $t\to\infty$ lo que no es compatible con el comportamieno esperado para una membrana. Hay que tener en cuenta que la validez física del modelo matemático alcanza solo a pequeñas oscilaciones, esto es valores chicos de $u$. De modo que podemos suponer $\lambda=\omega^2$. La ecuación resulta la ecuación del oscilador armónico, cuya solución general es
\[T(t)=k_1\cos(\omega t)+k_2\sen(\omega t)\]
Se debe cumplir \eqref{eq:cond_ini}
\[0=T'(0)=k_2\omega\Rightarrow \boxed{k_2=0}.\]
\emph{La membrana vibra en con frecuencia $\omega$} dado que el factor  $T$ de $u$ vibra en esa frecuencia y el otro factor que compone $u$, esto es $v$, no depende de $t$ es decir es estático. 
La ecuación \eqref{eq:eq_xy}, que ahora escribimos
\boxedeq{  v_{xx}(x, y) + v_{yy}(x, y) +\omega^2 v(x, y)=0,}{eq:helmotz}
se conoce como la ecuación de autovalores del Laplaciano o ecuación de Helmholtz.  Los valores de $\omega$ para los que esta ecuación tiene solución (no todo $\omega$ lo es)  son justamente los autovalores del operador de Laplace. Es decir los autovalores de Laplaciano resultan ser los cuadrados de las frecuencias de los tonos normales. Para encontrar estos autovalores vamos a escribir $v$ en coordenadas polares $v=v(r,\theta)$. Acorde a la fórmula \eqref{eq:lapla_polares} del Apéndice  \ref{eq:apendice} tenemos que \eqref{eq:helmotz} equivale a

\begin{equation}\label{eq:ecu_aux_1}v_{rr}+\frac{1}{r}v_r+\frac{1}{r^2}v_{\theta\theta}+\omega^2v=0
\end{equation}
Nuevamente vamos a considerar la técnica de separación de variables. Proponemos que
\[v(r,\theta)=R(r)\Theta(\theta)\]

Reemplazando en \eqref{eq:ecu_aux_1}
\begin{equation*}\label{eq:ecua_aux_2} 
R''\Theta +\frac{1}{r}R'\Theta+\frac{1}{r^2}R\Theta''+\omega^2R\Theta=0.
\end{equation*}
Multiplicando por $r^2/R\Theta$ y depejando los términos conteniendo $\Theta$
\begin{equation}\label{eq:ecua_aux_3} 
\frac{r^2R''}{R} +\frac{rR'}{R}+\omega^2r^2=-\frac{\Theta''}{\Theta}.
\end{equation}
Ahora se razona como lo hemos hecho con la ecuación \eqref{eq:sep_var}. Cada miembro depende de variables independientes entre si, esto es de $r$ el primer miembro y de $\theta$ el segundo. Para que la igualdad sea posible las funciones deben ser constantes. Luego debe existir $\lambda$ tal que
\begin{align}
%\begin{split} 
r^2R'' +rR'+(\omega^2r^2-\lambda) R&=0\label{eq:ecua_aux_4}\\
\Theta'' +\lambda\Theta=0\label{eq:ecua_aux_5}.
%\end{split}
\end{align}
Además teníamos la condición de contorno \eqref{eq:cond_contor} que se convierte en 
\begin{equation}\label{eq:cond_contor}
R(1)=0.
\end{equation}

La función $\Theta$ al depender de $\theta$ debería ser periódica de periodo $2\pi$. Para que esto sea así  $\lambda$ debe ser positivo, puesto que las soluciones de  \ref{eq:ecua_aux_4} son periódicas solo para estos valores de $\lambda$. Por consiguiente
\begin{equation}\label{sol_theta}
\Theta(\theta)=c_1\cos\sqrt{\lambda}\theta+c_2\sen\sqrt{\lambda}\theta.
\end{equation}
 Como la solución, además de períodica,  debe tener más específicamente  período $2\pi$, el valor de $\lambda$ debe ser un entero cuadrado, es decir que existe un entero positivo  $n$ tal que $\lambda=n^2$.   Así \eqref{sol_theta} se convierte en 
\[\Theta(\theta)= c_1\cos n\theta+c_2\sen n\theta.\]
Nuestras suposiciones no  determinan el valor de las constantes $c_1$ y $c_2$. Obtendremos dos soluciones linealmente independientes eligiendo $k_1=0$ y $k_2=1$ o permutando estos valores. Vamos a seguir ilustrando el método con la segunda elección, es decir
\boxedeq{\Theta(\theta)= \cos n\theta }{sol_theta2}
 Reemplazando $\lambda$ por $n^2$ en \eqref{eq:ecua_aux_4} 
\[r^2R'' +rR'+(\omega^2r^2-n^2) R=0\label{eq:ecua_aux_6},\]
que se parece mucho a la ecuación de Besssel. De hecho si $\omega$ fuese 1 sería exactamente la ecuación de Bessel. La podemos convertir facilmente en la ecuación de Bessel por el cambio de variable independiente $s=\omega r$. Tenemos 
\[\frac{dR}{dr}=\frac{dR}{ds}\frac{ds}{dr}=\frac{dR}{ds}\omega\]
y
\[\frac{d^2R}{dr^2}=\frac{d^2R}{ds^2}\omega^2\]
Reemplazando las igualdades anteriores y $r$ por $s/\omega$ en \eqref{eq:ecua_aux_6} llegamos a
\[s^2R''(s)+sR'+(s^2-n^2)R=0\]
Que es la ecuación de Bessel de orden $n$. Como el orden es precisamente un entero hemos visto que la ecuación tiene como par de soluciones linealmente independientes la función de Bessel de primera especie $J_n$ que es analítica en $0$ y la función de Bessel de segunda especie $Y_n$ que es no acotada en $0$. Debería ocurrir entonces que $R=c_1J_p+c_2Y_p$. Pero de ser $c_2\neq 0$ tendríamos que $R$ sería no acotada y esto implicaría que la menbrana sería la gráfica de una función no acotada, cosa que no se condice con nuestra idea de lo que es una membrana. De modo que estas soluciones, que matemáticamente son correctas, las descartamos por carecer de significado físico. Sin perder generalidad, supongamos $c_1=1$. Así tenemos que $R$ como función de $s$ es $J_n(s)$. Retornando a la variable $r$
\boxedeq{R(r)=J_n(\omega r)}{sol_R}
Ahora la condición de contorno \eqref{eq:cond_contor} implica que
\boxedeq{J(\omega)=0}{eq:cer_bessel}
Así llegamos a la sorprendente conclusión que cuando una membrana vibra en un tono normal su frecuencia es un cero de una función de Bessel de orden $n$ para algún entero $n$ positivo. Sabemos que los ceros de la ecuación de Bessel forman una sucesión $\omega_0<\omega_1<\cdots$ que tiende a infinito y que la distancia entre ceros sucesivos $\omega_{k+1}-\omega_k$ tiende a $\pi$ cuando $k\to\infty$. Hemos encontrado muchas soluciones del problema propuesto, tenemos una solución distinta por cada $n$ entero positivo y por cada $\omega_k$ en la lista de ceros de $J_n$. Al tono normal que obtenemos para un $n$ y $k$ dados lo llamamos el tono $(n,k)$ y resulta estar dado en coordenadas polares por

\boxedeq{u(r,\theta,t)=\cos(\omega_k t)\cos(n\theta)J_n(\omega_kr)}{eq:tono_normal}

Siendo $u$ una función de tres variables independientes no es posible graficarla en nuestro mundo 3d. Tampoco nos interesa la grafica, pues la interpretación de $u$ es que para un tiempo $t$ fijo la gráfica en $\mathbb{R}^3$ de  $u(r,\theta,t)$ es la posición de la membrana.  Con SAGE (ver el Apéndice \ref{apendiceB}) generamos un conjunto de gráficos para distintos valores de $t$ y luego los animamos con el paquete \texttt{animate} de \LaTeX. Los tonos normales representados son el $(0,1)$, $(0,2)$ y $(1,2)$.

\begin{figure}[h]
  \begin{center}
\animategraphics[controls, scale=.4]{15}{membrana/01/mem-}{0}{49}
\end{center} 
\caption{Modo $(0,1)$}
\end{figure}


  \begin{figure}
 \begin{center}
\animategraphics[controls, scale=.4]{15}{membrana/02/mem-}{0}{49}
\end{center}  
\caption{Modo $(0,2)$}
\end{figure}

    \begin{figure}
 \begin{center}
\animategraphics[controls, scale=.4]{15}{membrana/12/mem-}{0}{49}
\end{center}  
\caption{Modo $(1,2)$}
\end{figure}

\newpage
\section{Apéndice A: Laplaciano en coordenadas polares}\label{eq:apendice}
Escribamos el operador de Laplace en coordenadas polares. Las coordenadas cartesianas  $x,y$ se escriben en función de las coordenadas polares $r,\theta$ de la siguiente forma
\[x=r\cos\theta\quad y=r\sen\theta.\]
Tenemos que expresar las derivas tomadas respecto a las coordenadas cartesianas en términos de derivas tomadas respecto a coordenadas polares.  
\begin{align}
  u_x&=u_rr_x+u_{\theta}\theta_x\notag\\
  u_{xx}&=\left(u_r\right)_xr_x+u_rr_{xx}+\left(u_{\theta}\right)_x\theta_x+u_{\theta}\theta_{xx}\notag\\
&=u_{rr}r_x^2+u_{\theta\theta}\theta_x^2+2u_{r\theta}r_x\theta_x+u_rr_{xx}+u_{\theta}\theta_{xx}\label{eq:lapla_pol}
\end{align}

Usemos SAGE para calcular $r_x,\theta_x,r_{xx},\theta_{xx}$

\begin{lstlisting}
>>> from sympy import *
>>> x,y=symbols('x,y')
>>> r=sqrt(x**2+y**2)
>>> theta=atan2(y,x)
>>> rho=symbols('rho',positive=True)
>>> phi=symbols('phi') 
>>> r.diff(x).subs({x:rho*cos(phi),y:rho*sin(phi)}).simplify()
cos(phi)
\end{lstlisting}
Es decir 
\begin{equation}\label{eq:r_x}r_x=\cos(\theta).
\end{equation}
En el cálculo anterior \texttt{rho} y \texttt{phi} son introducidas en reemplazo de \texttt{r} y \texttt{theta} aunque son exactamente iguales. La razón de hacer esto es que SymPy reemplaza \texttt{r} y \texttt{theta} por sus expresiones en términos de \texttt{x} e \texttt{y}. Esto es así por que en las líneas 12 y 13 del código anterior hemos declarado \texttt{r} y \texttt{theta} como expresiones. Cada vez que uno invoca estas variables, SymPy nos devuelve las expresiones que representan. Oviamente SymPy  ``no se da cuenta'' que nosotros queremos expresar todo en función de \texttt{r} y \texttt{theta} considerando estas variables como independientes. Las restantes derivadas son
\begin{lstlisting}
>>> r.diff(x,2).subs({x:rho*cos(phi),y:rho*sin(phi)}).simplify()
sin(phi)**2/rho
\end{lstlisting}

\begin{equation}\label{eq:r_xx}r_{xx}=\frac{\sen^2(\theta)}{r}.
\end{equation}

\begin{lstlisting}
>>> theta.diff(x).subs({x:rho*cos(phi),y:rho*sin(phi)}).simplify()
-sin(phi)/rho
\end{lstlisting}
\begin{equation}\label{eq:theta_x}\theta_x=-\frac{\sen(\theta)}{r}.
\end{equation}
\begin{lstlisting}
>>> theta.diff(x,2).subs({x:rho*cos(phi),y:rho*sin(phi)}).simplify()
sin(2*phi)/rho**2
\end{lstlisting}
\begin{equation}\label{eq:theta_xx}\theta_{xx}=\frac{2\sen(\theta)\cos(\theta)}{r^2}.
\end{equation}
Sustituyendo en \eqref{eq:r_x}, \eqref{eq:r_xx}, \eqref{eq:theta_x} y \eqref{eq:theta_xx} en \eqref{eq:lapla_pol}
\boxedeq{u_{xx}=u_{rr}\cos^2\theta+ u_{\theta\theta}\frac{\sen^2\theta}{r^2}-2u_{r\theta}\frac{\cos\theta\sen\theta}{r}+u_r\frac{\sen^2(\theta)}{r}+u_{\theta}
\frac{2\sen(\theta)\cos(\theta)}{r^2}.}{eq:u_xx}

Para las derivadas respecto a $y$ tenemos
\[u_{yy}=u_{rr}r_y^2+u_{\theta\theta}\theta_y^2+2u_{r\theta}r_y\theta_y+u_rr_{yy}+u_{\theta}\theta_{yy}\]
Y las derivadas
 \begin{lstlisting}
>>> theta.diff(y).subs({x:rho*cos(phi),y:rho*sin(phi)}).simplify()
cos(phi)/rho
>>> r.diff(y).subs({x:rho*cos(phi),y:rho*sin(phi)}).simplify()
sin(phi)
>>> theta.diff(y,2).subs({x:rho*cos(phi),y:rho*sin(phi)}).simplify()
-sin(2*phi)/rho**2
>>> r.diff(y,2).subs({x:rho*cos(phi),y:rho*sin(phi)}).simplify()
cos(phi)**2/rho
\end{lstlisting}
Luego
\boxedeq{u_{yy}=u_{rr}\sin^2\theta+ u_{\theta\theta}\frac{\cos^2\theta}{r^2}+2u_{r\theta}\frac{\cos\theta\sen\theta}{r}+u_r\frac{\cos^2(\theta)}{r}-u_{\theta}
\frac{2\sen(\theta)\cos(\theta)}{r^2}.}{eq:u_yy}
De \eqref{eq:u_xx} y \eqref{eq:u_yy} obtenemos

\boxedeq{u_{xx}+u_{yy}=u_{rr}+\frac{1}{r}u_r+\frac{1}{r^2}u_{\theta\theta}}{eq:lapla_polares} 

\section{Apendice B: Código SAGE para animación de  tonos normales}\label{apendiceB}

\begin{sageblock}
P.<x>=RR[x]
n=2
l=0
J= sum([(-1)^k/factorial(k)/gamma(k+1)*(x/2)^(2*k+n) for k in range(20)])
Ceros=J.roots()
Ceros=[i[0] for i in Ceros if i[0]>0]
t,theta,r=var('t,theta,r')
u=cos(Ceros[l]*t)*cos(n*theta)*J(Ceros[l]*r)
gra=[]
Trans = Cylindrical('height', ['radius', 'azimuth'])
z=var('z')
Trans.transform(radius=r, azimuth=theta, height=z)
aux=parametric_plot3d([-1,-1,t],(t,-1,1),opacity=0)
for j in range(50):
    gra=plot3d(u(t=2*pi/Ceros[l]*j/50),(r,0,1),(theta,0,2*pi),\
transformation=Trans,viewer='tachyon')+aux
    str2='mem-'+str(j)
    #La sentencia siguiente guarda la gráfica 
    #\texttt{gra} en el directorio MiDirectorio
    #Hay que descomentarla para que funcione.
    #gra.save('/MiDirectorio/'+str2+'.png')

\end{sageblock}

 

\end{document}
