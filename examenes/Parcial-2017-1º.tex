\documentclass[a4paper,11pt]{article}
\usepackage{latexsym}
\usepackage{graphicx}
\usepackage[activeacute,spanish]{babel}
\usepackage[latin1]{inputenc}
\usepackage{epsfig}
\usepackage{amsmath}
\usepackage{amssymb}
\usepackage{array}
\usepackage{tabularx}
\usepackage[dvips]{color}
\setlength{\headheight}{27.08pt}
\usepackage{fancyhdr} % controla los headers and footers.


%%%%%%%%%%%% TAMANHO HOJA %%%%%%%%%%%%%%%%%%%%%%%%%%%%
 \vsize=27.9cm \hsize=21cm \setlength{\textwidth}{16.5cm}
 \setlength{\textheight}{22.5cm}
 %\addtolength{\textheight}{60pt}
 \addtolength{\topmargin}{-30pt}
 %\setlength{\leftmargin}{.5cm}
 \setlength{\oddsidemargin}{0.50cm}
 %\setlength{\oddsidemargin}{1.5in}
 \setlength{\evensidemargin}{-1.2cm}
 %\setlength{\evensidemargin}{1.5in}
 \parskip6pt


\begin{document}
\date{}
\begin{center}
 {\bf \large{Ecuaciones Diferenciales}}\\
 Final   15/05/2017
\end{center}

%\textbf{Nombre y Apellido:}
%\bigskip
%
%\textbf{D.N.I.:}

\begin{enumerate}

\item Tres tanques dispuestos en cascadas contienen $50$ litros de salmuera e inicialmente hay en cada uno de ellos $10$ kg. de sal.
\begin{itemize}
\item En el tanque $1$ ingresa agua pura a raz�n de $1$ litro/min. Y sale por su parte inferior $1$ litro/min de salmuera.

\item En el tanque $2$ ingresa la salmuera que sale del tanque $1$ y por otro  tubo un litro de agua pura. A su vez de este fluye hacia afuera $2$ litros/min de salmuera.

\item En el tanque $3$ ingresa la salmuera que sale del tanque $2$ y por un conducto diferente un litro de agua pura. Y de este fluye hacia afuera $3$ litros/min de salmuera.
\end{itemize} 

Demostrar que los tres tanques contiene la misma cantidad de sal en todo momento.

\item Reslover el problema 

\begin{equation*}
\left\{ \begin{array}{cc}
-u_{x}+u_{y}=u^2 & (x,y)\in\mathbb{R}^{2}\\
u(x,0)=\frac{1}{2}e^{-x} & x\in \mathbb{R}
\end{array}
\right.
\end{equation*}

\item Demostrar que toda recta que pasa por el origen interseca a las soluciones de una misma ecuaci�n homog�nea con el mismo �ngulo.





\item Dada la ecuaci�n 
\begin{equation}\label{ec1}
M(x,y)dx+N(x,y)dy=0.
\end{equation}

\begin{itemize}

\item[a)]
 Demostrar que si $\frac{\frac{\partial N}{\partial x}-\frac{\partial M}{\partial y}}{xN-My}$ es una funci�n radial entonces la ecuaci�n \eqref{ec1} tiene un factor integrante radial.

\item[b)] Usar el �tem \textrm{a)} para resolver $(y^3+x^2y)dx+(x^3+y^2x)dy=0$.

\end{itemize}







\item  Resolver la ecuaci�n  $\frac{dy}{dx}=\frac{x y^{4}}{3} - \frac{2 y}{3 x} + \frac{1}{3 x^{3} y^{2}}$. \textbf{Ayuda:} Plantear la \emph{condici�n de simetr�a linealizada} para encontrar los infinitesimales $\xi$ y $\eta$. Hacer  el Anzats $\xi=ax+c$ y $\eta=bx+d$.



\end{enumerate}

\end{document} 