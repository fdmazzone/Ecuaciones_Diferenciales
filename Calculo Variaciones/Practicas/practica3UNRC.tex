\documentclass{article}
\usepackage{amssymb,amsmath}

\usepackage[spanish, activeacute]{babel}
\usepackage{theorem}
\usepackage{times}
\usepackage{array}
\usepackage{graphicx}
\usepackage{hyperref}
\usepackage{multirow}
\usepackage[utf8x]{inputenc}
%\usepackage[cp1252]{inputenc}
\usepackage{hhline}
\usepackage{multicol}
\usepackage{bm}
\usepackage{natbib}
\usepackage{url}
\usepackage{tabularx}
%%%%%%%%%Estilo de la pagina%%%%%%%%%%%%%%%%%%%%%%%%%%%%%%%%%%%
%%%%%%%%%%%%%%%%%%%%%%%%%%%%%%%%%%%%%%%%%%%%%%%%%%%%%%%%%%%%%%%%%%
\newcounter{ejer}

{\theorembodyfont{\rmfamily}
\newtheorem{ejercicio}[ejer]{Ejercicio}}

\newcommand{\rr}{\mathbb{R}}
\newcommand{\qq}{\mathbb{Q}}
\newcommand{\nn}{\mathbb{N}}
\newcommand{\C}[1]{\overline{#1}}
\newcommand{\zz}{\mathbb{Z}}
\newcommand\eme {\mbox{\tiny\sc M}}
\newcommand\bfm {\mbox{\tiny\sc BFM}}
\newcommand\btuk {\mbox{\tiny\sc BT}}
\newcommand\rice {\mbox{\tiny\sc R}}
\newcommand\pc {\mbox{\tiny\sc PC}}
\newcommand\gasser {\mbox{\tiny\sc G}}
\newcommand\hall {\mbox{\tiny\sc H}}
\newcommand\h {\widehat{h}_n}
\newcommand\ta {\widehat{\tau}_n}
\newcommand\ho {h_{ {\mbox{\tiny opt}},n}}
\newcommand{\cs}{  \stackrel{c.s.}\longrightarrow  }
\newcommand{\pro}{  \stackrel{p}\longrightarrow  }
\newcommand{\di}{  \stackrel{d}\longrightarrow  }
\newcommand{\luno}{  \stackrel{L_1}\longrightarrow  }
\newcommand{\ldos}{  \stackrel{L_2}\longrightarrow  }
\newcommand{\comp}{  \stackrel{c}\longrightarrow  }
\newcommand{\seno}{\hbox{sen}}





\begin{document}


\hyphenation{excen-tri-ci-dad}
 %%%%%%%%%Estilo de la pagina%%%%%%%%%%%%%%%%%%%%%%%%%%%%%%%%%%%
%%%%%%%%%%%%%%%%%%%%%%%%%%%%%%%%%%%%%%%%%%%%%%%%%%%%%%%%%%%%%%%%%%
\setlength{\unitlength}{1cm}
%
\setlength{\extrarowheight}{5mm}
%

\setlength{\extrarowheight}{-5mm}
\noindent\begin{tabular}{m{.2\textwidth} m{.8\textwidth}}\hline\hline
\medskip

\includegraphics[scale=0.3]{imagenes/unrc.jpg} &
%\begin{large}
\begin{bfseries}  \begin{scshape}
Facultad de Cs. Exactas, Físico-Químicas y Naturales\par
        Depto de Matem\'atica.\par
        Segundo Cuatrimestre de 2015\par
        Cálculo Variaciones \par

        Práctica 3: Espacios de Sobolev.
				\end{scshape}
\end{bfseries}
%\end{large}
\\
\hline\hline
\end{tabular}
\renewcommand{\theenumi}{\alph{enumi}}


\begin{ejercicio}
Sea $I=(-1,1)$ y sea $u(x)=\frac{1}{2}(|x|+x)$.
Verificar que $u\in  W^{1,p}(I)$ para todo $1\leq p\leq \infty$ con $u'=H$ siendo \[H(x)=\left\{\begin{array}{ccc}
1&si&0<x<1
\\
0&si&-1<x<0
\end{array}
\right.\]
Justificar que  $H\notin W^{1,p} $ para $1\leq p \leq \infty$.
\end{ejercicio}





\begin{ejercicio} Considerar la función $u:\mathbb{R}\to\mathbb{R}$ dada por
\[u(x)=\frac{1}{(1+x^2)^{\alpha/2} \ln(2+x^2) }.\]
Demostrar que $u\in W^{1,p}(\rr)$ si $p\geq 1/\alpha$ y que $u\notin  L^{p}(\rr)$ si $1\leq p <1/\alpha$.

\end{ejercicio}

\begin{ejercicio} \textbf{(Espacios de Lebesgue y Sobolev de funciones con valores vectoriales)} Sea $I$ un intervalo abierto de $\rr$.
\begin{enumerate}
 \item Denotamos los espacios de Lebegue de funciones con valores en $\rr^n$ por
 \[ L^p(I,\rr^n):=\left\{u:I\to\rr^n\big| |u(t)|\in L^p(I)\right\},\]
 donde $ L^p(I)$ denota el espacio de Lebesgue escalar usual y $|\cdot|:\rr^n\to\rr_+$ la norma euclidea sobre $\rr^n$. Demostrar que $L^p(I,\rr^n)$ es un espacio de Banach con norma
 \[\|u\|_{L^p(I,\rr^n)}=\|\, |u| \,\|_{L^p(I)}.\]
 \item Si $u:I\to \rr^n$ pongamos $u=(u_1,\ldots,u_n)$ con $u_i:I\to\rr$, $i=1,\ldots,n$. Demostrar que $u\in L^p(I,\rr^n)\Leftrightarrow u_i\in L^p(I)$, $i=1,\ldots,n$.
 \item $L^p(I,\rr^n)\cong L^p(I)\times\cdots\times L^p(I)$, donde $\cong $ siginifica que existe un isomorfismo homeomorfo entre los espacios.
 \item  $L^p(I,\rr^n)$ es reflexivo cuando $1<p<\infty$.

\item Denotamos por $C_c^1(I,\rr^n)$ al conjunto (espacio vectorial) de todas las funciones test $\varphi$  continuamente diferenciables y de soporte compacto en $I$.  Para $1\leq p\leq \infty$, vamos a definir el espacio de Sobolev de funciones con valores en $\rr^n$, denotado por $W^{1,p}(I,\rr^n)$ por analogía con el caso escalar, es decir

\[
\begin{split}
W^{1,p}(I,\rr^n):=\bigg\{ u\in L^p(I,\rr^n) &\bigg| \exists v\in  L^p(I,\rr^n) \forall\varphi \in C_c^1(I,\rr^n)  : \\
&\int_I u\cdot \varphi' dt=-  \int_I v\cdot \varphi dt \bigg\},
\end{split}
\]

donde $a\cdot b$ denota el producto escalar entre vectores $a,b\in\rr^n$. El espacio $W^{1,p}(I,\rr^n)$ es Banach. Justificar esta afirmación.
\end{enumerate}
\end{ejercicio}


\begin{ejercicio} \textbf{(Espacios de Sobolev de Funciones Periódicas)} Sea $T>0$ y $C_T^1$ el espacio de las funciones continuas y con derivada continua sobre $[0,T]$
que además son $T$-periódicas, esto es $\varphi(0)=\varphi(T)=0$. Definimos el espacio de Sobolev de funciones $T$ periódicas como
\[
\begin{split}
W^{1,p}_T(I):=\bigg\{ u\in L^p(I) \bigg| \exists v\in  L^p(I) \forall\varphi \in C_T^1(I)  : \int_I u\cdot \varphi' dt=-  \int_I v\cdot \varphi dt \bigg\},
\end{split}
\]
Demostrar que
 \[
W^{1,p}_T(I)=\bigg\{ u\in L^p(I) \bigg| u \text{ es absolutamente continua, }  u'\in L^p(I), u(0)=u(T)
 \bigg\}.
\]
Generalizar a funciones con valores vectoriales.

\end{ejercicio}

\begin{ejercicio}  \textbf{(Desigualdad de Poincaré-Wiertinger)} Sea $I$ el intervalo acotado $(a,b)$.  Para $u:I\to\rr$ definimos el promedio
\[\overline{u}:=\frac{1}{b-a}\int_a^bu(t)dt.\]
Demostrar la desigualdad de Wirtinger, esto es que existe una contante $C>0$ tal que para toda  $u\in W^{1,p}(I)$, $1\leq p\leq \infty$ se satisface que
\[\|u-\overline{u}\|_{L^{\infty}}\leq C\|u'\|_{L^p}.\]
Generalizar a funciones con valores vetoriales.
\emph{Ayuda: } Notar que  por el teorema del valor medio para integrales debe existir $x_0\in I$ tal que $\overline{u}=u(x_0)$.

\end{ejercicio}



\begin{ejercicio} Sea $\{u_n\}_{n\in\nn}$ una sucesión acotada de $W^{1,p}(I)$ con $1<p\leq \infty$.
 \begin{enumerate}
  \item Utilizar el \href{https://es.wikipedia.org/wiki/Teorema_de_Arzel\%C3\%A1-Ascoli}{Teorema de Arzela-Ascoli} para demostrar que existe una subsucesión $\{u_{n_k}\}_{k\in\nn}$ y $u\in W^{1,p}(I)$ tal que $\|  u_{n_k}-u\|_{L^{\infty}}\to 0$.
  \item Si $1<p<\infty$ podemos asumir además que $ u'_{n_k} \rightharpoonup u'$ debilmente en $L^p(I)$
  \item Si, en cambio,  $p=\infty$  podemos asumir que $ u'_{n_k} \overset{*}{\rightharpoonup} u'$ en $\sigma(L^{\infty},L^1)$ .
 \end{enumerate}
\end{ejercicio}

\begin{ejercicio}  Sea $\varphi\in C_c^{\infty}(\rr)$ con $\varphi\not\equiv 0$. Definimos $u_n(x)=\varphi(x+n)$. Supongamos $1\leq p\leq \infty$. Demostrar que
 \begin{enumerate}
  \item $\{u_n\}_{n\in\nn}$ es acotada en $W^{1,p}(\rr)$.
  \item No exixte subsucesión que converge en la topología fuerte en  $L^q(\rr)$ para ningun  $1\leq q\leq \infty$.
   \item $ u'_{n} \rightharpoonup 0$ debilmente en $W^{1,p}(\rr)$.
 \end{enumerate}
\end{ejercicio}

\begin{ejercicio} Sea $1\leq p<\infty$ y $u\in W^{1,p}(\rr)$. Escribamos
\[D_hu(x)=\frac{u(x+h)-u(x)}{h},\quad\text{para } h\neq 0, x\in\rr.\]
Usar el hecho que $C_c^1(\rr)$ es denso en $W^{1,p}(\rr)$ para demostrar que $D_hu\to u'$ en $L^p(\rr)$ cuando $h\to 0$.
\end{ejercicio}












\end{document}
