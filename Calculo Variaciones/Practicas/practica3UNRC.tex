\documentclass{article}
\usepackage{amssymb,amsmath}

\usepackage[spanish, activeacute]{babel}
\usepackage{theorem}
\usepackage{times}
\usepackage{array}
\usepackage{graphicx}
\usepackage{hyperref}
\usepackage{multirow}
\usepackage[utf8x]{inputenc}
%\usepackage[cp1252]{inputenc}
\usepackage{hhline}
\usepackage{multicol}
\usepackage{bm}
\usepackage{natbib}
\usepackage{url}
\usepackage{tabularx}
%%%%%%%%%Estilo de la pagina%%%%%%%%%%%%%%%%%%%%%%%%%%%%%%%%%%%
%%%%%%%%%%%%%%%%%%%%%%%%%%%%%%%%%%%%%%%%%%%%%%%%%%%%%%%%%%%%%%%%%%
\newcounter{ejer}

{\theorembodyfont{\rmfamily}
\newtheorem{ejercicio}[ejer]{Ejercicio}}

\newcommand{\rr}{\mathbb{R}}
\newcommand{\qq}{\mathbb{Q}}
\newcommand{\nn}{\mathbb{N}}
\newcommand{\C}[1]{\overline{#1}}
\newcommand{\zz}{\mathbb{Z}}
\newcommand\eme {\mbox{\tiny\sc M}}
\newcommand\bfm {\mbox{\tiny\sc BFM}}
\newcommand\btuk {\mbox{\tiny\sc BT}}
\newcommand\rice {\mbox{\tiny\sc R}}
\newcommand\pc {\mbox{\tiny\sc PC}}
\newcommand\gasser {\mbox{\tiny\sc G}}
\newcommand\hall {\mbox{\tiny\sc H}}
\newcommand\h {\widehat{h}_n}
\newcommand\ta {\widehat{\tau}_n}
\newcommand\ho {h_{ {\mbox{\tiny opt}},n}}
\newcommand{\cs}{  \stackrel{c.s.}\longrightarrow  }
\newcommand{\pro}{  \stackrel{p}\longrightarrow  }
\newcommand{\di}{  \stackrel{d}\longrightarrow  }
\newcommand{\luno}{  \stackrel{L_1}\longrightarrow  }
\newcommand{\ldos}{  \stackrel{L_2}\longrightarrow  }
\newcommand{\comp}{  \stackrel{c}\longrightarrow  }
\newcommand{\seno}{\hbox{sen}}





\begin{document}


\hyphenation{excen-tri-ci-dad}
 %%%%%%%%%Estilo de la pagina%%%%%%%%%%%%%%%%%%%%%%%%%%%%%%%%%%%
%%%%%%%%%%%%%%%%%%%%%%%%%%%%%%%%%%%%%%%%%%%%%%%%%%%%%%%%%%%%%%%%%%
\setlength{\unitlength}{1cm}
%
\setlength{\extrarowheight}{5mm}
%

\setlength{\extrarowheight}{-5mm}
\noindent\begin{tabular}{m{.2\textwidth} m{.8\textwidth}}\hline\hline
\medskip

\includegraphics[scale=0.3]{imagenes/unrc.jpg} &
%\begin{large}
\begin{bfseries}  \begin{scshape}
Facultad de Cs. Exactas, Físico-Químicas y Naturales\par
        Depto de Matem\'atica.\par
        Segundo Cuatrimestre de 2015\par
        Cálculo Variaciones \par

        Práctica 3: Espacios de Sobolev.
				\end{scshape}
\end{bfseries}
%\end{large}
\\
\hline\hline
\end{tabular}
\renewcommand{\theenumi}{\alph{enumi}}


\begin{ejercicio}
Sea $I=(-1,1)$ y sea $u(x)=\frac{1}{2}(|x|+x)$.
Verificar que $u\in  W^{1,p}(I)$ para todo $1\leq p\leq \infty$ con $u'=H$ siendo \[H(x)=\left\{\begin{array}{ccc}
1&si&0<x<1
\\
0&si&-1<x<0
\end{array}
\right.\]
Justificar que  $H\notin W^{1,p} $ para $1\leq p \leq \infty$.
\end{ejercicio}




\begin{ejercicio}
Sea $G$ funci\'on de Lipschitz  tal que $G(0)=0$ y sea $u \in W^{1,p}$. Entonces
$G\circ u \in W^{1,p}$ y $\dot{(G\circ u)}(\dot{G}\circ u)\dot {u}$.
\end{ejercicio}

\begin{ejercicio} Considerar la función $u:\mathbb{R}\to\mathbb{R}$ dada por
\[u(x)=\frac{1}{(1+x^2)^{\alpha/2} \ln(2+x^2) }.\]
Demostrar que $u\in W^{1,p}(\rr)$ si $p\geq 1/\alpha$ y que $u\notin  L^{p}(\rr)$ si $1\leq p <1/\alpha$.

\end{ejercicio}


\begin{ejercicio} Sea $\{u_n\}_{n\in\nn}$ una sucesión acotada de $W^{1,p}(I)$ con $1<p\leq \infty$.
 \begin{enumerate}
  \item Utilizar el \href{https://es.wikipedia.org/wiki/Teorema_de_Arzel\%C3\%A1-Ascoli}{Teorema de Arzela-Ascoli} para demostrar que existe una subsucesión $\{u_{n_k}\}_{k\in\nn}$ y $u\in W^{1,p}(I)$ tal que $\|  u_{n_k}-u\|_{L^{\infty}}\to 0$.
  \item Si $1<p<\infty$ podemos asumir además que $ u'_{n_k} \rightharpoonup u'$ debilmente en $L^p(I)$
  \item Si, en cambio,  $p=\infty$  podemos asumir que $ u'_{n_k} \overset{*}{\rightharpoonup} u'$ en $\sigma(L^{\infty},L^1)$ .
 \end{enumerate}
\end{ejercicio}

\begin{ejercicio}  Sea $\varphi\in C_c^{\infty}(\rr)$ con $\varphi\not\equiv 0$. Definimos $u_n(x)=\varphi(x+n)$. Supongamos $1\leq p\leq \infty$. Demostrar que
 \begin{enumerate}
  \item $\{u_n\}_{n\in\nn}$ es acotada en $W^{1,p}(\rr)$.
  \item No exixte subsucesión que converge en la topología fuerte en  $L^q(\rr)$ para ningun  $1\leq q\leq \infty$.
   \item $ u'_{n} \rightharpoonup 0$ debilmente en $W^{1,p}(\rr)$.
 \end{enumerate}


\begin{ejercicio} Sea $1\leq p<\infty$ y $u\in W^{1,p}(\rr)$. Escribamos
\[D_hu(x)=\frac{u(x+h)-u(x)}{h},\quad\text{para } h\neq 0, x\in\rr.\]
Usar el hecho que $C_c^1(\rr)$ es denso en $W^{1,p}(\rr)$ para demostrar que $D_hu\to u'$ en $L^p(\rr)$ cuando $h\to 0$.
\end{ejercicio}






\end{ejercicio}




\end{document}
