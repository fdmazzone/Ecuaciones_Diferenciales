\documentclass{article}
\usepackage{amssymb,amsmath}

\usepackage[spanish, activeacute]{babel}
\usepackage{theorem}
\usepackage{times}
\usepackage{array}
\usepackage{graphicx}
\usepackage{hyperref}
\usepackage{multirow}
\usepackage[utf8x]{inputenc}
%\usepackage[cp1252]{inputenc}
\usepackage{hhline}
\usepackage{multicol}
\usepackage{bm}
\usepackage{natbib}
\usepackage{url}
\usepackage{tabularx}
%%%%%%%%%Estilo de la pagina%%%%%%%%%%%%%%%%%%%%%%%%%%%%%%%%%%%
%%%%%%%%%%%%%%%%%%%%%%%%%%%%%%%%%%%%%%%%%%%%%%%%%%%%%%%%%%%%%%%%%%
\newcounter{ejer}

{\theorembodyfont{\rmfamily}
\newtheorem{ejercicio}[ejer]{Ejercicio}}

\newcommand{\rr}{\mathbb{R}}
\newcommand{\qq}{\mathbb{Q}}
\newcommand{\nn}{\mathbb{N}}
\newcommand{\C}[1]{\overline{#1}}
\newcommand{\zz}{\mathbb{Z}}
\newcommand\eme {\mbox{\tiny\sc M}}
\newcommand\bfm {\mbox{\tiny\sc BFM}}
\newcommand\btuk {\mbox{\tiny\sc BT}}
\newcommand\rice {\mbox{\tiny\sc R}}
\newcommand\pc {\mbox{\tiny\sc PC}}
\newcommand\gasser {\mbox{\tiny\sc G}}
\newcommand\hall {\mbox{\tiny\sc H}}
\newcommand\h {\widehat{h}_n}
\newcommand\ta {\widehat{\tau}_n}
\newcommand\ho {h_{ {\mbox{\tiny opt}},n}}
\newcommand{\cs}{  \stackrel{c.s.}\longrightarrow  }
\newcommand{\pro}{  \stackrel{p}\longrightarrow  }
\newcommand{\di}{  \stackrel{d}\longrightarrow  }
\newcommand{\luno}{  \stackrel{L_1}\longrightarrow  }
\newcommand{\ldos}{  \stackrel{L_2}\longrightarrow  }
\newcommand{\comp}{  \stackrel{c}\longrightarrow  }
\newcommand{\seno}{\hbox{sen}}





\begin{document}


\hyphenation{excen-tri-ci-dad}
 %%%%%%%%%Estilo de la pagina%%%%%%%%%%%%%%%%%%%%%%%%%%%%%%%%%%%
%%%%%%%%%%%%%%%%%%%%%%%%%%%%%%%%%%%%%%%%%%%%%%%%%%%%%%%%%%%%%%%%%%
\setlength{\unitlength}{1cm}
%
\setlength{\extrarowheight}{5mm}
%

\setlength{\extrarowheight}{-5mm}
\noindent\begin{tabular}{m{.1\textwidth} m{.9\textwidth}}\hline\hline
\includegraphics[scale=.4]{imagenes/unrc.jpg} &
%\begin{large}
\begin{bfseries}  \begin{scshape}
Facultad de Cs. Exactas, Físico-Químicas y Naturales\par
        Depto de Matem\'atica.\par
        Segundo Cuatrimestre de 2015\par
        Cálculo Variaciones \par
        Práctica 2: Funciones absolutamente continuas y de variación acotada.

\end{scshape}
\end{bfseries}
%\end{large}
\\
\hline\hline
\end{tabular}
\renewcommand{\theenumi}{\alph{enumi}}





\begin{ejercicio} Supongamos que $f$ y $f$ son de variación acotada sobre $[a,b]$ y $a,b\in\rr$. Entonces $af+bg$ y $fg$ son de variación acotada en $[a,b]$. Si $g\geq \epsilon$ en $[a,b]$ para algún $\epsilon>0$ entonces $f/g$ es de variación acotada. ¿Se puede reemplazar  $g\geq \epsilon$ por  $g> 0$?
\end{ejercicio}

\begin{ejercicio} Demostrar que si $V_a^bf<\infty$ entonces
 \[V_a^bf=V_a^{b+}f + V_a^{b-}f\quad\hbox{y}\quad f(b)-f(a)=V_a^{b+}f -V_a^{b-}f\ .\]
\end{ejercicio}

\begin{ejercicio} Se define $f:[0,1]\to\rr$ de la siguiente forma:
 \[f(0)=0,\quad f\left(\frac{1}{2n+1}\right)=0\quad\text{y}\quad f\left(\frac{1}{2n}\right)=\frac{1}{2n},\quad n\in\nn,\]
 y en cada intervalo de la forma $[\frac{1}{n+1}, \frac{1}{n}]$ la función $f$ es lineal. Demostrar que $f$ no es de variación acotada.
\end{ejercicio}

\begin{ejercicio} Demostrar que
\[f(x)=\left\{\begin{array}{ll}
        x^2\cos\left(\frac{\pi}{x}\right) &\text{si } x\in [0,1]\\
        0                                 &\text{si } x=0,
       \end{array}
       \right.
\]
es de variación acotada.
\end{ejercicio}

\begin{ejercicio} Supongamos que la sucesión de funciones $f_n:[a,b]\to\rr$ converge puntualmente a $f$ cuando $n\to\infty$. Demostrar que
\[V_a^bf\leq \liminf_{n\to\infty}V_a^bf_n.\]
Demostrar que no vale en general la igualdad en desigualdad anterior. Para ello considerar la sucesión $f_n:[0,1]\to\rr$ definida por
\[f(x)=\left\{\begin{array}{ll}
        0               &\text{si } x=0\\
        1                &\text{si } x\in\left(0,\frac{1}{n}\right]\\
        0                             &\text{si } x\in\left(\frac{1}{n}, 1\right]\\
       \end{array}
       \right.
\]


\end{ejercicio}

\begin{ejercicio} Computar las derivadas de Dini $D^+$ y $D^-$ para la función del primer ejercicio.
 \end{ejercicio}

 \begin{ejercicio}Si $f$ tiene un máximo en $c\in (a,b)$ entonces $D^-f(c)\geq 0$.
   \end{ejercicio}

   \begin{ejercicio} Supongamos  $f:[a,b]\to\rr$ absolutamente continua. Entonces $f$ es Lipschitz en $[a,b]$ si y sólo si $f'\in L^{\infty}([a,b])$.

   \end{ejercicio}

   \begin{ejercicio} Demostrar que la función de Cantor es continua pero no absolutamente continua.

   \end{ejercicio}


\begin{ejercicio} Demostrar que si $f,g:[a,b]\to\rr$ son absolutamente continuas en $[a,b]$ entonces vale la fórmula integral por partes
 \[\int_a^bfg'dx=fg\Big|_a^b-\int_a^bf'gdx.\]
\end{ejercicio}






\end{document}
