\documentclass{article}
\usepackage{amssymb,amsmath}

\usepackage[spanish, activeacute]{babel}
\usepackage{theorem}
\usepackage{times}
\usepackage{array}
\usepackage{graphicx}
\usepackage{hyperref}
\usepackage{multirow}
\usepackage[utf8x]{inputenc}
%\usepackage[cp1252]{inputenc}
\usepackage{hhline}
\usepackage{multicol}
\usepackage{bm}
\usepackage{natbib}
\usepackage{url}
\usepackage{tabularx}
%%%%%%%%%Estilo de la pagina%%%%%%%%%%%%%%%%%%%%%%%%%%%%%%%%%%%
%%%%%%%%%%%%%%%%%%%%%%%%%%%%%%%%%%%%%%%%%%%%%%%%%%%%%%%%%%%%%%%%%%
\newcounter{ejer}

{\theorembodyfont{\rmfamily}
\newtheorem{ejercicio}[ejer]{Ejercicio}}

\newcommand{\rr}{\mathbb{R}}
\newcommand{\qq}{\mathbb{Q}}
\newcommand{\nn}{\mathbb{N}}
\newcommand{\C}[1]{\overline{#1}}
\newcommand{\zz}{\mathbb{Z}}
\newcommand\eme {\mbox{\tiny\sc M}}
\newcommand\bfm {\mbox{\tiny\sc BFM}}
\newcommand\btuk {\mbox{\tiny\sc BT}}
\newcommand\rice {\mbox{\tiny\sc R}}
\newcommand\pc {\mbox{\tiny\sc PC}}
\newcommand\gasser {\mbox{\tiny\sc G}}
\newcommand\hall {\mbox{\tiny\sc H}}
\newcommand\h {\widehat{h}_n}
\newcommand\ta {\widehat{\tau}_n}
\newcommand\ho {h_{ {\mbox{\tiny opt}},n}}
\newcommand{\cs}{  \stackrel{c.s.}\longrightarrow  }
\newcommand{\pro}{  \stackrel{p}\longrightarrow  }
\newcommand{\di}{  \stackrel{d}\longrightarrow  }
\newcommand{\luno}{  \stackrel{L_1}\longrightarrow  }
\newcommand{\ldos}{  \stackrel{L_2}\longrightarrow  }
\newcommand{\comp}{  \stackrel{c}\longrightarrow  }
\newcommand{\seno}{\hbox{sen}}





\begin{document}


\hyphenation{excen-tri-ci-dad}
 %%%%%%%%%Estilo de la pagina%%%%%%%%%%%%%%%%%%%%%%%%%%%%%%%%%%%
%%%%%%%%%%%%%%%%%%%%%%%%%%%%%%%%%%%%%%%%%%%%%%%%%%%%%%%%%%%%%%%%%%
\setlength{\unitlength}{1cm}
%
\setlength{\extrarowheight}{5mm}
%

\setlength{\extrarowheight}{-5mm}
\noindent\begin{tabular}{m{.2\textwidth} m{.8\textwidth}}\hline\hline
\medskip

\includegraphics[scale=0.3]{imagenes/unrc.jpg} &
%\begin{large}
\begin{bfseries}  \begin{scshape}
Facultad de Cs. Exactas, Físico-Químicas y Naturales\par
        Depto de Matem\'atica.\par
        Segundo Cuatrimestre de 2015\par
        Cálculo Variaciones \par

        Práctica 5: Teoremas de Existencia
				\end{scshape}
\end{bfseries}
%\end{large}
\\
\hline\hline
\end{tabular}
\renewcommand{\theenumi}{\alph{enumi}}

\begin{ejercicio} Sea $F:[a,b]\times \rr^n\to\rr$ una función tal que $F(\cdot,z)$ es medible para todo $z\in\rr^n$ y que $F(x,\cdot)$ es continuamente diferenciable para casi todo $x\in [a,b]$. Además supongamos que existe $a:[0,+\infty)\to [0,+\infty)$ continua y $0\leq b\in  L^1([a,b],\rr)$ tal que
\[|F(x,z)|+|D_{z}F(x,z)|\leq a(|z|)b(x).\]
Demostrar que para $1< p<\infty$ la integral de acción
\[I(u)=\int_a^b\frac{|u'(x)|^p}{p}+F(x,u(x))dx\]
satisface:
\begin{enumerate}
 \item $I$ es d.s.s.c.i. \emph{Ayuda:}   la intregral es suma de una funcional convexa y continua en la norma, y en el otro sumando se puede usar que $u_n\rightharpoonup u$ en $W^{1,p}((a,b),\rr^n)$ implica $u_n\to u$ uniformemente cuando $1<p$.

 \item Demostrar que $I$ es diferenciable G\^ateaux y expresar la diferencial de $I$.

 \item Supongamos que $I$ tiene un punto crítico $u$ sobre
 \[\{u\in W^{1,p}((a,b),\rr^n)| u(a)=\alpha, u(b)=\beta\},\]
 con $1<p$. Demostrar que $u$ satisface las ecuaciones de  Euler-Lagrange, que en este caso son
 \begin{equation}\label{ProbPrin}
    \left\{%
\begin{array}{ll}
   \frac{d}{dx}\big(|u(x)|^{p-2}u(x)\big)= D_zF(x,u(t)) \quad \hbox{a.e.}\ x \in (a,b)\\
    u(a)=\alpha\text{ y } u(b)=\beta\\
\end{array}%
\right.
\end{equation}

 \end{enumerate}


\end{ejercicio}




\begin{ejercicio} Supongamos que $L:(a,b)\times\rr^n\times\rr^n\to\rr$ satisface
\begin{enumerate}
 \item $L(x,u,p)$, $L_p(x,u,p)$ son continuas.
 \item $L(x,u,p)$ convexa respecto a $p$.
 \item \[L(x,u,p)\geq \theta(p)+a(x),\]
 donde $a\in L^1([a,b],\rr)$ y $\theta$ es super lineal. Demostrar que existe un mínimo de la integral de acción sobre el conjunto $\{u\in W^{1,1}((a,b),\rr^n)| u(a)=\alpha, u(b)=\beta\}$
\end{enumerate}


\end{ejercicio}



\begin{ejercicio} Encontrar condiciones suficientes sobre $V:(a,b)\to\rr$ para que
\[I(u)=\int_a^b\frac{|p|^2}{2}+V(x)dx\]
satisfaga el Teorema de existencia de Tonelli de modo que el problema
\[\min\left\{I(u)\bigg| u\in W^{1,1}((a,b),\rr), u(a)=\alpha, u(b)=\beta\right\}\]
tenga solución. Encontrar condiciones  sobre $V$ que garanticen que $I$ es diferenciable G\^ateaux y hallar las ecuaciones de Euler-Lagrange que satisface un mínimo.

\end{ejercicio}

\begin{ejercicio} Demostrar la siguiente afirmación. $L:(a,b)\times\rr^n\times\rr^n\to\rr$ es cuasiconvexa si y solo si para $(x_0,u_0)\in [a,b]\times\rr^n$ fijos la funcional
\[\int_a^bL(t_0,u_0,\phi'(x))dx,\]
alcanza un mínimo en $u$ sobre el conjunto
\[\left\{\phi\in W^{1,\infty}((a,b),\rr^n)| \phi(a)=\alpha, \phi(b)=\beta\right\},\]
si y solo si $u$ es la recta que une $\alpha$ con $\beta$. Una de las implicaciones fue demostrada durante las clases teóricas.

\end{ejercicio}



\end{document}
