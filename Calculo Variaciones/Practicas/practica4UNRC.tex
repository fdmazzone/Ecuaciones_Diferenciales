\documentclass{article}
\usepackage{amssymb,amsmath}

\usepackage[spanish, activeacute]{babel}
\usepackage{theorem}
\usepackage{times}
\usepackage{array}
\usepackage{graphicx}
\usepackage{hyperref}
\usepackage{multirow}
\usepackage[utf8x]{inputenc}
%\usepackage[cp1252]{inputenc}
\usepackage{hhline}
\usepackage{multicol}
\usepackage{bm}
\usepackage{natbib}
\usepackage{url}
\usepackage{tabularx}
%%%%%%%%%Estilo de la pagina%%%%%%%%%%%%%%%%%%%%%%%%%%%%%%%%%%%
%%%%%%%%%%%%%%%%%%%%%%%%%%%%%%%%%%%%%%%%%%%%%%%%%%%%%%%%%%%%%%%%%%
\newcounter{ejer}

{\theorembodyfont{\rmfamily}
\newtheorem{ejercicio}[ejer]{Ejercicio}}

\newcommand{\rr}{\mathbb{R}}
\newcommand{\qq}{\mathbb{Q}}
\newcommand{\nn}{\mathbb{N}}
\newcommand{\C}[1]{\overline{#1}}
\newcommand{\zz}{\mathbb{Z}}
\newcommand\eme {\mbox{\tiny\sc M}}
\newcommand\bfm {\mbox{\tiny\sc BFM}}
\newcommand\btuk {\mbox{\tiny\sc BT}}
\newcommand\rice {\mbox{\tiny\sc R}}
\newcommand\pc {\mbox{\tiny\sc PC}}
\newcommand\gasser {\mbox{\tiny\sc G}}
\newcommand\hall {\mbox{\tiny\sc H}}
\newcommand\h {\widehat{h}_n}
\newcommand\ta {\widehat{\tau}_n}
\newcommand\ho {h_{ {\mbox{\tiny opt}},n}}
\newcommand{\cs}{  \stackrel{c.s.}\longrightarrow  }
\newcommand{\pro}{  \stackrel{p}\longrightarrow  }
\newcommand{\di}{  \stackrel{d}\longrightarrow  }
\newcommand{\luno}{  \stackrel{L_1}\longrightarrow  }
\newcommand{\ldos}{  \stackrel{L_2}\longrightarrow  }
\newcommand{\comp}{  \stackrel{c}\longrightarrow  }
\newcommand{\seno}{\hbox{sen}}





\begin{document}


\hyphenation{excen-tri-ci-dad}
 %%%%%%%%%Estilo de la pagina%%%%%%%%%%%%%%%%%%%%%%%%%%%%%%%%%%%
%%%%%%%%%%%%%%%%%%%%%%%%%%%%%%%%%%%%%%%%%%%%%%%%%%%%%%%%%%%%%%%%%%
\setlength{\unitlength}{1cm}
%
\setlength{\extrarowheight}{5mm}
%

\setlength{\extrarowheight}{-5mm}
\noindent\begin{tabular}{m{.2\textwidth} m{.8\textwidth}}\hline\hline
\medskip

\includegraphics[scale=0.3]{imagenes/unrc.jpg} &
%\begin{large}
\begin{bfseries}  \begin{scshape}
Facultad de Cs. Exactas, Físico-Químicas y Naturales\par
        Depto de Matem\'atica.\par
        Segundo Cuatrimestre de 2015\par
        Cálculo Variaciones \par

        Práctica 4: Funcionales sobre Espacios de Banach .
				\end{scshape}
\end{bfseries}
%\end{large}
\\
\hline\hline
\end{tabular}
\renewcommand{\theenumi}{\alph{enumi}}


\begin{ejercicio} Sea $X$ un espacio de Banach
e $I:X\to (-\infty,\infty]$ una función convexa y acotada por superiormente  en
un entorno del punto $a\in X$.  Entonces  $I$ es continua en $a$.
\end{ejercicio}
\begin{ejercicio} Sea $X$ un espacio de Banach. Una funcional $I:X\to (-\infty,\infty]$ se llama \emph{estrictamente convexa} si
\[I(\lambda x+(1-\lambda)y)\leq \lambda I(x)+(1-\lambda)I(y),\quad\forall\lambda\in(0,1)\forall x,y\in X.\]
Demostrar que una función extrictamente convexa alcanza a lo sumo un punto mínimo.

\end{ejercicio}

\begin{ejercicio}
Sea $X$ un espacio de Banach
e $I:X\to (-\infty,\infty]$ una función convexa  y semicontinua inferiormente en la topología fuerte. Entonces $I$ es coercitiva ($I(u)\to\infty$ cuando $\|u\|\to\infty$) si y sólo si existen $\alpha>0$ y $\beta\geq 0$ tales que
\[I(u)\geq \alpha \|u\|-\beta.\]
\end{ejercicio}

\begin{ejercicio} Sea $X$ un espacio de topológico.
\begin{enumerate}
 \item  Si $I_1,I_2:X\to (-\infty,\infty]$ son convexas, o s.c.i., o s.s.c.i. y $\alpha\geq 0$ entonces $I_1+I_2$ y $\alpha I_1$ son convexas, o s.c.i., o s.s.c.i. respectivamente.
 \item Si $\{I_{\lambda}\}_{\lambda\in\Lambda}$ es un familía, posiblemente infinita, de funciones convexas, o s.c.i, o s.s.c.i. entonces
 \[\sup_{\lambda\in\Lambda}I_{\lambda}, \]
es convexa, o s.c.i., o s.s.c.i. respectivamente.

\end{enumerate}

\end{ejercicio}



\begin{ejercicio} Sea $X$ un espacio topológico. Una función $I:X\to (-\infty,\infty]$  es s.c.i si y sólo si $\{u\in X|I(u)\leq \alpha\}$ es cerrado para todo $\alpha\in\rr$.

\end{ejercicio}

\begin{ejercicio} Si $I:X\to (-\infty,\infty]$  es convexa, entonces $\{u\in X|I(u)\leq \alpha\}$ es convexo para todo $\alpha\in\rr$. Demostrar con un ejemplo que el recíproco no es en general cierto.

\end{ejercicio}

\begin{ejercicio}  Si $I:X\to (-\infty,\infty]$  es convexa, o si es  s.c.i.,  entonces el conjunto donde $I$ alcanza un mínimo es convexo, respectivamente cerrado, para todo $\alpha\in\rr$.
 \end{ejercicio}

\begin{ejercicio}   Si $I:X\to (-\infty,\infty]$  es convexa, entonces un mínimo local es mínimo global.

\end{ejercicio}


\begin{ejercicio} Sean $X,Y$ espacios de Banach y $F:X\to Y$ una función. Se dice que $F$ es \emph{diferenciable según Fréchet} en $u\in X$ si existe un operador lineal acotado $dF(u):X\to Y$ tal que
\[\lim_{\|v\|_X\to 0}\frac{\|F(u+v)-F(u)-dF(u)(v)\|_Y}{\|v\|_X}=0.\]
Demostrar que si $F:X\to\rr$ es diferenciable Fréchet entonces es diferenciable G\^ateaux. Demostrar, con un ejemplo de una  $F:\rr^2\to\rr$, que el recíproco no es cierto.
\end{ejercicio}

\end{document}
