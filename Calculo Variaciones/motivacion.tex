%\documentclass{article}
%\documentclass[hyperref={colorlinks=true}]{beamer}
\documentclass[handout,hyperref={colorlinks=true}]{beamer}

\usecolortheme{crane}
\usetheme{Darmstadt}
\usefonttheme{structuresmallcapsserif}
%%%%%%%%%%%%%%%%%%%%%%%%%%%%%%Paquetes%%%%%%%%%%%%%%%%%%%%%%%%%%%%%%%%%%%%%%%%%%%%%%%5
%%%%%%%%%%%%%%%%%%%%%%%%%%%%%%%%%%%%%%%%%%%%%%%%%%%%%%%%%%%%%%%%%%%%%%%%%%%%%%%%%%%%%
%\usepackage{pgfpages}
%\pgfpagesuselayout{2 on 1}[a4paper,border shrink=5mm]
\usepackage{empheq}
\usepackage[spanish]{babel}
\usepackage[utf8x]{inputenc}
\usepackage{times}
%\usepackage[T1]{fontenc}
\usepackage{amssymb,amsmath}
\usepackage{enumerate}
\usepackage{verbatim}
\usepackage{ esint }
%\usepackage{pst-all}
%\usepackage{pstricks-add}
\usepackage{array}
%\usepackage[T1]{fontenc}
%\usepackage{animate}
%\usepackage{media9}
\usepackage{xparse}
\usepackage{listings}
\usepackage{ wasysym }
%\usepackage{sagetex}
\usepackage{yfonts,mathrsfs,eufrak}
\usepackage{hyperref}

%%%%%%%%%%%%%%%%%%%%%%%%%Configuracion listing
\lstdefinelanguage{Sage}[]{Python}
{morekeywords={False,sage,True},sensitive=true}
\lstset{
frame=none,
showtabs=False,
showspaces=False,
showstringspaces=False,
commentstyle={\ttfamily\color{dgreencolor}},
keywordstyle={\ttfamily\color{dbluecolor}\bfseries},
stringstyle={\ttfamily\color{dgraycolor}\bfseries},
language=Sage,
basicstyle={\fontsize{8pt}{8pt}\ttfamily},
aboveskip=.3em,
belowskip=0.1em,
numbers=none,
numberstyle=\footnotesize
}

\AtBeginSection[]
{
  \begin{frame}
    \frametitle{Índice}
    \tableofcontents[currentsection]
  \end{frame}
}


%%%%%%%%%%%%%%%%%%%%%%%%Colores
\definecolor{myblue}{rgb}{.8, .8, 1}
\definecolor{dblackcolor}{rgb}{0.0,0.0,0.0}
\definecolor{dbluecolor}{rgb}{0.01,0.02,0.7}
\definecolor{dgreencolor}{rgb}{0.2,0.4,0.0}
\definecolor{dgraycolor}{rgb}{0.30,0.3,0.30}
\newcommand{\dblue}{\color{dbluecolor}\bf}
\newcommand{\dred}{\color{dredcolor}\bf}
\newcommand{\dblack}{\color{dblackcolor}\bf}
%%%%%%%%%%%%%%%%%%%%%%%%%%Nuevos comandos entornos%%%%%%%%%%%%%%%%%%%%%%%%%%%%%%%%
%%%%%%%%%%%%%%%%%%%%%%%%%%%%%%%%%%%%%%%%%%%%%%%%%%%%%%%%%%%%%%%%%%%%%%%%
\newenvironment{demo}{\noindent\emph{Dem.}}{$\square$ \newline\vspace{5pt}}
\newcommand{\com}{\mathbb{C}}
\newcommand{\dis}{\mathbb{D}}
\newcommand{\rr}{\mathbb{R}}
\newcommand{\oo}{\mathcal{O}}
\renewcommand{\emph}[1]{\textcolor[rgb]{1,0,0}{#1}}
\newcommand{\der}[2]{\frac{\partial #1}{\partial #2}}
\renewcommand{\v}[1]{\overrightarrow{#1}}
\renewcommand{\b}[1]{\boldsymbol{#1}}
\renewcommand{\epsilon}{\varepsilon}
\newlength\mytemplen
\newsavebox\mytempbox
\makeatletter
\newcommand\mybluebox{%
\@ifnextchar[%]
{\@mybluebox}%
{\@mybluebox[0pt]}}
\def\@mybluebox[#1]{%
\@ifnextchar[%]
{\@@mybluebox[#1]}%
{\@@mybluebox[#1][0pt]}}
\def\@@mybluebox[#1][#2]#3{
\sbox\mytempbox{#3}%
\mytemplen\ht\mytempbox
\advance\mytemplen #1\relax
\ht\mytempbox\mytemplen
\mytemplen\dp\mytempbox
\advance\mytemplen #2\relax
\dp\mytempbox\mytemplen
\colorbox{myblue}{\hspace{1em}\usebox{\mytempbox}\hspace{1em}}}
\makeatother
\DeclareDocumentCommand\boxedeq{ m g }{%
{\begin{empheq}[box={\mybluebox[2pt][2pt]}]{equation}% #1%
\IfNoValueF {#2} {\label{#2}}%
#1
\end{empheq}
}%
}
\DeclareMathOperator{\atan2}{atan2}
\DeclareMathOperator{\sen}{sen}
\newtheorem{teorema}{Teorema}[section]
\newtheorem{lema}[teorema]{Lema}
\newtheorem{corolario}[teorema]{Corolario}
\newtheorem{proposicion}[teorema]{Proposici\'on}
\newtheorem{definicion}[teorema]{Definici\'on}
%%%%%%%%%%%%%%%%%%%%%%%%%%%%%%%%%%%%%%%%%%%%%%%%%%%%%%%%%%%%%%%%%%%%%%%%%%%%%%%%%%%%%%%%%%%%%%%%%%%%%%%%%%%
%%%%%%%%%%Para escibir en clase articulo o similar
% \usepackage{color}
% \newcommand{\nl}{ }
% \renewenvironment{frame}[1]{}{}
% \newcommand{\qed}{$\square$}
% %\newcommand{\defverbatim}{\def{#1}}
% \newenvironment{block}[1]{\textbf{#1}}{}
% \title{Ecuaciones lineales de segundo orden}
% \author{Fernando Mazzone}
%
%%%%%%%%%%%%%%%%%%%%%%%Para clase beamer
\newcommand{\nl}{\onslide<+-> }

\title[Cálculo de variaciones] % (optional, nur bei langen Titeln nötig)
{%
Problemas del cálculo de variaciones
}



\author[] % (optional, nur bei vielen Autoren)
{Fernando Mazzone}

\institute[Depto de Matemática] % (optional, aber oft nötig)
{
 Dpto de Matemática\\
Facultad de Ciencias Exactas Físico-Químicas y Naturales\\
Universidad Nacional de Río Cuarto
 Dpto de Matemática\\
Facultad de Ciencias Exactas y Naturales\\
Universidad Nacional de La Pampa\\
CONICET}


\subject{Cálculo de Variaciones}

%%%%%%%%%%%%%%%%%%%%%%%%%%%%%%%%%%%%%%%%%%%%%%%%%%%%%%%%%%%%%%%%%%%%%%%%%%%%%%%%%%%%%%


\begin{document}
\begin{frame}
  \maketitle
  \begin{center}
   \includegraphics[scale=0.2]{imagenes/unrc.jpg}
   \end{center}
\end{frame}
\begin{frame}
    \frametitle{Índice}
\tableofcontents

\end{frame}

\section{Cálculo de variaciones y mecánica}
\begin{frame}{Ecuaciones de Newton}


\textbf{Sistema mecánico:} $n$-puntos masa en un espacio euclideano tridimensional. Supuesto un sistema de coordenadas cartesiano, sean $\b{x}_1,\b{x}_2,\ldots,\b{x}_n\in\rr^3$ las coordenadas de los puntos masa, $\b{x}_i=(x_{i,1},x_{i,2},x_{i,3})$, $i=1,\ldots,n$. Vamos a poner $\b{x}= (\b{x}_1,\b{x}_2,\ldots,\b{x}_n)\in\rr^{3n}$.

\textbf{Fuerzas:}  Supongamos que actúan fuerzas $\b{f}_i=\b{f}_i(t,\b{x}(t),\b{\dot{x}}(t))$ sobre cada masa $m_i$.

\begin{block}{Leyes de movimiento de Newton} Suponiendo que el sistema satisface la \href{https://es.wikipedia.org/wiki/Leyes_de_Newton\#Segunda_ley_de_Newton_o_ley_de_fuerza}{segunda ley de Newton}. 
\boxedeq{m_i\b{\ddot{x}}_i=\b{f}_i,\quad i=1,\ldots ,n.}{eq:mov_newton}
\end{block}


\end{frame}

\begin{frame}{Sistemas conservativos}

\begin{block}{Definición} El sistema se llama conservativo si existe una función $U=U(\b{x},\b{y})$, con  $U:\rr^{3n}\times \rr^{3n}\to\rr$ tal que
\boxedeq{\b{f}_i(\b{x},\b{\dot{x}})=-\left.\frac{\partial U}{\partial \b{x}_i}\right|_{(\b{x},\b{y})=(\b{x},\b{\dot{x}})},\quad i=1,\ldots,n.}{eq:conservativo}
El signo menos en el segundo miembro es sólo una convención. 

Las derivadas del miembro de  la derecha en \eqref{eq:conservativo} hay que entenderlas como que presuponen las tres identidades escalares 
\[f_{i,j}(\b{x},\b{\dot{x}})=\left.\frac{\partial U}{\partial x_{i,j}}\right|_{(\b{x},\b{y})=(\b{x},\b{\dot{x}})},\quad i=1,\ldots,n; j=1,2,3.\]

\end{block}

\end{frame}



\begin{frame}{Ecuaciones de Euler-Lagrange}
En un sistema conservativo se define la energía cinética $T:\rr^{3n}\to\rr$ por
\[T(\b{y})=\sum_{i=1}^nm_i\frac{|\b{y}_i|}{2},\quad \b{y}=(\b{y}_1,\ldots,\b{y}_n)\in\rr^{3n}\]
y la energía potencial por $U$. 



Vamos a definir la función de Lagrange o Lagrangiano por 
\boxedeq{L(t,\b{x},\b{y})=\sum_{i=1}^nm_i\frac{|\b{y}_i|}{2} - U(\b{x},\b{y})}{eq:energia_cinetica}

\href{https://es.wikipedia.org/wiki/Ecuaciones_de_Euler-Lagrange}{Ecuaciones de Euler-Lagrange}
\end{frame}


\end{document}
