%\documentclass{article}
%\documentclass[hyperref={colorlinks=true}]{beamer}
\documentclass[handout,hyperref={colorlinks=true}]{beamer}

\usecolortheme{crane}
\usetheme{Darmstadt}
\usefonttheme{structuresmallcapsserif}
%%%%%%%%%%%%%%%%%%%%%%%%%%%%%%Paquetes%%%%%%%%%%%%%%%%%%%%%%%%%%%%%%%%%%%%%%%%%%%%%%%5
%%%%%%%%%%%%%%%%%%%%%%%%%%%%%%%%%%%%%%%%%%%%%%%%%%%%%%%%%%%%%%%%%%%%%%%%%%%%%%%%%%%%%
%\usepackage{pgfpages}
%\pgfpagesuselayout{2 on 1}[a4paper,border shrink=5mm]
\usepackage{empheq}
\usepackage[spanish]{babel}
\usepackage[utf8x]{inputenc}
\usepackage{times}
%\usepackage[T1]{fontenc}
\usepackage{amssymb,amsmath}
\usepackage{enumerate}
\usepackage{verbatim}
\usepackage{ esint }
%\usepackage{pst-all}
%\usepackage{pstricks-add}
\usepackage{array}
%\usepackage[T1]{fontenc}
%\usepackage{animate}
%\usepackage{media9}
\usepackage{xparse}
\usepackage{listings}
\usepackage{ wasysym }
%\usepackage{sagetex}
\usepackage{yfonts,mathrsfs,eufrak}
\usepackage{hyperref}

%%%%%%%%%%%%%%%%%%%%%%%%%Configuracion listing
\lstdefinelanguage{Sage}[]{Python}
{morekeywords={False,sage,True},sensitive=true}
\lstset{
frame=none,
showtabs=False,
showspaces=False,
showstringspaces=False,
commentstyle={\ttfamily\color{dgreencolor}},
keywordstyle={\ttfamily\color{dbluecolor}\bfseries},
stringstyle={\ttfamily\color{dgraycolor}\bfseries},
language=Sage,
basicstyle={\fontsize{8pt}{8pt}\ttfamily},
aboveskip=.3em,
belowskip=0.1em,
numbers=none,
numberstyle=\footnotesize
}

\AtBeginSection[]
{
  \begin{frame}
    \frametitle{Índice}
    \tableofcontents[currentsection]
  \end{frame}
}


%%%%%%%%%%%%%%%%%%%%%%%%Colores
\definecolor{myblue}{rgb}{.8, .8, 1}
\definecolor{dblackcolor}{rgb}{0.0,0.0,0.0}
\definecolor{dbluecolor}{rgb}{0.01,0.02,0.7}
\definecolor{dgreencolor}{rgb}{0.2,0.4,0.0}
\definecolor{dgraycolor}{rgb}{0.30,0.3,0.30}
\newcommand{\dblue}{\color{dbluecolor}\bf}
\newcommand{\dred}{\color{dredcolor}\bf}
\newcommand{\dblack}{\color{dblackcolor}\bf}
%%%%%%%%%%%%%%%%%%%%%%%%%%Nuevos comandos entornos%%%%%%%%%%%%%%%%%%%%%%%%%%%%%%%%
%%%%%%%%%%%%%%%%%%%%%%%%%%%%%%%%%%%%%%%%%%%%%%%%%%%%%%%%%%%%%%%%%%%%%%%%
\newenvironment{demo}{\noindent\emph{Dem.}}{$\square$ \newline\vspace{5pt}}
\newcommand{\com}{\mathbb{C}}
\newcommand{\dis}{\mathbb{D}}
\newcommand{\rr}{\mathbb{R}}
\newcommand{\oo}{\mathcal{O}}
\renewcommand{\emph}[1]{\textcolor[rgb]{1,0,0}{#1}}
\newcommand{\der}[2]{\frac{\partial #1}{\partial #2}}
\renewcommand{\v}[1]{\overrightarrow{#1}}
\renewcommand{\b}[1]{\boldsymbol{#1}}
\renewcommand{\epsilon}{\varepsilon}
\newlength\mytemplen
\newsavebox\mytempbox
\makeatletter
\newcommand\mybluebox{%
\@ifnextchar[%]
{\@mybluebox}%
{\@mybluebox[0pt]}}
\def\@mybluebox[#1]{%
\@ifnextchar[%]
{\@@mybluebox[#1]}%
{\@@mybluebox[#1][0pt]}}
\def\@@mybluebox[#1][#2]#3{
\sbox\mytempbox{#3}%
\mytemplen\ht\mytempbox
\advance\mytemplen #1\relax
\ht\mytempbox\mytemplen
\mytemplen\dp\mytempbox
\advance\mytemplen #2\relax
\dp\mytempbox\mytemplen
\colorbox{myblue}{\hspace{1em}\usebox{\mytempbox}\hspace{1em}}}
\makeatother
\DeclareDocumentCommand\boxedeq{ m g }{%
{\begin{empheq}[box={\mybluebox[2pt][2pt]}]{equation}% #1%
\IfNoValueF {#2} {\label{#2}}%
#1
\end{empheq}
}%
}
\DeclareMathOperator{\atan2}{atan2}
\DeclareMathOperator{\sen}{sen}
\newtheorem{teorema}{Teorema}[section]
\newtheorem{lema}[teorema]{Lema}
\newtheorem{corolario}[teorema]{Corolario}
\newtheorem{proposicion}[teorema]{Proposici\'on}
\newtheorem{definicion}[teorema]{Definici\'on}
%%%%%%%%%%%%%%%%%%%%%%%%%%%%%%%%%%%%%%%%%%%%%%%%%%%%%%%%%%%%%%%%%%%%%%%%%%%%%%%%%%%%%%%%%%%%%%%%%%%%%%%%%%%
%%%%%%%%%%Para escibir en clase articulo o similar
% \usepackage{color}
% \newcommand{\nl}{ }
% \renewenvironment{frame}[1]{}{}
% \newcommand{\qed}{$\square$}
% %\newcommand{\defverbatim}{\def{#1}}
% \newenvironment{block}[1]{\textbf{#1}}{}
% \title{Ecuaciones lineales de segundo orden}
% \author{Fernando Mazzone}
%
%%%%%%%%%%%%%%%%%%%%%%%Para clase beamer
\newcommand{\nl}{\onslide<+-> }

\title[Cálculo de variaciones] % (optional, nur bei langen Titeln nötig)
{%
Introducción y motivaciones del cálculo de variaciones
}



\author[] % (optional, nur bei vielen Autoren)
{Fernando Mazzone}

\institute[Depto de Matemática] % (optional, aber oft nötig)
{
 Dpto de Matemática\\
Facultad de Ciencias Exactas Físico-Químicas y Naturales\\
Universidad Nacional de Río Cuarto
 Dpto de Matemática\\
Facultad de Ciencias Exactas y Naturales\\
Universidad Nacional de La Pampa\\
CONICET}


\subject{Cálculo de Variaciones}

%%%%%%%%%%%%%%%%%%%%%%%%%%%%%%%%%%%%%%%%%%%%%%%%%%%%%%%%%%%%%%%%%%%%%%%%%%%%%%%%%%%%%%


\begin{document}
\begin{frame}
  \maketitle
  \begin{center}
   \includegraphics[scale=0.2]{imagenes/unrc.jpg}
   \end{center}
\end{frame}
\begin{frame}
    \frametitle{Índice}
\tableofcontents

\end{frame}

\section{Cálculo de variaciones y mecánica}
\begin{frame}{Ecuaciones de Newton}


\onslide<+->\textbf{Sistema mecánico:} $n$-puntos masa en un espacio euclideano tridimensional. Supuesto un sistema de coordenadas cartesiano, sean $\b{x}_1,\b{x}_2,\ldots,\b{x}_n\in\rr^3$ las coordenadas de los puntos masa, $\b{x}_i=(x_{i,1},x_{i,2},x_{i,3})$, $i=1,\ldots,n$. Vamos a poner $\b{x}= (\b{x}_1,\b{x}_2,\ldots,\b{x}_n)\in\rr^{3n}$. Las variables $\b{x}_i$ dependen del tiempo $t$. A la función $t\mapsto \b{x}(t)$ la denominamos un movimiento.

\onslide<+->\textbf{Fuerzas:}  Supongamos que actúan fuerzas $\b{f}_i=\b{f}_i(t,\b{x}(t),\b{\dot{x}}(t))$ sobre cada masa $m_i$.

\onslide<+->\begin{block}{Leyes de movimiento de Newton} Suponiendo que el sistema satisface la \href{https://es.wikipedia.org/wiki/Leyes_de_Newton\#Segunda_ley_de_Newton_o_ley_de_fuerza}{segunda ley de Newton}. 
\boxedeq{m_i\b{\ddot{x}}_i=\b{f}_i,\quad i=1,\ldots ,n.}{eq:mov_newton}
\end{block}


\end{frame}

\begin{frame}{Sistemas conservativos}

\begin{block}{Definición} El sistema se llama conservativo si existe una función $U=U(\b{x},\b{y})$, con  $U:\rr^{3n}\times \rr^{3n}\to\rr$ tal que
\boxedeq{\b{f}_i(\b{x},\b{\dot{x}})=-\left.\frac{\partial U}{\partial \b{x}_i}\right|_{(\b{x},\b{y})=(\b{x},\b{\dot{x}})},\quad i=1,\ldots,n.}{eq:conservativo}
El signo menos en el segundo miembro es sólo una convención. 

Las derivadas del miembro de  la derecha en \eqref{eq:conservativo} hay que entenderlas como que presuponen las tres identidades escalares 
\[f_{i,j}(\b{x},\b{\dot{x}})=\left.\frac{\partial U}{\partial x_{i,j}}\right|_{(\b{x},\b{y})=(\b{x},\b{\dot{x}})},\quad i=1,\ldots,n; j=1,2,3.\]

\end{block}

\end{frame}



\begin{frame}{Lagrangiano}
En un sistema conservativo se define la energía cinética $T:\rr^{3n}\to\rr$ por
\[T(\b{y})=\sum_{i=1}^nm_i\frac{|\b{y}_i|}{2},\quad \b{y}=(\b{y}_1,\ldots,\b{y}_n)\in\rr^{3n}\]
y la energía potencial por $U$. 



Vamos a definir la función de Lagrange o Lagrangiano por 
\boxedeq{L(t,\b{x},\b{y})=T-U=\sum_{i=1}^nm_i\frac{|\b{y}_i|^2}{2} - U(\b{x},\b{y})}{eq:energia_cinetica}


\end{frame}



\begin{frame}{Ecuaciones de Euler-Lagrange}

Las ecuaciones de Newton \eqref{eq:mov_newton} ahora se pueden escribir
\[\frac{d}{dt}\left(\frac{\partial L}{\partial \b{y}_i }\right)=\frac{\partial L}{\partial \b{x}_i },\quad i=1,\ldots,n.\]
O más sintéticamente
\boxedeq{\frac{d}{dt}\left(\frac{\partial L}{\partial \b{y} }\right)=\frac{\partial L}{\partial \b{x} }}{eq:EqEuler-Lagr}
 Estas ecuaciones se denominan 
\href{https://es.wikipedia.org/wiki/Ecuaciones_de_Euler-Lagrange}{ecuaciones de Euler-Lagrange}

\textbf{Ejercicio} Definimos la energía total del sistema por $E=T+U$. Demostrar que si $\b{x}(t)$ resuelve \eqref{eq:mov_newton} entonces $E(\b{x}(t),\b{\dot{x}}(t))$ no depende de $t$, i.e. la energía total del sistema se conserva.  

\end{frame}

\begin{frame}{Ejemplo}
Consideremos la \href{https://es.wikipedia.org/wiki/Resorte}{ecuación del resorte} u
 \href{https://es.wikipedia.org/wiki/Oscilador_arm\%C3\%B3nico}{oscilador armónico}
\[\ddot{x}(t)=-\omega^2 x(t).\]
En estas ecuaciones hemos dividido por la masa. Aquí el movimiento se realiza sobre una línea, de modo que un eje de coordenadas es suficiente para describir el movimiento $x(t)\in\rr$. El sistema es conservativo, pues
\[f=-\omega^2 x=-\frac{dU}{dx},\quad\text{donde } U=\omega^2\frac{x^2}{2}.\]
El lagrangiano es
\[L(t,x,y)=\frac{y^2}{2}-\omega^2\frac{x^2}{2}.\]
\end{frame}


\section{Problemas de contorno}
\begin{frame}{Problemas de contorno}
La ecuación \eqref{eq:EqEuler-Lagr} no caracteriza una única solución, hace falta introducir condiciones adicionales. Estamos interesados en considerar  problemas de contorno. Hay de distinto tipo, aquí algunos ejemplos. Supongamos $\b{x}_0,\b{x}_1\in\rr^n$,

\begin{description}
\item[\href{https://es.wikipedia.org/wiki/Condici\%C3\%B3n_de_frontera_de_Dirichlet}{Dirichlet}] $\b{x}(a)=\b{x}_0$ y $\b{x}(b)=\b{x}_1$. 
\item[\href{https://es.wikipedia.org/wiki/Condici\%C3\%B3n_de_frontera_de_Neumann}{Neumann}] $\b{\dot{x}}(a)=\b{x}_0$ y $\b{\dot{x}}(b)=\b{x}_1$. 
\item[\href{https://es.wikipedia.org/wiki/Condici\%C3\%B3n_de_frontera_de_Robin}{Robin} (Mixta)] dados $\alpha$,$\beta$,$\gamma$ y $\delta$ en $\rr$, con $\alpha^2+\beta^2\neq 0$ y $\gamma^2+\delta^2\neq 0$,
\[\begin{array}{cc}
\alpha\b{x}(a)+\beta\b{\dot{x}}(a)&=\b{x}_0\\
\gamma\b{x}(b)+\delta\b{\dot{x}}(b)&=\b{x}_1.
  \end{array}
\]
\item[Periódicas] $\b{x}(a)=\b{x}(b)$ y $\b{\dot{x}}(a)=\b{\dot{x}}(b)$.
\end{description}

\end{frame}

\section{Principio de mínima acción de Hamilton}

\begin{frame}{Principio de mínima acción de Hamilton}

\begin{block}{\href{https://es.wikipedia.org/wiki/Principio_de_m\%C3\%ADnima_acci\%C3\%B3n}{Principio de Mínima Acción de Hamilton}} Las soluciones del problema de contorno 

\[\left\{\begin{array}{l}
\frac{d}{dt}\left(\frac{\partial L}{\partial \b{y} }\right)=\frac{\partial L}{\partial \b{x} }\\
\b{x}(a)=\b{x}_0,\quad \b{x}(b)=\b{x}_1\\
 \end{array}
       \right.
\]
\eqref{eq:EqEuler-Lagr}
son puntos críticos  de la \href{https://es.wikipedia.org/wiki/Principio_de_m\%C3\%ADnima_acci\%C3\%B3n\#La_integral_de_acci.C3.B3n_para_part.C3.ADculas}{integral de acción} 
\boxedeq{I(\b{x})=\int_a^bL(t,\b{x}(t),\b{\dot{x}}(t))dt}{eq:integral_accion} 
para  
\[\b{x}\in C:= \{\b{u}\in H^1([a,b],\rr^n)|\quad\b{x}(a)=\b{x}_0\quad\text{y}\quad \b{x}(b)=\b{x}_1\}.\]

\end{block}

\textbf{Observación:} Así los mínimos y máximos de $I$ son solución. 

\end{frame}


\begin{frame}{Principio de Hamilton, idea de justificación  } 

Más adelante desarrollaremos esto con  detalle. 

$\b{x}\in C$  es un punto crítico de $I$ si cierta derivada de $I$  es igual a cero. 

En este caso tomamos $\b{r}(t)\in H^1([0,T])$ con $\b{r}(a)=\b{r}(b)=0\in\rr^n$ y definimos la derivada de $I$ en $\b{x}$ en la dirección de $\b{r}$ por

 \[DI(\b{x})\cdot \b{r}=\lim_{\epsilon\to 0}\frac{I(\b{x}+\epsilon \b{r})-I(\b{x})}{\epsilon}.\]

Notar que $\b{x}\in C$,  $\b{r}(a)=\b{r}(b)=0$ y $\epsilon\in\rr$ implican  $\b{x}+\epsilon\b{r}\in C$
\end{frame}




\begin{frame}{Principio de Hamilton, idea de justificación  } 
Suponiendo que podemos intercambiar integrales con límites
\begin{align} DI(\b{x})&\cdot \b{r}=\lim_{\epsilon\to 0}\frac{I(\b{x}+\epsilon \b{r})-I(\b{x})}{\epsilon}. &\text{(definición)}\notag\\
=& \int_a^b\lim_{\epsilon\to 0}\frac{L(t,\b{x}+\epsilon \b{r},\b{\dot{x}}+\epsilon \b{\dot{r}}  )- L(t,\b{x},\b{\dot{x}} )  }{\epsilon}dt&\text{(TCM?)}\notag\\
=& \int_a^b D_{\b{x}}L(t,\b{x},\b{\dot{x}})\cdot \b{r}+  D_{\b{y}}L(t,\b{x},\b{\dot{x}})\cdot \b{\dot{r}}dt&\text{(derivamos $L$)}\notag\\
=& \int_a^b \left( D_{\b{x}}L(t,\b{x},\b{\dot{x}})-\frac{d}{dt} D_{\b{y}}L(t,\b{x},\b{\dot{x}})\right) \cdot \b{r}dt&\text{(integral}\notag\\
&+ D_{\b{y}}L(t,\b{x},\b{\dot{x}}) \cdot \b{r}\left.\right|_{t=a}^{t=b} &\text{por partes)}\notag\\
=& \int_a^b \left( D_{\b{x}}L(t,\b{x},\b{\dot{x}})-\frac{d}{dt} D_{\b{y}}L(t,\b{x},\b{\dot{x}})\right) \cdot \b{r}dt,& \small{(\b{r}(a)=\b{r}(b)=0)}\notag\\ \notag
\end{align}


\end{frame}
\begin{frame}{Principio de Hamilton, idea de justificación  } 
\begin{block}{Teorema} Sea $\b{f}:[a,b]\to\rr^n$ integrable, y supongamos que para toda $\b{r}\in C^{\infty}([a,b],\rr^n)$, con $\b{r}(a)=\b{r}(b)=0$, tenemos
\[\int_a^b\b{f}(t)\cdot \b{r}(t)dt=0,\]
entonces $\b{f}=0$, c.t.p.
\end{block}
\textbf{Ejercicio:} demostrar este teorema.

Aplicando el Teorema concluímos que 
\[ \forall \b{r}: DI(\b{x})\cdot \b{r}=0 \Leftrightarrow \frac{d}{dt} D_{\b{y}}L(t,\b{x},\b{\dot{x}})=D_{\b{x}}L(t,\b{x},\b{\dot{x}}),\, \text{c.t.p. } t\in[a,b] .\] \qed

Detalle pendiente: por qué se puede intercambiar los límtes con las intergales?
\end{frame}







\section{Ejemplos, problema de contorno}
\begin{frame}{Ecuaciones escalares}
   Las ecuaciones escalares de segundo orden
\[u''+f(u)=h(t),\]
$u:[a,b]\to\rr$ son siempre conservativas. Al menos si $f:\rr\to\rr$ es continua, pues $f$ tiene la primitiva
\[F(u)=\int_0^uf(s)ds\] 
Así tenemos el potencial
\[U(t,u)=F(u)-uh(t).\]
\end{frame}




\begin{frame}{Problema de Dirichlet ecuación lineal  no resonante}
\textbf{Ejemplo}
\[\left\{ \begin{array}{l}
            u''(t)+\frac12 u=\sen(t),\quad 0<t<\pi\\
            u(0)=u(\pi)=0
          \end{array}
          \right.
\]

Por medios elementales hallamos la solución general de la ecuación diferencial. Usamos el método de coeficientes indeterminados. Proponemos 
\[u(t)=A\cos(t)+B\sen(t).\]
reemplazamos en la ecuación y hallamos
\[\fbox{A=0}, \fbox{B=-2}\]


\end{frame}





\begin{frame}{Problema de Dirichlet ecuación lineal  no resonante}
La solución general es la solución particular más una general del homogéneo
\[u=c_1\cos\left(\frac{t}{\sqrt{2}}\right)+c_2\sen\left(\frac{t}{\sqrt{2}}\right)-2\sen(t).\]
La condición de contorno $ u(0)=u(\pi)=0$ implica $c_1=c_2=0$. El problema tiene una única solución

\[\boxed{u=-2\sen(t).}\]

El lagrangiano es

\[L(t,x,y)=\frac{y^2}{2}-\frac{x^2}{4}+\sen(t) x.\]
\end{frame}

\begin{frame}{Problema de Dirichlet ecuación lineal  no resonante}
La solución general es la solución particular más una general del homogéneo
\[u=c_1\cos\left(\frac{t}{\sqrt{2}}\right)+c_2\sen\left(\frac{t}{\sqrt{2}}\right)-2\sen(t).\]
La condición de contorno $ u(0)=u(\pi)=0$ implica $c_1=c_2=0$. El problema tiene una única solución

\[\boxed{u=-2\sen(t).}\]

El lagrangiano es

\[L(t,x,y)=\frac{y^2}{2}-\frac{x^2}{4}+\sen(t) x.\]
\end{frame}




\end{document}
