\documentclass{article}
\usepackage{amssymb,amsmath}

\usepackage[spanish, activeacute]{babel}
\usepackage{theorem}
\usepackage{times}
\usepackage{array}
\usepackage{graphicx}
\usepackage{hyperref}
\usepackage{multirow}
\usepackage[cp1252]{inputenc}
\usepackage{hhline}
\usepackage{multicol}

%%%%%%%%%Estilo de la pagina%%%%%%%%%%%%%%%%%%%%%%%%%%%%%%%%%%%
%%%%%%%%%%%%%%%%%%%%%%%%%%%%%%%%%%%%%%%%%%%%%%%%%%%%%%%%%%%%%%%%%%
\newcounter{ejer}

{\theorembodyfont{\rmfamily}
\newtheorem{ejercicio}[ejer]{Ejercicio}}

\newcommand{\rr}{\mathbb{R}}
\newcommand{\qq}{\mathbb{Q}}
\newcommand{\nn}{\mathbb{N}}
\newcommand{\C}[1]{\overline{#1}}
\newcommand{\zz}{\mathbb{Z}}






\begin{document}


\hyphenation{excen-tri-ci-dad}
 %%%%%%%%%Estilo de la pagina%%%%%%%%%%%%%%%%%%%%%%%%%%%%%%%%%%%
%%%%%%%%%%%%%%%%%%%%%%%%%%%%%%%%%%%%%%%%%%%%%%%%%%%%%%%%%%%%%%%%%%
\setlength{\unitlength}{1cm}
%
\setlength{\extrarowheight}{5mm}
%

\noindent\begin{tabular}{m{.1\textwidth} m{.9\textwidth}}\hline\hline
\includegraphics[scale=.4]{unrc.jpg} &
%\begin{large}
\begin{bfseries}  \begin{scshape}
Facultad de Cs. Exactas, F�sico-Qu�micas y Naturales\par
        Depto de Matem\'atica.\par
        Primer Cuatrimestre de 2015\par
        Ecuaciones Diferenciales (1913)\par
        Pr�ctica 3: Ecuaciones Diferenciales  Lineales 

\end{scshape}
\end{bfseries}
%\end{large}
\\ 
\hline\hline
\end{tabular}
\renewcommand{\theenumi}{\alph{enumi}}


\setlength{\extrarowheight}{-5mm}
\hyphenation{ho-meo-mor-fo}

Realizar los siguientes ejercicios de \cite[Cap�tulo 3]{simmons-esp} (notar que hago alusi�n a la versi�n en espa�ol). 

\begin{itemize}

\item  Secci�n 14, Pag 89-90, Ejercicios: 6a,6b,6c,9 y 10.
\item Secci�n 15, Pag 94-95, Ejercicios: 2, 4,8,9 y 10.
\item Secci�n 16, Pag 97-98, Ejercicios: 5, 6, 8 y 12.
\item Secci�n 17, Pag. 101-102, Ejercicios: 1,2 (elegir algunos) y Ejercicios: 3,4,5 y 6.
\item Secci�n 18, Pag. 107, Ejercicio 2.
\item Secci�n 19, Pag. 110, Ejercicios: 1,3 (elegir algunos) y 5.
\item Secci�n 20, Pag. 119, Ejercicio 3 (donde dice esf�rica poner cil�ndrica) , 4 y 5.
\item Secci�n 21, Pag. 128, Ejercicios 4 y 5.
\item Secci�n 22, Pag. 134, Ejercicios 17 y 18.
\item Se sabe que el resorte no amortiguado ni forzado, cuya ecuaci�n es $my''+ky=0$ es un sistema conservativo. Esto es que la energ�a se conserva

\[E=m\frac{{y'}^2}{2}+ky=\hbox{cte}.\]

Usar SAGE y estudiar qu� ocurre con la energ�a de un resorte amortiguado y no forzado y la de uno forzado y no amortiguado.


\end{itemize}

\bibliographystyle{plain}
\bibliography{biblio}

\end{document}
