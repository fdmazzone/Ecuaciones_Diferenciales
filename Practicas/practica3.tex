\documentclass{article}
\usepackage{amssymb,amsmath}

\usepackage[spanish, activeacute]{babel}
\usepackage{theorem}
\usepackage{times}
\usepackage{array}
\usepackage{graphicx}
\usepackage{hyperref}
\usepackage{multirow}
\usepackage[cp1252]{inputenc}
\usepackage{hhline}
\usepackage{multicol}
\usepackage{bm}
\usepackage{natbib}
\usepackage{url}
\usepackage{tabularx}
%%%%%%%%%Estilo de la pagina%%%%%%%%%%%%%%%%%%%%%%%%%%%%%%%%%%%
%%%%%%%%%%%%%%%%%%%%%%%%%%%%%%%%%%%%%%%%%%%%%%%%%%%%%%%%%%%%%%%%%%
\newcounter{ejer}

{\theorembodyfont{\rmfamily}
\newtheorem{ejercicio}[ejer]{Ejercicio}}

\newcommand{\rr}{\mathbb{R}}
\newcommand{\qq}{\mathbb{Q}}
\newcommand{\nn}{\mathbb{N}}
\newcommand{\C}[1]{\overline{#1}}
\newcommand{\zz}{\mathbb{Z}}
\newcommand\eme {\mbox{\tiny\sc M}}
\newcommand\bfm {\mbox{\tiny\sc BFM}}
\newcommand\btuk {\mbox{\tiny\sc BT}}
\newcommand\rice {\mbox{\tiny\sc R}}
\newcommand\pc {\mbox{\tiny\sc PC}}
\newcommand\gasser {\mbox{\tiny\sc G}}
\newcommand\hall {\mbox{\tiny\sc H}}
\newcommand\h {\widehat{h}_n}
\newcommand\ta {\widehat{\tau}_n}
\newcommand\ho {h_{ {\mbox{\tiny opt}},n}}
\newcommand{\cs}{  \stackrel{c.s.}\longrightarrow  }
\newcommand{\pro}{  \stackrel{p}\longrightarrow  }
\newcommand{\di}{  \stackrel{d}\longrightarrow  }
\newcommand{\luno}{  \stackrel{L_1}\longrightarrow  }
\newcommand{\ldos}{  \stackrel{L_2}\longrightarrow  }
\newcommand{\comp}{  \stackrel{c}\longrightarrow  }
\newcommand{\seno}{\hbox{sen}}
\DeclareMathOperator{\sen}{sen}




\begin{document}


\hyphenation{excen-tri-ci-dad}
 %%%%%%%%%Estilo de la pagina%%%%%%%%%%%%%%%%%%%%%%%%%%%%%%%%%%%
%%%%%%%%%%%%%%%%%%%%%%%%%%%%%%%%%%%%%%%%%%%%%%%%%%%%%%%%%%%%%%%%%%
\setlength{\unitlength}{1cm}
%
%\setlength{\extrarowheight}{5mm}
%

\noindent\begin{tabular}{m{.1\textwidth} m{.9\textwidth}}\hline\hline
\includegraphics[scale=.4]{unrc.jpg} &
%\begin{large}
\begin{bfseries}  \begin{scshape}
Facultad de Cs. Exactas, F�sico-Qu�micas y Naturales\par
        Depto de Matem\'atica.\par
        Primer Cuatrimestre de 2015\par
        Ecuaciones Diferenciales (1913) \par
        Pr�ctica 3: Simetr�as.

\end{scshape}
\end{bfseries}
%\end{large}
\\
\hline\hline
\end{tabular}
\renewcommand{\theenumi}{\alph{enumi}}



\setlength{\unitlength}{1cm}
%
%\setlength{\extrarowheight}{5mm}

\begin{ejercicio}\label{ejer: 1} Demostrar que las siguientes aplicaciones inducen grupos de Lie uniparam�tricos
\begin{enumerate}
\item\label{item:1_ejer1} $\Gamma_{\epsilon}(x,y)=(x+\epsilon,y)$ y $\Gamma_{\epsilon}(x,y)=(x,y+\epsilon)$.
\item\label{item:2_ejer1} $\Gamma_{\epsilon}(x,y)=(e^{\epsilon}x,y)$
\item$\Gamma_{\epsilon}(x,y)=\left(\frac{x}{1-\epsilon x},\frac{y}{1-\epsilon x} \right)$
\item\label{item:4_ejer1}$\Gamma_{\epsilon}(x,y)=\begin{pmatrix} \cos(\epsilon) & -\sen(\epsilon)
\\ \sen(\epsilon) & \cos(\epsilon)
\end{pmatrix} \begin{pmatrix} x\\ y
\end{pmatrix}
$
\end{enumerate}
\end{ejercicio}

\begin{ejercicio} Encontrar coordenadas can�nicas para las simetr�as de los incisos 
\ref{item:1_ejer1} y \ref{item:2_ejer1} del ejercicio \ref{ejer: 1}. Repetir este mismo c�lculo pero usando \texttt{SymPy} (o \texttt{SAGE}) con los incisos \ref{item:1_ejer1}, \ref{item:2_ejer1} y \ref{item:4_ejer1}.
\end{ejercicio}

\begin{ejercicio} Ejercicios 1.1, 1.2, 1.4 y 1.5 de \cite{hydon2000symmetry}
\end{ejercicio}
\begin{ejercicio}  Ejercicio 2.1, 2.2, 2.3, 2.5 y 2.6 de \cite{hydon2000symmetry}
\end{ejercicio}


\bibliographystyle{plain}
\bibliography{biblio}
\end{document}
